%% Generated by Sphinx.
\def\sphinxdocclass{report}
\documentclass[letterpaper,12pt,english,openany,oneside]{sphinxmanual}
\ifdefined\pdfpxdimen
   \let\sphinxpxdimen\pdfpxdimen\else\newdimen\sphinxpxdimen
\fi \sphinxpxdimen=.75bp\relax
%% turn off hyperref patch of \index as sphinx.xdy xindy module takes care of
%% suitable \hyperpage mark-up, working around hyperref-xindy incompatibility
\PassOptionsToPackage{hyperindex=false}{hyperref}

\PassOptionsToPackage{warn}{textcomp}

\catcode`^^^^00a0\active\protected\def^^^^00a0{\leavevmode\nobreak\ }
\usepackage{cmap}
\usepackage{fontspec}
\defaultfontfeatures[\rmfamily,\sffamily,\ttfamily]{}
\usepackage{amsmath,amssymb,amstext}
\usepackage{polyglossia}
\setmainlanguage{english}



\setmainfont{FreeSerif}[
  Extension      = .otf,
  UprightFont    = *,
  ItalicFont     = *Italic,
  BoldFont       = *Bold,
  BoldItalicFont = *BoldItalic
]
\setsansfont{FreeSans}[
  Extension      = .otf,
  UprightFont    = *,
  ItalicFont     = *Oblique,
  BoldFont       = *Bold,
  BoldItalicFont = *BoldOblique,
]
\setmonofont{FreeMono}[
  Extension      = .otf,
  UprightFont    = *,
  ItalicFont     = *Oblique,
  BoldFont       = *Bold,
  BoldItalicFont = *BoldOblique,
]


\usepackage[Bjarne]{fncychap}
\usepackage{sphinx}

\fvset{fontsize=\small}
\usepackage{geometry}


% Include hyperref last.
\usepackage{hyperref}
% Fix anchor placement for figures with captions.
\usepackage{hypcap}% it must be loaded after hyperref.
% Set up styles of URL: it should be placed after hyperref.
\urlstyle{same}


\usepackage{sphinxmessages}
\setcounter{tocdepth}{1}

        \usepackage{tocloft}
        \usepackage{tabularx}
        \usepackage{titlesec}
        \usepackage{titling}
        \usepackage{fancyhdr}
        \pagestyle{fancy}
        \usepackage{graphicx}
        \usepackage{fontspec}
        \setmainfont{AdobeClean-Regular}
        \makeatletter
    \fancypagestyle{normal}{
        \fancyhf{}
        \fancyfoot[LE,RO]{{\py@HeaderFamily\thepage}}
        \fancyfoot[LO,RE]{{\textcopyright\ 2021, Adobe Inc.}}
        \fancyhead[LE,RO]{{\py@HeaderFamily \@title\sphinxheadercomma\py@release}}
        \renewcommand{\headrulewidth}{0pt}
        \renewcommand{\footrulewidth}{0.4pt}
    }
    \fancypagestyle{plain}{
        \fancyhf{}
        \fancyfoot[LE,RO]{{\py@HeaderFamily\thepage}}
        \renewcommand{\headrulewidth}{0pt}
        \renewcommand{\footrulewidth}{0.4pt}
        \fancyfoot[LO,RE]{{\textcopyright\ 2021, Adobe Inc.}}
    }
\makeatother
    

\title{PDFMark Reference}
\date{Apr 26, 2022}
\release{}
\author{unknown}
\newcommand{\sphinxlogo}{\vbox{}}
\renewcommand{\releasename}{}
\makeindex
\begin{document}

\pagestyle{empty}

    \begin{titlepage}
        \begin{figure}[h]
        \centering{\includegraphics[scale=1.5]{../images/adobelogo.png}}
        \end{figure}
        \centering
        \vspace*{40mm}
        \textbf{\Huge PDFMark Reference}

        \vspace{15mm}
        \Large \textbf{{This PDF is programmatically generated: Review copy only}}
        \vfill
        \small \textcopyright\ 2022, Adobe Inc.
    \end{titlepage}
    \clearpage
    \tableofcontents
    \clearpage
    
\pagestyle{plain}

\pagestyle{normal}
\phantomsection\label{\detokenize{toc::doc}}





\chapter{PDFMark Reference}
\label{\detokenize{index:pdfmark-reference}}\label{\detokenize{index::doc}}
The Adobe Acrobat Distiller application is a PostScript language interpreter that converts PostScript language files into Adobe PDF files. Because PDF has features (such as annotations, bookmarks, articles, and forms) that are not expressible using the standard PostScript operators, some PostScript language extension is necessary to describe features that are present in PDF but not in standard PostScript.

To satisfy this need, Distiller supports the pdfmark operator, which is not supported by standard PostScript interpreters. The use of pdfmark allows an independent software vendor (ISV) already using the PostScript language to express, in PostScript syntax, idioms that are converted by Distiller to PDF without having to write PDF files directly.


\section{What’s in this guide?}
\label{\detokenize{index:whats-in-this-guide}}
This document describes the syntax and use of the pdfmark operator, and contains examples of many of the features that can be implemented using pdfmark.


\section{Who should read this guide?}
\label{\detokenize{index:who-should-read-this-guide}}
This document is intended for those who use pdfmark constructs in PostScript code intended for conversion to PDF by Distiller.

pdfmark Reference


\chapter{Introduction}
\label{\detokenize{pdfmark_Syntax:introduction}}\label{\detokenize{pdfmark_Syntax::doc}}
This chapter describes the pdfmark operator, its syntax, and its use by Distiller and other PostScript interpreters. It also describes how built\sphinxhyphen{}in and user\sphinxhyphen{}defined PDF objects are referred to and defined.

The pdfmark operator is a PostScript\sphinxhyphen{}language extension that describes features that are present in PDF, but not in standard PostScript. The pdfmark operator has been available beginning with Distiller 3.0, and, as an extensible operator, has evolved with each release of the PDF specification. This document describes pdfmark as it applies to the Acrobat 9.0 and later suite of products.

\begin{sphinxadmonition}{note}{Note:}
While the pdfmark operator provides for greater extensibility, it is not intended to define every feature that is present in PDF but not in standard PostScript.
\end{sphinxadmonition}

Those using pdfmark typically do so in one of the following ways:
\begin{itemize}
\item {} 
By manual creation or modification of PostScript code

\item {} 
By filtering or post\sphinxhyphen{}processing existing PostScript code

\item {} 
By an application that directly generates pdfmark constructs as part of its PostScript code generation

\end{itemize}


\section{Syntax of pdfmark operators}
\label{\detokenize{pdfmark_Syntax:syntax-of-pdfmark-operators}}
The pdfmark operator requires the following syntax:

\begin{sphinxVerbatim}[commandchars=\\\{\}]
\PYG{p}{[} \PYG{n+nb}{any}
\PYG{l+m+mi}{1}
 \PYG{o}{.}\PYG{o}{.}\PYG{o}{.} \PYG{n+nb}{any}
\PYG{n}{n}
 \PYG{n}{feature} \PYG{n}{pdfmark}
\end{sphinxVerbatim}

The syntax has the following characteristics:
\begin{itemize}
\item {} 
It begins with a mark object (either \sphinxcode{\sphinxupquote{mark}} or \sphinxcode{\sphinxupquote{{[}}} ).

\item {} 
It is followed by zero or more PostScript objects called the arguments of the pdfmark operator.

\item {} 
It concludes with a name object that indicates the particular feature that the pdfmark operator is to apply.

\end{itemize}

Any instance of the pdfmark operator, the mark, its arguments, and the feature name in a PostScript program is referred to as a pdfmark in this document.

Frequently, the arguments for a given feature are sequences of key\sphinxhyphen{}value pairs. Many of the pdfmark keys correspond directly to PDF dictionary keys. However, some keys may be new, entirely different, or abbreviated forms of keys as found in PDF dictionaries. For example, the PDF \sphinxcode{\sphinxupquote{Subtype}} key may become the pdfmark key \sphinxcode{\sphinxupquote{S}} , the \sphinxcode{\sphinxupquote{Dest}} key may become \sphinxcode{\sphinxupquote{D}} , and the \sphinxcode{\sphinxupquote{File}} key may become \sphinxcode{\sphinxupquote{F}} , and so forth. See the \sphinxstyleemphasis{PDF Reference} for more information on PDF keys.

The pdfmark operator does not change the operand nor the dictionary stacks, but may alter the execution, graphics state, or clipping stacks, depending on the particular pdfmark feature.


\section{Usage with standard PostScript interpreters}
\label{\detokenize{pdfmark_Syntax:usage-with-standard-postscript-interpreters}}
Support for the pdfmark operator is implemented in Distiller, but is not available in many other PostScript products. Therefore, if a PostScript program containing pdfmark constructs is to be used by other such products, they must be able to respond appropriately when they encounter the pdfmark operator.

One reasonable response is to ignore pdfmark constructs. This can be accomplished by defining a pdfmark procedure that discards the pdfmark code for interpreters in which the pdfmark operator does not exist. One way to do this is to place the following code in the prolog of a PostScript program.
\begin{enumerate}
\sphinxsetlistlabels{\arabic}{enumi}{enumii}{}{.}%
\item {} 
Ignoring pdfmark constructs

\end{enumerate}

\begin{sphinxVerbatim}[commandchars=\\\{\}]
 \PYGZpc{}\PYGZpc{}BeginProlog
  /pdfmark where
\PYGZpc{} Is pdfmark already available?

      \PYGZob{} pop \PYGZcb{}
\PYGZpc{} Yes: do nothing (use that definition)

      \PYGZob{}
\PYGZpc{} No: define pdfmark as follows:

      /globaldict where
\PYGZpc{} globaldict is preferred because

          \PYGZob{} pop globaldict \PYGZcb{}
\PYGZpc{} globaldict is always visible; else,

          \PYGZob{} userdict \PYGZcb{}
\PYGZpc{} use userdict otherwise.

      ifelse
      /pdfmark /cleartomark load put
      \PYGZcb{}
\PYGZpc{} Define pdfmark to remove all objects

  ifelse
\PYGZpc{} up to and including the mark object.

  \PYGZpc{}\PYGZpc{}EndProlog
\end{sphinxVerbatim}

This code example works on PostScript Level 1 and higher interpreters. To simplify the presentation of the following examples, PostScript Level 2 or higher is assumed.

Here is a similar example.
\begin{enumerate}
\sphinxsetlistlabels{\arabic}{enumi}{enumii}{}{.}%
\item {} 
Ignoring pdfmark if not defined in the PostScript interpreter

\end{enumerate}

\begin{sphinxVerbatim}[commandchars=\\\{\}]
\PYGZpc{}\PYGZpc{}BeginProlog
/pdfmark where
  \PYGZob{} pop globaldict /?pdfmark /exec load put \PYGZcb{}
\PYGZpc{} pdfmark is built\PYGZhy{}in: exec code.

  \PYGZob{}
  globaldict
      begin
      /?pdfmark /pop load def
\PYGZpc{} pdfmark is absent: ignore code.

      /pdfmark /cleartomark load def
      end
  \PYGZcb{}
ifelse
\PYGZpc{}\PYGZpc{}EndProlog
\end{sphinxVerbatim}

Most pdfmark features are atomic. That is, the pdfmark construct stands alone and, if removed, does not affect surrounding PostScript code. A few pdfmark features, on the other hand, are modal. A modal feature is one that, once completed, leaves the interpreter in a different state. Most modal features are paired: one feature shifts to a new state and a corresponding feature shifts back to the previous state. For example, consider:

\begin{sphinxVerbatim}[commandchars=\\\{\}]
\PYG{p}{[} \PYG{n+nb}{any}
\PYG{l+m+mi}{1}
 \PYG{o}{.}\PYG{o}{.}\PYG{o}{.} \PYG{n+nb}{any}
\PYG{n}{n}
 \PYG{o}{/}\PYG{n}{BeginFeature} \PYG{n}{pdfmark}

\PYG{n}{additional} \PYG{n}{PostScript} \PYG{n}{code}

 \PYG{p}{[} \PYG{n+nb}{any}
\PYG{l+m+mi}{1}
 \PYG{o}{.}\PYG{o}{.}\PYG{o}{.} \PYG{n+nb}{any}
\PYG{n}{m}
 \PYG{o}{/}\PYG{n}{EndFeature} \PYG{n}{pdfmark}
\end{sphinxVerbatim}

If you want to make the additional PostScript code conditional on the availability of the pdfmark operator, then the above definition of pdfmark needs to be improved.

\begin{sphinxVerbatim}[commandchars=\\\{\}]
 \PYGZpc{}\PYGZpc{}BeginProlog
  /pdfmark where
      \PYGZob{} pop globaldict /?pdfmark /exec load put \PYGZcb{}
\PYGZpc{} pdfmark is built\PYGZhy{}in: exec code.

      \PYGZob{}
      globaldict
          begin
          /?pdfmark /pop load def
\PYGZpc{} pdfmark is absent: ignore code.

          /pdfmark /cleartomark load def
          end
      \PYGZcb{}
  ifelse
  \PYGZpc{}\PYGZpc{}EndProlog
\end{sphinxVerbatim}

With this, the handling of modal code can be performed as:

\begin{sphinxVerbatim}[commandchars=\\\{\}]
[
any

1

 ... any

n
 /BeginFeature pdfmark
  \PYGZob{}
additional PostScript code
 \PYGZcb{} ?pdfmark
 [
any

1

 ... any

m
 /EndFeature pdfmark
\end{sphinxVerbatim}

While the above solution is sufficient in most circumstances, you might want to define a pdfmark procedure to handle individual features. The following example demonstrates a simple framework for handling individual pdfmark features:
\begin{enumerate}
\sphinxsetlistlabels{\arabic}{enumi}{enumii}{}{.}%
\item {} 
Handling individual pdfmark features

\end{enumerate}

\begin{sphinxVerbatim}[commandchars=\\\{\}]
 \PYGZpc{}\PYGZpc{}BeginProlog
  currentglobal currentpacking
\PYGZpc{} Because the pdfmark definition below uses

  true setglobal true setpacking
\PYGZpc{} composite objects, we need to make sure


\PYGZpc{} the procedure is defined in global VM mode.

  /pdfmark where
      \PYGZob{} pop globaldict /?pdfmark /exec load put\PYGZcb{}
      \PYGZob{}
      globaldict
          begin
          /?pdfmark /pop load def
          /pdfmark
              \PYGZob{}
                  \PYGZob{} counttomark pop \PYGZcb{}
\PYGZpc{} Check to see that a mark is on the stack.

              stopped
                  \PYGZob{} /pdfmark errordict /unmatchedmark get exec stop \PYGZcb{}
              if
\PYGZpc{} Raise an error if no mark is found.

              dup type /nametype ne
\PYGZpc{} The topmost argument must be the feature.

                  \PYGZob{} /pdfmark errordict /typecheck get exec stop \PYGZcb{}
              if
\PYGZpc{} The feature must be a name object.

                  \PYGZob{}
                  dup /FEATURE
1
 eq
                      \PYGZob{} (Interpreting FEATURE 1n) print cleartomark exit \PYGZcb{}
                  if
\PYGZpc{} Replace the above code with actual code

                  dup /
FEATURE

2
 eq
                      \PYGZob{} (Interpreting FEATURE 2n) print cleartomark exit \PYGZcb{}
                  if
\PYGZpc{} Replace the above code with actual code

                  (Feature not supported: ) print == cleartomark exit

\PYGZpc{} Replace the above code with actual code

                  \PYGZcb{}
              loop
              \PYGZcb{} bind def
          end
      \PYGZcb{}
  ifelse
  setpacking setglobal
\PYGZpc{} Restore to original modes.

  \PYGZpc{}\PYGZpc{}EndProlog
\end{sphinxVerbatim}

In the preceding code, the name objects \sphinxcode{\sphinxupquote{FEATURE}} \sphinxstyleemphasis{n} would be replaced with actual pdfmark feature names and the code that follows the \sphinxcode{\sphinxupquote{dup /FEATURE}} \sphinxstyleemphasis{n} \sphinxcode{\sphinxupquote{eq}} would be replaced with code that consumes all of the arguments and the mark object.

In the examples that follow in this document, the \sphinxcode{\sphinxupquote{?pdfmark}} definition is assumed to be as shown above. To work correctly with non\sphinxhyphen{}Distiller PostScript interpreters, any production implementation of these or additional definitions must take into account factors such as PostScript level, VM allocation modes, packing modes, and others.


\section{Syntax for private keys}
\label{\detokenize{pdfmark_Syntax:syntax-for-private-keys}}
Some features can accept arbitrary key–value pairs, providing a way of placing private data into PDF files. All keys must be name objects. Unless otherwise stated, values must be Boolean, number, string, name, array, or dictionary objects. Array elements must be Boolean, number, string, or name objects.

When specifying arbitrary key–value pairs, key names must contain a specific prefix to ensure that they do not collide with key names used by other developers. Contact Adobe to obtain a prefix to be used by your company or organization.

\begin{sphinxadmonition}{note}{Note:}
The private key names in this document use the Adobe prefix \sphinxcode{\sphinxupquote{ADBE}} .
\end{sphinxadmonition}


\section{Named objects}
\label{\detokenize{pdfmark_Syntax:named-objects}}
This section describes how built\sphinxhyphen{}in and user\sphinxhyphen{}defined PDF objects are referred to and defined.


\subsection{Built\sphinxhyphen{}in named objects}
\label{\detokenize{pdfmark_Syntax:built-in-named-objects}}
A PDF file contains built\sphinxhyphen{}in objects such as the Catalog and Page dictionaries. To refer to one of these dictionaries in a pdfmark construct, a syntax called a named object is used:

\begin{sphinxVerbatim}[commandchars=\\\{\}]
\PYG{p}{\PYGZob{}}\PYG{n}{objname}\PYG{p}{\PYGZcb{}}
\end{sphinxVerbatim}

The objname is one of:
\begin{itemize}
\item {} 
\sphinxstyleemphasis{Catalog} — The PDF file’s Catalog dictionary

\item {} 
\sphinxstyleemphasis{DocInfo} — The PDF file’s Info dictionary

\item {} 
\sphinxstyleemphasis{Page} \sphinxstyleemphasis{N} — The dictionary for page N (where N is a positive integer)

\item {} 
\sphinxstyleemphasis{ThisPage} — The dictionary for the current page being processed in the PostScript stream

\item {} 
\sphinxstyleemphasis{PrevPage} — The dictionary for the page before the current page

\item {} 
\sphinxstyleemphasis{NextPage} — The dictionary for the page after the current page

\end{itemize}

\begin{sphinxadmonition}{note}{Note:}
The objname used here is not a standard PostScript name object. It does not start with a slash “\sphinxcode{\sphinxupquote{/}} ” but instead is surrounded with braces “\sphinxcode{\sphinxupquote{\{\}}} “. It otherwise follows the syntax of PostScript name objects. The objname serves as a reference name to identify particular PDF objects and has no relationship to any identifier created in the resultant PDF file.
\end{sphinxadmonition}


\subsection{User\sphinxhyphen{}defined named objects}
\label{\detokenize{pdfmark_Syntax:user-defined-named-objects}}
In addition to built\sphinxhyphen{}in named objects, user\sphinxhyphen{}defined named objects can be created. The syntax to specify a user\sphinxhyphen{}defined named object is the following:

\begin{sphinxVerbatim}[commandchars=\\\{\}]
\PYG{p}{[} \PYG{o}{/}\PYG{n}{\PYGZus{}objdef} \PYG{p}{\PYGZob{}}
\PYG{n}{objname}
\PYG{p}{\PYGZcb{}} \PYG{o}{/}\PYG{n+nb}{type}
\PYG{n}{objtype}
\PYG{o}{/}\PYG{n}{OBJ} \PYG{n}{pdfmark}
\end{sphinxVerbatim}

The name \sphinxcode{\sphinxupquote{/\_objdef}} indicates that a named object is to be defined and is followed by the \{objname\}. The object type, objtype, specifies the PDF type of the named object that is to be created and must be one of the following name objects:
\begin{itemize}
\item {} 
\sphinxstyleemphasis{/array} — Creates an array.

\item {} 
\sphinxstyleemphasis{/dict} — Creates a dictionary.

\item {} 
\sphinxstyleemphasis{/stream} — Creates a stream.

\end{itemize}

\begin{sphinxadmonition}{note}{Note:}
The feature \sphinxcode{\sphinxupquote{/OBJ}} is used only to declare a particular objname and its associated type. Other pdfmark features are required to associate this objname with actual content and to have some existing PDF object refer to it.
\end{sphinxadmonition}

Here is an example in which the named object “galaxy” is declared to be a dictionary type:

\begin{sphinxVerbatim}[commandchars=\\\{\}]
\PYG{p}{[} \PYG{o}{/}\PYG{n}{\PYGZus{}objdef} \PYG{p}{\PYGZob{}}\PYG{n}{galaxy}\PYG{p}{\PYGZcb{}} \PYG{o}{/}\PYG{n+nb}{type} \PYG{o}{/}\PYG{n+nb}{dict} \PYG{o}{/}\PYG{n}{OBJ} \PYG{n}{pdfmark}
\end{sphinxVerbatim}

A few pdfmark features allow for the definition of a named object as part of the argument list. In these cases, the modified syntax is as follows:

\begin{sphinxVerbatim}[commandchars=\\\{\}]
 \PYG{p}{[} \PYG{o}{/}\PYG{n}{\PYGZus{}objdef} \PYG{p}{\PYGZob{}}
\PYG{n}{objname}
\PYG{p}{\PYGZcb{}}
\PYG{n+nb}{any}

\PYG{l+m+mi}{1}

 \PYG{o}{.}\PYG{o}{.}\PYG{o}{.} \PYG{n+nb}{any}

\PYG{n}{n}

 \PYG{n}{feature}
 \PYG{n}{pdfmark}
\end{sphinxVerbatim}

In this case, the objname is not only created, but also refers to the PDF object created as a result of the pdfmark feature. The \sphinxcode{\sphinxupquote{type}} entry is not used because the feature implicitly provides this information. The following features support this syntax:
\begin{itemize}
\item {} 
\sphinxstyleemphasis{ANN} — Annotation

\item {} 
\sphinxstyleemphasis{BP} — Encapsulated graphic

\item {} 
\sphinxstyleemphasis{DEST} — Named destination

\item {} 
\sphinxstyleemphasis{NI} — Encapsulated image

\item {} 
\sphinxstyleemphasis{StPNE} — Structure element

\end{itemize}

Named objects created in any of the preceding ways can be used in the definition of other named objects. That is, an \{objname\} can be used as an argument in a pdfmark construct as the value of a key–value pair or as an element in an array. In these cases, Distiller places an indirect reference to the object with which \{objname\} is associated in the PDF file.

\begin{sphinxadmonition}{note}{Note:}
A pdfmark construct can make an object reference to \{objname\} before defining the object \{objname\}. That is, the \{objname\} can be in the argument list of a pdfmark construct before it is defined. If \{objname\} is never defined, it is left as an unresolved reference in the cross\sphinxhyphen{}reference table. Hence, any consumer of such a PDF file must be able to handle unresolved references.
\end{sphinxadmonition}


\subsection{Namespaces}
\label{\detokenize{pdfmark_Syntax:namespaces}}
When using named objects in PostScript programs, it is possible that the same name might be used more than once. To avoid conflicts in name object definitions, Distiller provides a means for specifying the scope in which named objects have well\sphinxhyphen{}defined meaning.

In addition to the standard five PostScript stacks, Distiller maintains a stack of namespaces. The namespace stack is similar to the PostScript dictionary stack, except that only the top\sphinxhyphen{}most namespace name objects are visible. The namespace stack is also similar to the graphics state stack, except that no \sphinxcode{\sphinxupquote{currentgstate}} analog is provided. For more information on PostScript stacks, see the \sphinxstyleemphasis{PostScript Language Reference} .

A namespace contains:
\begin{itemize}
\item {} 
Names for user\sphinxhyphen{}defined named objects (see \sphinxhref{pdfmark\_Syntax.html\#50454537\_72745}{User\sphinxhyphen{}defined named objects})

\item {} 
Names for stored implicit parent stacks (see \sphinxhref{pdfmark\_Logical.html\#50454553\_60367}{StStore})

\item {} 
Names for images (see \sphinxhref{pdfmark\_Basic.html\#50454556\_14682}{Named images (NI)})

\end{itemize}

The appropriate use of namespaces can help ensure that there are no named\sphinxhyphen{}object conflicts when you use pdfmark constructs from various sources to create a PostScript file. A common example is the handling of Encapsulated PostScript files (see \sphinxhref{pdfmark\_Logical.html\#50454553\_89401}{EPS considerations}).

\begin{sphinxadmonition}{note}{Note:}
The built\sphinxhyphen{}in named objects are managed separately from the namespace stack and are always visible.
\end{sphinxadmonition}

The following pdfmark features are available for manipulating namespaces:
\begin{itemize}
\item {} 
\sphinxcode{\sphinxupquote{NamespacePush}} causes a new, empty namespace to be pushed onto the namespace stack and causes all other namespaces to be hidden. The syntax for pushing a namespace is as follows:

\end{itemize}

\begin{sphinxVerbatim}[commandchars=\\\{\}]
\PYG{p}{[} \PYG{o}{/}\PYG{n}{NamespacePush} \PYG{n}{pdfmark}
\end{sphinxVerbatim}
\begin{itemize}
\item {} 
\sphinxcode{\sphinxupquote{NamespacePop}} pops the topmost namespace from the stack. Once a namespace has been popped, it cannot be accessed again. The next lower namespace on the stack becomes the current namespace.

\end{itemize}

The syntax for popping a namespace is as follows:

\begin{sphinxVerbatim}[commandchars=\\\{\}]
\PYG{p}{[} \PYG{o}{/}\PYG{n}{NamespacePop} \PYG{n}{pdfmark}
\end{sphinxVerbatim}

A warning is issued by Distiller if \sphinxcode{\sphinxupquote{NamespacePop}} is encountered when the namespace stack is empty.

\begin{sphinxVerbatim}[commandchars=\\\{\}]
\PYG{o}{\PYGZpc{}}\PYG{o}{\PYGZpc{}}\PYG{p}{[} \PYG{n+ne}{Warning}\PYG{p}{:} \PYG{o}{/}\PYG{n}{NamespacePop} \PYG{n}{pdfmark} \PYG{n}{ignored}\PYG{p}{:} \PYG{n}{No} \PYG{n}{matching} \PYG{n}{NamespacePush} \PYG{p}{]}\PYG{o}{\PYGZpc{}}\PYG{o}{\PYGZpc{}}
\end{sphinxVerbatim}

\begin{sphinxadmonition}{note}{Note:}
There are no pdfmark features to save or restore namespaces.
\end{sphinxadmonition}


\subsection{Adding content to named objects}
\label{\detokenize{pdfmark_Syntax:adding-content-to-named-objects}}
Once a named object has been declared, content can be added to the PDF object that it refers to. There are several pdfmark features to accomplish this for each of the types of named objects:
\begin{itemize}
\item {} 
\sphinxhref{pdfmark\_Syntax.html\#50454537\_34588}{Arrays}

\item {} 
\sphinxhref{pdfmark\_Syntax.html\#50454537\_41622}{Dictionaries}

\item {} 
\sphinxhref{pdfmark\_Syntax.html\#50454537\_54618}{Streams}

\end{itemize}


\subsubsection{Arrays}
\label{\detokenize{pdfmark_Syntax:arrays}}
There are several methods for adding content to arrays that are named objects. The most basic of these is the \sphinxcode{\sphinxupquote{PUT}} feature, using this syntax:

\begin{sphinxVerbatim}[commandchars=\\\{\}]
 \PYG{p}{[}   \PYG{p}{\PYGZob{}}
\PYG{n}{arrayname}
\PYG{p}{\PYGZcb{}}
\PYG{n}{index} \PYG{n}{value}
 \PYG{o}{/}\PYG{n}{PUT} \PYG{n}{pdfmark}
\end{sphinxVerbatim}

The \sphinxcode{\sphinxupquote{PUT}} feature inserts the value argument at the location index. Indices start at 0, and the array grows automatically to hold the largest index specified. Unspecified entries are created as NULL objects. For example:

\begin{sphinxVerbatim}[commandchars=\\\{\}]
\PYG{p}{[} \PYG{o}{/}\PYG{n}{\PYGZus{}objdef} \PYG{p}{\PYGZob{}}\PYG{n}{MoonInfo}\PYG{p}{\PYGZcb{}} \PYG{o}{/}\PYG{n+nb}{type} \PYG{o}{/}\PYG{n}{array} \PYG{o}{/}\PYG{n}{OBJ} \PYG{n}{pdfmark}
\PYG{p}{[} \PYG{p}{\PYGZob{}}\PYG{n}{MoonInfo}\PYG{p}{\PYGZcb{}} \PYG{l+m+mi}{0} \PYG{p}{(}\PYG{n}{Earth} \PYG{n}{to} \PYG{n}{Moon}\PYG{p}{)} \PYG{o}{/}\PYG{n}{PUT} \PYG{n}{pdfmark}
\PYG{p}{[} \PYG{p}{\PYGZob{}}\PYG{n}{MoonInfo}\PYG{p}{\PYGZcb{}} \PYG{l+m+mi}{1} \PYG{l+m+mi}{238855} \PYG{o}{/}\PYG{n}{PUT} \PYG{n}{pdfmark}
\PYG{p}{[} \PYG{p}{\PYGZob{}}\PYG{n}{MoonInfo}\PYG{p}{\PYGZcb{}} \PYG{l+m+mi}{2} \PYG{o}{/}\PYG{n}{miles} \PYG{o}{/}\PYG{n}{PUT} \PYG{n}{pdfmark}
\end{sphinxVerbatim}

The above code creates an array object and populates it with objects of various types. At this point, the named object cannot be reached because there are no entries in the PDF file’s cross\sphinxhyphen{}reference table or file trailer that lead to it.

Adding array objects as above can become tedious. When adding objects to contiguous array index positions, the pdfmark feature \sphinxcode{\sphinxupquote{PUTINTERVAL}} can simplify this task. The syntax for this feature is as follows:

\begin{sphinxVerbatim}[commandchars=\\\{\}]
 \PYG{p}{[} \PYG{p}{\PYGZob{}}
\PYG{n}{arrayname}
\PYG{p}{\PYGZcb{}}
\PYG{n}{index}
 \PYG{p}{[}
\PYG{n}{value}

\PYG{l+m+mi}{1}
 \PYG{o}{.}\PYG{o}{.}\PYG{o}{.} \PYG{n}{value}
\PYG{n}{n}
\PYG{p}{]} \PYG{o}{/}\PYG{n}{PUTINTERVAL} \PYG{n}{pdfmark}
\end{sphinxVerbatim}

The operation of this feature is the same as in PostScript: value \sphinxstyleemphasis{1} is placed in arrayname \sphinxstyleemphasis{index} , value \sphinxstyleemphasis{2} is placed in arrayname \sphinxstyleemphasis{index+1} , and so forth. The array is resized if necessary to hold the objects added. The previous example can be simplified to:

\begin{sphinxVerbatim}[commandchars=\\\{\}]
\PYG{p}{[} \PYG{o}{/}\PYG{n}{\PYGZus{}objdef} \PYG{p}{\PYGZob{}}\PYG{n}{MoonInfo}\PYG{p}{\PYGZcb{}} \PYG{o}{/}\PYG{n+nb}{type} \PYG{o}{/}\PYG{n}{array} \PYG{o}{/}\PYG{n}{OBJ} \PYG{n}{pdfmark}
\PYG{p}{[} \PYG{p}{\PYGZob{}}\PYG{n}{MoonInfo}\PYG{p}{\PYGZcb{}} \PYG{l+m+mi}{0} \PYG{p}{[}\PYG{p}{(}\PYG{n}{Earth} \PYG{n}{to} \PYG{n}{Moon}\PYG{p}{)} \PYG{l+m+mi}{238855} \PYG{o}{/}\PYG{n}{miles}\PYG{p}{]} \PYG{o}{/}\PYG{n}{PUTINTERVAL} \PYG{n}{pdfmark}
\end{sphinxVerbatim}

One additional convenience for adding objects to an array is available: the \sphinxcode{\sphinxupquote{APPEND}} feature. This feature adds one additional entry immediately after the end of the array. Its syntax is as follows:

\begin{sphinxVerbatim}[commandchars=\\\{\}]
\PYG{p}{[} \PYG{p}{\PYGZob{}}
\PYG{n}{arrayname}
\PYG{p}{\PYGZcb{}}
\PYG{n}{value}
\PYG{o}{/}\PYG{n}{APPEND} \PYG{n}{pdfmark}
\end{sphinxVerbatim}


\subsubsection{Dictionaries}
\label{\detokenize{pdfmark_Syntax:dictionaries}}
The \sphinxcode{\sphinxupquote{PUT}} feature can also be used to add dictionary content. The named object can be either a built\sphinxhyphen{}in name, such as \sphinxcode{\sphinxupquote{\{Catalog\}}} or \sphinxcode{\sphinxupquote{\{Page37\}}} , or a user\sphinxhyphen{}defined object name.

For dictionary named objects, the syntax of the \sphinxcode{\sphinxupquote{PUT}} feature is as follows:

\begin{sphinxVerbatim}[commandchars=\\\{\}]
\PYG{p}{[}   \PYG{p}{\PYGZob{}}
\PYG{n}{dictname}
\PYG{p}{\PYGZcb{}} \PYG{o}{\PYGZlt{}\PYGZlt{}}
\PYG{n}{key}

\PYG{l+m+mi}{1}

\PYG{n}{value}

\PYG{l+m+mi}{1}
 \PYG{o}{.}\PYG{o}{.}\PYG{o}{.}
\PYG{n}{key}

\PYG{n}{n}

\PYG{n}{value}

\PYG{n}{n}
 \PYG{o}{\PYGZgt{}\PYGZgt{}} \PYG{o}{/}\PYG{n}{PUT} \PYG{n}{pdfmark}
\end{sphinxVerbatim}

For dictionary named objects, \sphinxcode{\sphinxupquote{PUT}} adds the key–value pairs provided as arguments. Continuing the previous example:

\begin{sphinxVerbatim}[commandchars=\\\{\}]
\PYG{p}{[} \PYG{p}{\PYGZob{}}\PYG{n}{Catalog}\PYG{p}{\PYGZcb{}} \PYG{o}{\PYGZlt{}\PYGZlt{}} \PYG{o}{/}\PYG{n}{TheMoon} \PYG{p}{\PYGZob{}}\PYG{n}{MoonInfo}\PYG{p}{\PYGZcb{}} \PYG{o}{\PYGZgt{}\PYGZgt{}} \PYG{o}{/}\PYG{n}{PUT} \PYG{n}{pdfmark}
\end{sphinxVerbatim}

This adds a key–value pair to the PDF Catalog dictionary. The inserted key is \sphinxcode{\sphinxupquote{/TheMoon}} and the value is an indirect object. To illustrate this, the resultant PDF file might have the following content:

\begin{sphinxVerbatim}[commandchars=\\\{\}]
trailer
 \PYGZlt{}\PYGZlt{} … /Root 9 0 R … \PYGZgt{}\PYGZgt{}
 …
 9 0 obj \PYGZlt{}\PYGZlt{} … /Type /Catalog … /TheMoon 3 0 R … \PYGZgt{}\PYGZgt{}
 endobj
 3 0 obj [(Earth to Moon)238855/miles]
 endobj
\end{sphinxVerbatim}

The named object \sphinxcode{\sphinxupquote{MoonInfo}} does not appear in the resultant PDF file, but the object it referred to, \sphinxcode{\sphinxupquote{3 0 obj}} in this case, does.


\subsubsection{Streams}
\label{\detokenize{pdfmark_Syntax:streams}}
For stream named objects, the syntax can take several forms:

\begin{sphinxVerbatim}[commandchars=\\\{\}]
\PYG{p}{[}   \PYG{p}{\PYGZob{}}
\PYG{n}{streamname}
\PYG{p}{\PYGZcb{}}
\PYG{n}{string}
 \PYG{o}{/}\PYG{n}{PUT} \PYG{n}{pdfmark}

   \PYG{p}{[}   \PYG{p}{\PYGZob{}}
\PYG{n}{streamname}
\PYG{p}{\PYGZcb{}}
\PYG{n}{file}
 \PYG{o}{/}\PYG{n}{PUT} \PYG{n}{pdfmark}
 \PYG{p}{[}   \PYG{p}{\PYGZob{}}
\PYG{n}{streamname}
\PYG{p}{\PYGZcb{}} \PYG{o}{\PYGZlt{}\PYGZlt{}}
\PYG{n}{key}

\PYG{l+m+mi}{1}

\PYG{n}{value}

\PYG{l+m+mi}{1}
 \PYG{o}{.}\PYG{o}{.}\PYG{o}{.}
\PYG{n}{key}

\PYG{n}{n}

\PYG{n}{value}

\PYG{n}{n}
 \PYG{o}{\PYGZgt{}\PYGZgt{}} \PYG{o}{/}\PYG{n}{PUT} \PYG{n}{pdfmark}
\end{sphinxVerbatim}

A stream object consists of a sequence of bytes, its character data, and an associated dictionary. When the stream named object is created, the character data is empty. The source of stream data can come from an explicit string or can be read from a PostScript file object (a file or filter), in which case reading proceeds until the end of file is reached.

In addition to the character data, a stream has an associated PDF dictionary. Some dictionary entries such as \sphinxcode{\sphinxupquote{Length}} are created automatically. Key–value pairs that do not conflict with the keys common to PDF stream dictionaries can be added to this dictionary. The resultant PDF object associated with the stream named object is always compressed using a lossless method that can be specified in Distiller’s Adobe PDF Settings dialog box.

The \sphinxcode{\sphinxupquote{CLOSE}} feature closes a stream object created by pdfmark and has the syntax:

\begin{sphinxVerbatim}[commandchars=\\\{\}]
\PYG{p}{[}   \PYG{p}{\PYGZob{}}
\PYG{n}{streamname}
\PYG{p}{\PYGZcb{}} \PYG{o}{/}\PYG{n}{CLOSE} \PYG{n}{pdfmark}
\end{sphinxVerbatim}

The named stream object is closed and written to the PDF file. The \{streamname\} is still valid and may be referenced by other objects, but it can no longer be written to. When Distiller completes writing a PDF file, any open streams are closed and written automatically.

For example:

\begin{sphinxVerbatim}[commandchars=\\\{\}]
\PYG{p}{[}   \PYG{o}{/}\PYG{n}{\PYGZus{}objdef} \PYG{p}{\PYGZob{}}\PYG{n}{MoonNotes}\PYG{p}{\PYGZcb{}} \PYG{o}{/}\PYG{n+nb}{type} \PYG{o}{/}\PYG{n}{stream} \PYG{o}{/}\PYG{n}{OBJ} \PYG{n}{pdfmark}
\PYG{p}{[}   \PYG{p}{\PYGZob{}}\PYG{n}{MoonNotes}\PYG{p}{\PYGZcb{}} \PYG{p}{(}\PYG{n}{Hipparchus} \PYG{n}{around} \PYG{l+m+mi}{129} \PYG{n}{BC} \PYG{n}{calculated} \PYG{n}{the} \PYG{n}{distance} \PYG{n}{to} \PYG{n}{the} \PYG{n}{Moon}\PYG{o}{.}\PYG{n}{n}\PYG{p}{)}
    \PYG{o}{/}\PYG{n}{PUT} \PYG{n}{pdfmark}
\PYG{p}{[}   \PYG{p}{\PYGZob{}}\PYG{n}{MoonNotes}\PYG{p}{\PYGZcb{}} \PYG{p}{(}\PYG{n}{The} \PYG{n}{Moon} \PYG{n}{was} \PYG{n}{first} \PYG{n}{touched} \PYG{n}{by} \PYG{n}{Armstrong} \PYG{n}{on} \PYG{n}{July} \PYG{l+m+mi}{20}\PYG{p}{,} \PYG{l+m+mf}{1969.}\PYG{n}{n}\PYG{p}{)}
    \PYG{o}{/}\PYG{n}{PUT} \PYG{n}{pdfmark}
\PYG{p}{[}   \PYG{p}{\PYGZob{}}\PYG{n}{MoonNotes}\PYG{p}{\PYGZcb{}} \PYG{o}{\PYGZlt{}\PYGZlt{}} \PYG{o}{/}\PYG{n}{Author} \PYG{p}{(}\PYG{n}{Steve} \PYG{n}{Amerige}\PYG{p}{)} \PYG{o}{/}\PYG{n}{Company} \PYG{p}{(}\PYG{n}{Adobe}\PYG{p}{)} \PYG{o}{\PYGZgt{}\PYGZgt{}} \PYG{o}{/}\PYG{n}{PUT} \PYG{n}{pdfmark}
\PYG{p}{[}   \PYG{p}{\PYGZob{}}\PYG{n}{Catalog}\PYG{p}{\PYGZcb{}} \PYG{o}{\PYGZlt{}\PYGZlt{}} \PYG{o}{/}\PYG{n}{MoonNotes} \PYG{p}{\PYGZob{}}\PYG{n}{MoonNotes}\PYG{p}{\PYGZcb{}} \PYG{o}{\PYGZgt{}\PYGZgt{}} \PYG{o}{/}\PYG{n}{PUT} \PYG{n}{pdfmark}
\PYG{p}{[}   \PYG{p}{\PYGZob{}}\PYG{n}{MoonNotes}\PYG{p}{\PYGZcb{}} \PYG{o}{/}\PYG{n}{CLOSE} \PYG{n}{pdfmark}
\end{sphinxVerbatim}


\chapter{Basic Features}
\label{\detokenize{pdfmark_Basic:basic-features}}\label{\detokenize{pdfmark_Basic::doc}}
This chapter describes the basic pdfmark features. In general, the key–value pairs used as arguments for pdfmark follow closely the key–value pairs that appear in the PDF file. For a description of the PDF file format, see the \sphinxstyleemphasis{PDF Reference} .

The following features are described in this chapter:
\begin{itemize}
\item {} 
\sphinxhref{pdfmark\_Basic.html\#50454556\_26704}{Annotations (ANN)}

\item {} 
\sphinxhref{pdfmark\_Basic.html\#50454556\_30435}{Articles (ARTICLE)}

\item {} 
\sphinxhref{pdfmark\_Basic.html\#50454556\_44268}{Bookmarks (OUT)}

\item {} 
\sphinxhref{pdfmark\_Basic.html\#50454556\_33876}{Document Info dictionary (DOCINFO)}

\item {} 
\sphinxhref{pdfmark\_Basic.html\#50454556\_62753}{Document open options (DOCVIEW)}

\item {} 
\sphinxhref{pdfmark\_Basic.html\#50454556\_76251}{Embedded file content (EMBED)}

\item {} 
\sphinxhref{pdfmark\_Basic.html\#50454556\_78698}{Graphics encapsulation (BP, EP, SP)}

\item {} 
\sphinxhref{pdfmark\_Basic.html\#50454556\_41438}{Marked content (MP, DP, BMC, BDC, EMC)}

\item {} 
\sphinxhref{pdfmark\_Basic.html\#50454556\_82967}{Metadata (Metadata)}

\item {} 
\sphinxhref{pdfmark\_Basic.html\#50454556\_14682}{Named images (NI)}

\item {} 
\sphinxhref{pdfmark\_Basic.html\#50454556\_42582}{Page crops (PAGE, PAGES)}

\item {} 
\sphinxhref{pdfmark\_Basic.html\#50454556\_78357}{Page label and plate color (PAGELABEL)}

\item {} 
\sphinxhref{pdfmark\_Basic.html\#50454556\_93185}{Transparency (SetTransparency)}

\end{itemize}

Other pdfmark features are defined in other chapters of this document.


\section{Annotations (ANN)}
\label{\detokenize{pdfmark_Basic:annotations-ann}}
PDF supports several types of annotations. The properties of each annotation are specified in an annotation dictionary containing various key–value pairs. The \sphinxstyleemphasis{PDF Reference} describes all the types of annotations, and their required and optional dictionary entries.

The pdfmark operator using the feature name \sphinxcode{\sphinxupquote{ANN}} is used to specify an annotation in a PostScript file. The general syntax is as follows:

\begin{sphinxVerbatim}[commandchars=\\\{\}]
[ /Rect [
xll yll xur yur
]

/Subtype
name
        …Optional key–value pairs…
/ANN pdfmark
\end{sphinxVerbatim}

The following table describes the two required keys for annotations.

\sphinxstylestrong{Required annotation keys}


\begin{savenotes}\sphinxattablestart
\centering
\begin{tabulary}{\linewidth}[t]{|T|T|T|}
\hline
\sphinxstyletheadfamily 
Key
&\sphinxstyletheadfamily 
Type
&\sphinxstyletheadfamily 
Semantics
\\
\hline
Rect
&
array
&
An array of four numbers {[} xll yll xur yur {]} specifying the lower\sphinxhyphen{}left x, lower\sphinxhyphen{}left y, upper\sphinxhyphen{}right x, and upper\sphinxhyphen{}right y coordinates—in user space—of the rectangle defining the open note window or link button.
\\
\hline
Subtype
&
name
&
The annotation’s PDF subtype. If omitted, the value defaults to \sphinxcode{\sphinxupquote{Text}} , indicating a note annotation. See the table \sphinxhref{pdfmark\_Basic.html\#50454556\_36966}{PDF annotation types} for the possible subtypes that can be used.
\\
\hline
\end{tabulary}
\par
\sphinxattableend\end{savenotes}

As of PDF 1.3, the following annotation types are supported:


\begin{savenotes}\sphinxattablestart
\centering
\phantomsection\label{\detokenize{pdfmark_Basic:section-1}}\nobreak
\begin{tabulary}{\linewidth}[t]{|T|T|}
\hline
\sphinxstyletheadfamily 
Value of subtype key
&\sphinxstyletheadfamily 
Description
\\
\hline
Circle
&
Circle annotation
\\
\hline
FileAttachment
&
File attachment annotation
\\
\hline
FreeText
&
Free text annotation
\\
\hline
Highlight
&
Highlight annotation
\\
\hline
Ink
&
Ink annotation
\\
\hline
Line
&
Line annotation
\\
\hline
Link
&
Link annotation
\\
\hline
Movie
&
Movie annotation
\\
\hline
Popup
&
Pop\sphinxhyphen{}up annotation
\\
\hline
Sound
&
Sound annotation
\\
\hline
Square
&
Square annotation
\\
\hline
Stamp
&
Rubber stamp annotation
\\
\hline
StrikeOut
&
Strikeout annotation
\\
\hline
Text
&
Text annotation (note)
\\
\hline
TrapNet
&
Trap network annotation
\\
\hline
Underline
&
Underline annotation
\\
\hline
Widget
&
Widget annotation
\\
\hline
\end{tabulary}
\par
\sphinxattableend\end{savenotes}

Each type has its own set of key\sphinxhyphen{}value pairs that can be specified, as described in the \sphinxstyleemphasis{PDF Reference.} Future versions of PDF may introduce new types.

In addition to these types, annotations with unrecognized \sphinxcode{\sphinxupquote{Subtype}} values, called custom annotations, are supported. Custom annotations can contain, in addition to the \sphinxcode{\sphinxupquote{Rect}} and \sphinxcode{\sphinxupquote{Subtype}} keys, arbitrary key\sphinxhyphen{}value pairs.
\begin{enumerate}
\sphinxsetlistlabels{\arabic}{enumi}{enumii}{}{.}%
\item {} 
Custom annotation

\end{enumerate}

\begin{sphinxVerbatim}[commandchars=\\\{\}]
\PYG{p}{[}\PYG{o}{/}\PYG{n}{Rect} \PYG{p}{[} \PYG{l+m+mi}{400} \PYG{l+m+mi}{435} \PYG{l+m+mi}{500} \PYG{l+m+mi}{535} \PYG{p}{]}
   \PYG{o}{/}\PYG{n}{Subtype} \PYG{o}{/}\PYG{n}{ADBETest\PYGZus{}DummyType}
   \PYG{o}{/}\PYG{n}{ADBETest\PYGZus{}F8Array} \PYG{p}{[} \PYG{l+m+mi}{0} \PYG{l+m+mi}{1} \PYG{l+m+mi}{1} \PYG{l+m+mi}{2} \PYG{l+m+mi}{3} \PYG{l+m+mi}{5} \PYG{l+m+mi}{8} \PYG{l+m+mi}{13} \PYG{p}{]}
   \PYG{o}{/}\PYG{n}{ANN} \PYG{n}{pdfmark}
\end{sphinxVerbatim}

When viewed with Acrobat Viewer, this annotation appears with an unknown annotation icon.

The following table lists optional keys that are common to all annotations. Specific annotation types have additional keys that they use. See the \sphinxstyleemphasis{PDF Reference} for complete information.

\sphinxstylestrong{Optional annotation keys}


\begin{savenotes}\sphinxattablestart
\centering
\phantomsection\label{\detokenize{pdfmark_Basic:section-2}}\nobreak
\begin{tabular}[t]{|*{3}{\X{1}{3}|}}
\hline
\sphinxstyletheadfamily 
Key
&\sphinxstyletheadfamily 
Type
&\sphinxstyletheadfamily 
Semantics
\\
\hline\begin{quote}

Action
\end{quote}

(PDF key = \sphinxcode{\sphinxupquote{A}} )
&
name or dictionary
&
An action to be performed when the annotation is activated. See \sphinxhref{pdfmark\_Actions.html\#50454557\_92618}{Actions} for details.

For links, this key is not permitted if the \sphinxcode{\sphinxupquote{Dest}} key is present.
\\
\hline
AP
&
dictionary
&
An appearance dictionary specifying how the annotation is presented visually. See the \sphinxstyleemphasis{PDF Reference} for details.
\\
\hline
AS
&
name
&
The annotation’s appearance state. See the \sphinxstyleemphasis{PDF Reference} for details.
\\
\hline
Border
&
array
&
The link’s border properties. \sphinxcode{\sphinxupquote{Border}} is an array containing three numbers and, optionally, an array. All elements are specified in user space coordinates.

If \sphinxcode{\sphinxupquote{Border}} is of the form {[} \sphinxstyleemphasis{bx by c} {]}, the numbers specify the horizontal corner radius ( \sphinxstyleemphasis{bx} ), the vertical corner radius ( \sphinxstyleemphasis{by} ), and the width ( \sphinxstyleemphasis{c} ) of the link’s border. The link has a solid border.

If it is of the form {[} \sphinxstyleemphasis{bx by c {[}d{]}} {]}, the fourth element ( \sphinxstyleemphasis{d} ) is a dash array that specifies the lengths of dashes and gaps in the link’s border.

The default value for \sphinxcode{\sphinxupquote{Border}} is {[}0 0 1{]}.
\\
\hline\begin{quote}

Color
\end{quote}

(PDF key = \sphinxcode{\sphinxupquote{C}} )
&
array
&
A color value used for the background of the annotation’s icon when closed; the title bar of the annotation’s pop\sphinxhyphen{}up window; and the border of a link annotation.

The value is an array containing three numbers (red, green, and blue), each of which must be between 0 and 1, inclusive, specifying a color in the DeviceRGB color space. (See the \sphinxstyleemphasis{PDF Reference} for a description of this color space.) If omitted, a default color is used.
\\
\hline
F
&
integer
&
A set of flags specifying various characteristics. See the \sphinxstyleemphasis{PDF Reference} for details \sphinxstyleemphasis{.}
\\
\hline\begin{quote}

ModDate
\end{quote}

(PDF key = \sphinxcode{\sphinxupquote{M}} )
&
string
&
The date and time the note was last modified. It should be of the form:

( \sphinxstyleemphasis{D:YYYYMMDDHHmmSSOHH’mm’} )

\sphinxstyleemphasis{D:} is an optional but strongly recommended prefix. \sphinxstyleemphasis{YYYY} is the year. All fields after the year are optional. \sphinxstyleemphasis{MM} is the month (01\sphinxhyphen{}12), \sphinxstyleemphasis{DD} is the day (01\sphinxhyphen{}31), \sphinxstyleemphasis{HH} is the hour (00\sphinxhyphen{}23), \sphinxstyleemphasis{mm} are the minutes (00\sphinxhyphen{}59), and \sphinxstyleemphasis{SS} are the seconds (00\sphinxhyphen{}59). The remainder of the string defines the relation of local time to GMT. \sphinxstyleemphasis{O} is either + for a positive difference (local time is later than GMT) or \sphinxhyphen{} (minus) for a negative difference. \sphinxstyleemphasis{HH’} is the absolute value of the offset from GMT in hours, and \sphinxstyleemphasis{mm’} is the absolute value of the offset in minutes. If no GMT information is specified, the relation between the specified time and GMT is considered unknown. Regardless of whether or not GMT information is specified, the remainder of the string should specify the local time.
\\
\hline
SrcPg
&
integer
&
The sequence number of the page on which the annotation appears. (The first page in a document is always page 1.) If this key is used, the pdfmark can be placed anywhere in the PostScript language file. If omitted, the pdfmark must occur within the PostScript language description for the page on which the annotation is to appear.
\\
\hline\begin{quote}

Title
\end{quote}

(PDF key = \sphinxcode{\sphinxupquote{T}} )
&
string
&
The text label to be displayed in the title bar of the annotation’s pop\sphinxhyphen{}up window when open and active

The encoding and character set used is either PDFDocEncoding (as described in the \sphinxstyleemphasis{PDF Reference} ) or Unicode. If Unicode, the string must begin with <FEFF>. For example, the string “ABC” is represented as (ABC) in PDFDocEncoding and <FEFF004100420043> in Unicode. \sphinxcode{\sphinxupquote{Title}} has a maximum length of 255 PDFDocEncoding characters or 126 Unicode values, although a practical limit of 32 characters is advised so that it can be read easily in the Acrobat viewer.
\\
\hline
\end{tabular}
\par
\sphinxattableend\end{savenotes}

\sphinxhref{pdfmark\_Basic.html\#50454556\_16273}{Text annotations (notes)} and \sphinxhref{pdfmark\_Basic.html\#50454556\_13483}{Links} describe the syntax for two of the original and most commonly used annotation types in more detail.


\subsection{Text annotations (notes)}
\label{\detokenize{pdfmark_Basic:text-annotations-notes}}
Notes are known as text annotations in PDF. The syntax for creating a note is as follows:

\begin{sphinxVerbatim}[commandchars=\\\{\}]
\PYG{p}{[} \PYG{o}{/}\PYG{n}{Contents} \PYG{n}{string}
       \PYG{o}{/}\PYG{n}{Rect} \PYG{p}{[}\PYG{n}{xll} \PYG{n}{yll} \PYG{n}{xur} \PYG{n}{yur}\PYG{p}{]}
       \PYG{o}{/}\PYG{n}{SrcPg} \PYG{n}{pagenum}
       \PYG{o}{/}\PYG{n}{Open} \PYG{n}{boolean}
       \PYG{o}{/}\PYG{n}{Color} \PYG{n}{array}
       \PYG{o}{/}\PYG{n}{Title} \PYG{n}{string}
       \PYG{o}{/}\PYG{n}{ModDate} \PYG{n}{datestring}
       \PYG{o}{/}\PYG{n}{Name} \PYG{n}{name}
       \PYG{o}{/}\PYG{n}{Subtype} \PYG{o}{/}\PYG{n}{Text}
       \PYG{o}{/}\PYG{n}{ANN} \PYG{n}{pdfmark}
\end{sphinxVerbatim}

In addition to the keys described in the tables \sphinxhref{pdfmark\_Basic.html\#50454556\_23949}{Required annotation keys} and \sphinxhref{pdfmark\_Basic.html\#50454556\_78817}{Optional annotation keys}, the keys specific to text annotations are listed in the following table. In addition to these keys, notes may also specify arbitrary key–value pairs.

\sphinxstylestrong{Keys specific to text annotations}


\begin{savenotes}\sphinxattablestart
\centering
\phantomsection\label{\detokenize{pdfmark_Basic:section-3}}\nobreak
\begin{tabulary}{\linewidth}[t]{|T|T|T|}
\hline
\sphinxstyletheadfamily 
Key
&\sphinxstyletheadfamily 
Type
&\sphinxstyletheadfamily 
Semantics
\\
\hline
Contents
&
string
&
Required. Contains the note’s text string. The maximum length of the \sphinxcode{\sphinxupquote{Contents}} string is 65,535 characters. The encoding and character set used is the PDFDocEncoding (described in the \sphinxstyleemphasis{PDF Reference} ) or Unicode. If Unicode, the string must begin with <FEFF>.
\\
\hline
Open
&
Boolean
&
Optional. If \sphinxstyleemphasis{true} , the note is open (that is, the text is visible). If \sphinxstyleemphasis{false} (the default if omitted), the note is closed (that is, displayed as an icon).
\\
\hline
Name
&
name
&
Optional. The name of an icon to be used in displaying the note. The values are: \sphinxcode{\sphinxupquote{Note}} (default), \sphinxcode{\sphinxupquote{Comment}} , \sphinxcode{\sphinxupquote{Help}} , \sphinxcode{\sphinxupquote{Insert}} , \sphinxcode{\sphinxupquote{Key}} , \sphinxcode{\sphinxupquote{NewParagraph}} , \sphinxcode{\sphinxupquote{Paragraph}} .
\\
\hline
\end{tabulary}
\par
\sphinxattableend\end{savenotes}

The following examples demonstrate the use of notes.
\begin{enumerate}
\sphinxsetlistlabels{\arabic}{enumi}{enumii}{}{.}%
\item {} 
Text annotation

\end{enumerate}

\begin{sphinxVerbatim}[commandchars=\\\{\}]
\PYG{p}{[} \PYG{o}{/}\PYG{n}{Contents} \PYG{p}{(}\PYG{n}{My} \PYG{n}{unimaginative} \PYG{n}{contents}\PYG{p}{)}
       \PYG{o}{/}\PYG{n}{Rect} \PYG{p}{[}\PYG{l+m+mi}{400} \PYG{l+m+mi}{550} \PYG{l+m+mi}{500} \PYG{l+m+mi}{650}\PYG{p}{]}
       \PYG{o}{/}\PYG{n}{Open} \PYG{n}{false}
       \PYG{o}{/}\PYG{n}{Title} \PYG{p}{(}\PYG{n}{My} \PYG{n}{Boring} \PYG{n}{Title}\PYG{p}{)}

       \PYG{o}{\PYGZpc{}} \PYG{n}{The} \PYG{n}{following} \PYG{o+ow}{is} \PYG{n}{private} \PYG{n}{data}\PYG{o}{.} \PYG{n}{Keys} \PYG{n}{within} \PYG{n}{the} \PYG{n}{private}
       \PYG{o}{\PYGZpc{}} \PYG{n}{dictionary} \PYG{n}{do} \PYG{o+ow}{not} \PYG{n}{need} \PYG{n}{to} \PYG{n}{use} \PYG{n}{the} \PYG{n}{organization}\PYG{l+s+s1}{\PYGZsq{}}\PYG{l+s+s1}{s prefix}
       \PYG{o}{\PYGZpc{}} \PYG{n}{because} \PYG{n}{the} \PYG{n}{dictionary} \PYG{n}{encapsulates} \PYG{n}{them}\PYG{o}{.}

       \PYG{o}{/}\PYG{n}{ADBETest\PYGZus{}MyInfo}
           \PYG{o}{\PYGZlt{}\PYGZlt{}}
           \PYG{o}{/}\PYG{n}{Routing} \PYG{p}{[}\PYG{p}{(}\PYG{n}{Me}\PYG{p}{)} \PYG{p}{(}\PYG{n}{You}\PYG{p}{)}\PYG{p}{]}
           \PYG{o}{/}\PYG{n}{Test\PYGZus{}Privileges} \PYG{o}{\PYGZlt{}\PYGZlt{}} \PYG{o}{/}\PYG{n}{Me} \PYG{o}{/}\PYG{n}{All} \PYG{o}{/}\PYG{n}{You} \PYG{o}{/}\PYG{n}{ReadOnly} \PYG{o}{\PYGZgt{}\PYGZgt{}}
           \PYG{o}{\PYGZgt{}\PYGZgt{}}

       \PYG{o}{/}\PYG{n}{ADBETest\PYGZus{}PrivFlags} \PYG{l+m+mi}{42}
       \PYG{o}{/}\PYG{n}{ANN} \PYG{n}{pdfmark}
\end{sphinxVerbatim}
\begin{enumerate}
\sphinxsetlistlabels{\arabic}{enumi}{enumii}{}{.}%
\item {} 
Simple note

\end{enumerate}

\begin{sphinxVerbatim}[commandchars=\\\{\}]
\PYG{p}{[} \PYG{o}{/}\PYG{n}{Rect} \PYG{p}{[}\PYG{l+m+mi}{75} \PYG{l+m+mi}{586} \PYG{l+m+mi}{456} \PYG{l+m+mi}{663}\PYG{p}{]}
    \PYG{o}{/}\PYG{n}{Contents} \PYG{p}{(}\PYG{n}{This} \PYG{o+ow}{is} \PYG{n}{an} \PYG{n}{example} \PYG{n}{of} \PYG{n}{a} \PYG{n}{note}\PYG{o}{.} \PYG{n}{You} \PYG{n}{can} \PYG{n+nb}{type} \PYG{n}{text} \PYG{n}{directly} \PYG{n}{into} \PYG{n}{a} \PYG{n}{note} \PYG{o+ow}{or} \PYG{n}{copy} \PYG{n}{text} \PYG{k+kn}{from} \PYG{n+nn}{the} \PYG{n}{clipboard}\PYG{o}{.}\PYG{p}{)}
    \PYG{o}{/}\PYG{n}{ANN} \PYG{n}{pdfmark}
\end{sphinxVerbatim}
\begin{enumerate}
\sphinxsetlistlabels{\arabic}{enumi}{enumii}{}{.}%
\item {} 
Fancy note

\end{enumerate}

\begin{sphinxVerbatim}[commandchars=\\\{\}]
\PYG{p}{[} \PYG{o}{/}\PYG{n}{Rect} \PYG{p}{[}\PYG{l+m+mi}{75} \PYG{l+m+mi}{425} \PYG{l+m+mi}{350} \PYG{l+m+mi}{563}\PYG{p}{]}
    \PYG{o}{/}\PYG{n}{Open} \PYG{n}{true}
    \PYG{o}{/}\PYG{n}{Title} \PYG{p}{(}\PYG{n}{John} \PYG{n}{Doe}\PYG{p}{)}
    \PYG{o}{/}\PYG{n}{Contents} \PYG{p}{(}\PYG{n}{This} \PYG{o+ow}{is} \PYG{n}{an} \PYG{n}{example} \PYG{n}{of} \PYG{n}{a} \PYG{n}{note}\PYG{o}{.} \PYG{n}{Here} \PYG{o+ow}{is} \PYG{n}{some} \PYG{n}{text} \PYG{n}{after} \PYG{n}{a} \PYG{n}{forced} \PYG{n}{line} \PYG{k}{break}\PYG{o}{.}

\PYG{n}{This} \PYG{o+ow}{is} \PYG{n}{another} \PYG{n}{way} \PYG{n}{to} \PYG{n}{do} \PYG{n}{line} \PYG{n}{breaks}\PYG{o}{.}\PYG{p}{)}

    \PYG{o}{/}\PYG{n}{Color} \PYG{p}{[}\PYG{l+m+mi}{1} \PYG{l+m+mi}{0} \PYG{l+m+mi}{0}\PYG{p}{]}
    \PYG{o}{/}\PYG{n}{Border} \PYG{p}{[}\PYG{l+m+mi}{0} \PYG{l+m+mi}{0} \PYG{l+m+mi}{1}\PYG{p}{]}
    \PYG{o}{/}\PYG{n}{ANN} \PYG{n}{pdfmark}
\end{sphinxVerbatim}
\begin{enumerate}
\sphinxsetlistlabels{\arabic}{enumi}{enumii}{}{.}%
\item {} 
Private data in note

\end{enumerate}

\begin{sphinxVerbatim}[commandchars=\\\{\}]
\PYG{p}{[} \PYG{o}{/}\PYG{n}{Contents} \PYG{p}{(}\PYG{n}{My} \PYG{n}{unimaginative} \PYG{n}{contents}\PYG{p}{)}
    \PYG{o}{/}\PYG{n}{Rect} \PYG{p}{[}\PYG{l+m+mi}{400} \PYG{l+m+mi}{550} \PYG{l+m+mi}{500} \PYG{l+m+mi}{650}\PYG{p}{]}
    \PYG{o}{/}\PYG{n}{Open} \PYG{n}{false}
    \PYG{o}{/}\PYG{n}{Title} \PYG{p}{(}\PYG{n}{My} \PYG{n}{Boring} \PYG{n}{Title}\PYG{p}{)}

\PYG{o}{\PYGZpc{}} \PYG{n}{The} \PYG{n}{following} \PYG{o+ow}{is} \PYG{n}{private} \PYG{n}{data}\PYG{o}{.} \PYG{n}{Keys} \PYG{n}{within} \PYG{n}{the} \PYG{n}{private}
\PYG{o}{\PYGZpc{}} \PYG{n}{dictionary} \PYG{n}{do} \PYG{o+ow}{not} \PYG{n}{need} \PYG{n}{to} \PYG{n}{use} \PYG{n}{the} \PYG{n}{organization}\PYG{l+s+s1}{\PYGZsq{}}\PYG{l+s+s1}{s prefix}
\PYG{o}{\PYGZpc{}} \PYG{n}{because} \PYG{n}{the} \PYG{n}{dictionary} \PYG{n}{encapsulates} \PYG{n}{them}\PYG{o}{.}

    \PYG{o}{/}\PYG{n}{ADBETest\PYGZus{}MyInfo}
        \PYG{o}{\PYGZlt{}\PYGZlt{}}
        \PYG{o}{/}\PYG{n}{Routing} \PYG{p}{[}\PYG{p}{(}\PYG{n}{Me}\PYG{p}{)} \PYG{p}{(}\PYG{n}{You}\PYG{p}{)}\PYG{p}{]}
        \PYG{o}{/}\PYG{n}{Test\PYGZus{}Privileges} \PYG{o}{\PYGZlt{}\PYGZlt{}} \PYG{o}{/}\PYG{n}{Me} \PYG{o}{/}\PYG{n}{All} \PYG{o}{/}\PYG{n}{You} \PYG{o}{/}\PYG{n}{ReadOnly} \PYG{o}{\PYGZgt{}\PYGZgt{}}
        \PYG{o}{\PYGZgt{}\PYGZgt{}}

    \PYG{o}{/}\PYG{n}{ADBETest\PYGZus{}PrivFlags} \PYG{l+m+mi}{42}
    \PYG{o}{/}\PYG{n}{ANN} \PYG{n}{pdfmark}
\end{sphinxVerbatim}


\subsection{Links}
\label{\detokenize{pdfmark_Basic:links}}
A link annotation represents either a hypertext link to a destination in the document, or an action to be performed.

The usual syntax for creating a link is as follows:

\begin{sphinxVerbatim}[commandchars=\\\{\}]
[/Rect [
xll yll xur yur
]
/Border [
bx by c [d]
]
/SrcPg
pagenum

/Color
array

/Subtype /Link
… Action\PYGZhy{}or\PYGZhy{}destination\PYGZhy{}specifying key–value pairs …

/ANN pdfmark
\end{sphinxVerbatim}

In addition to the keys described in the tables \sphinxhref{pdfmark\_Basic.html\#50454556\_23949}{Required annotation keys} and \sphinxhref{pdfmark\_Basic.html\#50454556\_78817}{Optional annotation keys}, a link may also contain keys specifying destinations or actions, described in \sphinxhref{pdfmark\_Actions.html\#50454557\_94191}{Actions and Destinations}.

The following examples demonstrate the use of links.
\begin{enumerate}
\sphinxsetlistlabels{\arabic}{enumi}{enumii}{}{.}%
\item {} 
Link annotation

\end{enumerate}

\begin{sphinxVerbatim}[commandchars=\\\{\}]
\PYG{p}{[} \PYG{o}{/}\PYG{n}{Rect} \PYG{p}{[}\PYG{l+m+mi}{70} \PYG{l+m+mi}{550} \PYG{l+m+mi}{210} \PYG{l+m+mi}{575}\PYG{p}{]}
       \PYG{o}{/}\PYG{n}{Border} \PYG{p}{[}\PYG{l+m+mi}{0} \PYG{l+m+mi}{0} \PYG{l+m+mi}{2} \PYG{p}{[}\PYG{l+m+mi}{3}\PYG{p}{]}\PYG{p}{]}
       \PYG{o}{/}\PYG{n}{Color} \PYG{p}{[}\PYG{l+m+mi}{0} \PYG{l+m+mi}{1} \PYG{l+m+mi}{0}\PYG{p}{]}
       \PYG{o}{/}\PYG{n}{Page} \PYG{o}{/}\PYG{n}{Next}
       \PYG{o}{/}\PYG{n}{View} \PYG{p}{[}\PYG{o}{/}\PYG{n}{XYZ} \PYG{o}{\PYGZhy{}}\PYG{l+m+mi}{5} \PYG{l+m+mi}{797} \PYG{l+m+mf}{1.5}\PYG{p}{]}
       \PYG{o}{/}\PYG{n}{Subtype} \PYG{o}{/}\PYG{n}{Link}
       \PYG{o}{/}\PYG{n}{ANN} \PYG{n}{pdfmark}
\end{sphinxVerbatim}
\begin{enumerate}
\sphinxsetlistlabels{\arabic}{enumi}{enumii}{}{.}%
\item {} 
Simple link (old style, compatible with all Distiller application versions)

\end{enumerate}

\begin{sphinxVerbatim}[commandchars=\\\{\}]
\PYG{p}{[} \PYG{o}{/}\PYG{n}{Rect} \PYG{p}{[}\PYG{l+m+mi}{70} \PYG{l+m+mi}{650} \PYG{l+m+mi}{210} \PYG{l+m+mi}{675}\PYG{p}{]}
    \PYG{o}{/}\PYG{n}{Page} \PYG{l+m+mi}{3}
    \PYG{o}{/}\PYG{n}{View} \PYG{p}{[}\PYG{o}{/}\PYG{n}{XYZ} \PYG{o}{\PYGZhy{}}\PYG{l+m+mi}{5} \PYG{l+m+mi}{797} \PYG{l+m+mf}{1.5}\PYG{p}{]}
    \PYG{o}{/}\PYG{n}{LNK} \PYG{n}{pdfmark}
\end{sphinxVerbatim}
\begin{enumerate}
\sphinxsetlistlabels{\arabic}{enumi}{enumii}{}{.}%
\item {} 
Simple link

\end{enumerate}

\begin{sphinxVerbatim}[commandchars=\\\{\}]
\PYG{p}{[} \PYG{o}{/}\PYG{n}{Rect} \PYG{p}{[}\PYG{l+m+mi}{70} \PYG{l+m+mi}{650} \PYG{l+m+mi}{210} \PYG{l+m+mi}{675}\PYG{p}{]}
    \PYG{o}{/}\PYG{n}{Border} \PYG{p}{[}\PYG{l+m+mi}{16} \PYG{l+m+mi}{16} \PYG{l+m+mi}{1}\PYG{p}{]}
    \PYG{o}{/}\PYG{n}{Color} \PYG{p}{[}\PYG{l+m+mi}{1} \PYG{l+m+mi}{0} \PYG{l+m+mi}{0}\PYG{p}{]}
    \PYG{o}{/}\PYG{n}{Page} \PYG{l+m+mi}{1}
    \PYG{o}{/}\PYG{n}{View} \PYG{p}{[}\PYG{o}{/}\PYG{n}{FitH} \PYG{l+m+mi}{5}\PYG{p}{]}
    \PYG{o}{/}\PYG{n}{Subtype} \PYG{o}{/}\PYG{n}{Link}
    \PYG{o}{/}\PYG{n}{ANN} \PYG{n}{pdfmark}
\end{sphinxVerbatim}
\begin{enumerate}
\sphinxsetlistlabels{\arabic}{enumi}{enumii}{}{.}%
\item {} 
Fancy link

\end{enumerate}

\begin{sphinxVerbatim}[commandchars=\\\{\}]
\PYG{p}{[} \PYG{o}{/}\PYG{n}{Rect} \PYG{p}{[}\PYG{l+m+mi}{70} \PYG{l+m+mi}{550} \PYG{l+m+mi}{210} \PYG{l+m+mi}{575}\PYG{p}{]}
    \PYG{o}{/}\PYG{n}{Border} \PYG{p}{[}\PYG{l+m+mi}{0} \PYG{l+m+mi}{0} \PYG{l+m+mi}{2} \PYG{p}{[}\PYG{l+m+mi}{3}\PYG{p}{]}\PYG{p}{]}
    \PYG{o}{/}\PYG{n}{Color} \PYG{p}{[}\PYG{l+m+mi}{0} \PYG{l+m+mi}{1} \PYG{l+m+mi}{0}\PYG{p}{]}
    \PYG{o}{/}\PYG{n}{Page} \PYG{o}{/}\PYG{n}{Next}
    \PYG{o}{/}\PYG{n}{View} \PYG{p}{[}\PYG{o}{/}\PYG{n}{XYZ} \PYG{o}{\PYGZhy{}}\PYG{l+m+mi}{5} \PYG{l+m+mi}{797} \PYG{l+m+mf}{1.5}\PYG{p}{]}
    \PYG{o}{/}\PYG{n}{Subtype} \PYG{o}{/}\PYG{n}{Link}
    \PYG{o}{/}\PYG{n}{ANN} \PYG{n}{pdfmark}
\end{sphinxVerbatim}
\begin{enumerate}
\sphinxsetlistlabels{\arabic}{enumi}{enumii}{}{.}%
\item {} 
Link that launches another file

\end{enumerate}

\begin{sphinxVerbatim}[commandchars=\\\{\}]
\PYG{p}{[} \PYG{o}{/}\PYG{n}{Rect} \PYG{p}{[}\PYG{l+m+mi}{70} \PYG{l+m+mi}{600} \PYG{l+m+mi}{210} \PYG{l+m+mi}{625}\PYG{p}{]}
    \PYG{o}{/}\PYG{n}{Border} \PYG{p}{[}\PYG{l+m+mi}{16} \PYG{l+m+mi}{16} \PYG{l+m+mi}{1}\PYG{p}{]}
    \PYG{o}{/}\PYG{n}{Color} \PYG{p}{[}\PYG{l+m+mi}{0} \PYG{l+m+mi}{0} \PYG{l+m+mi}{1}\PYG{p}{]}
    \PYG{o}{/}\PYG{n}{Action} \PYG{o}{/}\PYG{n}{Launch}
    \PYG{o}{/}\PYG{n}{File} \PYG{p}{(}\PYG{n}{test}\PYG{o}{.}\PYG{n}{doc}\PYG{p}{)}
    \PYG{o}{/}\PYG{n}{Subtype} \PYG{o}{/}\PYG{n}{Link}
    \PYG{o}{/}\PYG{n}{ANN} \PYG{n}{pdfmark}
\end{sphinxVerbatim}
\begin{enumerate}
\sphinxsetlistlabels{\arabic}{enumi}{enumii}{}{.}%
\item {} 
Custom link action (URI link for the Acrobat WebLink plug\sphinxhyphen{}in)

\end{enumerate}

\begin{sphinxVerbatim}[commandchars=\\\{\}]
\PYG{p}{[} \PYG{o}{/}\PYG{n}{Rect} \PYG{p}{[}\PYG{l+m+mi}{50} \PYG{l+m+mi}{425} \PYG{l+m+mi}{295} \PYG{l+m+mi}{445}\PYG{p}{]}
    \PYG{o}{/}\PYG{n}{Action} \PYG{o}{\PYGZlt{}\PYGZlt{}} \PYG{o}{/}\PYG{n}{Subtype} \PYG{o}{/}\PYG{n}{URI} \PYG{o}{/}\PYG{n}{URI} \PYG{p}{(}\PYG{n}{http}\PYG{p}{:}\PYG{o}{/}\PYG{o}{/}\PYG{n}{www}\PYG{o}{.}\PYG{n}{adobe}\PYG{o}{.}\PYG{n}{com}\PYG{p}{)} \PYG{o}{\PYGZgt{}\PYGZgt{}}
    \PYG{o}{/}\PYG{n}{Border} \PYG{p}{[}\PYG{l+m+mi}{0} \PYG{l+m+mi}{0} \PYG{l+m+mi}{2}\PYG{p}{]}
    \PYG{o}{/}\PYG{n}{Color} \PYG{p}{[}\PYG{o}{.}\PYG{l+m+mi}{7} \PYG{l+m+mi}{0} \PYG{l+m+mi}{0}\PYG{p}{]}
    \PYG{o}{/}\PYG{n}{Subtype} \PYG{o}{/}\PYG{n}{Link}
    \PYG{o}{/}\PYG{n}{ANN} \PYG{n}{pdfmark}


\PYG{o}{\PYGZpc{}} \PYG{n}{Equivalent} \PYG{n}{link} \PYG{n}{using} \PYG{n}{Launch} \PYG{n}{action}

\PYG{p}{[} \PYG{o}{/}\PYG{n}{Rect} \PYG{p}{[}\PYG{l+m+mi}{50} \PYG{l+m+mi}{425} \PYG{l+m+mi}{295} \PYG{l+m+mi}{445}\PYG{p}{]}
    \PYG{o}{/}\PYG{n}{Action} \PYG{o}{/}\PYG{n}{Launch}
    \PYG{o}{/}\PYG{n}{Border} \PYG{p}{[}\PYG{l+m+mi}{0} \PYG{l+m+mi}{0} \PYG{l+m+mi}{2}\PYG{p}{]}
    \PYG{o}{/}\PYG{n}{Color} \PYG{p}{[}\PYG{o}{.}\PYG{l+m+mi}{7} \PYG{l+m+mi}{0} \PYG{l+m+mi}{0}\PYG{p}{]}
    \PYG{o}{/}\PYG{n}{URI} \PYG{p}{(}\PYG{n}{http}\PYG{p}{:}\PYG{o}{/}\PYG{o}{/}\PYG{n}{www}\PYG{o}{.}\PYG{n}{adobe}\PYG{o}{.}\PYG{n}{com}\PYG{p}{)}
    \PYG{o}{/}\PYG{n}{Subtype} \PYG{o}{/}\PYG{n}{Link}
    \PYG{o}{/}\PYG{n}{ANN} \PYG{n}{pdfmark}


\PYG{o}{\PYGZpc{}} \PYG{n}{URI} \PYG{n}{link} \PYG{k}{with} \PYG{n}{a} \PYG{n}{named} \PYG{n}{destination}

\PYG{p}{[} \PYG{o}{/}\PYG{n}{Rect} \PYG{p}{[}\PYG{l+m+mi}{50} \PYG{l+m+mi}{425} \PYG{l+m+mi}{295} \PYG{l+m+mi}{445}\PYG{p}{]}
    \PYG{o}{/}\PYG{n}{Action} \PYG{o}{\PYGZlt{}\PYGZlt{}} \PYG{o}{/}\PYG{n}{Subtype} \PYG{o}{/}\PYG{n}{URI} \PYG{o}{/}\PYG{n}{URI} \PYG{p}{(}\PYG{n}{http}\PYG{p}{:}\PYG{o}{/}\PYG{o}{/}\PYG{n}{www}\PYG{o}{.}\PYG{n}{adobe}\PYG{o}{.}\PYG{n}{com}\PYG{c+c1}{\PYGZsh{}YourDestination) \PYGZgt{}\PYGZgt{}}
    \PYG{o}{/}\PYG{n}{Border} \PYG{p}{[}\PYG{l+m+mi}{0} \PYG{l+m+mi}{0} \PYG{l+m+mi}{2}\PYG{p}{]}
    \PYG{o}{/}\PYG{n}{Color} \PYG{p}{[}\PYG{o}{.}\PYG{l+m+mi}{7} \PYG{l+m+mi}{0} \PYG{l+m+mi}{0}\PYG{p}{]}
    \PYG{o}{/}\PYG{n}{Subtype} \PYG{o}{/}\PYG{n}{Link}
    \PYG{o}{/}\PYG{n}{ANN} \PYG{n}{pdfmark}
\end{sphinxVerbatim}
\begin{enumerate}
\sphinxsetlistlabels{\arabic}{enumi}{enumii}{}{.}%
\item {} 
Custom link action (named action)

\end{enumerate}

\begin{sphinxVerbatim}[commandchars=\\\{\}]
\PYGZpc{} Link with a named action—executes a menu item

[ /Rect [50 425 295 445]
    /Action \PYGZlt{}\PYGZlt{} /Subtype /Named /N /GeneralInfo \PYGZgt{}\PYGZgt{}
    /Border [0 0 2]
    /Color [.7 0 0]
    /Subtype /Link
    /ANN pdfmark
\end{sphinxVerbatim}


\subsection{Other annotations}
\label{\detokenize{pdfmark_Basic:other-annotations}}
A number of other annotation types are available. For example, consider the following movie annotation.
\begin{enumerate}
\sphinxsetlistlabels{\arabic}{enumi}{enumii}{}{.}%
\item {} 
Movie annotation

\end{enumerate}

\begin{sphinxVerbatim}[commandchars=\\\{\}]
\PYG{p}{[} \PYG{o}{/}\PYG{n}{Subtype} \PYG{o}{/}\PYG{n}{Movie}
       \PYG{o}{/}\PYG{n}{Rect} \PYG{p}{[} \PYG{l+m+mi}{216} \PYG{l+m+mi}{503} \PYG{l+m+mi}{361} \PYG{l+m+mi}{612} \PYG{p}{]}
       \PYG{o}{/}\PYG{n}{T} \PYG{p}{(}\PYG{n}{Title}\PYG{p}{)}
       \PYG{o}{/}\PYG{n}{F} \PYG{l+m+mi}{1}

       \PYG{o}{\PYGZpc{}} \PYG{n}{The} \PYG{n}{specified} \PYG{n}{file} \PYG{n}{may} \PYG{n}{be} \PYG{n}{a} \PYG{n}{movie} \PYG{o+ow}{or} \PYG{n}{sound} \PYG{n}{file}
       \PYG{o}{\PYGZpc{}} \PYG{n}{Add} \PYG{n}{your} \PYG{n}{movie} \PYG{o+ow}{in} \PYG{n}{place} \PYG{n}{of} \PYG{l+s+s2}{\PYGZdq{}}\PYG{l+s+s2}{(/Disk/moviefile)}\PYG{l+s+s2}{\PYGZdq{}}

       \PYG{o}{/}\PYG{n}{Movie} \PYG{o}{\PYGZlt{}\PYGZlt{}} \PYG{o}{/}\PYG{n}{F} \PYG{p}{(}\PYG{o}{/}\PYG{n}{Disk}\PYG{o}{/}\PYG{n}{moviefile}\PYG{p}{)} \PYG{o}{/}\PYG{n}{Aspect} \PYG{p}{[} \PYG{l+m+mi}{160} \PYG{l+m+mi}{120} \PYG{p}{]} \PYG{o}{\PYGZgt{}\PYGZgt{}}
       \PYG{o}{/}\PYG{n}{A} \PYG{o}{\PYGZlt{}\PYGZlt{}} \PYG{o}{/}\PYG{n}{ShowControls} \PYG{n}{true} \PYG{o}{\PYGZgt{}\PYGZgt{}}
       \PYG{o}{/}\PYG{n}{Border} \PYG{p}{[}\PYG{l+m+mi}{0} \PYG{l+m+mi}{0} \PYG{l+m+mi}{3}\PYG{p}{]}
       \PYG{o}{/}\PYG{n}{C} \PYG{p}{[}\PYG{l+m+mi}{0} \PYG{l+m+mi}{0} \PYG{l+m+mi}{1}\PYG{p}{]}
       \PYG{o}{/}\PYG{n}{ANN} \PYG{n}{pdfmark}
\end{sphinxVerbatim}

For a complete list of available annotation types, see \sphinxhref{pdfmark\_Basic.html\#50454556\_36966}{PDF annotation types}.

One useful type of annotation is the widget annotation. Widgets are used by PDF interactive forms to represent the appearance of fields and to manage user interactions. See the \sphinxstyleemphasis{PDF Reference} for detailed information on using interactive forms.

For examples of using widget annotations to create interactive forms, see \sphinxhref{pdfmark\_Examples.html\#50454555\_89212}{Define the Widget annotations, which are also field dictionaries for this form}.

The following example appears with an unknown annotation icon in the Acrobat viewers, because they do not know how to interpret this annotation type.
\begin{enumerate}
\sphinxsetlistlabels{\arabic}{enumi}{enumii}{}{.}%
\item {} 
Custom annotation type

\end{enumerate}

\begin{sphinxVerbatim}[commandchars=\\\{\}]
\PYG{p}{[} \PYG{o}{/}\PYG{n}{Rect} \PYG{p}{[}\PYG{l+m+mi}{400} \PYG{l+m+mi}{435} \PYG{l+m+mi}{500} \PYG{l+m+mi}{535}\PYG{p}{]}
    \PYG{o}{/}\PYG{n}{Subtype} \PYG{o}{/}\PYG{n}{ADBETest\PYGZus{}DummyType}
    \PYG{o}{/}\PYG{n}{ADBETest\PYGZus{}F8Array} \PYG{p}{[}\PYG{l+m+mi}{0} \PYG{l+m+mi}{1} \PYG{l+m+mi}{1} \PYG{l+m+mi}{2} \PYG{l+m+mi}{3} \PYG{l+m+mi}{5} \PYG{l+m+mi}{8} \PYG{l+m+mi}{13}\PYG{p}{]}
    \PYG{o}{/}\PYG{n}{ANN} \PYG{n}{pdfmark}
\end{sphinxVerbatim}


\section{Articles (ARTICLE)}
\label{\detokenize{pdfmark_Basic:articles-article}}
Articles consist of a title and a list of rectangular areas called beads. Each bead is specified by the pdfmark operator in conjunction with the feature name \sphinxcode{\sphinxupquote{ARTICLE}} . Beads are added to the article in the order that they are encountered in the PostScript language file.

The syntax for a bead pdfmark is as follows:

\begin{sphinxVerbatim}[commandchars=\\\{\}]
 \PYG{p}{[} \PYG{o}{/}\PYG{n}{Title}
\PYG{n}{string}

        \PYG{o}{/}\PYG{n}{Rect} \PYG{p}{[}
\PYG{n}{xll} \PYG{n}{yll} \PYG{n}{xur} \PYG{n}{yur}
\PYG{p}{]}
        \PYG{o}{/}\PYG{n}{Page}
\PYG{n}{pagenum}

        \PYG{o}{/}\PYG{n}{ARTICLE} \PYG{n}{pdfmark}
\end{sphinxVerbatim}

\sphinxstylestrong{Article bead attributes}


\begin{savenotes}\sphinxattablestart
\centering
\phantomsection\label{\detokenize{pdfmark_Basic:section-4}}\nobreak
\begin{tabulary}{\linewidth}[t]{|T|T|T|}
\hline
\sphinxstyletheadfamily 
Key
&\sphinxstyletheadfamily 
Type
&\sphinxstyletheadfamily 
Semantics
\\
\hline
Title
&
string
&
Required. The title of the article to which a bead belongs. The encoding and character set used is either PDFDocEncoding (as described in the \sphinxstyleemphasis{PDF Reference)} or Unicode. If Unicode, the string must begin with <FEFF>. For example, the Unicode string for (ABC) is <FEFF004100420043>. \sphinxcode{\sphinxupquote{Title}} has a maximum length of 255 PDFDocEncoding characters or 126 Unicode values, although a practical limit of 32 characters is advised so that it can be read easily in the Acrobat viewer.
\\
\hline
Rect
&
array
&
Required. An array of four numbers {[}xll, yll, xur, yur{]} specifying the lower\sphinxhyphen{}left x, lower\sphinxhyphen{}left y, upper\sphinxhyphen{}right x, and upper\sphinxhyphen{}right y coordinates—in user space—of the rectangle defining the bead.
\\
\hline
Page
&
integer
&
Optional. The sequence number of the page on which the bead is located. A bead pdfmark that contains the optional \sphinxcode{\sphinxupquote{Page}} key can be placed anywhere in the PostScript language file. A bead pdfmark that does not contain this key must occur within the PostScript language description for the page on which the article bead is to appear.
\\
\hline
\end{tabulary}
\par
\sphinxattableend\end{savenotes}

In addition to the keys listed in the preceding table, the first bead in an article can also specify arbitrary key–value pairs. Suggested keys are \sphinxcode{\sphinxupquote{Subject}} , \sphinxcode{\sphinxupquote{Author}} , and \sphinxcode{\sphinxupquote{Keywords}} .

\begin{sphinxadmonition}{note}{Note:}
Articles do not support dictionaries as values in arbitrary key–value pairs.
\end{sphinxadmonition}

The following examples demonstrate the use of articles.
\begin{enumerate}
\sphinxsetlistlabels{\arabic}{enumi}{enumii}{}{.}%
\item {} 
Article action

\end{enumerate}

\begin{sphinxVerbatim}[commandchars=\\\{\}]
\PYG{p}{[} \PYG{o}{/}\PYG{n}{Action} \PYG{o}{/}\PYG{n}{Article} \PYG{o}{/}\PYG{n}{Dest} \PYG{p}{(}\PYG{n}{Now} \PYG{o+ow}{is} \PYG{n}{the} \PYG{n}{Time}\PYG{p}{)}
    \PYG{o}{/}\PYG{n}{Title} \PYG{p}{(}\PYG{n}{Now} \PYG{o+ow}{is} \PYG{n}{the} \PYG{n}{Time}\PYG{p}{)}
    \PYG{o}{/}\PYG{n}{OUT} \PYG{n}{pdfmark}
\end{sphinxVerbatim}
\begin{enumerate}
\sphinxsetlistlabels{\arabic}{enumi}{enumii}{}{.}%
\item {} 
Create text for the article “Now is the Time”

\end{enumerate}

\begin{sphinxVerbatim}[commandchars=\\\{\}]
\PYG{o}{/}\PYG{n}{Helvetica} \PYG{l+m+mi}{12} \PYG{n}{selectfont}
\PYG{p}{(}\PYG{n}{Now} \PYG{o+ow}{is} \PYG{n}{the} \PYG{n}{Time} \PYG{p}{(}\PYG{n}{Article}\PYG{p}{)}\PYG{p}{)} \PYG{l+m+mi}{230} \PYG{l+m+mi}{690}  \PYG{n}{moveto} \PYG{n}{show}
\PYG{p}{(}\PYG{n}{Now} \PYG{o+ow}{is} \PYG{n}{the} \PYG{n}{time} \PYG{k}{for} \PYG{n+nb}{all} \PYG{n}{good} \PYG{n}{men} \PYG{n}{to} \PYG{n}{come} \PYG{n}{to} \PYG{n}{the} \PYG{n}{aid} \PYG{n}{of} \PYG{n}{their}
\PYG{n}{country}\PYG{o}{.}\PYG{p}{)} \PYG{l+m+mi}{230} \PYG{l+m+mi}{670}  \PYG{n}{moveto} \PYG{n}{show}
\PYG{p}{(}\PYG{n}{Now} \PYG{o+ow}{is} \PYG{n}{the} \PYG{n}{time} \PYG{k}{for} \PYG{n+nb}{all} \PYG{n}{good} \PYG{n}{people} \PYG{n}{to} \PYG{n}{come} \PYG{n}{to} \PYG{n}{the} \PYG{n}{aid} \PYG{n}{of} \PYG{n}{their}
\PYG{n}{country}\PYG{o}{.}\PYG{p}{)} \PYG{l+m+mi}{230} \PYG{l+m+mi}{655} \PYG{n}{moveto} \PYG{n}{show}

\PYG{o}{\PYGZpc{}} \PYG{o}{.}\PYG{o}{.}\PYG{o}{.} \PYG{n}{additional} \PYG{n}{text} \PYG{o}{.}\PYG{o}{.}\PYG{o}{.}

\PYG{p}{(}\PYG{n}{Click} \PYG{n}{here} \PYG{n}{to} \PYG{n}{go} \PYG{n}{to} \PYG{n}{Adobe}\PYG{l+s+s1}{\PYGZsq{}}\PYG{l+s+s1}{s Home Page on the Web) 55 430 moveto show}
\end{sphinxVerbatim}
\begin{enumerate}
\sphinxsetlistlabels{\arabic}{enumi}{enumii}{}{.}%
\item {} 
Article containing two beads

\end{enumerate}

\begin{sphinxVerbatim}[commandchars=\\\{\}]
\PYG{p}{[} \PYG{o}{/}\PYG{n}{Title} \PYG{p}{(}\PYG{n}{Now} \PYG{o+ow}{is} \PYG{n}{the} \PYG{n}{Time}\PYG{p}{)}
    \PYG{o}{/}\PYG{n}{Author} \PYG{p}{(}\PYG{n}{John} \PYG{n}{Doe}\PYG{p}{)}
    \PYG{o}{/}\PYG{n}{Subject} \PYG{p}{(}\PYG{n}{Coming} \PYG{n}{to} \PYG{n}{the} \PYG{n}{aid} \PYG{n}{of} \PYG{n}{your} \PYG{n}{country}\PYG{p}{)}
    \PYG{o}{/}\PYG{n}{Keywords} \PYG{p}{(}\PYG{n}{Time}\PYG{p}{,} \PYG{n}{Country}\PYG{p}{,} \PYG{n}{Aid}\PYG{p}{)}
    \PYG{o}{/}\PYG{n}{Rect} \PYG{p}{[} \PYG{l+m+mi}{225} \PYG{l+m+mi}{500} \PYG{l+m+mi}{535} \PYG{l+m+mi}{705} \PYG{p}{]}
    \PYG{o}{/}\PYG{n}{Page} \PYG{l+m+mi}{2}
    \PYG{o}{/}\PYG{n}{ARTICLE} \PYG{n}{pdfmark}
\PYG{p}{[} \PYG{o}{/}\PYG{n}{Title} \PYG{p}{(}\PYG{n}{Now} \PYG{o+ow}{is} \PYG{n}{the} \PYG{n}{Time}\PYG{p}{)}
    \PYG{o}{/}\PYG{n}{Rect} \PYG{p}{[}\PYG{l+m+mi}{225} \PYG{l+m+mi}{500} \PYG{l+m+mi}{535} \PYG{l+m+mi}{705}\PYG{p}{]}
    \PYG{o}{/}\PYG{n}{Page} \PYG{l+m+mi}{3}
    \PYG{o}{/}\PYG{n}{ARTICLE} \PYG{n}{pdfmark}
\end{sphinxVerbatim}


\section{Bookmarks (OUT)}
\label{\detokenize{pdfmark_Basic:bookmarks-out}}
Bookmarks are known as outline items in PDF. They are specified by using the pdfmark operator with the feature name \sphinxcode{\sphinxupquote{OUT}} .

The syntax for a bookmark pdfmark is as follows:

\begin{sphinxVerbatim}[commandchars=\\\{\}]
 [ /Title
string

        /Count
int

        /Color
array

        /F
integer
        …Action\PYGZhy{}specifying key–value pairs…

        /OUT pdfmark
\end{sphinxVerbatim}

\sphinxstylestrong{Bookmark attributes}


\begin{savenotes}\sphinxattablestart
\centering
\phantomsection\label{\detokenize{pdfmark_Basic:section-5}}\nobreak
\begin{tabular}[t]{|*{3}{\X{1}{3}|}}
\hline
\sphinxstyletheadfamily 
Key
&\sphinxstyletheadfamily 
Type
&\sphinxstyletheadfamily 
Semantics
\\
\hline
Title
&
string
&
Required. The bookmark’s text. The encoding and character set used is either PDFDocEncoding (as described in the \sphinxstyleemphasis{PDF Reference)} or Unicode. If Unicode, the string must begin with <FEFF>. For example, the Unicode string for (ABC) is <FEFF004100420043>. \sphinxcode{\sphinxupquote{Title}} has a maximum length of 255 PDFDocEncoding characters or 126 Unicode values, although a practical limit of 32 characters is advised so that it can be read easily in the Acrobat viewer.
\\
\hline
Count
&
integer
&
Required if the bookmark has subordinate bookmarks, omitted otherwise. This key’s absolute value is the number of bookmarks immediately subordinate—that is, excluding subordinates of subordinates. If the value is positive, the bookmark is open, revealing its subordinates; if negative, the bookmark is closed, hiding its subordinates.
\begin{itemize}
\item {} 
This differs from the PDF \sphinxcode{\sphinxupquote{Count}} key, which represents the total number of open descendants at all lower levels of the outline hierarchy.

\end{itemize}
\\
\hline
Color
&
array
&
Optional. The bookmark’s color. The value is an array containing three numbers (red, green, and blue), each of which must be between 0 and 1, inclusive, specifying a color in the DeviceRGB color space. (See the \sphinxstyleemphasis{PDF Reference} for a description of this color space.)
\\
\hline
F
&
integer
&
Optional. The style of the bookmark. Four styles are implemented:
\begin{itemize}
\item {} 
0 — Plain (the default)

\item {} 
1 — Italic

\item {} 
2 — Bold

\item {} 
3 — Bold and Italic

\end{itemize}
\\
\hline
\end{tabular}
\par
\sphinxattableend\end{savenotes}

In addition to the keys listed in the table \sphinxhref{pdfmark\_Basic.html\#50454556\_29668}{Bookmark attributes}, a bookmark must contain key–value pairs that specify an action. See \sphinxhref{pdfmark\_Actions.html\#50454557\_94191}{Actions and Destinations} for more information.

The bookmark pdfmarks can begin anywhere in the PostScript language file. However, they must appear in sequential order.
\begin{enumerate}
\sphinxsetlistlabels{\arabic}{enumi}{enumii}{}{.}%
\item {} 
Bookmark examples

\end{enumerate}

\begin{sphinxVerbatim}[commandchars=\\\{\}]
\PYG{p}{[} \PYG{o}{/}\PYG{n}{Count} \PYG{l+m+mi}{2} \PYG{o}{/}\PYG{n}{Page} \PYG{l+m+mi}{1} \PYG{o}{/}\PYG{n}{View} \PYG{p}{[}\PYG{o}{/}\PYG{n}{XYZ} \PYG{l+m+mi}{44} \PYG{l+m+mi}{730} \PYG{l+m+mf}{1.0}\PYG{p}{]} \PYG{o}{/}\PYG{n}{Title} \PYG{p}{(}\PYG{n}{Open} \PYG{n}{Actions}\PYG{p}{)} \PYG{o}{/}\PYG{n}{OUT} \PYG{n}{pdfmark}
\PYG{p}{[} \PYG{o}{/}\PYG{n}{Action} \PYG{o}{/}\PYG{n}{Launch} \PYG{o}{/}\PYG{n}{File} \PYG{p}{(}\PYG{n}{test}\PYG{o}{.}\PYG{n}{doc}\PYG{p}{)} \PYG{o}{/}\PYG{n}{Title} \PYG{p}{(}\PYG{n}{Open} \PYG{n}{test}\PYG{o}{.}\PYG{n}{doc}\PYG{p}{)} \PYG{o}{/}\PYG{n}{OUT} \PYG{n}{pdfmark}
\PYG{p}{[} \PYG{o}{/}\PYG{n}{Action} \PYG{o}{/}\PYG{n}{GoToR} \PYG{o}{/}\PYG{n}{File} \PYG{p}{(}\PYG{n}{test}\PYG{o}{.}\PYG{n}{pdf}\PYG{p}{)} \PYG{o}{/}\PYG{n}{Page} \PYG{l+m+mi}{2} \PYG{o}{/}\PYG{n}{View} \PYG{p}{[}\PYG{o}{/}\PYG{n}{FitR} \PYG{l+m+mi}{30} \PYG{l+m+mi}{648} \PYG{l+m+mi}{209} \PYG{l+m+mi}{761}\PYG{p}{]}
    \PYG{o}{/}\PYG{n}{Title} \PYG{p}{(}\PYG{n}{Open} \PYG{n}{test}\PYG{o}{.}\PYG{n}{pdf} \PYG{n}{on} \PYG{n}{page} \PYG{l+m+mi}{2}\PYG{p}{)} \PYG{o}{/}\PYG{n}{OUT} \PYG{n}{pdfmark}


\PYG{p}{[} \PYG{o}{/}\PYG{n}{Count} \PYG{l+m+mi}{2} \PYG{o}{/}\PYG{n}{Page} \PYG{l+m+mi}{2} \PYG{o}{/}\PYG{n}{View} \PYG{p}{[}\PYG{o}{/}\PYG{n}{XYZ} \PYG{l+m+mi}{44} \PYG{l+m+mi}{730} \PYG{l+m+mf}{1.0}\PYG{p}{]} \PYG{o}{/}\PYG{n}{Title} \PYG{p}{(}\PYG{n}{Fixed} \PYG{n}{Zoom}\PYG{p}{)} \PYG{o}{/}\PYG{n}{OUT} \PYG{n}{pdfmark}
\PYG{p}{[} \PYG{o}{/}\PYG{n}{Page} \PYG{l+m+mi}{2} \PYG{o}{/}\PYG{n}{View} \PYG{p}{[}\PYG{o}{/}\PYG{n}{XYZ} \PYG{l+m+mi}{44} \PYG{l+m+mi}{730} \PYG{l+m+mf}{2.0}\PYG{p}{]} \PYG{o}{/}\PYG{n}{Title} \PYG{p}{(}\PYG{l+m+mi}{200}\PYG{o}{\PYGZpc{}} \PYG{n}{Magnification}\PYG{p}{)} \PYG{o}{/}\PYG{n}{OUT} \PYG{n}{pdfmark}
\PYG{p}{[} \PYG{o}{/}\PYG{n}{Count} \PYG{l+m+mi}{1} \PYG{o}{/}\PYG{n}{Page} \PYG{l+m+mi}{2} \PYG{o}{/}\PYG{n}{View} \PYG{p}{[}\PYG{o}{/}\PYG{n}{XYZ} \PYG{l+m+mi}{44} \PYG{l+m+mi}{730} \PYG{l+m+mf}{4.0}\PYG{p}{]} \PYG{o}{/}\PYG{n}{Title} \PYG{p}{(}\PYG{l+m+mi}{400}\PYG{o}{\PYGZpc{}} \PYG{n}{Magnification}\PYG{p}{)}
    \PYG{o}{/}\PYG{n}{OUT} \PYG{n}{pdfmark}
\PYG{p}{[} \PYG{o}{/}\PYG{n}{Page} \PYG{l+m+mi}{2} \PYG{o}{/}\PYG{n}{View} \PYG{p}{[}\PYG{o}{/}\PYG{n}{XYZ} \PYG{l+m+mi}{44} \PYG{l+m+mi}{730} \PYG{l+m+mf}{5.23}\PYG{p}{]} \PYG{o}{/}\PYG{n}{Title} \PYG{p}{(}\PYG{l+m+mi}{523}\PYG{o}{\PYGZpc{}} \PYG{n}{Magnification}\PYG{p}{)} \PYG{o}{/}\PYG{n}{OUT} \PYG{n}{pdfmark}


\PYG{p}{[} \PYG{o}{/}\PYG{n}{Count} \PYG{l+m+mi}{3} \PYG{o}{/}\PYG{n}{Page} \PYG{l+m+mi}{1} \PYG{o}{/}\PYG{n}{View} \PYG{p}{[}\PYG{o}{/}\PYG{n}{XYZ} \PYG{l+m+mi}{44} \PYG{l+m+mi}{730} \PYG{l+m+mf}{1.0}\PYG{p}{]} \PYG{o}{/}\PYG{n}{Title} \PYG{p}{(}\PYG{n}{Table} \PYG{n}{of} \PYG{n}{Contents} \PYG{c+c1}{\PYGZsh{}1)}
    \PYG{o}{/}\PYG{n}{OUT} \PYG{n}{pdfmark}
\PYG{p}{[} \PYG{o}{/}\PYG{n}{Page} \PYG{l+m+mi}{1} \PYG{o}{/}\PYG{n}{View} \PYG{p}{[}\PYG{o}{/}\PYG{n}{XYZ} \PYG{l+m+mi}{44} \PYG{l+m+mi}{730} \PYG{l+m+mf}{1.0}\PYG{p}{]} \PYG{o}{/}\PYG{n}{Title} \PYG{p}{(}\PYG{n}{Page} \PYG{l+m+mi}{1} \PYG{o}{\PYGZhy{}} \PYG{l+m+mi}{100}\PYG{o}{\PYGZpc{}}\PYG{p}{)} \PYG{o}{/}\PYG{n}{OUT} \PYG{n}{pdfmark}
\PYG{p}{[} \PYG{o}{/}\PYG{n}{Page} \PYG{l+m+mi}{2} \PYG{o}{/}\PYG{n}{View} \PYG{p}{[}\PYG{o}{/}\PYG{n}{XYZ} \PYG{l+m+mi}{44} \PYG{l+m+mi}{730} \PYG{l+m+mf}{2.25}\PYG{p}{]} \PYG{o}{/}\PYG{n}{Title} \PYG{p}{(}\PYG{n}{Page} \PYG{l+m+mi}{2} \PYG{o}{\PYGZhy{}} \PYG{l+m+mi}{225}\PYG{o}{\PYGZpc{}}\PYG{p}{)} \PYG{o}{/}\PYG{n}{OUT} \PYG{n}{pdfmark}
\PYG{p}{[} \PYG{o}{/}\PYG{n}{Page} \PYG{l+m+mi}{3} \PYG{o}{/}\PYG{n}{View} \PYG{p}{[}\PYG{o}{/}\PYG{n}{Fit}\PYG{p}{]} \PYG{o}{/}\PYG{n}{Title} \PYG{p}{(}\PYG{n}{Page} \PYG{l+m+mi}{3} \PYG{o}{\PYGZhy{}} \PYG{n}{Fit} \PYG{n}{Page}\PYG{p}{)} \PYG{o}{/}\PYG{n}{OUT} \PYG{n}{pdfmark}


\PYG{p}{[} \PYG{o}{/}\PYG{n}{Count} \PYG{o}{\PYGZhy{}}\PYG{l+m+mi}{3} \PYG{o}{/}\PYG{n}{Page} \PYG{l+m+mi}{1} \PYG{o}{/}\PYG{n}{View} \PYG{p}{[}\PYG{o}{/}\PYG{n}{XYZ} \PYG{l+m+mi}{44} \PYG{l+m+mi}{730} \PYG{l+m+mf}{1.0}\PYG{p}{]} \PYG{o}{/}\PYG{n}{Title} \PYG{p}{(}\PYG{n}{Table} \PYG{n}{of} \PYG{n}{Contents} \PYG{c+c1}{\PYGZsh{}2)}
    \PYG{o}{/}\PYG{n}{OUT} \PYG{n}{pdfmark}
\PYG{p}{[} \PYG{o}{/}\PYG{n}{Page} \PYG{l+m+mi}{1} \PYG{o}{/}\PYG{n}{View} \PYG{p}{[}\PYG{o}{/}\PYG{n}{XYZ} \PYG{n}{null} \PYG{n}{null} \PYG{l+m+mi}{0}\PYG{p}{]} \PYG{o}{/}\PYG{n}{Title} \PYG{p}{(}\PYG{n}{Page} \PYG{l+m+mi}{1} \PYG{o}{\PYGZhy{}} \PYG{n}{Inherit}\PYG{p}{)} \PYG{o}{/}\PYG{n}{OUT} \PYG{n}{pdfmark}
\PYG{p}{[} \PYG{o}{/}\PYG{n}{Page} \PYG{l+m+mi}{2} \PYG{o}{/}\PYG{n}{View} \PYG{p}{[}\PYG{o}{/}\PYG{n}{XYZ} \PYG{n}{null} \PYG{n}{null} \PYG{l+m+mi}{0}\PYG{p}{]} \PYG{o}{/}\PYG{n}{Title} \PYG{p}{(}\PYG{n}{Page} \PYG{l+m+mi}{2} \PYG{o}{\PYGZhy{}} \PYG{n}{Inherit}\PYG{p}{)} \PYG{o}{/}\PYG{n}{OUT} \PYG{n}{pdfmark}
\PYG{p}{[} \PYG{o}{/}\PYG{n}{Page} \PYG{l+m+mi}{3} \PYG{o}{/}\PYG{n}{View} \PYG{p}{[}\PYG{o}{/}\PYG{n}{XYZ} \PYG{n}{null} \PYG{n}{null} \PYG{l+m+mi}{0}\PYG{p}{]} \PYG{o}{/}\PYG{n}{Title} \PYG{p}{(}\PYG{n}{Page} \PYG{l+m+mi}{3} \PYG{o}{\PYGZhy{}} \PYG{n}{Inherit}\PYG{p}{)} \PYG{o}{/}\PYG{n}{OUT} \PYG{n}{pdfmark}


\PYG{p}{[} \PYG{o}{/}\PYG{n}{Count} \PYG{l+m+mi}{1} \PYG{o}{/}\PYG{n}{Page} \PYG{l+m+mi}{0} \PYG{o}{/}\PYG{n}{Title} \PYG{p}{(}\PYG{n}{Articles}\PYG{p}{)} \PYG{o}{/}\PYG{n}{OUT} \PYG{n}{pdfmark}
\PYG{p}{[} \PYG{o}{/}\PYG{n}{Action} \PYG{o}{/}\PYG{n}{Article} \PYG{o}{/}\PYG{n}{Dest} \PYG{p}{(}\PYG{n}{Now} \PYG{o+ow}{is} \PYG{n}{the} \PYG{n}{Time}\PYG{p}{)} \PYG{o}{/}\PYG{n}{Title} \PYG{p}{(}\PYG{n}{Now} \PYG{o+ow}{is} \PYG{n}{the} \PYG{n}{Time}\PYG{p}{)} \PYG{o}{/}\PYG{n}{OUT} \PYG{n}{pdfmark}


\PYG{o}{\PYGZpc{}} \PYG{n}{Bookmark} \PYG{k}{with} \PYG{n}{color} \PYG{o+ow}{and} \PYG{n}{style} \PYG{p}{(}\PYG{n}{new} \PYG{o+ow}{in} \PYG{n}{Acrobat} \PYG{l+m+mf}{5.0}\PYG{p}{)}

\PYG{p}{[} \PYG{o}{/}\PYG{n}{Count} \PYG{l+m+mi}{0}
    \PYG{o}{/}\PYG{n}{Title} \PYG{p}{(}\PYG{n}{The} \PYG{n}{Adobe} \PYG{n}{home} \PYG{n}{page}\PYG{p}{)}
    \PYG{o}{/}\PYG{n}{Action} \PYG{o}{/}\PYG{n}{Launch}
    \PYG{o}{/}\PYG{n}{URI} \PYG{p}{(}\PYG{n}{http}\PYG{p}{:}\PYG{o}{/}\PYG{o}{/}\PYG{n}{www}\PYG{o}{.}\PYG{n}{adobe}\PYG{o}{.}\PYG{n}{com}\PYG{p}{)}
    \PYG{o}{/}\PYG{n}{C} \PYG{p}{[}\PYG{l+m+mi}{1} \PYG{l+m+mi}{0} \PYG{l+m+mi}{0}\PYG{p}{]}
    \PYG{o}{/}\PYG{n}{F} \PYG{l+m+mi}{3}
    \PYG{o}{/}\PYG{n}{OUT} \PYG{n}{pdfmark}


\PYG{o}{\PYGZpc{}} \PYG{n}{Bookmark} \PYG{k}{with} \PYG{n}{a} \PYG{n}{URI} \PYG{k}{as} \PYG{n}{an} \PYG{n}{action}

\PYG{p}{[}  \PYG{o}{/}\PYG{n}{Count} \PYG{l+m+mi}{0} \PYG{o}{/}\PYG{n}{Title} \PYG{p}{(}\PYG{n}{The} \PYG{n}{Adobe} \PYG{n}{home} \PYG{n}{page}\PYG{p}{)}
    \PYG{o}{/}\PYG{n}{Action} \PYG{o}{\PYGZlt{}\PYGZlt{}} \PYG{o}{/}\PYG{n}{Subtype} \PYG{o}{/}\PYG{n}{URI} \PYG{o}{/}\PYG{n}{URI} \PYG{p}{(}\PYG{n}{http}\PYG{p}{:}\PYG{o}{/}\PYG{o}{/}\PYG{n}{www}\PYG{o}{.}\PYG{n}{adobe}\PYG{o}{.}\PYG{n}{com}\PYG{p}{)}\PYG{o}{\PYGZgt{}\PYGZgt{}} \PYG{o}{/}\PYG{n}{OUT} \PYG{n}{pdfmark}
\end{sphinxVerbatim}


\section{Document Info dictionary (DOCINFO)}
\label{\detokenize{pdfmark_Basic:document-info-dictionary-docinfo}}
A document’s Info dictionary contains key–value pairs that provide various pieces of information about the document. Info dictionary information is specified by using the pdfmark operator in conjunction with the name \sphinxcode{\sphinxupquote{DOCINFO}} .

The syntax for specifying Info dictionary entries is as follows:

\begin{sphinxVerbatim}[commandchars=\\\{\}]
 \PYG{p}{[} \PYG{o}{/}\PYG{n}{Author}
\PYG{n}{string}


        \PYG{o}{/}\PYG{n}{CreationDate}
\PYG{n}{string}


        \PYG{o}{/}\PYG{n}{Creator}
\PYG{n}{string}


        \PYG{o}{/}\PYG{n}{Producer}
\PYG{n}{string}


        \PYG{o}{/}\PYG{n}{Title}
\PYG{n}{string}


        \PYG{o}{/}\PYG{n}{Subject}
\PYG{n}{string}


        \PYG{o}{/}\PYG{n}{Keywords}
\PYG{n}{string}


        \PYG{o}{/}\PYG{n}{ModDate}
\PYG{n}{string}


        \PYG{o}{/}\PYG{n}{DOCINFO} \PYG{n}{pdfmark}
\end{sphinxVerbatim}

All the allowable keys are strings, and they are all optional. In addition to the keys listed in the following table, arbitrary keys (which must also take string values) can be specified.

Info dictionary attributes


\begin{savenotes}\sphinxattablestart
\centering
\phantomsection\label{\detokenize{pdfmark_Basic:section-6}}\nobreak
\begin{tabulary}{\linewidth}[t]{|T|T|T|}
\hline
\sphinxstyletheadfamily 
Key
&\sphinxstyletheadfamily 
Type
&\sphinxstyletheadfamily 
Semantics
\\
\hline
Author
&
string
&
Optional. The document’s author
\\
\hline
CreationDate
&
string
&
Optional. The date the document was created. See the description of the \sphinxcode{\sphinxupquote{ModDate}} key for information on the string’s format.
\\
\hline
Creator
&
string
&
Optional. If the document was converted to PDF from another form, the name of the application that originally created the document
\\
\hline
Producer
&
string
&
Optional. The name of the application that converted the document from its native form to PDF.

\begin{sphinxadmonition}{note}{Note:}
Distiller ignores the setting of this attribute.
\end{sphinxadmonition}
\\
\hline
Title
&
string
&
Optional. The document’s title.
\\
\hline
Subject
&
string
&
Optional. The document’s subject.
\\
\hline
Keywords
&
string
&
Optional. Keywords relevant for this document. These are used primarily in cross\sphinxhyphen{}document searches.
\\
\hline
ModDate
&
string
&
Optional. The date and time the document was last modified. It should be of the form:

( \sphinxstyleemphasis{D:YYYYMMDDHHmmSSOHH’mm’} )

\sphinxstyleemphasis{D:} is an optional prefix. \sphinxstyleemphasis{YYYY} is the year. All fields after the year are optional. \sphinxstyleemphasis{MM} is the month (01\sphinxhyphen{}12), \sphinxstyleemphasis{DD} is the day (01\sphinxhyphen{}31), \sphinxstyleemphasis{HH} is the hour (00\sphinxhyphen{}23), \sphinxstyleemphasis{mm} are the minutes (00\sphinxhyphen{}59), and \sphinxstyleemphasis{SS} are the seconds (00\sphinxhyphen{}59). The remainder of the string defines the relation of local time to GMT. O is either + for a positive difference (local time is later than GMT) or \sphinxhyphen{} (minus) for a negative difference. \sphinxstyleemphasis{HH’} is the absolute value of the offset from GMT in hours, and \sphinxstyleemphasis{mm’} is the absolute value of the offset in minutes. If no GMT information is specified, the relation between the specified time and GMT is considered unknown. Regardless of whether or not GMT information is specified, the remainder of the string should specify the local time.
\\
\hline
\end{tabulary}
\par
\sphinxattableend\end{savenotes}

Info dictionary pdfmarks can occur anywhere in the PostScript language file.
\begin{enumerate}
\sphinxsetlistlabels{\arabic}{enumi}{enumii}{}{.}%
\item {} 
Info dictionary

\end{enumerate}

\begin{sphinxVerbatim}[commandchars=\\\{\}]
\PYG{p}{[} \PYG{o}{/}\PYG{n}{Title} \PYG{p}{(}\PYG{n}{My} \PYG{n}{Test} \PYG{n}{Document}\PYG{p}{)}
    \PYG{o}{/}\PYG{n}{Author} \PYG{p}{(}\PYG{n}{John} \PYG{n}{Doe}\PYG{p}{)}
    \PYG{o}{/}\PYG{n}{Subject} \PYG{p}{(}\PYG{n}{pdfmark} \PYG{l+m+mf}{3.0}\PYG{p}{)}
    \PYG{o}{/}\PYG{n}{Keywords} \PYG{p}{(}\PYG{n}{pdfmark}\PYG{p}{,} \PYG{n}{example}\PYG{p}{,} \PYG{n}{test}\PYG{p}{)}
    \PYG{o}{/}\PYG{n}{Creator} \PYG{p}{(}\PYG{n}{Hand} \PYG{n}{programmed}\PYG{p}{)}
    \PYG{o}{/}\PYG{n}{ModificationDate} \PYG{p}{(}\PYG{n}{D}\PYG{p}{:}\PYG{l+m+mi}{19940912205731}\PYG{p}{)}
    \PYG{o}{/}\PYG{n}{ADBETest\PYGZus{}MyKey} \PYG{p}{(}\PYG{n}{My} \PYG{n}{private} \PYG{n}{information}\PYG{p}{)}
    \PYG{o}{/}\PYG{n}{DOCINFO} \PYG{n}{pdfmark}
\end{sphinxVerbatim}


\section{Document open options (DOCVIEW)}
\label{\detokenize{pdfmark_Basic:document-open-options-docview}}
A PDF file can specify the following to determine what happens when it is opened:
\begin{itemize}
\item {} 
The way the document is displayed. The options are: the document only, the document plus thumbnail images, the document plus bookmarks, or just the document in full screen mode.

\item {} 
A location other than the first page that is to be displayed.

\item {} 
An optional action that occurs.

\end{itemize}

The above information is contained in key–value pairs in the document’s Catalog dictionary. This information can be set using the pdfmark operator in conjunction with the name \sphinxcode{\sphinxupquote{DOCVIEW}} .

The syntax for specifying Catalog dictionary entries is as follows:

\begin{sphinxVerbatim}[commandchars=\\\{\}]
[ /PageMode name
       …Action\PYGZhy{}specifying key–value pairs…
       /DOCVIEW pdfmark
\end{sphinxVerbatim}

The \sphinxcode{\sphinxupquote{PageMode}} key specifies how the document is to be displayed when opened. It can take the following values:
\begin{itemize}
\item {} 
UseNone — Open the document, displaying neither bookmarks nor thumbnail images.

\item {} 
UseOutlines — Open the document and display bookmarks.

\item {} 
UseThumbs — Open the document and display thumbnail images.

\item {} 
FullScreen — Open the document in full screen mode.

\end{itemize}

If \sphinxcode{\sphinxupquote{PageMode}} is not specified, the value defaults to \sphinxcode{\sphinxupquote{UseNone}} .

The \sphinxcode{\sphinxupquote{DOCVIEW}} pdfmark can also specify a destination (a page to which the document should be opened) or an action, by using additional key–value pairs. See \sphinxhref{pdfmark\_Actions.html\#50454557\_94191}{Actions and Destinations} for details about the key–value pairs that can be used.

\sphinxcode{\sphinxupquote{DOCVIEW}} pdfmarks can occur anywhere in the PostScript language file.
\begin{enumerate}
\sphinxsetlistlabels{\arabic}{enumi}{enumii}{}{.}%
\item {} 
File Open action

\end{enumerate}

\begin{sphinxVerbatim}[commandchars=\\\{\}]
\PYG{p}{[} \PYG{o}{/}\PYG{n}{PageMode} \PYG{o}{/}\PYG{n}{UseOutlines}
    \PYG{o}{/}\PYG{n}{Page} \PYG{l+m+mi}{2} \PYG{o}{/}\PYG{n}{View} \PYG{p}{[}\PYG{o}{/}\PYG{n}{XYZ} \PYG{n}{null} \PYG{n}{null} \PYG{n}{null}\PYG{p}{]}
    \PYG{o}{/}\PYG{n}{DOCVIEW} \PYG{n}{pdfmark}
\end{sphinxVerbatim}


\section{Embedded file content (EMBED)}
\label{\detokenize{pdfmark_Basic:embedded-file-content-embed}}
The pdfmark feature \sphinxcode{\sphinxupquote{EMBED}} enables the embedding of file content into a PDF document.

The syntax for specifying \sphinxcode{\sphinxupquote{EMBED}} dictionary entries is as follows:

\begin{sphinxVerbatim}[commandchars=\\\{\}]
\PYG{p}{[} \PYG{o}{/}\PYG{n}{Name} \PYG{p}{(}\PYG{n}{Unicode} \PYG{n}{Name}\PYG{p}{)}
       \PYG{o}{/}\PYG{n}{FS} \PYG{o}{\PYGZlt{}\PYGZlt{}} \PYG{o}{/}\PYG{n}{Type} \PYG{o}{/}\PYG{n}{Filespec} \PYG{o}{/}\PYG{n}{F} \PYG{p}{(}\PYG{n}{name}\PYG{p}{)} \PYG{o}{/}\PYG{n}{EF} \PYG{o}{\PYGZlt{}\PYGZlt{}} \PYG{o}{/}\PYG{n}{F} \PYG{p}{\PYGZob{}}\PYG{n}{streamName}\PYG{p}{\PYGZcb{}} \PYG{o}{\PYGZgt{}\PYGZgt{}} \PYG{o}{\PYGZgt{}\PYGZgt{}}
       \PYG{n}{EMBED} \PYG{n}{pdfmark}
\end{sphinxVerbatim}

The \sphinxcode{\sphinxupquote{EMBED}} pdfmark directs Adobe Distiller to embed files in the \sphinxcode{\sphinxupquote{EmbeddedFiles}} dictionary of the PDF document’s name tree. The following PDF segment is an example of an \sphinxcode{\sphinxupquote{EmbeddedFiles}} dictionary.

\begin{sphinxVerbatim}[commandchars=\\\{\}]
\PYG{o}{\PYGZlt{}\PYGZlt{}}    \PYG{o}{/}\PYG{n}{Type} \PYG{o}{/}\PYG{n}{Catalog}    \PYG{o}{\PYGZpc{}} \PYG{n}{The} \PYG{n}{catalog} \PYG{n}{dictionary}

       \PYG{o}{/}\PYG{n}{Names}    \PYG{o}{\PYGZpc{}} \PYG{n}{The} \PYG{n}{name} \PYG{n}{dictionary}

       \PYG{o}{\PYGZlt{}\PYGZlt{}}    \PYG{o}{/}\PYG{n}{EmbeddedFiles}    \PYG{o}{\PYGZpc{}} \PYG{n}{One} \PYG{n}{particular} \PYG{n}{name} \PYG{n}{tree}

           \PYG{o}{\PYGZlt{}\PYGZlt{}}    \PYG{o}{/}\PYG{n}{Names}    \PYG{o}{\PYGZpc{}} \PYG{n}{The} \PYG{n}{name} \PYG{n}{tree} \PYG{n}{node}

               \PYG{p}{[}
               \PYG{p}{(}\PYG{n}{Unicode} \PYG{n}{Name}\PYG{p}{)}    \PYG{o}{\PYGZpc{}} \PYG{n}{Unique} \PYG{n}{Unicode} \PYG{n}{string} \PYG{n}{used} \PYG{k}{for} \PYG{n}{Acrobat} \PYG{n}{access}

                   \PYG{o}{\PYGZlt{}\PYGZlt{}}    \PYG{o}{\PYGZpc{}} \PYG{n}{The} \PYG{n}{file} \PYG{n}{specification} \PYG{n}{dictionary}

                   \PYG{o}{/}\PYG{n}{F} \PYG{p}{(}\PYG{n}{name}\PYG{p}{)}    \PYG{o}{\PYGZpc{}} \PYG{n}{The} \PYG{n}{file} \PYG{n}{name} \PYG{k}{for} \PYG{n}{export}

                   \PYG{o}{/}\PYG{n}{EF} \PYG{o}{\PYGZlt{}\PYGZlt{}} \PYG{o}{.}\PYG{o}{.}\PYG{o}{.} \PYG{o}{\PYGZgt{}\PYGZgt{}}    \PYG{o}{\PYGZpc{}} \PYG{n}{Embedded} \PYG{n}{file} \PYG{n}{stream} \PYG{n}{dictionary}

                   \PYG{o}{\PYGZgt{}\PYGZgt{}}
               \PYG{p}{]}
           \PYG{o}{\PYGZgt{}\PYGZgt{}}
       \PYG{o}{\PYGZgt{}\PYGZgt{}}
   \PYG{o}{\PYGZgt{}\PYGZgt{}}
\end{sphinxVerbatim}

For example:

\begin{sphinxVerbatim}[commandchars=\\\{\}]
 \PYG{p}{[} \PYG{o}{/}\PYG{n}{NamespacePush} \PYG{n}{pdfmark}
    \PYG{p}{[} \PYG{o}{/}\PYG{n}{\PYGZus{}objdef} \PYG{p}{\PYGZob{}}\PYG{n}{fstream}\PYG{p}{\PYGZcb{}} \PYG{o}{/}\PYG{n+nb}{type} \PYG{o}{/}\PYG{n}{stream} \PYG{o}{/}\PYG{n}{OBJ} \PYG{n}{pdfmark}
    \PYG{p}{[}   \PYG{p}{\PYGZob{}}\PYG{n}{fstream}\PYG{p}{\PYGZcb{}} \PYG{o}{\PYGZlt{}\PYGZlt{}} \PYG{o}{/}\PYG{n}{Type} \PYG{o}{/}\PYG{n}{EmbeddedFile} \PYG{o}{\PYGZgt{}\PYGZgt{}} \PYG{o}{/}\PYG{n}{PUT} \PYG{n}{pdfmark}
    \PYG{p}{[}   \PYG{p}{\PYGZob{}}\PYG{n}{fstream}\PYG{p}{\PYGZcb{}} \PYG{p}{(}\PYG{n}{Simulating} \PYG{n}{file} \PYG{n}{content} \PYG{n}{here}\PYG{p}{)} \PYG{o}{/}\PYG{n}{PUT} \PYG{n}{pdfmark}
    \PYG{p}{[} \PYG{o}{/}\PYG{n}{Name} \PYG{p}{(}
\PYG{n}{Unicode} \PYG{n}{Unique} \PYG{n}{Name}
\PYG{p}{)}
\PYG{o}{\PYGZpc{}} \PYG{n}{e}\PYG{o}{.}\PYG{n}{g}\PYG{o}{.}\PYG{p}{,} \PYG{o}{\PYGZlt{}}\PYG{n}{feff} \PYG{l+m+mi}{0041} \PYG{l+m+mi}{0073}\PYG{o}{\PYGZgt{}} \PYG{o+ow}{is} \PYG{n}{Unicode} \PYG{k}{for} \PYG{l+s+s2}{\PYGZdq{}}\PYG{l+s+s2}{As}\PYG{l+s+s2}{\PYGZdq{}}

        \PYG{o}{/}\PYG{n}{FS} \PYG{o}{\PYGZlt{}\PYGZlt{}}
            \PYG{o}{/}\PYG{n}{Type} \PYG{o}{/}\PYG{n}{Filespec}
            \PYG{o}{/}\PYG{n}{F} \PYG{p}{(}\PYG{n}{myfile}\PYG{o}{.}\PYG{n}{txt}\PYG{p}{)}
            \PYG{o}{/}\PYG{n}{EF} \PYG{o}{\PYGZlt{}\PYGZlt{}} \PYG{o}{/}\PYG{n}{F} \PYG{p}{\PYGZob{}}\PYG{n}{fstream}\PYG{p}{\PYGZcb{}} \PYG{o}{\PYGZgt{}\PYGZgt{}}
            \PYG{o}{\PYGZgt{}\PYGZgt{}}
        \PYG{o}{/}\PYG{n}{EMBED} \PYG{n}{pdfmark}
    \PYG{p}{[}   \PYG{p}{\PYGZob{}}\PYG{n}{fstream}\PYG{p}{\PYGZcb{}} \PYG{o}{/}\PYG{n}{CLOSE} \PYG{n}{pdfmark}
    \PYG{p}{[} \PYG{o}{/}\PYG{n}{NamespacePop} \PYG{n}{pdfmark}
\end{sphinxVerbatim}


\subsection{Distiller command line options to enable file embedding}
\label{\detokenize{pdfmark_Basic:distiller-command-line-options-to-enable-file-embedding}}
Acrobat Distiller 8.1 and later permit PostScript operators to access only font files, char map files, and files within the installation directory. In contrast, Acrobat Distiller 8.0 and earlier permit unlimited file access. This change was introduced to address security concerns.

To reflect the change in file access behavior, the Distiller command line option (Windows and UNIX ) or user preference (Mac) related to file embedding were also reversed, but in the opposite direction. In Acrobat Distiller 8.1 and later, these Distiller command line options enable unlimited file access, overriding the normal mode of restricting file access. In Acrobat Distiller 8.0 and earlier, these command line options specified limited file access (restricted to fonts, char map files, and files within the installation directory), overriding the normal mode of unrestricted file access.

To use the \sphinxcode{\sphinxupquote{EMBED}} pdfMark directive to embed files other than fonts and char maps, follow these version\sphinxhyphen{}specific guidelines on invoking Acrobat Distiller:

\sphinxstylestrong{Acrobat Distiller 8.1 and later:} Include the Distiller command line option (Windows and UNIX ) or user preference (Mac) that enables unlimited file access. You should be aware that such unlimited access can pose security problems. The following Windows command line invokes Acrobat Distiller with the option that specifies unlimited file access.

\begin{sphinxVerbatim}[commandchars=\\\{\}]
\PYG{n}{acrodist} \PYG{o}{\PYGZhy{}}\PYG{n}{F} \PYG{n}{MyFileContainingPDFMarkEMBED}\PYG{o}{.}\PYG{n}{ps}
\end{sphinxVerbatim}

\sphinxstylestrong{Acrobat Distiller 8.0 and earlier:} Omit the file\sphinxhyphen{}embedding Distiller command line option (Windows and UNIX ) or user preference (Mac) that restricts unlimited file access. The following Windows command line invokes Acrobat Distiller. The omission of the \sphinxhyphen{}F command line option specifies unlimited file access.

\begin{sphinxVerbatim}[commandchars=\\\{\}]
\PYG{n}{acrodist} \PYG{n}{MyFileContainingPDFMarkEMBED}\PYG{o}{.}\PYG{n}{ps}
\end{sphinxVerbatim}

To summarize, in 8.0 and earlier the command line switch “restricts” unlimited file access. In 8.1 and later the command line switch “enables” unlimited file access.

For information on the file\sphinxhyphen{}embedding Distiller command line option (Windows and UNIX ) and user preference (Mac), see ` <../DistillerAPIReference/Distiller\_AutomationIntro.html\#90081>`\_\_.


\section{Graphics encapsulation (BP, EP, SP)}
\label{\detokenize{pdfmark_Basic:graphics-encapsulation-bp-ep-sp}}
Distiller allows a PostScript language program to specify that a given set of graphical operations should be encapsulated and treated as a single object. The pdfmark features \sphinxcode{\sphinxupquote{BP}} (Begin Picture) and \sphinxcode{\sphinxupquote{EP}} (End Picture) enclose a set of graphic operations. The \sphinxcode{\sphinxupquote{SP}} (Show Picture) pdfmark indicates where to insert an object (which may be inserted in more than one place).

The syntax for the graphics encapsulation commands is as follows:

\begin{sphinxVerbatim}[commandchars=\\\{\}]
\PYG{p}{[} \PYG{o}{/}\PYG{n}{\PYGZus{}objdef} \PYG{p}{\PYGZob{}}\PYG{n}{objname}\PYG{p}{\PYGZcb{}} \PYG{o}{/}\PYG{n}{BBox} \PYG{p}{[}\PYG{n}{xll} \PYG{n}{yll} \PYG{n}{xur} \PYG{n}{yur}\PYG{p}{]} \PYG{o}{/}\PYG{n}{BP} \PYG{n}{pdfmark}
   \PYG{o}{.}\PYG{o}{.}\PYG{o}{.} \PYG{n}{page} \PYG{n}{marking} \PYG{n}{instructions} \PYG{o}{.}\PYG{o}{.}\PYG{o}{.}
   \PYG{p}{[} \PYG{o}{/}\PYG{n}{EP} \PYG{n}{pdfmark}
   \PYG{p}{[}   \PYG{p}{\PYGZob{}}\PYG{n}{objname}\PYG{p}{\PYGZcb{}} \PYG{o}{/}\PYG{n}{SP} \PYG{n}{pdfmark}
\end{sphinxVerbatim}

The \sphinxcode{\sphinxupquote{\_objdef}} \{objname\} key–value pair in the \sphinxcode{\sphinxupquote{BP}} pdfmark names the picture objname. Any subsequent pdfmark can refer to this object.

\begin{sphinxadmonition}{note}{Note:}
Graphics names are in the namespace governed by \sphinxcode{\sphinxupquote{NamespacePush}} and \sphinxcode{\sphinxupquote{NamespacePop}} , defined in \sphinxhref{pdfmark\_Syntax.html\#50454537\_19819}{Namespaces}.
\end{sphinxadmonition}

The \sphinxcode{\sphinxupquote{BBox}} key is an array of four numbers {[}xll, yll, xur, yur{]} specifying the lower\sphinxhyphen{}left x, lower\sphinxhyphen{}left y, upper\sphinxhyphen{}right x, and upper\sphinxhyphen{}right y coordinates—in user space—of the rectangle defining the graphic’s bounding box.

When Distiller sees a \sphinxcode{\sphinxupquote{BP pdfmark}} , it forks the distillation from the current context and distills subsequent graphics into a PDF Form object. When it encounters an \sphinxcode{\sphinxupquote{EP}} pdfmark, Distiller finishes the Form object, and distillation continues in the original context. \sphinxcode{\sphinxupquote{BP}} and \sphinxcode{\sphinxupquote{EP}} pdfmark operators can be nested.

The \sphinxcode{\sphinxupquote{SP}} pdfmark tells Distiller to insert a use of a named picture in the current context—in the same manner as if it were a cached PostScript form painted with the \sphinxcode{\sphinxupquote{execform}} PostScript language operator. It includes the picture in the current context (page, form, and so forth) using the current transformation matrix (CTM) to position the graphic.

In addition to using \sphinxcode{\sphinxupquote{SP}} to insert pictures, other pdfmark features that allow specifying named objects can add pictures built using \sphinxcode{\sphinxupquote{BP}} and \sphinxcode{\sphinxupquote{EP}} to a page.

The following examples demonstrate graphic encapsulation.
\begin{enumerate}
\sphinxsetlistlabels{\arabic}{enumi}{enumii}{}{.}%
\item {} 
Creating a picture

\end{enumerate}

This PostScript language sample draws a gray rectangle, then builds a picture enclosed by the \sphinxcode{\sphinxupquote{BP}} and \sphinxcode{\sphinxupquote{EP}} pdfmarks. The picture is simply an X. It shows the picture in three places on the page using the \sphinxcode{\sphinxupquote{SP}} pdfmark, then draws another gray rectangle.

\begin{sphinxVerbatim}[commandchars=\\\{\}]
\PYG{o}{\PYGZpc{}} \PYG{n}{draw} \PYG{n}{a} \PYG{n}{gray} \PYG{n}{rectangle}
\PYG{l+m+mf}{0.5} \PYG{n}{setgray}
\PYG{l+m+mi}{0} \PYG{l+m+mi}{0} \PYG{l+m+mi}{100} \PYG{l+m+mi}{100} \PYG{n}{rectfill}

\PYG{o}{\PYGZpc{}} \PYG{n}{create} \PYG{n}{a} \PYG{n}{picture}
\PYG{p}{[} \PYG{o}{/}\PYG{n}{BBox} \PYG{p}{[}\PYG{l+m+mi}{0} \PYG{l+m+mi}{0} \PYG{l+m+mi}{100} \PYG{l+m+mi}{100}\PYG{p}{]} \PYG{o}{/}\PYG{n}{\PYGZus{}objdef} \PYG{p}{\PYGZob{}}\PYG{n}{MyPicture}\PYG{p}{\PYGZcb{}} \PYG{o}{/}\PYG{n}{BP} \PYG{n}{pdfmark}

\PYG{l+m+mi}{0} \PYG{n}{setgray}
\PYG{l+m+mi}{0} \PYG{l+m+mi}{0} \PYG{n}{moveto} \PYG{l+m+mi}{100} \PYG{l+m+mi}{100} \PYG{n}{lineto} \PYG{n}{stroke}
\PYG{l+m+mi}{100} \PYG{l+m+mi}{0} \PYG{n}{moveto} \PYG{l+m+mi}{0} \PYG{l+m+mi}{100} \PYG{n}{lineto} \PYG{n}{stroke}
\PYG{p}{[} \PYG{o}{/}\PYG{n}{EP} \PYG{n}{pdfmark}

\PYG{o}{\PYGZpc{}} \PYG{n}{make} \PYG{n}{the} \PYG{n}{picture} \PYG{n}{appear} \PYG{n}{on} \PYG{n}{the} \PYG{n}{page}

\PYG{p}{[}   \PYG{p}{\PYGZob{}}\PYG{n}{MyPicture}\PYG{p}{\PYGZcb{}} \PYG{o}{/}\PYG{n}{SP} \PYG{n}{pdfmark}
\PYG{o}{\PYGZpc{}} \PYG{n}{make} \PYG{n}{the} \PYG{n}{picture} \PYG{n}{appear} \PYG{o+ow}{in} \PYG{n}{another} \PYG{n}{place} \PYG{n}{on} \PYG{n}{the} \PYG{n}{page}
\PYG{n}{gsave}
\PYG{l+m+mi}{200} \PYG{l+m+mi}{200} \PYG{n}{translate}
\PYG{p}{[}   \PYG{p}{\PYGZob{}}\PYG{n}{MyPicture}\PYG{p}{\PYGZcb{}} \PYG{o}{/}\PYG{n}{SP} \PYG{n}{pdfmark}
\PYG{n}{grestore}


\PYG{o}{\PYGZpc{}} \PYG{n}{make} \PYG{n}{the} \PYG{n}{picture} \PYG{n}{appear} \PYG{o+ow}{in} \PYG{n}{another} \PYG{n}{place} \PYG{n}{on} \PYG{n}{the} \PYG{n}{page} \PYG{n}{at} \PYG{n}{a} \PYG{n}{different} \PYG{n}{size}

\PYG{n}{gsave}
\PYG{l+m+mi}{100} \PYG{l+m+mi}{400} \PYG{n}{translate}
\PYG{o}{.}\PYG{l+m+mi}{5} \PYG{o}{.}\PYG{l+m+mi}{5} \PYG{n}{scale}
\PYG{p}{[}\PYG{p}{\PYGZob{}}\PYG{n}{MyPicture}\PYG{p}{\PYGZcb{}} \PYG{o}{/}\PYG{n}{SP} \PYG{n}{pdfmark}
\PYG{n}{grestore}


\PYG{o}{\PYGZpc{}} \PYG{n}{draw} \PYG{n}{another} \PYG{n}{gray} \PYG{n}{rectangle}

\PYG{l+m+mf}{0.5} \PYG{n}{setgray}
\PYG{l+m+mi}{512} \PYG{l+m+mi}{692} \PYG{l+m+mi}{100} \PYG{l+m+mi}{100} \PYG{n}{rectfill} \PYG{n}{showpage}
\end{sphinxVerbatim}

The resulting page stream in the PDF file contains the following:

\begin{sphinxVerbatim}[commandchars=\\\{\}]
\PYG{l+m+mf}{0.5} \PYG{n}{g}
   \PYG{l+m+mi}{0} \PYG{l+m+mi}{0} \PYG{l+m+mi}{100} \PYG{l+m+mi}{100} \PYG{n}{re} \PYG{n}{f}
   \PYG{n}{q} \PYG{l+m+mi}{1} \PYG{l+m+mi}{0} \PYG{l+m+mi}{0} \PYG{l+m+mi}{1} \PYG{l+m+mi}{0} \PYG{l+m+mi}{0} \PYG{n}{cm} \PYG{o}{/}\PYG{n}{Fm1} \PYG{n}{Do} \PYG{n}{Q}
   \PYG{n}{q} \PYG{l+m+mi}{1} \PYG{l+m+mi}{0} \PYG{l+m+mi}{0} \PYG{l+m+mi}{1} \PYG{l+m+mi}{200} \PYG{l+m+mi}{200} \PYG{n}{cm} \PYG{o}{/}\PYG{n}{Fm1} \PYG{n}{Do} \PYG{n}{Q}
   \PYG{n}{q} \PYG{l+m+mf}{0.5} \PYG{l+m+mi}{0} \PYG{l+m+mi}{0} \PYG{l+m+mf}{0.5} \PYG{l+m+mi}{100} \PYG{l+m+mi}{400} \PYG{n}{cm} \PYG{o}{/}\PYG{n}{Fm1} \PYG{n}{Do} \PYG{n}{Q}
   \PYG{l+m+mi}{512} \PYG{l+m+mi}{692} \PYG{l+m+mi}{100} \PYG{l+m+mi}{100} \PYG{n}{re} \PYG{n}{f}
\end{sphinxVerbatim}

The graphics between the \sphinxcode{\sphinxupquote{BP}} and the \sphinxcode{\sphinxupquote{EP}} pdfmarks have been saved in a Form object, which has this stream:

\begin{sphinxVerbatim}[commandchars=\\\{\}]
\PYG{l+m+mi}{0} \PYG{n}{g}
   \PYG{l+m+mi}{0} \PYG{l+m+mi}{0} \PYG{n}{m}
   \PYG{l+m+mi}{100} \PYG{l+m+mi}{100} \PYG{n}{l}
   \PYG{l+m+mi}{100} \PYG{l+m+mi}{0} \PYG{n}{m}
   \PYG{l+m+mi}{0} \PYG{l+m+mi}{100} \PYG{n}{l}
   \PYG{n}{S}
\end{sphinxVerbatim}

The resulting page looks like this:

\noindent\sphinxincludegraphics{{square}.png}
\begin{enumerate}
\sphinxsetlistlabels{\arabic}{enumi}{enumii}{}{.}%
\item {} 
Using BP and EP pdfmarks to define button faces for forms

\end{enumerate}

Even if you define the pdfmark operator so that a PostScript interpreter ignores any text between a mark and a pdfmark, any PostScript operators between the \sphinxcode{\sphinxupquote{BP}} and \sphinxcode{\sphinxupquote{EP}} pdfmarks are still processed. To avoid printing anything between the \sphinxcode{\sphinxupquote{BP}} and \sphinxcode{\sphinxupquote{EP}} pdfmarks, use a conditional construct like the one shown in this example.

This code defines common objects that can be used by widgets for forms.

\begin{sphinxVerbatim}[commandchars=\\\{\}]
  \PYGZpc{} AcroForm Begin
  [ /BBox [0 0 100 100] /\PYGZus{}objdef \PYGZob{}Check\PYGZcb{} /BP pdfmark
      \PYGZob{}0 0 1 setrgbcolor /ZapfDingbats 119 selectfont 0 7 moveto (4) show\PYGZcb{}
  ?pdfmark
  [ /EP pdfmark

  [ /BBox [0 0 100 100] /\PYGZus{}objdef \PYGZob{}Cross\PYGZcb{} /BP pdfmark
      \PYGZob{}0 0 1 setrgbcolor /ZapfDingbats 119 selectfont 9.7 7.3 moveto (8) show\PYGZcb{}
  ?pdfmark
  [ /EP pdfmark


  \PYGZpc{} Up/Down button appearances

  [ /BBox [0 0 200 100] /\PYGZus{}objdef \PYGZob{}Up\PYGZcb{} /BP pdfmark
      \PYGZob{}
      0.3 setgray 0 0 200 100 rectfill 1 setgray 2 2 moveto
      2 98 lineto 198 98 lineto 196 96 lineto 4 96 lineto 4 4 lineto fill
      0.34 setgray 198 98 moveto
      198 2 lineto 2 2 lineto 4 4 lineto 196 4 lineto 196 96 lineto fill
      0 setgray 8 22.5 moveto 1 0 0 setrgbcolor /Helvetica 72 selectfont (Up) show
  \PYGZcb{}
if
[ /EP pdfmark

[ /BBox [0 0 200 100] /\PYGZus{}objdef \PYGZob{}Down\PYGZcb{} /BP pdfmark
    \PYGZob{}
      0.7 setgray 0 0 200 100 rectfill 1 setgray 2 2 moveto
      2 98 lineto 198 98 lineto 196 96 lineto 4 96 lineto 4 4 lineto fill
      0.34 setgray 198 98 moveto
      198 2 lineto 2 2 lineto 4 4 lineto 196 4 lineto 196 96 lineto fill
      0 setgray 8 22.5 moveto 0 0 1 setrgbcolor /Helvetica 72 selectfont (Down) show
      \PYGZcb{}
  ?pdfmark
  [ /EP pdfmark

  \PYGZpc{} Submit button appearances

  [ /BBox [0 0 250 100] /\PYGZus{}objdef \PYGZob{}Submit\PYGZcb{} /BP pdfmark
      \PYGZob{}
      0.6 setgray 0 0 250 100 rectfill 1 setgray 2 2 moveto
      2 98 lineto 248 98 lineto 246 96 lineto 4 96 lineto 4 4 lineto fill
      0.34 setgray 248 98 moveto
      248 2 lineto 2 2 lineto 4 4 lineto 246 4 lineto 246 96 lineto fill
      /Helvetica 76 selectfont 0 setgray 8 22.5 moveto (Submit) show
      \PYGZcb{}
  ?pdfmark
  [ /EP pdfmark

  [ /BBox [0 0 250 100] /\PYGZus{}objdef \PYGZob{}SubmitP\PYGZcb{} /BP pdfmark
      \PYGZob{}
      0.6 setgray 0 0 250 100 rectfill 0.34 setgray 2 2 moveto
      2 98 lineto 248 98 lineto 246 96 lineto 4 96 lineto 4 4 lineto fill
      1 setgray 248 98 moveto
      248 2 lineto 2 2 lineto 4 4 lineto 246 4 lineto 246 96 lineto fill
      /Helvetica 76 selectfont 0 setgray 10 20.5 moveto (Submit) show
      \PYGZcb{}
  ?pdfmark
  [ /EP pdfmark
\end{sphinxVerbatim}

For more information on forms, see \sphinxhref{pdfmark\_Examples.html\#50454555\_65255}{Structure examples}. For the definition of \sphinxcode{\sphinxupquote{?pdfmark,}} see \sphinxhref{pdfmark\_Syntax.html\#50454537\_17070}{Usage with standard PostScript interpreters}.


\section{Marked content (MP, DP, BMC, BDC, EMC)}
\label{\detokenize{pdfmark_Basic:marked-content-mp-dp-bmc-bdc-emc}}
PDF 1.2 introduced \sphinxstyleemphasis{marked content operators} , which identify (mark) a portion of a PDF document as elements that can be processed by an application or plug\sphinxhyphen{}in.

Several pdfmark names can be used to specify marked content:
\begin{itemize}
\item {} 
\sphinxcode{\sphinxupquote{MP}} and \sphinxcode{\sphinxupquote{DP}} designate a single marked\sphinxhyphen{}content point in the document’s content stream.

\item {} 
\sphinxcode{\sphinxupquote{BMC}} , \sphinxcode{\sphinxupquote{BDC}} , and \sphinxcode{\sphinxupquote{EMC}} bracket a marked\sphinxhyphen{}content sequence of objects in the content stream. These are complete graphics objects, not just a sequence of bytes.

\end{itemize}

\begin{sphinxadmonition}{note}{Note:}
Marked content can also be used in conjunction with PDF’s logical structure facilities. See \sphinxhref{pdfmark\_Logical.html\#50454553\_46213}{Logical Structure} for information about pdfmark features that implement logical structure.
\end{sphinxadmonition}


\subsection{Marked\sphinxhyphen{}content points}
\label{\detokenize{pdfmark_Basic:marked-content-points}}
\sphinxcode{\sphinxupquote{MP}} creates a marked\sphinxhyphen{}content point in the PDF file. \sphinxcode{\sphinxupquote{DP}} creates a marked\sphinxhyphen{}content point, with an associated property list. Their syntax is as follows:

\begin{sphinxVerbatim}[commandchars=\\\{\}]
 \PYG{p}{[}
\PYG{n}{tag}

        \PYG{o}{/}\PYG{n}{MP} \PYG{n}{pdfmark}
    \PYG{p}{[}
\PYG{n}{tag}
        \PYG{n+nb}{property}\PYG{o}{\PYGZhy{}}\PYG{n+nb}{list}

        \PYG{o}{/}\PYG{n}{DP} \PYG{n}{pdfmark}
\end{sphinxVerbatim}

The tag is an optional name object indicating the role or significance of the point. The property\sphinxhyphen{}list is a dictionary containing key\sphinxhyphen{}value pairs that are meaningful to the program creating the marked content.


\subsection{Marked\sphinxhyphen{}content sequences}
\label{\detokenize{pdfmark_Basic:marked-content-sequences}}
\sphinxcode{\sphinxupquote{BMC}} and \sphinxcode{\sphinxupquote{BDC}} begin a marked\sphinxhyphen{}content sequence, and \sphinxcode{\sphinxupquote{EMC}} ends a sequence. Their syntax is as follows:

\begin{sphinxVerbatim}[commandchars=\\\{\}]
 \PYG{p}{[}
\PYG{n}{tag}
        \PYG{o}{/}\PYG{n}{BMC} \PYG{n}{pdfmark}
    \PYG{p}{[}
\PYG{n}{tag}
        \PYG{n+nb}{property}\PYG{o}{\PYGZhy{}}\PYG{n+nb}{list}

        \PYG{o}{/}\PYG{n}{BDC} \PYG{n}{pdfmark}
    \PYG{p}{[} \PYG{o}{/}\PYG{n}{EMC} \PYG{n}{pdfmark}
\end{sphinxVerbatim}

The tag is an optional name for the sequence. The property\sphinxhyphen{}list is a dictionary containing key\sphinxhyphen{}value pairs that are meaningful to the program creating the marked content.


\section{Metadata (Metadata)}
\label{\detokenize{pdfmark_Basic:metadata-metadata}}
The ability to add metadata to the \sphinxcode{\sphinxupquote{Catalog}} was added in Distiller 6.0. The syntax for the \sphinxcode{\sphinxupquote{Metadata}} feature is as follows:

\begin{sphinxVerbatim}[commandchars=\\\{\}]
 \PYG{p}{[}   \PYG{p}{\PYGZob{}}\PYG{n}{Catalog}\PYG{p}{\PYGZcb{}} \PYG{p}{\PYGZob{}}
\PYG{n}{XMPStreamName}
\PYG{p}{\PYGZcb{}} \PYG{o}{/}\PYG{n}{Metadata} \PYG{n}{pdfmark}
\end{sphinxVerbatim}

In future releases of Distiller, metadata may be attached to objects other than the \sphinxcode{\sphinxupquote{Catalog}} object.

If the target is not the \sphinxcode{\sphinxupquote{Catalog}} object or if DSC processing is enabled and this feature is located within Encapsulated PostScript (EPS), then this feature is ignored. Otherwise, the metadata associated with the stream \sphinxcode{\sphinxupquote{XMPStreamName}} is added to the \sphinxcode{\sphinxupquote{Catalog}} object with the key \sphinxcode{\sphinxupquote{Metadata}} . See the \sphinxstyleemphasis{PDF Reference} for more information.
\begin{enumerate}
\sphinxsetlistlabels{\arabic}{enumi}{enumii}{}{.}%
\item {} 
Metadata example

\end{enumerate}

\begin{sphinxVerbatim}[commandchars=\\\{\}]
[ /\PYGZus{}objdef \PYGZob{}myMetadata\PYGZcb{} /type stream /OBJ pdfmark
   [   \PYGZob{}myMetadata\PYGZcb{} currentfile 0 (\PYGZpc{} \PYGZhy{}\PYGZhy{} end \PYGZhy{}\PYGZhy{}) /SubFileDecode filter /PUT pdfmark
   \PYGZlt{}?xpacket begin=\PYGZsq{}\PYGZsq{} id=\PYGZsq{}W5M0MpCehiHzreSzNTczkc9d\PYGZsq{}?\PYGZgt{}
   \PYGZlt{}rdf:RDF xmlns:rdf=\PYGZsq{}http://www.w3.org/1999/02/22\PYGZhy{}rdf\PYGZhy{}syntax\PYGZhy{}ns\PYGZsh{}\PYGZsq{}
   ...
   \PYGZpc{} \PYGZhy{}\PYGZhy{} end \PYGZhy{}\PYGZhy{}
   [   \PYGZob{}myMetadata\PYGZcb{} \PYGZlt{}\PYGZlt{} /Type /Metadata /Subtype /XML\PYGZgt{}\PYGZgt{} /PUT pdfmark
   [   \PYGZob{}Catalog\PYGZcb{} \PYGZob{}myMetadata\PYGZcb{} /Metadata pdfmark
\end{sphinxVerbatim}


\section{Named images (NI)}
\label{\detokenize{pdfmark_Basic:named-images-ni}}
The \sphinxcode{\sphinxupquote{NI}} pdfmark gives a name to a PostScript image. Subsequently, the name can be used to refer to the image in the same way that a named object is referenced. For example, an image can be included in PDF logical structure with \sphinxcode{\sphinxupquote{StOBJ}} (see \sphinxhref{pdfmark\_Logical.html\#50454553\_43183}{StOBJ}) so that it can be included later in element content. The example in \sphinxhref{pdfmark\_Examples.html\#50454555\_59092}{Using OBJ and PUT pdfmarks to create an alternate image} shows using \sphinxcode{\sphinxupquote{NI}} with an alternate image.

The syntax for defining an image name is as follows:

\begin{sphinxVerbatim}[commandchars=\\\{\}]
 \PYG{p}{[} \PYG{o}{/}\PYG{n}{\PYGZus{}objdef} \PYG{p}{\PYGZob{}}
\PYG{n}{objname}
\PYG{p}{\PYGZcb{}}
 \PYG{o}{/}\PYG{n}{NI} \PYG{n}{pdfmark}
\end{sphinxVerbatim}

\sphinxcode{\sphinxupquote{NI}} takes the standard \sphinxcode{\sphinxupquote{\_objdef}} key to name the image within Distiller. Image names are in the namespace governed by \sphinxcode{\sphinxupquote{NamespacePush}} and \sphinxcode{\sphinxupquote{NamespacePop}} , defined in \sphinxhref{pdfmark\_Syntax.html\#50454537\_19819}{Namespaces}.

The image named by an \sphinxcode{\sphinxupquote{NI}} command is to be found subsequently in the PostScript source file, but it does not need to immediately follow the \sphinxcode{\sphinxupquote{NI}} . An image is assigned the name given by the most recent \sphinxcode{\sphinxupquote{NI}} not yet paired with an image.

In other words, Distiller maintains a stack of names pushed by \sphinxcode{\sphinxupquote{NI}} and popped by the occurrence of an image. If an image is encountered when this stack is empty, it is not an error: the image simply does not receive a name.


\section{Page crops (PAGE, PAGES)}
\label{\detokenize{pdfmark_Basic:page-crops-page-pages}}
Page cropping is used to specify the dimensions of a page or pages in a PDF file that will be displayed or printed, without altering the actual data in the file. Cropping is specified by using the pdfmark operator with the names \sphinxcode{\sphinxupquote{PAGE}} (for an individual page) or \sphinxcode{\sphinxupquote{PAGES}} (for the entire document).

The syntax for specifying a non\sphinxhyphen{}default page cropping for a particular page in a document is as follows:

\begin{sphinxVerbatim}[commandchars=\\\{\}]
 \PYG{p}{[} \PYG{o}{/}\PYG{n}{CropBox} \PYG{p}{[}
\PYG{n}{xll} \PYG{n}{yll} \PYG{n}{xur} \PYG{n}{yur}
\PYG{p}{]}

 \PYG{o}{/}\PYG{n}{PAGE} \PYG{n}{pdfmark}
\end{sphinxVerbatim}

The syntax for specifying the default page cropping for a document is as follows:

\begin{sphinxVerbatim}[commandchars=\\\{\}]
 \PYG{p}{[} \PYG{o}{/}\PYG{n}{CropBox} \PYG{p}{[}
\PYG{n}{xll} \PYG{n}{yll} \PYG{n}{xur} \PYG{n}{yur}
\PYG{p}{]}

 \PYG{o}{/}\PYG{n}{PAGES} \PYG{n}{pdfmark}
\end{sphinxVerbatim}

The \sphinxcode{\sphinxupquote{CropBox}} key is an array representing the location and size of the viewable area of the page. \sphinxcode{\sphinxupquote{CropBox}} is an array of four numbers {[}xll, yll, xur, yur{]} specifying the lower\sphinxhyphen{}left x, lower\sphinxhyphen{}left y, upper\sphinxhyphen{}right x, and upper\sphinxhyphen{}right y coordinates—measured in default user space—of the rectangle defining the cropped page. The minimum allowed page size is .04 x .04 inch (3 x 3 units) and the maximum allowed page size is 200 x 200 inches (14,400 x 14,400 units) in the default user space coordinate system.

The \sphinxcode{\sphinxupquote{PAGE}} pdfmark must be placed before the \sphinxcode{\sphinxupquote{showpage}} operator for the page it is to affect. It is recommended that it be placed before any marks are made on the page. For example, it affects only the first page of a document if it is placed before any marks are made on the first page.

The \sphinxcode{\sphinxupquote{PAGES}} pdfmark can be placed anywhere in the PostScript language program, but it is recommended that it be placed at the beginning of the file, in the Document Setup section between the document structuring comments \sphinxcode{\sphinxupquote{\%\%BeginSetup}} and \sphinxcode{\sphinxupquote{\%\%EndSetup}} , before any marks are placed on the first page.
\begin{enumerate}
\sphinxsetlistlabels{\arabic}{enumi}{enumii}{}{.}%
\item {} 
Crop this page

\end{enumerate}

\begin{sphinxVerbatim}[commandchars=\\\{\}]
\PYG{o}{\PYGZpc{}} \PYG{o}{.}\PYG{o}{.}\PYG{o}{.}

\PYG{p}{[} \PYG{o}{/}\PYG{n}{CropBox} \PYG{p}{[}\PYG{l+m+mi}{0} \PYG{l+m+mi}{0} \PYG{l+m+mi}{288} \PYG{l+m+mi}{288}\PYG{p}{]} \PYG{o}{/}\PYG{n}{PAGE} \PYG{n}{pdfmark}
\PYG{o}{/}\PYG{n}{Helvetica} \PYG{n}{findfont} \PYG{l+m+mi}{12} \PYG{n}{scalefont} \PYG{n}{setfont}
\PYG{o}{/}\PYG{n}{DrawBorder}
    \PYG{p}{\PYGZob{}}
    \PYG{l+m+mi}{10} \PYG{l+m+mi}{278} \PYG{n}{moveto} \PYG{l+m+mi}{278} \PYG{l+m+mi}{278} \PYG{n}{lineto} \PYG{l+m+mi}{278} \PYG{l+m+mi}{10} \PYG{n}{lineto}
    \PYG{l+m+mi}{10} \PYG{l+m+mi}{10} \PYG{n}{lineto} \PYG{n}{closepath} \PYG{n}{stroke}
    \PYG{p}{\PYGZcb{}} \PYG{n}{bind} \PYG{k}{def}
\PYG{o}{\PYGZpc{}}\PYG{o}{\PYGZpc{}}\PYG{n}{EndSetup}
\PYG{o}{\PYGZpc{}}\PYG{o}{\PYGZpc{}}\PYG{n}{Page}\PYG{p}{:} \PYG{l+m+mi}{1} \PYG{l+m+mi}{1}
\PYG{n}{DrawBorder}
\PYG{l+m+mi}{75} \PYG{l+m+mi}{250} \PYG{n}{moveto} \PYG{p}{(}\PYG{n}{This} \PYG{o+ow}{is} \PYG{n}{Page} \PYG{l+m+mi}{3}\PYG{p}{)} \PYG{n}{show}
\PYG{l+m+mi}{75} \PYG{l+m+mi}{230} \PYG{n}{moveto} \PYG{p}{(}\PYG{n}{Click} \PYG{n}{here} \PYG{n}{to} \PYG{n}{go} \PYG{n}{to} \PYG{n}{page} \PYG{l+m+mf}{1.}\PYG{p}{)} \PYG{n}{show}
\PYG{l+m+mi}{75} \PYG{l+m+mi}{200} \PYG{n}{moveto} \PYG{p}{(}\PYG{n}{Click} \PYG{n}{here} \PYG{n}{to} \PYG{n+nb}{open} \PYG{n}{test}\PYG{o}{.}\PYG{n}{doc}\PYG{o}{.}\PYG{p}{)} \PYG{n}{show}
\end{sphinxVerbatim}
\begin{enumerate}
\sphinxsetlistlabels{\arabic}{enumi}{enumii}{}{.}%
\item {} 
Crop all pages

\end{enumerate}

\begin{sphinxVerbatim}[commandchars=\\\{\}]
\PYG{o}{\PYGZpc{}} \PYG{o}{.}\PYG{o}{.}\PYG{o}{.}

\PYG{p}{[} \PYG{o}{/}\PYG{n}{CropBox} \PYG{p}{[}\PYG{l+m+mi}{54} \PYG{l+m+mi}{403} \PYG{l+m+mi}{558} \PYG{l+m+mi}{720}\PYG{p}{]} \PYG{o}{/}\PYG{n}{PAGES} \PYG{n}{pdfmark}
\PYG{o}{/}\PYG{n}{DrawBorder}
    \PYG{p}{\PYGZob{}}
    \PYG{l+m+mi}{58} \PYG{l+m+mi}{407} \PYG{n}{moveto} \PYG{l+m+mi}{554} \PYG{l+m+mi}{407} \PYG{n}{lineto} \PYG{l+m+mi}{554} \PYG{l+m+mi}{716} \PYG{n}{lineto}
    \PYG{l+m+mi}{58} \PYG{l+m+mi}{716} \PYG{n}{lineto} \PYG{n}{closepath} \PYG{n}{stroke}
    \PYG{p}{\PYGZcb{}} \PYG{n}{bind} \PYG{k}{def}
\PYG{o}{/}\PYG{n}{Helvetica} \PYG{n}{findfont} \PYG{l+m+mi}{10} \PYG{n}{scalefont} \PYG{n}{setfont}
\PYG{o}{\PYGZpc{}}\PYG{o}{\PYGZpc{}}\PYG{n}{EndSetup}
\PYG{o}{\PYGZpc{}}\PYG{o}{\PYGZpc{}}\PYG{n}{Page}\PYG{p}{:} \PYG{l+m+mi}{1} \PYG{l+m+mi}{1}
\PYG{n}{DrawBorder}
\PYG{l+m+mi}{75} \PYG{l+m+mi}{690} \PYG{n}{moveto} \PYG{p}{(}\PYG{n}{This} \PYG{o+ow}{is} \PYG{n}{Page} \PYG{l+m+mi}{1}\PYG{p}{)} \PYG{n}{show}
\PYG{l+m+mi}{75} \PYG{l+m+mi}{670} \PYG{n}{moveto} \PYG{p}{(}\PYG{n}{Below} \PYG{o+ow}{is} \PYG{n}{a} \PYG{n}{closed}\PYG{p}{,} \PYG{n}{default} \PYG{n}{note} \PYG{n}{created} \PYG{n}{using} \PYG{n}{pdfmark}\PYG{p}{:}\PYG{p}{)} \PYG{n}{show}
\PYG{l+m+mi}{75} \PYG{l+m+mi}{570} \PYG{n}{moveto} \PYG{p}{(}\PYG{n}{Below} \PYG{o+ow}{is} \PYG{n}{an} \PYG{n+nb}{open} \PYG{n}{note} \PYG{k}{with} \PYG{n}{a} \PYG{n}{custom} \PYG{n}{color} \PYG{o+ow}{and} \PYG{n}{label}\PYG{p}{:}\PYG{p}{)} \PYG{n}{show}
\PYG{l+m+mi}{400} \PYG{l+m+mi}{670} \PYG{n}{moveto} \PYG{p}{(}\PYG{n}{Below} \PYG{o+ow}{is} \PYG{n}{a} \PYG{n}{closed} \PYG{n}{note}\PYG{p}{)} \PYG{n}{show}
\PYG{l+m+mi}{400} \PYG{l+m+mi}{655} \PYG{n}{moveto} \PYG{p}{(}\PYG{n}{containing} \PYG{n}{private} \PYG{n}{data}\PYG{p}{:}\PYG{p}{)} \PYG{n}{show}
\PYG{l+m+mi}{400} \PYG{l+m+mi}{570} \PYG{n}{moveto} \PYG{p}{(}\PYG{n}{Below} \PYG{o+ow}{is} \PYG{n}{a} \PYG{n}{custom} \PYG{n}{annotation}\PYG{o}{.}\PYG{p}{)} \PYG{n}{show}
\PYG{l+m+mi}{400} \PYG{l+m+mi}{555} \PYG{n}{moveto} \PYG{p}{(}\PYG{n}{It} \PYG{n}{should} \PYG{n}{appear} \PYG{k}{as} \PYG{n}{an} \PYG{n}{unknown}\PYG{p}{)} \PYG{n}{show}
\PYG{l+m+mi}{400} \PYG{l+m+mi}{540} \PYG{n}{moveto} \PYG{p}{(}\PYG{n}{annotation} \PYG{n}{icon}\PYG{p}{:}\PYG{p}{)} \PYG{n}{show}
\end{sphinxVerbatim}


\section{Page label and plate color (PAGELABEL)}
\label{\detokenize{pdfmark_Basic:page-label-and-plate-color-pagelabel}}
The \sphinxcode{\sphinxupquote{PAGELABEL}} pdfmark allows specification of the \sphinxstyleemphasis{page label} for a given page. Page labels can be strings like “iv” or “3\sphinxhyphen{}24”, and do not necessarily correspond to the actual page numbers, which run consecutively. See the \sphinxstyleemphasis{PDF Reference} for details.

Its syntax is as follows:

\begin{sphinxVerbatim}[commandchars=\\\{\}]
 \PYG{p}{[} \PYG{o}{/}\PYG{n}{Label}
\PYG{n}{string}

        \PYG{o}{/}\PYG{n}{PlateColor}
\PYG{n}{string}

        \PYG{o}{/}\PYG{n}{PAGELABEL} \PYG{n}{pdfmark}
\end{sphinxVerbatim}

Both the \sphinxcode{\sphinxupquote{Label}} and \sphinxcode{\sphinxupquote{PlateColor}} keys are optional. \sphinxcode{\sphinxupquote{Label}} takes a string representing the page label for the page on which the \sphinxcode{\sphinxupquote{pdfmark}} appears.

\sphinxcode{\sphinxupquote{PlateColor}} takes an optional string representing a device colorant. It is used in high\sphinxhyphen{}end printing situations where the pages are pre\sphinxhyphen{}separated prior to generating PDF. This means that there are multiple page objects in the PDF file (each representing a different colorant) corresponding to a single physical page.The color for each separation must be specified in a \sphinxstyleemphasis{separation dictionary} ; see the \sphinxstyleemphasis{PDF Reference} for details.

Consecutive pages that specify \sphinxcode{\sphinxupquote{PlateColor}} , with the same value for \sphinxcode{\sphinxupquote{Label}} , are placed in the same separation group. The last instance of a \sphinxcode{\sphinxupquote{Label}} or \sphinxcode{\sphinxupquote{PlateColor}} on a page overrides any earlier settings of the same key on the same page.
\begin{enumerate}
\sphinxsetlistlabels{\arabic}{enumi}{enumii}{}{.}%
\item {} 
Page Label

\end{enumerate}

\begin{sphinxVerbatim}[commandchars=\\\{\}]
\PYG{o}{\PYGZpc{}}\PYG{o}{\PYGZpc{}}\PYG{n}{Page}\PYG{p}{:} \PYG{n}{Sec1}\PYG{p}{:}\PYG{l+m+mi}{2} \PYG{l+m+mi}{1}
\PYG{o}{\PYGZpc{}}\PYG{o}{\PYGZpc{}}\PYG{n}{PlateColor}\PYG{p}{:} \PYG{n}{Cyan}
\PYG{p}{[} \PYG{o}{/}\PYG{n}{Label} \PYG{p}{(}\PYG{n}{Sec1}\PYG{p}{:}\PYG{l+m+mi}{1}\PYG{p}{)} \PYG{o}{/}\PYG{n}{PlateColor} \PYG{p}{(}\PYG{n}{Cyan}\PYG{p}{)} \PYG{o}{/}\PYG{n}{PAGELABEL} \PYG{n}{pdfmark}

\PYG{o}{\PYGZpc{}}\PYG{o}{\PYGZpc{}}\PYG{n}{Page}\PYG{p}{:} \PYG{n}{iii} \PYG{l+m+mi}{3}
\PYG{p}{[} \PYG{o}{/}\PYG{n}{Label} \PYG{p}{(}\PYG{n}{iii}\PYG{p}{)} \PYG{o}{/}\PYG{n}{PAGELABEL} \PYG{n}{pdfmark}
\end{sphinxVerbatim}


\section{Transparency (SetTransparency)}
\label{\detokenize{pdfmark_Basic:transparency-settransparency}}
PDF 1.4 extended the Adobe imaging model to include the notion of transparency. See the \sphinxstyleemphasis{PDF Reference} for complete information on transparency. To produce PDF files with transparency from PostScript files, use the \sphinxcode{\sphinxupquote{SetTransparency}} pdfmark feature. This feature provides a mechanism for specifying the following graphics state parameters:

Graphics state parameters for transparency


\begin{savenotes}\sphinxattablestart
\centering
\phantomsection\label{\detokenize{pdfmark_Basic:section-7}}\nobreak
\begin{tabulary}{\linewidth}[t]{|T|T|T|}
\hline
\sphinxstyletheadfamily 
Key
&\sphinxstyletheadfamily 
Value
&\sphinxstyletheadfamily 
Meaning
\\
\hline
AIS
&
Boolean
&
The alpha source flag (“alpha is shape”), specifying whether the current soft mask and alpha constant are to be interpreted as shape values (\sphinxcode{\sphinxupquote{true}} ) or opacity values (\sphinxcode{\sphinxupquote{false}} ). Default is \sphinxcode{\sphinxupquote{false}} .
\\
\hline
BM
&
name or array of names
&
Current blend mode. Default is \sphinxcode{\sphinxupquote{Normal}} .
\\
\hline
CA
&
number
&
Current stroking alpha constant, specifying the constant shape or constant opacity value to be used for stroking operations. Default is \sphinxcode{\sphinxupquote{1.0}} .
\\
\hline
ca
&
number
&
Same as \sphinxcode{\sphinxupquote{CA}} , but for nonstroking operation. Default is \sphinxcode{\sphinxupquote{1.0}} .
\\
\hline
SMask
&
dictionary or \sphinxcode{\sphinxupquote{None}}
&
Current soft mask, specifying the mask shape or mask opacity values. Default is \sphinxcode{\sphinxupquote{None}} .
\\
\hline
TK
&
Boolean
&
The text knockout flag, which determines the behavior of overlapping glyphs within a text object. Default is \sphinxcode{\sphinxupquote{true}} .
\\
\hline
\end{tabulary}
\par
\sphinxattableend\end{savenotes}

The syntax of the \sphinxcode{\sphinxupquote{SetTransparency}} feature is as follows:

\begin{sphinxVerbatim}[commandchars=\\\{\}]
 [
key–value pairs
    /SetTransparency pdfmark
\end{sphinxVerbatim}

where recognized key\sphinxhyphen{}value pairs are found in the table \sphinxhref{pdfmark\_Basic.html\#50454556\_14514}{Graphics state parameters for transparency}.

\begin{sphinxadmonition}{note}{Note:}
The keys used by this pdfmark feature are the same as are found in PDF documents.
\end{sphinxadmonition}

The arguments to the SetTransparency feature are checked for correct types and values. Unrecognized keys are ignored and their values are neither checked nor written to the PDF document. If no recognized key\sphinxhyphen{}value pairs are presented, then this feature adds no transparency information to the PDF document.

The values set by this feature are subject to \sphinxcode{\sphinxupquote{gsave/grestore}} . For example:

\begin{sphinxVerbatim}[commandchars=\\\{\}]
\PYG{p}{[} \PYG{o}{/}\PYG{n}{ca} \PYG{o}{.}\PYG{l+m+mi}{8} \PYG{o}{/}\PYG{n}{SetTransparency} \PYG{n}{pdfmark}      \PYG{o}{\PYGZpc{}} \PYG{n}{Nonstroking} \PYG{n}{alpha} \PYG{o+ow}{is} \PYG{n}{now} \PYG{o}{.}\PYG{l+m+mi}{8}
       \PYG{n}{gsave}
       \PYG{p}{[} \PYG{o}{/}\PYG{n}{ca} \PYG{o}{.}\PYG{l+m+mi}{7} \PYG{o}{/}\PYG{n}{SetTransparency} \PYG{n}{pdfmark}    \PYG{o}{\PYGZpc{}} \PYG{n}{Nonstroking} \PYG{n}{alpha} \PYG{o+ow}{is} \PYG{n}{now} \PYG{o}{.}\PYG{l+m+mi}{7}
       \PYG{n}{grestore}
                                            \PYG{o}{\PYGZpc{}} \PYG{n}{Nonstroking} \PYG{n}{alpha} \PYG{o+ow}{is} \PYG{n}{now} \PYG{o}{.}\PYG{l+m+mi}{8}
\end{sphinxVerbatim}

The \sphinxcode{\sphinxupquote{initgraphics}} operator resets all of the graphics state parameters for transparency to the defaults as shown in the table \sphinxhref{pdfmark\_Basic.html\#50454556\_14514}{Graphics state parameters for transparency}.

The following PostScript code demonstrates a use of the \sphinxcode{\sphinxupquote{SetTransparency}} feature using Normal blend mode with differing opacities.
\begin{enumerate}
\sphinxsetlistlabels{\arabic}{enumi}{enumii}{}{.}%
\item {} 
Transparencies

\end{enumerate}

\begin{sphinxVerbatim}[commandchars=\\\{\}]
\PYG{o}{/}\PYG{n}{DeviceCMYK} \PYG{n}{setcolorspace} \PYG{l+m+mi}{15} \PYG{n}{setlinewidth}
   \PYG{p}{[} \PYG{o}{/}\PYG{n}{ca} \PYG{o}{.}\PYG{l+m+mi}{6} \PYG{o}{/}\PYG{n}{CA} \PYG{o}{.}\PYG{l+m+mi}{3} \PYG{o}{/}\PYG{n}{BM} \PYG{o}{/}\PYG{n}{Normal} \PYG{o}{/}\PYG{n}{SetTransparency} \PYG{n}{pdfmark}
   \PYG{l+m+mi}{0} \PYG{l+m+mi}{1} \PYG{l+m+mi}{1} \PYG{l+m+mi}{0} \PYG{n}{setcolor} \PYG{l+m+mi}{220} \PYG{l+m+mi}{330} \PYG{l+m+mi}{150} \PYG{l+m+mi}{0} \PYG{l+m+mi}{360} \PYG{n}{arc} \PYG{n}{fill}    \PYG{o}{\PYGZpc{}} \PYG{n}{red}
   \PYG{l+m+mi}{0} \PYG{l+m+mi}{0} \PYG{l+m+mi}{1} \PYG{l+m+mi}{0} \PYG{n}{setcolor} \PYG{l+m+mi}{320} \PYG{l+m+mi}{503} \PYG{l+m+mi}{150} \PYG{l+m+mi}{0} \PYG{l+m+mi}{360} \PYG{n}{arc} \PYG{n}{fill}    \PYG{o}{\PYGZpc{}} \PYG{n}{yellow}
   \PYG{l+m+mi}{1} \PYG{l+m+mi}{1} \PYG{l+m+mi}{0} \PYG{l+m+mi}{0} \PYG{n}{setcolor} \PYG{l+m+mi}{420} \PYG{l+m+mi}{330} \PYG{l+m+mi}{150} \PYG{l+m+mi}{0} \PYG{l+m+mi}{360} \PYG{n}{arc} \PYG{n}{fill}    \PYG{o}{\PYGZpc{}} \PYG{n}{blue}
   \PYG{l+m+mi}{1} \PYG{l+m+mi}{0} \PYG{l+m+mi}{0} \PYG{l+m+mi}{0} \PYG{n}{setcolor} \PYG{l+m+mi}{230} \PYG{l+m+mi}{440} \PYG{l+m+mi}{104} \PYG{l+m+mi}{0} \PYG{l+m+mi}{360} \PYG{n}{arc} \PYG{n}{stroke}    \PYG{o}{\PYGZpc{}} \PYG{n}{cyan}
   \PYG{l+m+mi}{0} \PYG{l+m+mi}{1} \PYG{l+m+mi}{0} \PYG{l+m+mi}{0} \PYG{n}{setcolor} \PYG{l+m+mi}{410} \PYG{l+m+mi}{440} \PYG{l+m+mi}{104} \PYG{l+m+mi}{0} \PYG{l+m+mi}{360} \PYG{n}{arc} \PYG{n}{stroke}    \PYG{o}{\PYGZpc{}} \PYG{n}{magenta}
   \PYG{l+m+mi}{0} \PYG{l+m+mi}{0} \PYG{l+m+mi}{1} \PYG{l+m+mi}{0} \PYG{n}{setcolor} \PYG{l+m+mi}{320} \PYG{l+m+mi}{284} \PYG{l+m+mi}{104} \PYG{l+m+mi}{0} \PYG{l+m+mi}{360} \PYG{n}{arc} \PYG{n}{stroke}    \PYG{o}{\PYGZpc{}} \PYG{n}{yellow}
\end{sphinxVerbatim}

\noindent\sphinxincludegraphics{{circle1}.png}

Compare this to the following in which the blend mode has been changed:

\begin{sphinxVerbatim}[commandchars=\\\{\}]
\PYG{o}{/}\PYG{n}{DeviceCMYK} \PYG{n}{setcolorspace} \PYG{l+m+mi}{15} \PYG{n}{setlinewidth}
   \PYG{p}{[} \PYG{o}{/}\PYG{n}{ca} \PYG{o}{.}\PYG{l+m+mi}{6} \PYG{o}{/}\PYG{n}{CA} \PYG{o}{.}\PYG{l+m+mi}{3} \PYG{o}{/}\PYG{n}{BM} \PYG{o}{/}\PYG{n}{Difference} \PYG{o}{/}\PYG{n}{SetTransparency} \PYG{n}{pdfmark}

   \PYG{l+m+mi}{0} \PYG{l+m+mi}{1} \PYG{l+m+mi}{1} \PYG{l+m+mi}{0} \PYG{n}{setcolor} \PYG{l+m+mi}{220} \PYG{l+m+mi}{330} \PYG{l+m+mi}{150} \PYG{l+m+mi}{0} \PYG{l+m+mi}{360} \PYG{n}{arc} \PYG{n}{fill}    \PYG{o}{\PYGZpc{}} \PYG{n}{red}
   \PYG{l+m+mi}{0} \PYG{l+m+mi}{0} \PYG{l+m+mi}{1} \PYG{l+m+mi}{0} \PYG{n}{setcolor} \PYG{l+m+mi}{320} \PYG{l+m+mi}{503} \PYG{l+m+mi}{150} \PYG{l+m+mi}{0} \PYG{l+m+mi}{360} \PYG{n}{arc} \PYG{n}{fill}    \PYG{o}{\PYGZpc{}} \PYG{n}{yellow}
   \PYG{l+m+mi}{1} \PYG{l+m+mi}{1} \PYG{l+m+mi}{0} \PYG{l+m+mi}{0} \PYG{n}{setcolor} \PYG{l+m+mi}{420} \PYG{l+m+mi}{330} \PYG{l+m+mi}{150} \PYG{l+m+mi}{0} \PYG{l+m+mi}{360} \PYG{n}{arc} \PYG{n}{fill}    \PYG{o}{\PYGZpc{}} \PYG{n}{blue}
   \PYG{l+m+mi}{1} \PYG{l+m+mi}{0} \PYG{l+m+mi}{0} \PYG{l+m+mi}{0} \PYG{n}{setcolor} \PYG{l+m+mi}{230} \PYG{l+m+mi}{440} \PYG{l+m+mi}{104} \PYG{l+m+mi}{0} \PYG{l+m+mi}{360} \PYG{n}{arc} \PYG{n}{stroke}    \PYG{o}{\PYGZpc{}} \PYG{n}{cyan}
   \PYG{l+m+mi}{0} \PYG{l+m+mi}{1} \PYG{l+m+mi}{0} \PYG{l+m+mi}{0} \PYG{n}{setcolor} \PYG{l+m+mi}{410} \PYG{l+m+mi}{440} \PYG{l+m+mi}{104} \PYG{l+m+mi}{0} \PYG{l+m+mi}{360} \PYG{n}{arc} \PYG{n}{stroke}    \PYG{o}{\PYGZpc{}} \PYG{n}{magenta}
   \PYG{l+m+mi}{0} \PYG{l+m+mi}{0} \PYG{l+m+mi}{1} \PYG{l+m+mi}{0} \PYG{n}{setcolor} \PYG{l+m+mi}{320} \PYG{l+m+mi}{284} \PYG{l+m+mi}{104} \PYG{l+m+mi}{0} \PYG{l+m+mi}{360} \PYG{n}{arc} \PYG{n}{stroke}    \PYG{o}{\PYGZpc{}} \PYG{n}{yellow}
\end{sphinxVerbatim}

\noindent\sphinxincludegraphics{{circle2}.png}

Note that filling and stroking the \sphinxstyleemphasis{same} path results in the use of the PDF \sphinxcode{\sphinxupquote{f}} and \sphinxcode{\sphinxupquote{S}} operators and not the \sphinxcode{\sphinxupquote{B}} operator. This produces a “double border” effect and is not usually desirable.

\begin{sphinxVerbatim}[commandchars=\\\{\}]
\PYG{o}{/}\PYG{n}{DeviceCMYK} \PYG{n}{setcolorspace} \PYG{l+m+mi}{15} \PYG{n}{setlinewidth}
   \PYG{p}{[} \PYG{o}{/}\PYG{n}{ca} \PYG{o}{.}\PYG{l+m+mi}{6} \PYG{o}{/}\PYG{n}{CA} \PYG{o}{.}\PYG{l+m+mi}{3} \PYG{o}{/}\PYG{n}{BM} \PYG{o}{/}\PYG{n}{Normal} \PYG{o}{/}\PYG{n}{SetTransparency} \PYG{n}{pdfmark}

   \PYG{l+m+mi}{0} \PYG{l+m+mi}{1} \PYG{l+m+mi}{1} \PYG{l+m+mi}{0} \PYG{n}{setcolor} \PYG{l+m+mi}{220} \PYG{l+m+mi}{330} \PYG{l+m+mi}{150} \PYG{l+m+mi}{0} \PYG{l+m+mi}{360} \PYG{n}{arc}    \PYG{o}{\PYGZpc{}} \PYG{n}{red} \PYG{n}{path}
                   \PYG{n}{gsave} \PYG{n}{fill} \PYG{n}{grestore} \PYG{n}{stroke}    \PYG{o}{\PYGZpc{}} \PYG{n}{fill}\PYG{p}{,} \PYG{n}{then} \PYG{n}{stroke}
   \PYG{l+m+mi}{0} \PYG{l+m+mi}{0} \PYG{l+m+mi}{1} \PYG{l+m+mi}{0} \PYG{n}{setcolor} \PYG{l+m+mi}{320} \PYG{l+m+mi}{503} \PYG{l+m+mi}{150} \PYG{l+m+mi}{0} \PYG{l+m+mi}{360} \PYG{n}{arc}    \PYG{o}{\PYGZpc{}} \PYG{n}{yellow} \PYG{n}{path}
                   \PYG{n}{gsave} \PYG{n}{fill} \PYG{n}{grestore} \PYG{n}{stroke}    \PYG{o}{\PYGZpc{}} \PYG{n}{fill}\PYG{p}{,} \PYG{n}{then} \PYG{n}{stroke}
   \PYG{l+m+mi}{1} \PYG{l+m+mi}{1} \PYG{l+m+mi}{0} \PYG{l+m+mi}{0} \PYG{n}{setcolor} \PYG{l+m+mi}{420} \PYG{l+m+mi}{330} \PYG{l+m+mi}{150} \PYG{l+m+mi}{0} \PYG{l+m+mi}{360} \PYG{n}{arc}    \PYG{o}{\PYGZpc{}} \PYG{n}{blue} \PYG{n}{path}
                   \PYG{n}{gsave} \PYG{n}{fill} \PYG{n}{grestore} \PYG{n}{stroke}    \PYG{o}{\PYGZpc{}} \PYG{n}{fill}\PYG{p}{,} \PYG{n}{then} \PYG{n}{stroke}
\end{sphinxVerbatim}

\noindent\sphinxincludegraphics{{circle3}.png}


\subsection{Transparency group XObject and soft mask}
\label{\detokenize{pdfmark_Basic:transparency-group-xobject-and-soft-mask}}
To specify a soft mask dictionary in a graphics state, it is necessary to \sphinxstyleemphasis{define} and \sphinxstyleemphasis{access} a transparency group XObject—a form XObject with a \sphinxcode{\sphinxupquote{Group}} entry.

\sphinxhref{pdfmark\_Basic.html\#50454556\_89255}{Transparency group XObject}

\sphinxhref{pdfmark\_Basic.html\#50454556\_17842}{Soft mask dictionaries}

\sphinxhref{pdfmark\_Basic.html\#50454556\_57844}{Soft mask images}

See the \sphinxstyleemphasis{PDF Reference} for complete information.


\subsubsection{Transparency group XObject}
\label{\detokenize{pdfmark_Basic:transparency-group-xobject}}
There are two PostScript idioms that create a Form XObject with Distiller: the \sphinxcode{\sphinxupquote{execform}} operator and the \sphinxcode{\sphinxupquote{BP}} pdfmark feature. In Distiller 6.0 and later, each of these recognize the \sphinxcode{\sphinxupquote{Group}} key that is used to indicate a transparency group. Two forms with differing \sphinxcode{\sphinxupquote{Group}} content are considered to be different. The syntax for these two idioms are:

\begin{sphinxVerbatim}[commandchars=\\\{\}]
 \PYG{o}{\PYGZlt{}\PYGZlt{}}    \PYG{o}{/}\PYG{n}{FormType} \PYG{l+m+mi}{1}
        \PYG{o}{/}\PYG{n}{BBox} \PYG{p}{[}\PYG{n}{xll} \PYG{n}{yll} \PYG{n}{xur} \PYG{n}{yur}\PYG{p}{]}
        \PYG{o}{/}\PYG{n}{Group}
\PYG{n}{group}\PYG{o}{\PYGZhy{}}\PYG{n}{dictionary}

        \PYG{o}{.}\PYG{o}{.}\PYG{o}{.}
    \PYG{o}{\PYGZgt{}\PYGZgt{}}

  \PYG{p}{[} \PYG{o}{/}\PYG{n}{\PYGZus{}objdef} \PYG{p}{\PYGZob{}}\PYG{n}{myForm}\PYG{p}{\PYGZcb{}}
        \PYG{o}{/}\PYG{n}{BBox} \PYG{p}{[}\PYG{n}{xll} \PYG{n}{yll} \PYG{n}{xur} \PYG{n}{yur}\PYG{p}{]}
        \PYG{o}{/}\PYG{n}{Group}
\PYG{n}{group}\PYG{o}{\PYGZhy{}}\PYG{n}{dictionary}

        \PYG{o}{.}\PYG{o}{.}\PYG{o}{.}
        \PYG{o}{/}\PYG{n}{BP} \PYG{n}{pdfmark}
\end{sphinxVerbatim}


\subsubsection{Soft mask dictionaries}
\label{\detokenize{pdfmark_Basic:soft-mask-dictionaries}}
Because Distiller is configured to use \sphinxcode{\sphinxupquote{execform}} (not \sphinxcode{\sphinxupquote{/Form defineresource}} ), there is no direct way for Distiller to access a PostScript form dictionary if it is not used by \sphinxcode{\sphinxupquote{execform}} . But a form used by \sphinxcode{\sphinxupquote{execform}} will always leave marks on the page. So the way to create a soft mask dictionary is to create a transparency group form XObject using the \sphinxcode{\sphinxupquote{BP}} pdfmark feature, then to refer to this form in the soft mask dictionary in the Graphics state. For example:

\begin{sphinxVerbatim}[commandchars=\\\{\}]
 \PYG{p}{[} \PYG{o}{/}\PYG{n}{\PYGZus{}objdef} \PYG{p}{\PYGZob{}}\PYG{n}{myForm}\PYG{p}{\PYGZcb{}}    \PYG{o}{\PYGZpc{}} \PYG{n}{Name} \PYG{n}{to} \PYG{n}{be} \PYG{n}{used} \PYG{n}{by} \PYG{n}{G} \PYG{o+ow}{in} \PYG{n}{Soft} \PYG{n}{Mask} \PYG{n}{below}

        \PYG{o}{/}\PYG{n}{BBox} \PYG{p}{[}
\PYG{n}{xll} \PYG{n}{yll} \PYG{n}{xur} \PYG{n}{yur}
\PYG{p}{]}
        \PYG{o}{/}\PYG{n}{Group}
\PYG{n+nb}{dict}

        \PYG{o}{/}\PYG{n}{BP} \PYG{n}{pdfmark}
        \PYG{o}{.}\PYG{o}{.}\PYG{o}{.}
\PYG{n}{define} \PYG{n}{the} \PYG{n}{shapes} \PYG{n}{here}

        \PYG{o}{/}\PYG{n}{EP} \PYG{n}{pdfmark}


    \PYG{o}{\PYGZpc{}} \PYG{n}{Set} \PYG{n}{the} \PYG{n}{soft} \PYG{n}{mask} \PYG{o+ow}{in} \PYG{n}{Graphics} \PYG{n}{state}

    \PYG{p}{[} \PYG{o}{/}\PYG{n}{SMask} \PYG{o}{\PYGZlt{}\PYGZlt{}} \PYG{o}{/}\PYG{n}{S} \PYG{o}{/}\PYG{n}{Alpha} \PYG{o}{/}\PYG{n}{G} \PYG{p}{\PYGZob{}}\PYG{n}{myForm}\PYG{p}{\PYGZcb{}} \PYG{o}{\PYGZgt{}\PYGZgt{}} \PYG{o}{/}\PYG{n}{SetTransparency} \PYG{n}{pdfmark}
\end{sphinxVerbatim}

Here is another example.
\begin{enumerate}
\sphinxsetlistlabels{\arabic}{enumi}{enumii}{}{.}%
\item {} 
Soft mask dictionaries

\end{enumerate}

\begin{sphinxVerbatim}[commandchars=\\\{\}]
\PYG{l+m+mi}{280} \PYG{l+m+mi}{0} \PYG{n}{translate}
   \PYG{o}{/}\PYG{n}{DeviceCMYK} \PYG{n}{setcolorspace}

   \PYG{o}{\PYGZpc{}} \PYG{n}{Draw} \PYG{n}{the} \PYG{n}{background}\PYG{o}{.}\PYG{o}{.}\PYG{o}{.}

   \PYG{l+m+mi}{0} \PYG{l+m+mi}{0} \PYG{l+m+mi}{0} \PYG{l+m+mf}{0.2} \PYG{n}{setcolor} \PYG{l+m+mi}{10} \PYG{l+m+mi}{540} \PYG{l+m+mi}{100} \PYG{l+m+mi}{200} \PYG{n}{rectfill}
   \PYG{l+m+mi}{1} \PYG{l+m+mi}{1} \PYG{l+m+mi}{1} \PYG{l+m+mi}{0} \PYG{n}{setcolor} \PYG{l+m+mi}{10} \PYG{l+m+mi}{540} \PYG{l+m+mi}{200} \PYG{l+m+mi}{200} \PYG{n}{rectstroke}

   \PYG{o}{\PYGZpc{}} \PYG{n}{Define} \PYG{n}{the} \PYG{n}{form}\PYG{o}{.}\PYG{o}{.}\PYG{o}{.}

   \PYG{p}{[} \PYG{o}{/}\PYG{n}{\PYGZus{}objdef} \PYG{p}{\PYGZob{}}\PYG{n}{aForm}\PYG{p}{\PYGZcb{}} \PYG{o}{/}\PYG{n}{BBox} \PYG{p}{[} \PYG{l+m+mi}{10} \PYG{l+m+mi}{540} \PYG{l+m+mi}{210} \PYG{l+m+mi}{740}\PYG{p}{]}
   \PYG{o}{/}\PYG{n}{Group} \PYG{o}{\PYGZlt{}\PYGZlt{}} \PYG{o}{/}\PYG{n}{S} \PYG{o}{/}\PYG{n}{Transparency} \PYG{o}{/}\PYG{n}{K} \PYG{n}{true}\PYG{o}{\PYGZgt{}\PYGZgt{}} \PYG{o}{/}\PYG{n}{BP} \PYG{n}{pdfmark}
   \PYG{o}{/}\PYG{n}{DeviceCMYK} \PYG{n}{setcolorspace}
   \PYG{l+m+mf}{0.14} \PYG{l+m+mf}{0.85} \PYG{l+m+mf}{0.77} \PYG{l+m+mf}{0.03} \PYG{n}{setcolor} \PYG{l+m+mi}{72} \PYG{l+m+mi}{600} \PYG{l+m+mi}{50} \PYG{l+m+mi}{0} \PYG{l+m+mi}{360} \PYG{n}{arc} \PYG{n}{fill}
   \PYG{l+m+mf}{0.12} \PYG{l+m+mf}{0.02} \PYG{l+m+mf}{0.78} \PYG{l+m+mi}{0} \PYG{n}{setcolor} \PYG{l+m+mi}{110} \PYG{l+m+mi}{650} \PYG{l+m+mi}{50} \PYG{l+m+mi}{0} \PYG{l+m+mi}{360} \PYG{n}{arc} \PYG{n}{fill}
   \PYG{l+m+mf}{0.93} \PYG{l+m+mf}{0.69} \PYG{l+m+mf}{0.07} \PYG{l+m+mf}{0.01} \PYG{n}{setcolor} \PYG{l+m+mi}{147} \PYG{l+m+mi}{600} \PYG{l+m+mi}{50} \PYG{l+m+mi}{0} \PYG{l+m+mi}{360} \PYG{n}{arc} \PYG{n}{fill}
   \PYG{p}{[} \PYG{o}{/}\PYG{n}{EP} \PYG{n}{pdfmark}

   \PYG{o}{\PYGZpc{}} \PYG{n}{Draw} \PYG{n}{the} \PYG{n}{form}\PYG{o}{.}\PYG{o}{.}\PYG{o}{.}

   \PYG{n}{gsave}
   \PYG{p}{[} \PYG{o}{/}\PYG{n}{ca} \PYG{l+m+mf}{0.5} \PYG{o}{/}\PYG{n}{BM} \PYG{o}{/}\PYG{n}{Normal} \PYG{o}{/}\PYG{n}{SetTransparency} \PYG{n}{pdfmark}
   \PYG{p}{[}   \PYG{p}{\PYGZob{}}\PYG{n}{aForm}\PYG{p}{\PYGZcb{}} \PYG{o}{/}\PYG{n}{SP} \PYG{n}{pdfmark}
   \PYG{n}{grestore}

   \PYG{o}{\PYGZpc{}} \PYG{n}{Use} \PYG{n}{the} \PYG{n}{Form} \PYG{k}{as} \PYG{n}{Soft} \PYG{n}{Mask}\PYG{o}{.}\PYG{o}{.}\PYG{o}{.}

   \PYG{p}{[} \PYG{o}{/}\PYG{n}{SMask} \PYG{o}{\PYGZlt{}\PYGZlt{}} \PYG{o}{/}\PYG{n}{S} \PYG{o}{/}\PYG{n}{Alpha} \PYG{o}{/}\PYG{n}{G} \PYG{p}{\PYGZob{}}\PYG{n}{aForm}\PYG{p}{\PYGZcb{}} \PYG{o}{\PYGZgt{}\PYGZgt{}} \PYG{o}{/}\PYG{n}{SetTransparency} \PYG{n}{pdfmark}
   \PYG{o}{.}\PYG{o}{.}\PYG{o}{.}
\end{sphinxVerbatim}


\subsubsection{Soft mask images}
\label{\detokenize{pdfmark_Basic:soft-mask-images}}
There are two ways to specify a soft mask in PDF: a soft\sphinxhyphen{}mask dictionary in the Graphics state as described above, and a soft\sphinxhyphen{}mask image associated with another image XObject (as an \sphinxcode{\sphinxupquote{SMask}} entry).

A soft\sphinxhyphen{}mask image XObject has the same entries as an ordinary image XObject, with some restrictions:
\begin{itemize}
\item {} 
\sphinxcode{\sphinxupquote{ColorSpace}} must be \sphinxcode{\sphinxupquote{DeviceGray}} .

\item {} 
\sphinxcode{\sphinxupquote{Matte}} is an array of component values in the color space of the parent image.

\item {} 
\sphinxcode{\sphinxupquote{Width}} and \sphinxcode{\sphinxupquote{Height}} must be the same as in the parent image if \sphinxcode{\sphinxupquote{Matte}} is present.

\item {} 
\sphinxcode{\sphinxupquote{ImageMask}} must be \sphinxcode{\sphinxupquote{false}} or absent.

\item {} 
\sphinxcode{\sphinxupquote{Mask}} and \sphinxcode{\sphinxupquote{SMask}} must be absent.

\item {} 
\sphinxcode{\sphinxupquote{BitsPerComponent}} is required.

\end{itemize}

Distiller has a mechanism for naming and identifying image objects using the \sphinxcode{\sphinxupquote{NI}} pdfmark feature. To support soft masks, \sphinxcode{\sphinxupquote{NI}} also recognizes three additional entries: \sphinxcode{\sphinxupquote{IsSoftMask}} , \sphinxcode{\sphinxupquote{Matte}} , and \sphinxcode{\sphinxupquote{SMask}} .

\sphinxstylestrong{NI pdfmark}


\begin{savenotes}\sphinxattablestart
\centering
\phantomsection\label{\detokenize{pdfmark_Basic:section-8}}\nobreak
\begin{tabulary}{\linewidth}[t]{|T|T|T|}
\hline
\sphinxstyletheadfamily 
Key
&\sphinxstyletheadfamily 
Value
&\sphinxstyletheadfamily 
Comments
\\
\hline
/\_objdef
&
\{nameobject\}
&
A name object assigned to the next image.
\\
\hline
IsSoftMask
&
Boolean
&
Default is \sphinxcode{\sphinxupquote{false}} .
\\
\hline
Matte
&
array
&
Array of component values specifying matte color with which the parent image data has been pre\sphinxhyphen{}blended.
\\
\hline
SMask
&
\{SoftMaskImageName\}
&
\{SoftMaskImageName\} must be defined already by another \sphinxcode{\sphinxupquote{NI}} pdfmark. If \sphinxcode{\sphinxupquote{SMask}} is present, \sphinxcode{\sphinxupquote{IsSoftMask}} must be \sphinxcode{\sphinxupquote{false}} .
\\
\hline
\end{tabulary}
\par
\sphinxattableend\end{savenotes}

Using the \sphinxcode{\sphinxupquote{NI}} pdfmark feature, you must define the soft\sphinxhyphen{}mask image first and then use it as the \sphinxcode{\sphinxupquote{SMask}} entry for the parent image. For example:
\begin{enumerate}
\sphinxsetlistlabels{\arabic}{enumi}{enumii}{}{.}%
\item {} 
Soft mask images

\end{enumerate}

\begin{sphinxVerbatim}[commandchars=\\\{\}]
 \PYGZbs{}\PYG{p}{[} \PYG{o}{/}\PYG{n}{\PYGZus{}objdef} \PYG{p}{\PYGZob{}}\PYG{n}{mySoftMask}\PYG{p}{\PYGZcb{}}
   \PYG{o}{\PYGZpc{}} \PYG{n}{Name} \PYG{n}{assigned} \PYG{n}{to} \PYG{n}{the} \PYG{n+nb}{next} \PYG{n}{image}\PYG{o}{.}

        \PYG{o}{/}\PYG{n}{SoftMask} \PYG{n}{true}    \PYG{o}{\PYGZpc{}} \PYG{n}{Next} \PYG{n}{image} \PYG{p}{\PYGZob{}}\PYG{n}{mySoftMask123}\PYG{p}{\PYGZcb{}} \PYG{o+ow}{is} \PYG{n}{a} \PYG{n}{soft} \PYG{n}{mask}\PYG{o}{.}

        \PYG{o}{/}\PYG{n}{Matte} \PYG{p}{[}\PYG{o}{.}\PYG{l+m+mi}{1} \PYG{o}{.}\PYG{l+m+mi}{2} \PYG{o}{.}\PYG{l+m+mi}{3}\PYG{p}{]}
        \PYG{o}{/}\PYG{n}{NI} \PYG{n}{pdfmark}

    \PYG{o}{.}\PYG{o}{.}\PYG{o}{.} \PYG{n}{define} \PYG{n}{the} \PYG{n}{soft} \PYG{n}{mask} \PYG{n}{image} \PYG{p}{(}\PYG{n}{ColorSpace} \PYG{n}{must} \PYG{n}{be} \PYG{o}{/}\PYG{n}{DeviceGray}\PYG{p}{)}


    \PYG{p}{[} \PYG{o}{/}\PYG{n}{\PYGZus{}objdef} \PYG{p}{\PYGZob{}}\PYG{n}{myImage}\PYG{p}{\PYGZcb{}}
    \PYG{o}{\PYGZpc{}} \PYG{n}{Name} \PYG{n}{assigned} \PYG{n}{to} \PYG{n+nb}{next} \PYG{n}{image}\PYG{o}{.}

        \PYG{o}{/}\PYG{n}{SMask} \PYG{p}{\PYGZob{}}\PYG{n}{mySoftMask}\PYG{p}{\PYGZcb{}}    \PYG{o}{\PYGZpc{}} \PYG{n}{Associate} \PYG{n}{soft} \PYG{n}{mask}
\PYG{p}{\PYGZob{}}\PYG{n}{mySoftMask123}\PYG{p}{\PYGZcb{}}


        \PYG{o}{/}\PYG{n}{NI} \PYG{n}{pdfmark}

\PYG{o}{.}\PYG{o}{.}\PYG{o}{.} \PYG{n}{define} \PYG{n}{the} \PYG{n}{image} \PYG{n}{here}
\end{sphinxVerbatim}

In this example, the image’s \sphinxcode{\sphinxupquote{ColorSpace}} must have three components and the image data must be preblended with \sphinxcode{\sphinxupquote{{[}.1 .2 .3{]}}} .


\chapter{Actions and Destinations}
\label{\detokenize{pdfmark_Actions:actions-and-destinations}}\label{\detokenize{pdfmark_Actions::doc}}
When a user opens a file, clicks on a link, or clicks on a bookmark, several types of information need to be specified to indicate what should happen. Different pdfmark types require one or more of the following:
\begin{itemize}
\item {} 
\sphinxstyleemphasis{Actions} specify what type of action should be taken. They are indicated by the \sphinxcode{\sphinxupquote{Action}} key in a pdfmark. See \sphinxhref{pdfmark\_Actions.html\#50454557\_92618}{Actions}. \sphinxstyleemphasis{File specifiers} indicate the target of an action when it is not the current file. See the table \sphinxhref{pdfmark\_Actions.html\#50454557\_44931}{File specifier keys}.

\item {} 
\sphinxstyleemphasis{Destinations} specify a particular location in a file, and a zoom factor. See \sphinxhref{pdfmark\_Actions.html\#50454557\_30628}{Destinations}.

\end{itemize}


\section{Actions}
\label{\detokenize{pdfmark_Actions:actions}}
PDF defines several types of actions that can be specified for bookmarks and annotations. The following table outlines the types defined as of PDF 1.3.

Action types


\begin{savenotes}\sphinxattablestart
\centering
\begin{tabulary}{\linewidth}[t]{|T|T|}
\hline
\sphinxstyletheadfamily 
Action type
&\sphinxstyletheadfamily 
Description
\\
\hline
GoTo
&
Go to a destination in the current document.
\\
\hline
GoToR
&
Go to a destination in another document.
\\
\hline
Hide
&
Set an annotation’s Hidden flag.
\\
\hline
ImportData
&
Import field values from a files.
\\
\hline
JavaScript
&
Execute a JavaScript™ script.
\\
\hline
Launch
&
Launch an application, usually to open a file.
\\
\hline
Movie
&
Play a movie.
\\
\hline
Named
&
Execute an action predefined by the viewer application.
\\
\hline
ResetForm
&
Set fields to their default values.
\\
\hline
Sound
&
Play a sound.
\\
\hline
SubmitForm
&
Send data to a URL.
\\
\hline
Thread
&
Begin reading an article thread.
\\
\hline
URI
&
Resolve a uniform resource identifier.
\\
\hline
\end{tabulary}
\par
\sphinxattableend\end{savenotes}

When using pdfmark, the type of action for the annotation or bookmark is specified by the \sphinxcode{\sphinxupquote{Action}} key. It takes one of the following values:
\begin{itemize}
\item {} 
A predefined name corresponding to one of the first four items in the table \sphinxhref{pdfmark\_Actions.html\#50454557\_74461}{Action types}: \sphinxcode{\sphinxupquote{GoTo}} , \sphinxcode{\sphinxupquote{GoToR}} , \sphinxcode{\sphinxupquote{Launch}} , or \sphinxcode{\sphinxupquote{Article}} (which corresponds to the \sphinxcode{\sphinxupquote{Thread}} type in PDF).

\item {} 
A dictionary specifying one of the other types, or a custom action. This dictionary must contain the key–value pairs that are to be placed into the \sphinxstyleemphasis{action dictionary} in the PDF file. See the \sphinxstyleemphasis{PDF Reference} for a detailed description of all the actions and their dictionaries. The syntax for this type of \sphinxcode{\sphinxupquote{Action}} key is as follows:

\end{itemize}

\begin{sphinxVerbatim}[commandchars=\\\{\}]
     /Action \PYGZlt{}\PYGZlt{} / Subtype
actiontype
        ...other action dictionary key–value pairs...
\PYGZgt{}\PYGZgt{}
\end{sphinxVerbatim}

\sphinxhref{pdfmark\_Basic.html\#50454556\_25340}{Custom link action (URI link for the Acrobat WebLink plug\sphinxhyphen{}in)} shows a note pdfmark containing a \sphinxcode{\sphinxupquote{URI}} action.

If the \sphinxcode{\sphinxupquote{Action}} key is not present, the action is assumed to be the equivalent of \sphinxcode{\sphinxupquote{GoTo}} ; that is, jumping to a location in the current document. Actions other than \sphinxcode{\sphinxupquote{GoTo}} may require a file\sphinxhyphen{}specifier key to specify an external document (see the table \sphinxhref{pdfmark\_Actions.html\#50454557\_44931}{File specifier keys}).


\subsection{GoTo actions}
\label{\detokenize{pdfmark_Actions:goto-actions}}
\sphinxcode{\sphinxupquote{GoTo}} actions jump to a specified page and zoom factor within the current document. They require the \sphinxcode{\sphinxupquote{Dest}} key, or both the \sphinxcode{\sphinxupquote{Page}} and \sphinxcode{\sphinxupquote{View}} keys. See \sphinxhref{pdfmark\_Actions.html\#50454557\_30628}{Destinations} for more information on these keys.


\subsection{GoToR actions}
\label{\detokenize{pdfmark_Actions:gotor-actions}}
\sphinxcode{\sphinxupquote{GoToR}} actions specify a location in another PDF file. They require the \sphinxcode{\sphinxupquote{Dest}} key, or both the \sphinxcode{\sphinxupquote{Page}} and \sphinxcode{\sphinxupquote{View}} keys, plus one or more file\sphinxhyphen{}specifier keys (see the table \sphinxhref{pdfmark\_Actions.html\#50454557\_44931}{File specifier keys}).

See \sphinxhref{pdfmark\_Basic.html\#50454556\_44268}{Bookmarks (OUT)} for an example of a \sphinxcode{\sphinxupquote{GoToR}} action.

The following table specifies keys that can be used with the \sphinxcode{\sphinxupquote{GoToR}} , \sphinxcode{\sphinxupquote{Launch}} , and \sphinxcode{\sphinxupquote{Article}} actions to specify the target file.

File specifier keys


\begin{savenotes}\sphinxattablestart
\centering
\phantomsection\label{\detokenize{pdfmark_Actions:section-1}}\nobreak
\begin{tabular}[t]{|*{3}{\X{1}{3}|}}
\hline
\sphinxstyletheadfamily 
Key
&\sphinxstyletheadfamily 
Type
&\sphinxstyletheadfamily 
Semantics
\\
\hline
DOSFile
&
string
&
Optional. The MS\sphinxhyphen{}DOS path (in the PDF path format), of the PDF file. Acrobat viewer applications in Windows and DOS ignore the \sphinxcode{\sphinxupquote{File}} key if the \sphinxcode{\sphinxupquote{DOSFile}} key is present.
\\
\hline
File
&
string
&
Required. The device\sphinxhyphen{}independent path of the PDF file.
\\
\hline
ID
&
array
&
Optional. An array of two strings specifying the PDF file ID. This key can be used to ensure the correct version of the destination file is found. If present, the destination PDF file’s ID is compared with \sphinxcode{\sphinxupquote{ID}} , and the user is warned if they are different.
\\
\hline
MacFile
&
string
&
Optional. The Mac OS file name (in the PDF path format) of the PDF file. Acrobat viewer applications in Mac OS ignore the \sphinxcode{\sphinxupquote{File}} key if the \sphinxcode{\sphinxupquote{MacFile}} key is present.
\\
\hline
UnixFile
&
string
&
Optional. The UNIX file name (in the PDF path format) of the PDF file. Acrobat viewer applications in UNIX ignore the \sphinxcode{\sphinxupquote{File}} key if the \sphinxcode{\sphinxupquote{UnixFile}} key is present.
\\
\hline
URI
&
string
&
Optional. The uniform resource identifier (URI) of a file on the Internet. It can be either an HTML or PDF file. Acrobat viewer applications ignore the \sphinxcode{\sphinxupquote{File}} key if the \sphinxcode{\sphinxupquote{URI}} key is present.

Named destinations may be appended to URLs, following a “\#” character, as in \sphinxcode{\sphinxupquote{http://www.example.com/example.pdf\#name}} . The Acrobat viewer displays the part of the PDF file specified by the named destination.
\begin{itemize}
\item {} 
This key is used with the \sphinxcode{\sphinxupquote{Launch}} action. URIs can also be specified with an action dictionary where the value of the \sphinxcode{\sphinxupquote{Subtype}} key is \sphinxcode{\sphinxupquote{/URI}} (see \sphinxhref{pdfmark\_Basic.html\#50454556\_25340}{Custom link action (URI link for the Acrobat WebLink plug\sphinxhyphen{}in)}.)

\end{itemize}
\\
\hline
\end{tabular}
\par
\sphinxattableend\end{savenotes}

The \sphinxstyleemphasis{PDF Reference} provides more information about the above specifiers.


\subsection{Launch actions}
\label{\detokenize{pdfmark_Actions:launch-actions}}
\sphinxcode{\sphinxupquote{Launch}} actions launch an arbitrary application or document, specified by the \sphinxcode{\sphinxupquote{File}} key. If an application is specified, some platforms allow passing options or filenames to the application that is launched. See \sphinxhref{pdfmark\_Basic.html\#50454556\_25749}{Link that launches another file} for an example of a launch action.

See the table \sphinxhref{pdfmark\_Actions.html\#50454557\_44931}{File specifier keys} for the file specifier keys that can be used by Launch actions. In addition, the following optional keys can be used.

Optional keys for Launch actions


\begin{savenotes}\sphinxattablestart
\centering
\phantomsection\label{\detokenize{pdfmark_Actions:section-2}}\nobreak
\begin{tabulary}{\linewidth}[t]{|T|T|T|}
\hline
\sphinxstyletheadfamily 
Key
&\sphinxstyletheadfamily 
Type
&\sphinxstyletheadfamily 
Semantics
\\
\hline
Dir
&
string
&
Optional. The default directory of a Windows application.
\\
\hline
Op
&
string
&
Optional. The operation to perform; used only under Windows. The string must be open (the default) or print. If \sphinxcode{\sphinxupquote{WinFile}} specifies an application, not a document, this key is ignored and the application is launched.
\\
\hline
Params
&
string
&
Optional. The parameters passed to a Windows application started with the Launch action. If the \sphinxcode{\sphinxupquote{WinFile}} key specifies an application, \sphinxcode{\sphinxupquote{Params}} must not be present.
\\
\hline
WinFile
&
string
&
Optional. The MS\sphinxhyphen{}DOS file name of the document or application to launch.
\\
\hline
\end{tabulary}
\par
\sphinxattableend\end{savenotes}

\begin{sphinxadmonition}{note}{Note:}
Acrobat viewer applications running under Windows use the Windows function \sphinxcode{\sphinxupquote{ShellExecute}} to launch an application specified using the Launch action. The keys \sphinxcode{\sphinxupquote{WinFile}} , \sphinxcode{\sphinxupquote{Dir}} , \sphinxcode{\sphinxupquote{Op}} , and \sphinxcode{\sphinxupquote{Params}} correspond to the parameters of \sphinxcode{\sphinxupquote{ShellExecute}} .
\end{sphinxadmonition}


\subsection{Article actions}
\label{\detokenize{pdfmark_Actions:article-actions}}
Article actions set the Acrobat viewer to article\sphinxhyphen{}reading mode, at the beginning of a specified article in the current document or another PDF document.

They require the \sphinxcode{\sphinxupquote{Dest}} key, which takes one of the following values:
\begin{itemize}
\item {} 
An integer that specifies the article’s index in the document (the first article in a document has an index of 0).

\item {} 
A string that matches the article’s Title.

\end{itemize}

In addition, article actions require one or more file\sphinxhyphen{}specifier keys if the article is in a different PDF file (see the table \sphinxhref{pdfmark\_Actions.html\#50454557\_44931}{File specifier keys}).

See \sphinxhref{pdfmark\_Basic.html\#50454556\_47055}{Article action} for an example of an article action.


\section{Destinations}
\label{\detokenize{pdfmark_Actions:destinations}}
There are two ways of specifying a location within a document that is the target of an action:
\begin{itemize}
\item {} 
\sphinxstyleemphasis{View destinations} explicitly specify a page, a location on the page, and a fit type. View destinations require a \sphinxcode{\sphinxupquote{Page}} key and a \sphinxcode{\sphinxupquote{View}} key. Typically they are used along with an \sphinxcode{\sphinxupquote{Action}} key; if there is no \sphinxcode{\sphinxupquote{Action}} key, the action is the equivalent of \sphinxcode{\sphinxupquote{GoTo}} , meaning to jump to the destination in the current file. See \sphinxhref{pdfmark\_Actions.html\#50454557\_52160}{View destinations}.

\item {} 
\sphinxstyleemphasis{Named destinations} specify the target as a name which has been defined. Named destinations are specified by the \sphinxcode{\sphinxupquote{Dest}} key. They specify a destination in the same file or another file, by name. See \sphinxhref{pdfmark\_Actions.html\#50454557\_47823}{Defining named destinations}.

\end{itemize}


\subsection{View destinations}
\label{\detokenize{pdfmark_Actions:view-destinations}}
View destinations require the following two keys.

Keys for view destinations


\begin{savenotes}\sphinxattablestart
\centering
\phantomsection\label{\detokenize{pdfmark_Actions:section-3}}\nobreak
\begin{tabulary}{\linewidth}[t]{|T|T|T|}
\hline
\sphinxstyletheadfamily 
Key
&\sphinxstyletheadfamily 
Type
&\sphinxstyletheadfamily 
Semantics
\\
\hline
Page
&
integer or name
&
The destination page. An integer value represents the sequence number of the page within the PDF file. The first page in a file is page 1, not page 0.

The name objects \sphinxcode{\sphinxupquote{Next}} and \sphinxcode{\sphinxupquote{Prev}} are valid destination page values for links and articles.

If the destination of a link is on the same page, the \sphinxcode{\sphinxupquote{Page}} key should be omitted. If the value of the \sphinxcode{\sphinxupquote{Page}} key is 0, the bookmark or link has a \sphinxcode{\sphinxupquote{NULL}} destination.
\\
\hline
View
&
array
&
Specifies a link or bookmark’s destination on a page, and its fit type. The first array entry is one of the fit type names shown in the table \sphinxhref{pdfmark\_Actions.html\#50454557\_47544}{Fit type names and parameters}. The remaining entries, if any, specify the location as either a rectangle, a point, or an x– or y–coordinate, depending on the fit type.
\\
\hline
\end{tabulary}
\par
\sphinxattableend\end{savenotes}

All distances and coordinates specified in the following table are in default user space.

Fit type names and parameters


\begin{savenotes}\sphinxattablestart
\centering
\phantomsection\label{\detokenize{pdfmark_Actions:section-4}}\nobreak
\begin{tabular}[t]{|*{3}{\X{1}{3}|}}
\hline
\sphinxstyletheadfamily 
Name
&\sphinxstyletheadfamily 
Parameters
&\sphinxstyletheadfamily 
Description
\\
\hline
Fit
&
None
&
Fit the page to the window. This is a shortcut for specifying \sphinxcode{\sphinxupquote{FitR}} with the rectangle being the crop box for the page.
\\
\hline
FitB
&
None
&
Fit the bounding box of the page contents to the window.
\\
\hline
FitBH
&
top
&
Fit the width of the bounding box of the page contents to the window. top specifies the distance from the page origin to the top of the window.
\\
\hline
FitBV
&
left
&
Fit the height of the bounding box of the page contents to the window. left specifies the distance from the page origin to the left edge of the window.
\\
\hline
FitH
&
top
&
Fit the width of the page to the window. \sphinxstyleemphasis{top} specifies the distance from the page origin to the top of the window. This is a shortcut for specifying \sphinxcode{\sphinxupquote{FitR}} with the rectangle having the width of the page, and both y\sphinxhyphen{}coordinates equal to top.
\\
\hline
FitR
&
x1 y1 x2 y2
&
Fit the rectangle specified by the parameters to the window.
\\
\hline
FitV
&
left
&
Fit the height of the page to the window. \sphinxstyleemphasis{left} specifies the distance in from the page origin to the left edge of the window. This is a shortcut for specifying \sphinxcode{\sphinxupquote{FitR}} with the rectangle having the height of the page, and both x\sphinxhyphen{}coordinates equal to left.
\\
\hline
XYZ
&
left top zoom
&
left and top specify the distance from the origin of the page to the top\sphinxhyphen{}left corner of the window. zoom specifies the zoom factor, with 1 being 100\% magnification. If left, top or zoom is \sphinxcode{\sphinxupquote{NULL}} , the current value of that parameter is retained. For example, specifying a view destination of
\begin{quote}

/View {[}/XYZ NULL NULL NULL{]}
\end{quote}

goes to the specified page and retains the same horizontal and vertical offset and zoom as the current page. A zoom of 0 has the same meaning as a zoom of \sphinxcode{\sphinxupquote{NULL}} .
\\
\hline
\end{tabular}
\par
\sphinxattableend\end{savenotes}

The zoom factors for the horizontal and vertical directions are identical; there are not separate zoom factors for the two directions. As a result, more of the page may be shown than specified by the destination. For example, when using \sphinxcode{\sphinxupquote{FitR}} , portions of the page outside the destination rectangle appear in the window unless the window happens to have the same aspect ratio (height\sphinxhyphen{}to\sphinxhyphen{}width ratio) as the destination rectangle.

A common destination is “upper left corner of the specified page, with a zoom factor of 1.” This can be obtained using the \sphinxcode{\sphinxupquote{XYZ}} destination form, with a left of \sphinxhyphen{}4 and a top equal to the top of the \sphinxcode{\sphinxupquote{CropBox}} (or the page size if no \sphinxcode{\sphinxupquote{CropBox}} was specified) plus 4. The offset of 4 is used to slightly move the page corner from the corner of the window, to provide a visual cue that the corner of the page is being shown.

The following sections have examples related to destinations: \sphinxhref{pdfmark\_Basic.html\#50454556\_13483}{Links}, \sphinxhref{pdfmark\_Basic.html\#50454556\_44268}{Bookmarks (OUT)}, \sphinxhref{pdfmark\_Basic.html\#50454556\_35664}{File Open action}, and \sphinxhref{pdfmark\_Actions.html\#50454557\_47823}{Defining named destinations}.


\subsection{Defining named destinations}
\label{\detokenize{pdfmark_Actions:defining-named-destinations}}
Locations in PDF files can be specified by name instead of by page number and view. These names can then be used as destinations of bookmarks or links. Using named destinations is particularly advantageous for cross\sphinxhyphen{}document links, because if the document containing a link’s destination is revised, the link still works, regardless of whether its location in the file has changed.

A named destination is specified by using the pdfmark operator with the name \sphinxcode{\sphinxupquote{DEST}} . The syntax for a named destination pdfmark is as follows:

\begin{sphinxVerbatim}[commandchars=\\\{\}]
 \PYG{p}{[} \PYG{o}{/}\PYG{n}{Dest}
\PYG{n}{name}

        \PYG{o}{/}\PYG{n}{Page}
\PYG{n}{pagenum}

        \PYG{o}{/}\PYG{n}{View}
\PYG{n}{destination}

        \PYG{o}{/}\PYG{n}{DEST} \PYG{n}{pdfmark}
\end{sphinxVerbatim}

Named destination attributes


\begin{savenotes}\sphinxattablestart
\centering
\phantomsection\label{\detokenize{pdfmark_Actions:section-5}}\nobreak
\begin{tabulary}{\linewidth}[t]{|T|T|T|}
\hline
\sphinxstyletheadfamily 
Key
&\sphinxstyletheadfamily 
Type
&\sphinxstyletheadfamily 
Description
\\
\hline
Dest
&
name
&
Required. The destination’s name.
\\
\hline
Page
&
integer
&
Optional. The sequence number of the destination page. If present, the named destination pdfmark can be placed anywhere in the PostScript language file. If omitted, the pdfmark must occur within the PostScript language description for the destination page.
\\
\hline
View
&
array
&
Optional. The view to display on the destination page. If omitted, defaults to a null destination (lower left corner of the page at a zoom of 100\%). See \sphinxhref{pdfmark\_Actions.html\#50454557\_30628}{Destinations} for information on specifying a view destination.
\\
\hline
\end{tabulary}
\par
\sphinxattableend\end{savenotes}

In addition to the keys listed in the table \sphinxhref{pdfmark\_Actions.html\#50454557\_52162}{Named destination attributes}, named destinations can also specify arbitrary key–value pairs.

Named destinations can be appended to URLs, following a “\#” character, as in \sphinxcode{\sphinxupquote{http://www.example.com/example.pdf\#nameddest=name}} . The Acrobat viewer displays the part of the PDF file specified in the named destination.
\begin{enumerate}
\sphinxsetlistlabels{\arabic}{enumi}{enumii}{}{.}%
\item {} 
Definition of named destination

\end{enumerate}

\begin{sphinxVerbatim}[commandchars=\\\{\}]
\PYG{p}{[} \PYG{o}{/}\PYG{n}{Dest} \PYG{o}{/}\PYG{n}{MyNamedDest}
    \PYG{o}{/}\PYG{n}{Page} \PYG{l+m+mi}{1}
    \PYG{o}{/}\PYG{n}{View} \PYG{p}{[}\PYG{o}{/}\PYG{n}{FitH} \PYG{l+m+mi}{5}\PYG{p}{]}
    \PYG{o}{/}\PYG{n}{DEST} \PYG{n}{pdfmark}
\end{sphinxVerbatim}
\begin{enumerate}
\sphinxsetlistlabels{\arabic}{enumi}{enumii}{}{.}%
\item {} 
Link to a named destination

\end{enumerate}

\begin{sphinxVerbatim}[commandchars=\\\{\}]
\PYG{p}{[} \PYG{o}{/}\PYG{n}{Rect} \PYG{p}{[}\PYG{l+m+mi}{70} \PYG{l+m+mi}{650} \PYG{l+m+mi}{210} \PYG{l+m+mi}{675}\PYG{p}{]}
    \PYG{o}{/}\PYG{n}{Border} \PYG{p}{[}\PYG{l+m+mi}{16} \PYG{l+m+mi}{16} \PYG{l+m+mi}{1} \PYG{p}{[}\PYG{l+m+mi}{3} \PYG{l+m+mi}{10}\PYG{p}{]}\PYG{p}{]}
    \PYG{o}{/}\PYG{n}{Color} \PYG{p}{[}\PYG{l+m+mi}{0} \PYG{o}{.}\PYG{l+m+mi}{7} \PYG{l+m+mi}{1}\PYG{p}{]}
    \PYG{o}{/}\PYG{n}{Dest} \PYG{o}{/}\PYG{n}{MyNamedDest}
    \PYG{o}{/}\PYG{n}{Subtype} \PYG{o}{/}\PYG{n}{Link}
    \PYG{o}{/}\PYG{n}{ANN} \PYG{n}{pdfmark}
\end{sphinxVerbatim}


\subsection{Referencing named destinations}
\label{\detokenize{pdfmark_Actions:referencing-named-destinations}}
Named destinations that have been defined with the \sphinxcode{\sphinxupquote{DEST}} pdfmark can be used as the target of a bookmark or link, or by the optional open action in a document’s Catalog dictionary. They are specified using the \sphinxcode{\sphinxupquote{Dest}} key.

See \sphinxhref{pdfmark\_Actions.html\#50454557\_47823}{Defining named destinations} for examples of named destinations.

\begin{sphinxadmonition}{note}{Note:}
When used with the \sphinxcode{\sphinxupquote{Article}} action, \sphinxcode{\sphinxupquote{Dest}} has a different syntax. See \sphinxhref{pdfmark\_Actions.html\#50454557\_38957}{Article actions}.
\end{sphinxadmonition}


\chapter{Logical Structure}
\label{\detokenize{pdfmark_Logical:logical-structure}}\label{\detokenize{pdfmark_Logical::doc}}
PDF files (in versions 1.3 and later) can contain \sphinxstyleemphasis{structure trees} giving a logical structure to the information in a document. The facilities for logical structure in PDF are described in the \sphinxstyleemphasis{PDF Reference} .

A \sphinxstyleemphasis{structure suite} of names is used with the pdfmark operator that can be used to specify logical structure within PDF files.

\sphinxhref{pdfmark\_Examples.html\#50454555\_65255}{Structure examples} gives a variety of examples of using the structure suite.


\section{Elements and parents}
\label{\detokenize{pdfmark_Logical:elements-and-parents}}
A document’s logical structure consists of a hierarchy of \sphinxstyleemphasis{structure elements.} Elements can contain contents and attributes. At the root of the hierarchy is a dictionary object called the Structure Tree Root.

When using the structure suite, the hierarchy is established by means of the \sphinxstyleemphasis{implicit parent stack} of elements. Elements can be pushed onto or popped off of this stack. When an element is created, its parent is the current top item on the stack. If the stack is empty, the document’s Structure Tree Root is made the parent; the Structure Tree Root is created if it does not already exist. When element content is created, its containing element is the current top item on the stack.

\begin{sphinxadmonition}{note}{Note:}
Some operators that specify an element cannot accept the Structure Tree Root as the implicit argument; therefore these commands are in error if the implicit parent stack is empty when they are encountered or if the top item on the stack is the Structure Tree Root. These cases are noted in the command descriptions.
\end{sphinxadmonition}


\section{Structure operators}
\label{\detokenize{pdfmark_Logical:structure-operators}}
This section lists the pdfmark names that make up the structure suite. Most of these are directly related to PDF logical structure features, but some only manipulate the state of the PDF creation process, without corresponding to any particular output.
\begin{itemize}
\item {} 
Structure Tree Root
\begin{itemize}
\item {} 
\sphinxhref{pdfmark\_Logical.html\#50454553\_91312}{StRoleMap} adds entries to the role map.

\item {} 
\sphinxhref{pdfmark\_Logical.html\#50454553\_StClassMap}{StClassMap} adds entries to the class map.

\end{itemize}

\item {} 
Elements
\begin{itemize}
\item {} 
\sphinxhref{pdfmark\_Logical.html\#50454553\_48437}{StPNE} creates a new structure element.

\item {} 
\sphinxhref{pdfmark\_Logical.html\#50454553\_34079}{StBookmarkRoot} creates a root bookmark for a structure bookmark tree.

\item {} 
\sphinxhref{pdfmark\_Logical.html\#50454553\_75789}{StPush} pushes an existing element onto the implicit parent stack.

\item {} 
\sphinxhref{pdfmark\_Logical.html\#50454553\_69556}{StPop} pops an element off the implicit parent stack.

\item {} 
\sphinxhref{pdfmark\_Logical.html\#50454553\_66687}{StPopAll} empties the implicit parent stack.

\end{itemize}

\item {} 
Element Content
\begin{itemize}
\item {} 
\sphinxhref{pdfmark\_Logical.html\#50454553\_53028}{StBMC} indicates the beginning of marked content.

\item {} 
\sphinxhref{pdfmark\_Logical.html\#50454553\_73021}{StBDC} indicates the beginning of marked content with a dictionary.

\item {} 
\sphinxhref{pdfmark\_Logical.html\#50454553\_75598}{EMC} delimits the end of marked content.

\item {} 
\sphinxhref{pdfmark\_Logical.html\#50454553\_43183}{StOBJ} adds an existing PDF object as part of an element’s content.

\end{itemize}

\item {} 
Attributes
\begin{itemize}
\item {} 
\sphinxhref{pdfmark\_Logical.html\#50454553\_10527}{StAttr} enables the attachment of attribute objects to elements.

\end{itemize}

\item {} 
Saving and restoring the stack
\begin{itemize}
\item {} 
\sphinxhref{pdfmark\_Logical.html\#50454553\_60367}{StStore} saves the current state of the implicit parent stack.

\item {} 
\sphinxhref{pdfmark\_Logical.html\#50454553\_89182}{StRetrieve} restores the implicit parent stack from a saved state.

\end{itemize}

\end{itemize}

The following sections provide details about the structure suite.


\section{Structure Tree Root}
\label{\detokenize{pdfmark_Logical:structure-tree-root}}
Distiller automatically creates a new Structure Tree Root the first time it creates a new element with \sphinxcode{\sphinxupquote{StPNE}} (see \sphinxhref{pdfmark\_Logical.html\#50454553\_48437}{StPNE}).

The Structure Tree Root contains a \sphinxstyleemphasis{role map} and a \sphinxstyleemphasis{class map} (see the \sphinxstyleemphasis{PDF Reference} for details). The following two pdfmark features can be used to add information to these maps.


\subsection{StRoleMap}
\label{\detokenize{pdfmark_Logical:strolemap}}
\sphinxcode{\sphinxupquote{StRoleMap}} specifies key\sphinxhyphen{}value pairs to be added as dictionary entries to the Structure Tree Root’s role map. If the Structure Tree Root doesn’t already exist, it is created; if the Structure Tree Root doesn’t have a role map dictionary, one is created. A given key–value pair always modifies the role map, even if the key is already in the dictionary.

The syntax for adding entries to a role map is as follows:

\begin{sphinxVerbatim}[commandchars=\\\{\}]
 \PYG{p}{[} \PYG{o}{/}
\PYG{n}{new}\PYG{o}{\PYGZhy{}}\PYG{n}{element}\PYG{o}{\PYGZhy{}}\PYG{n}{subtype}\PYG{o}{\PYGZhy{}}\PYG{n}{name}\PYG{p}{[} \PYG{o}{/}
            \PYG{o}{/}
\PYG{n}{standard}\PYG{o}{\PYGZhy{}}\PYG{n}{structural}\PYG{o}{\PYGZhy{}}\PYG{n}{subtype}\PYG{o}{\PYGZhy{}}\PYG{n}{name}\PYG{p}{[} \PYG{o}{/}
        \PYG{o}{.}\PYG{o}{.}\PYG{o}{.}\PYG{p}{[} \PYG{o}{/}           \PYG{o}{/}
\PYG{n}{new}\PYG{o}{\PYGZhy{}}\PYG{n}{element}\PYG{o}{\PYGZhy{}}\PYG{n}{subtype}\PYG{o}{\PYGZhy{}}\PYG{n}{name}\PYG{p}{[} \PYG{o}{/}
            \PYG{o}{/}
\PYG{n}{standard}\PYG{o}{\PYGZhy{}}\PYG{n}{structural}\PYG{o}{\PYGZhy{}}\PYG{n}{subtype}\PYG{o}{\PYGZhy{}}\PYG{n}{name}\PYG{p}{[} \PYG{o}{/}
        \PYG{o}{/}\PYG{n}{StRoleMap} \PYG{n}{pdfmark}
\end{sphinxVerbatim}


\subsection{StClassMap}
\label{\detokenize{pdfmark_Logical:stclassmap}}
\sphinxcode{\sphinxupquote{StClassMap}} behaves like \sphinxcode{\sphinxupquote{StRoleMap}} , except that it adds entries to the Structure Tree Root’s class map, rather than the role map. The syntax for adding entries to a class map is as follows:

\begin{sphinxVerbatim}[commandchars=\\\{\}]
 \PYG{p}{[} \PYG{o}{/}
\PYG{n}{class}\PYG{o}{\PYGZhy{}}\PYG{n}{name}
\PYG{o}{/}
\PYG{n}{attribute}\PYG{o}{\PYGZhy{}}\PYG{n+nb}{object}\PYG{o}{\PYGZhy{}}\PYG{n}{name}\PYG{p}{[} \PYG{o}{/}
        \PYG{o}{.}\PYG{o}{.}\PYG{o}{.}\PYG{p}{[} \PYG{o}{/}           \PYG{o}{/}
\PYG{n}{class}\PYG{o}{\PYGZhy{}}\PYG{n}{name}
\PYG{o}{/}
\PYG{n}{attribute}\PYG{o}{\PYGZhy{}}\PYG{n+nb}{object}\PYG{o}{\PYGZhy{}}\PYG{n}{name}\PYG{p}{[} \PYG{o}{/}
        \PYG{o}{/}\PYG{n}{StClassMap} \PYG{n}{pdfmark}
\end{sphinxVerbatim}


\section{Elements}
\label{\detokenize{pdfmark_Logical:elements}}
The structure suite provides several commands to create elements and link them into structure trees.


\subsection{StPNE}
\label{\detokenize{pdfmark_Logical:stpne}}
\sphinxcode{\sphinxupquote{StPNE}} (“Push New Element”) creates a new element whose parent is the element on the top of the implicit parent stack. Its syntax is as follows:

\begin{sphinxVerbatim}[commandchars=\\\{\}]
 \PYG{p}{[} \PYG{o}{/}\PYG{n}{Subtype}
\PYG{n}{name}\PYG{p}{[} \PYG{o}{/}
        \PYG{o}{/}\PYG{n}{\PYGZus{}objdef} \PYG{p}{\PYGZob{}}
\PYG{n}{objname}
\PYG{p}{\PYGZcb{}}\PYG{p}{[} \PYG{o}{/}           \PYG{o}{/}\PYG{n}{Title}
\PYG{n}{string}\PYG{p}{[} \PYG{o}{/}
        \PYG{o}{/}\PYG{n}{Alt}
\PYG{n}{string}\PYG{p}{[} \PYG{o}{/}
        \PYG{o}{/}\PYG{n}{ID}
\PYG{n}{string}\PYG{p}{[} \PYG{o}{/}
        \PYG{o}{/}\PYG{n}{Class}
\PYG{n}{name}\PYG{p}{[} \PYG{o}{/}
        \PYG{o}{/}\PYG{n}{At}
\PYG{n}{integer}\PYG{p}{[} \PYG{o}{/}
        \PYG{o}{/}\PYG{n}{Bookmark}
\PYG{n}{dictionary}\PYG{p}{[} \PYG{o}{/}
        \PYG{o}{/}\PYG{n}{StPNE} \PYG{n}{pdfmark}
\end{sphinxVerbatim}

These keys are described in the following table.

Common element keys


\begin{savenotes}\sphinxattablestart
\centering
\begin{tabulary}{\linewidth}[t]{|T|T|T|}
\hline
\sphinxstyletheadfamily 
Key
&\sphinxstyletheadfamily 
Type
&\sphinxstyletheadfamily 
Description
\\
\hline
Subtype
&
name
&
Required. The element type, such as Link or Section.
\\
\hline
Title
&
string
&
Optional. A human\sphinxhyphen{}readable name for the particular element.
\\
\hline
Alt
&
string
&
Optional. An alternate representation of the element’s contents as human\sphinxhyphen{}readable text
\\
\hline
ID
&
string
&
Optional. A unique identifier for the element. The identifier must be unique within the document in which the element occurs. It is an error to specify an element with the same ID as an existing element in the same tree.
\\
\hline
Class
&
name
&
Optional. The class name to be associated with the element
\\
\hline
At
&
integer
&
Optional. Index at which to insert this item within its parent. If omitted, or greater than or equal to the parent’s current number of children, the item is added as the \sphinxstyleemphasis{last} child of its parent, retaining all existing items in their original positions. If less than or equal to zero, the new item becomes the \sphinxstyleemphasis{first} child of its parent. If the index is any other number, the item is inserted at that index within the container, and all items that had indices greater than or equal to the given index are shifted to the position with index one greater. An item may be an element, marked content, or a PDF object.
\\
\hline
Bookmark
&
dictionary
&
Optional. Specifies a bookmark that is generated for this structural element. The table \sphinxhref{pdfmark\_Logical.html\#50454553\_33804}{Bookmark dictionary / bookmark tree root} describes this dictionary.
\\
\hline
\end{tabulary}
\par
\sphinxattableend\end{savenotes}

A new element is added to its parent at the index specified with the \sphinxcode{\sphinxupquote{At}} key. The newly\sphinxhyphen{}created element is pushed onto the implicit parent stack.

\begin{sphinxadmonition}{note}{Note:}
If the implicit parent stack is empty, the Structure Tree Root is pushed onto the stack and used as the new element’s parent. If there is no Structure Tree Root, one is created, pushed onto the stack, and used as the new element’s parent.
\end{sphinxadmonition}

\sphinxcode{\sphinxupquote{StPNE}} may also take the key \sphinxcode{\sphinxupquote{\_objdef}} to specify an object name for the element. Once an element is named, it can be referenced with the \sphinxcode{\sphinxupquote{E}} key of the \sphinxcode{\sphinxupquote{StPush}} pdfmark (see \sphinxhref{pdfmark\_Logical.html\#50454553\_75789}{StPush}).

The \sphinxcode{\sphinxupquote{Bookmark}} key allows a bookmark to be automatically generated for an element and added to the Structured Bookmark subtree. Its value is a bookmark dictionary, which may contain the \sphinxcode{\sphinxupquote{Title}} and \sphinxcode{\sphinxupquote{Open}} keys described in the following table.

Bookmark dictionary / bookmark tree root


\begin{savenotes}\sphinxattablestart
\centering
\phantomsection\label{\detokenize{pdfmark_Logical:section-1}}\nobreak
\begin{tabulary}{\linewidth}[t]{|T|T|T|}
\hline
\sphinxstyletheadfamily 
Key
&\sphinxstyletheadfamily 
Type
&\sphinxstyletheadfamily 
Semantics
\\
\hline
Open
&
Boolean
&
Optional. If \sphinxcode{\sphinxupquote{true}} , the bookmark is open, that is, its children are visible. If \sphinxcode{\sphinxupquote{false}} , the bookmark is closed. If this key is absent, the bookmark is closed.
\\
\hline
Title
&
string
&
Optional. The bookmark title. The encoding and character set used is either PDFDocEncoding (as described in the \sphinxstyleemphasis{PDF Reference} ) or Unicode. If Unicode, the string must begin with <FEFF>. For example, the Unicode string for (ABC) is <FEFF004100420043>. \sphinxcode{\sphinxupquote{Title}} has a maximum length of 255 PDFDocEncoding characters or 126 Unicode values, although a practical limit of 32 characters is advised so that it can be read easily in the Acrobat viewer.
\\
\hline
\end{tabulary}
\par
\sphinxattableend\end{savenotes}

If the \sphinxcode{\sphinxupquote{Title}} key is absent, the title is the title of the element or its subtype.

The bookmark dictionary may also contain key\sphinxhyphen{}value pairs that specify an action to be taken when the bookmark is activated (see \sphinxhref{pdfmark\_Actions.html\#50454557\_94191}{Actions and Destinations}). If none of the action keys are present, the bookmark’s action is to go to either the first page where marked content is a child of this element or a child in one of its descendant elements.

The example \sphinxhref{pdfmark\_Examples.html\#50454555\_70689}{A bookmark for a structural element} defines a bookmark for an element.


\subsection{StBookmarkRoot}
\label{\detokenize{pdfmark_Logical:stbookmarkroot}}
\sphinxcode{\sphinxupquote{StBookmarkRoot}} creates the root bookmark for structure bookmarks added by a \sphinxcode{\sphinxupquote{StPNE}} with a \sphinxcode{\sphinxupquote{Bookmark}} key. Its syntax is as follows:

\begin{sphinxVerbatim}[commandchars=\\\{\}]
 \PYG{p}{[} \PYG{o}{/}\PYG{n}{Title}
\PYG{n}{string}\PYG{p}{[} \PYG{o}{/}
        \PYG{o}{/}\PYG{n}{Open}
\PYG{n}{boolean}\PYG{p}{[} \PYG{o}{/}           \PYG{o}{.}\PYG{o}{.}\PYG{o}{.} \PYG{n}{action}\PYG{o}{\PYGZhy{}}\PYG{n}{specifying}\PYG{o}{\PYGZhy{}}\PYG{n}{keys} \PYG{o}{.}\PYG{o}{.}\PYG{o}{.}\PYG{p}{[} \PYG{o}{/}
        \PYG{o}{/}\PYG{n}{StBookmarkRoot} \PYG{n}{pdfmark}
\end{sphinxVerbatim}

It contains the \sphinxcode{\sphinxupquote{Title}} and \sphinxcode{\sphinxupquote{Open}} keys shown in the table \sphinxhref{pdfmark\_Logical.html\#50454553\_33804}{Bookmark dictionary / bookmark tree root}. If the \sphinxcode{\sphinxupquote{Title}} key is absent, the title is “Untitled”.

It may also contain the action keys in \sphinxhref{pdfmark\_Actions.html\#50454557\_94191}{Actions and Destinations} if none of these keys are present, the bookmark root has no action associated with it.

An operator with \sphinxcode{\sphinxupquote{StBookmarkRoot}} \sphinxstyleemphasis{must} appear before any \sphinxcode{\sphinxupquote{StPNE}} with a \sphinxcode{\sphinxupquote{Bookmark}} key; otherwise the default (“Untitled”, closed, no action) is used for the structured bookmark subtree.


\subsection{StPush}
\label{\detokenize{pdfmark_Logical:stpush}}
\sphinxcode{\sphinxupquote{StPush}} pushes an existing element onto the implicit parent stack. The syntax for pushing an element is as follows:

\begin{sphinxVerbatim}[commandchars=\\\{\}]
 \PYG{p}{[}\PYG{o}{/}\PYG{n}{E} \PYG{p}{\PYGZob{}}
\PYG{n}{objname}
\PYG{p}{\PYGZcb{}}

    \PYG{o}{/}\PYG{n}{StPush} \PYG{n}{pdfmark}
\end{sphinxVerbatim}

The \sphinxcode{\sphinxupquote{E}} key specifies an existing element, given as an object name of the special form \{objname\} used to refer to Cos objects. It must be a name that was created by a previous \sphinxcode{\sphinxupquote{StPNE}} using the \sphinxcode{\sphinxupquote{\_objdef}} key (see \sphinxhref{pdfmark\_Logical.html\#50454553\_48437}{StPNE}).

\begin{sphinxadmonition}{note}{Note:}
If the \sphinxcode{\sphinxupquote{E}} key is omitted, the Structure Tree Root of the document is specified. The Structure Tree Root is created if it does not already exist.
\end{sphinxadmonition}


\subsection{StPop}
\label{\detokenize{pdfmark_Logical:stpop}}
\sphinxcode{\sphinxupquote{StPop}} removes the element at the top of the implicit parent stack. It is an error for \sphinxcode{\sphinxupquote{StPop}} to be encountered when the implicit parent stack is empty.

The syntax for popping an element is as follows:

\begin{sphinxVerbatim}[commandchars=\\\{\}]
\PYG{p}{[} \PYG{o}{/}\PYG{n}{StPop} \PYG{n}{pdfmark}
\end{sphinxVerbatim}


\subsection{StPopAll}
\label{\detokenize{pdfmark_Logical:stpopall}}
\sphinxcode{\sphinxupquote{StPopAll}} completely empties the implicit parent stack. The syntax for emptying the stack is as follows:

\begin{sphinxVerbatim}[commandchars=\\\{\}]
\PYG{p}{[} \PYG{o}{/}\PYG{n}{StPopAll} \PYG{n}{pdfmark}
\end{sphinxVerbatim}


\subsection{StUpdate}
\label{\detokenize{pdfmark_Logical:stupdate}}
\sphinxcode{\sphinxupquote{StUpdate}} updates the entries of the current structure element. The syntax is as follows:

\begin{sphinxVerbatim}[commandchars=\\\{\}]
 \PYG{p}{[}   \PYG{o}{\PYGZlt{}\PYGZlt{}}
\PYG{o}{/}\PYG{n}{S} \PYG{o}{/}\PYG{n}{Span}\PYG{o}{.}\PYG{o}{.}\PYG{o}{.}
 \PYG{o}{\PYGZgt{}\PYGZgt{}} \PYG{o}{/}\PYG{n}{StUpdate} \PYG{n}{pdfmark}
\end{sphinxVerbatim}


\section{Element content}
\label{\detokenize{pdfmark_Logical:element-content}}
Elements can have two kinds of document content: marked content and references to PDF objects.

Use \sphinxcode{\sphinxupquote{StBDC}} and \sphinxcode{\sphinxupquote{StBMC}} to indicate the beginning of marked content and \sphinxcode{\sphinxupquote{EMC}} to delimit the end of marked content. These operators combine the creation of the marked content region in the PDF content stream with the creation of marked content and its placement within the structure hierarchy.

\begin{sphinxadmonition}{note}{Note:}
Marked content can be specified independently of the structure suite, using the operators described in \sphinxhref{pdfmark\_Basic.html\#50454556\_41438}{Marked content (MP, DP, BMC, BDC, EMC)}.
\end{sphinxadmonition}

It is possible to nest marked content by nesting the \sphinxcode{\sphinxupquote{StBMC/BDC}} and \sphinxcode{\sphinxupquote{EMC}} operators. This is different from the nesting maintained by the tree structure of elements, which is implemented using \sphinxcode{\sphinxupquote{StPNE}} and \sphinxcode{\sphinxupquote{StPop}} . Note that nested marked content may belong to elements in different branches of a Structure Tree.

To specify references to PDF objects, use the \sphinxcode{\sphinxupquote{StOBJ}} operator.


\subsection{StBMC}
\label{\detokenize{pdfmark_Logical:stbmc}}
\sphinxcode{\sphinxupquote{StBMC}} marks the beginning of a sequence of marked content objects. Its syntax is as follows:

\begin{sphinxVerbatim}[commandchars=\\\{\}]
 \PYG{p}{[} \PYG{o}{/}\PYG{n}{T}
\PYG{n}{tag}\PYG{p}{[} \PYG{o}{/}
        \PYG{o}{/}\PYG{n}{At}
\PYG{n}{integer}\PYG{p}{[} \PYG{o}{/}
        \PYG{o}{/}\PYG{n}{StBMC} \PYG{n}{pdfmark}
\end{sphinxVerbatim}

The marked content is added to its containing element (the top element of the implicit parent stack) at the position optionally specified by the \sphinxcode{\sphinxupquote{At}} key (see the table \sphinxhref{pdfmark\_Logical.html\#50454553\_65055}{Common element keys}). The \sphinxcode{\sphinxupquote{T}} key is described in the following table. It is an error if the implicit parent stack is empty when \sphinxcode{\sphinxupquote{StBMC}} is encountered.

Specifying tags and property list entries for marked content


\begin{savenotes}\sphinxattablestart
\centering
\phantomsection\label{\detokenize{pdfmark_Logical:section-2}}\nobreak
\begin{tabulary}{\linewidth}[t]{|T|T|T|}
\hline
\sphinxstyletheadfamily 
Key
&\sphinxstyletheadfamily 
Type
&\sphinxstyletheadfamily 
Description
\\
\hline
P (Properties)
&
dictionary
&
Optional. Key–value pairs that are entered into the properties dictionary of the marked content being created. If this key is omitted, no properties other than those required by the implementation of logical structure in PDF are entered into the properties dictionary. This key is supported only with StBDC.
\\
\hline
T (Tag)
&
name
&
Optional. The tag to be given to the marked content being created. If this key is omitted, the subtype of the containing element is used.
\\
\hline
\end{tabulary}
\par
\sphinxattableend\end{savenotes}


\subsection{StBDC}
\label{\detokenize{pdfmark_Logical:stbdc}}
\sphinxcode{\sphinxupquote{StBDC}} marks the beginning of a sequence of page content objects with an associated property list, given by a dictionary. \sphinxcode{\sphinxupquote{StBDC}} behaves just like \sphinxcode{\sphinxupquote{StBMC}} , with the addition of a property list. Its syntax is as follows:

\begin{sphinxVerbatim}[commandchars=\\\{\}]
 \PYG{p}{[} \PYG{o}{/}\PYG{n}{T}
\PYG{n}{tag}\PYG{p}{[} \PYG{o}{/}
        \PYG{o}{/}\PYG{n}{P}
\PYG{n}{properties}\PYG{o}{\PYGZhy{}}\PYG{n}{dictionary}\PYG{p}{[} \PYG{o}{/}
        \PYG{o}{/}\PYG{n}{At}
\PYG{n}{integer}\PYG{p}{[} \PYG{o}{/}
        \PYG{o}{/}\PYG{n}{StBDC} \PYG{n}{pdfmark}
\end{sphinxVerbatim}

The marked content is added to its containing element (the element on top of the implicit parent stack) at the position optionally specified by the \sphinxcode{\sphinxupquote{At}} key (see the table \sphinxhref{pdfmark\_Logical.html\#50454553\_65055}{Common element keys}). The \sphinxcode{\sphinxupquote{P (Properties)}} and \sphinxcode{\sphinxupquote{T (Tag)}} keys are described in the table \sphinxhref{pdfmark\_Logical.html\#50454553\_43535}{Specifying tags and property list entries for marked content}. It is an error if the implicit parent stack is empty when \sphinxcode{\sphinxupquote{StBDC}} is encountered.


\subsection{EMC}
\label{\detokenize{pdfmark_Logical:emc}}
\sphinxcode{\sphinxupquote{EMC}} signals the end of a marked sequence of page content operators. Its syntax is as follows:

\begin{sphinxVerbatim}[commandchars=\\\{\}]
\PYG{p}{[} \PYG{o}{/}\PYG{n}{EMC} \PYG{n}{pdfmark}
\end{sphinxVerbatim}


\subsection{StOBJ}
\label{\detokenize{pdfmark_Logical:stobj}}
\sphinxcode{\sphinxupquote{StOBJ}} adds an existing PDF object to the content of the top element of the implicit parent stack, using the Cos object reference mechanism. Its syntax is as follows:

\begin{sphinxVerbatim}[commandchars=\\\{\}]
 \PYG{p}{[} \PYG{o}{/}\PYG{n}{Obj} \PYG{p}{\PYGZob{}}
\PYG{n}{objname}
\PYG{p}{\PYGZcb{}}\PYG{p}{[} \PYG{o}{/}           \PYG{o}{/}\PYG{n}{At}
\PYG{n}{integer}\PYG{p}{[} \PYG{o}{/}
        \PYG{o}{/}\PYG{n}{StOBJ} \PYG{n}{pdfmark}
\end{sphinxVerbatim}

The \sphinxcode{\sphinxupquote{Obj}} key specifies the object to be added as data to the specified element, given as an object name of the special form \{objname\} used to refer to Cos objects. This object must have been created previously and must be a dictionary or stream.

The \sphinxcode{\sphinxupquote{At}} key (see the table \sphinxhref{pdfmark\_Logical.html\#50454553\_65055}{Common element keys}) specifies the position of the new content within the containing element.

It is an error if the implicit parent stack is empty when \sphinxcode{\sphinxupquote{StOBJ}} is encountered.


\section{Attribute objects}
\label{\detokenize{pdfmark_Logical:attribute-objects}}
Elements can have additional information, or attributes, associated with them. Attributes are held in attribute objects, which can be associated with either a single element by using \sphinxcode{\sphinxupquote{StAttr}} (see \sphinxhref{pdfmark\_Logical.html\#50454553\_10527}{StAttr}), or with a group of objects by storing it in the \sphinxcode{\sphinxupquote{ClassMap}} of the Structure Tree Root, using \sphinxcode{\sphinxupquote{StClassMap}} (see \sphinxhref{pdfmark\_Logical.html\#50454553\_StClassMap}{StClassMap}).


\subsection{StAttr}
\label{\detokenize{pdfmark_Logical:stattr}}
\sphinxcode{\sphinxupquote{StAttr}} creates a new attribute object and adds it to the element on top of the implicit parent stack.

The syntax to create a new attribute object is as follows:

\begin{sphinxVerbatim}[commandchars=\\\{\}]
 \PYG{p}{[} \PYG{o}{/}\PYG{n}{Obj} \PYG{p}{\PYGZob{}}
\PYG{n}{objname}
\PYG{p}{\PYGZcb{}}

        \PYG{o}{/}\PYG{n}{StAttr} \PYG{n}{pdfmark}
\end{sphinxVerbatim}

The \sphinxcode{\sphinxupquote{Obj}} key specifies the object to be added as an attribute object to the specified element, given as an object name of the special form \{objname\} used to refer to Cos objects. This object must have been created previously and must be a dictionary or stream.

\begin{sphinxadmonition}{note}{Note:}
In the PDF file, the attribute object is stored in the \sphinxcode{\sphinxupquote{A}} key in the element’s dictionary.
\end{sphinxadmonition}

It is an error if the implicit parent stack is empty when \sphinxcode{\sphinxupquote{StAttr}} is encountered.


\section{Storage and retrieval of the implicit parent stack}
\label{\detokenize{pdfmark_Logical:storage-and-retrieval-of-the-implicit-parent-stack}}
Structure suite operators specify parents implicitly by means of the stack. However, it is not always possible to mimic a tree’s structure by nesting the structure within the document. For example, a paragraph may be represented by regions on more than one page, or it may be interrupted by other page content.

To allow applications flexibility in their page output while allowing them the convenience of specifying tree structure, the structure suite provides a way of storing and later retrieving the tree’s context.

See \sphinxhref{pdfmark\_Examples.html\#50454555\_16737}{Interrupted structure} for an example of storing and retrieving the implicit parent stack.

\begin{sphinxadmonition}{note}{Note:}
The names under which implicit parent stacks are stored and retrieved are in the current namespace governed by the stack operators \sphinxcode{\sphinxupquote{NamespacePush}} and \sphinxcode{\sphinxupquote{NamespacePop}} , defined in \sphinxhref{pdfmark\_Syntax.html\#50454537\_19819}{Namespaces}.
\end{sphinxadmonition}


\subsection{StStore}
\label{\detokenize{pdfmark_Logical:ststore}}
\sphinxcode{\sphinxupquote{StStore}} saves the current state of the implicit parent stack (without changing it). Its syntax is as follows:

\begin{sphinxVerbatim}[commandchars=\\\{\}]
 \PYG{p}{[} \PYG{o}{/}\PYG{n}{StoreName}
\PYG{n}{name}
        \PYG{o}{/}\PYG{n}{StStore} \PYG{n}{pdfmark}
\end{sphinxVerbatim}

The \sphinxcode{\sphinxupquote{StoreName}} key specifies a name object to be associated with the saved implicit parent stack state. Storing an implicit parent stack state under a previously used name completely replaces the implicit parent stack state already stored under that name.


\subsection{StRetrieve}
\label{\detokenize{pdfmark_Logical:stretrieve}}
\sphinxcode{\sphinxupquote{StRetrieve}} restores the implicit parent stack from a saved state, whose name is specified by the \sphinxcode{\sphinxupquote{StoreName}} key (as described in \sphinxhref{pdfmark\_Logical.html\#50454553\_60367}{StStore}). The syntax for a restoring the current state is as follows:

\begin{sphinxVerbatim}[commandchars=\\\{\}]
 \PYG{p}{[} \PYG{o}{/}\PYG{n}{StoreName}
\PYG{n}{name}
 \PYG{o}{/}\PYG{n}{StRetrieve} \PYG{n}{pdfmark}
\end{sphinxVerbatim}

The previous state of the implicit parent stack is overwritten by the restored state. It is an error to try to retrieve a nonexistent state, that is, to use a name that was not associated with a stack state by a previous \sphinxcode{\sphinxupquote{StStore}} .


\section{EPS considerations}
\label{\detokenize{pdfmark_Logical:eps-considerations}}
Encapsulated PostScript (EPS) is a form of PostScript used to embed graphics created in one application in a document created in another application. Applications can create EPS files containing structure elements without knowing anything about the environment into which the EPS file is to be embedded, which complicates the processing of a structure inside embedded EPS.

The logical structure design allows structure within an embedded EPS to be connected to the structure of the surrounding file by way of the implicit parent stack, while insulating the namespace of the containing file from accidents due to naming coincidences in embedded EPS files.

It is strongly recommended that applications embedding EPS files wrap the embedded PostScript between NamespacePush and NamespacePop to insulate the overall PostScript document from the consequences of multiply\sphinxhyphen{}defined object names.


\section{Tagged PDF}
\label{\detokenize{pdfmark_Logical:tagged-pdf}}
PDF 1.4 introduced the concept of \sphinxstyleemphasis{tagged PDF.} Tagged PDF is a type of structured PDF that allows page content to be extracted and reused for various purposes, such as reflow of text and graphics, conversion to various file formats such as HTML and XML, and accessibility to the visually impaired.

For detailed information on tagged PDF, see the \sphinxstyleemphasis{PDF Reference} .

In PDF 1.4, the Catalog dictionary contains a \sphinxcode{\sphinxupquote{MarkInfo}} entry whose value is a dictionary. That dictionary has a single key called \sphinxcode{\sphinxupquote{Marked}} whose value is a Boolean; a value of \sphinxcode{\sphinxupquote{true}} indicates that the document is a tagged PDF.

The syntax for indicating tagged PDF using pdfmark is as follows:

\begin{sphinxVerbatim}[commandchars=\\\{\}]
\PYG{p}{[}   \PYG{p}{\PYGZob{}}\PYG{n}{Catalog}\PYG{p}{\PYGZcb{}} \PYG{o}{\PYGZlt{}\PYGZlt{}}\PYG{o}{/}\PYG{n}{MarkInfo} \PYG{o}{\PYGZlt{}\PYGZlt{}}\PYG{o}{/}\PYG{n}{Marked} \PYG{n}{true} \PYG{o}{\PYGZgt{}\PYGZgt{}} \PYG{o}{\PYGZgt{}\PYGZgt{}} \PYG{o}{/}\PYG{n}{PUT} \PYG{n}{pdfmark}
\end{sphinxVerbatim}
\begin{enumerate}
\sphinxsetlistlabels{\arabic}{enumi}{enumii}{}{.}%
\item {} 
Tagged PDF

\end{enumerate}

This is a sample PostScript file that illustrates the use of tagged PDF.

Three items should be added to this example for completeness:
\begin{enumerate}
\sphinxsetlistlabels{\arabic}{enumi}{enumii}{}{.}%
\item {} 
A small table (just two rows, three column)

\item {} 
A figure (either standalone, or actually embedded in the text)

\item {} 
If possible, the encoding of a font so that the soft hyphen really works without the “actual text”

\end{enumerate}

\begin{sphinxVerbatim}[commandchars=\\\{\}]
\PYG{p}{[} \PYG{o}{/}\PYG{n}{Creator} \PYG{p}{(}\PYG{n}{Hand} \PYG{n}{Created}\PYG{p}{)}
    \PYG{o}{/}\PYG{n}{CreationDate} \PYG{p}{(}\PYG{n}{D}\PYG{p}{:}\PYG{l+m+mi}{20010508130548}\PYG{p}{)}
    \PYG{o}{/}\PYG{n}{ModDate} \PYG{p}{(}\PYG{n}{D}\PYG{p}{:}\PYG{l+m+mi}{20010508145339}\PYG{p}{)}
    \PYG{o}{/}\PYG{n}{Author} \PYG{p}{(}\PYG{n}{Adobe} \PYG{n}{Developer}\PYG{p}{)}
    \PYG{o}{/}\PYG{n}{Title} \PYG{p}{(}\PYG{n}{Sample} \PYG{n}{Document} \PYG{l+m+mi}{1} \PYG{k}{for} \PYG{n}{tagged} \PYG{n}{PDF} \PYG{n}{creation}\PYG{p}{)}
    \PYG{o}{/}\PYG{n}{Subject} \PYG{p}{(}\PYG{n}{A} \PYG{n}{base} \PYG{n}{document} \PYG{k}{for} \PYG{n}{the} \PYG{n}{creation} \PYG{n}{of} \PYG{n}{some} \PYG{n}{simple} \PYG{n}{PostScript} \PYG{o+ow}{and}

  \PYG{n}{PDFMarks} \PYG{n}{to} \PYG{n}{show} \PYG{n}{tagged} \PYG{n}{PDF}\PYG{p}{)}
    \PYG{o}{/}\PYG{n}{Session} \PYG{p}{(}\PYG{n}{Tagged} \PYG{n}{PDF} \PYG{n}{Dev} \PYG{n}{Tech} \PYG{n}{Seminar}\PYG{p}{)}
    \PYG{o}{/}\PYG{n}{Purpose} \PYG{p}{(}\PYG{n}{Demonstration}\PYG{p}{)}
    \PYG{o}{/}\PYG{n}{DOCINFO} \PYG{n}{pdfmark}

\PYG{p}{[}   \PYG{p}{\PYGZob{}}\PYG{n}{Catalog}\PYG{p}{\PYGZcb{}} \PYG{o}{\PYGZlt{}\PYGZlt{}}\PYG{o}{/}\PYG{n}{MarkInfo} \PYG{o}{\PYGZlt{}\PYGZlt{}}\PYG{o}{/}\PYG{n}{Marked} \PYG{n}{true}\PYG{o}{\PYGZgt{}\PYGZgt{}}\PYG{o}{\PYGZgt{}\PYGZgt{}} \PYG{o}{/}\PYG{n}{PUT} \PYG{n}{pdfmark}
\end{sphinxVerbatim}

Layout class for documenttitle below

\begin{sphinxVerbatim}[commandchars=\\\{\}]
\PYG{p}{[} \PYG{o}{/}\PYG{n}{\PYGZus{}objdef} \PYG{p}{\PYGZob{}}\PYG{n}{C1}\PYG{p}{\PYGZcb{}} \PYG{o}{/}\PYG{n+nb}{type} \PYG{o}{/}\PYG{n+nb}{dict} \PYG{o}{/}\PYG{n}{OBJ} \PYG{n}{pdfmark}
 \PYG{p}{[}   \PYG{p}{\PYGZob{}}\PYG{n}{C1}\PYG{p}{\PYGZcb{}} \PYG{o}{\PYGZlt{}\PYGZlt{}}\PYG{o}{/}\PYG{n}{O} \PYG{o}{/}\PYG{n}{Layout} \PYG{o}{/}\PYG{n}{SpaceAfter} \PYG{l+m+mi}{10} \PYG{o}{/}\PYG{n}{SpaceBefore} \PYG{l+m+mi}{10} \PYG{o}{/}\PYG{n}{TextAlign} \PYG{o}{/}\PYG{n}{Center}\PYG{o}{\PYGZgt{}\PYGZgt{}}
 \PYG{o}{/}   \PYG{n}{PUT} \PYG{n}{pdfmark}
 \PYG{p}{[} \PYG{o}{/}\PYG{n}{CM1} \PYG{p}{\PYGZob{}}\PYG{n}{C1}\PYG{p}{\PYGZcb{}} \PYG{o}{/}\PYG{n}{StClassMap} \PYG{n}{pdfmark}
\end{sphinxVerbatim}

Layout class for topichead

\begin{sphinxVerbatim}[commandchars=\\\{\}]
\PYG{p}{[} \PYG{o}{/}\PYG{n}{\PYGZus{}objdef} \PYG{p}{\PYGZob{}}\PYG{n}{C2}\PYG{p}{\PYGZcb{}} \PYG{o}{/}\PYG{n+nb}{type} \PYG{o}{/}\PYG{n+nb}{dict} \PYG{o}{/}\PYG{n}{OBJ} \PYG{n}{pdfmark}
 \PYG{p}{[}   \PYG{p}{\PYGZob{}}\PYG{n}{C2}\PYG{p}{\PYGZcb{}} \PYG{o}{\PYGZlt{}\PYGZlt{}}\PYG{o}{/}\PYG{n}{O} \PYG{o}{/}\PYG{n}{Layout} \PYG{o}{/}\PYG{n}{SpaceAfter} \PYG{l+m+mi}{5} \PYG{o}{/}\PYG{n}{SpaceBefore} \PYG{l+m+mi}{5} \PYG{o}{/}\PYG{n}{TextAlign} \PYG{o}{/}\PYG{n}{Left}\PYG{o}{\PYGZgt{}\PYGZgt{}}
     \PYG{o}{/}\PYG{n}{PUT} \PYG{n}{pdfmark}
 \PYG{p}{[} \PYG{o}{/}\PYG{n}{CM2} \PYG{p}{\PYGZob{}}\PYG{n}{C2}\PYG{p}{\PYGZcb{}} \PYG{o}{/}\PYG{n}{StClassMap} \PYG{n}{pdfmark}
\end{sphinxVerbatim}

Layout class for topichead2

\begin{sphinxVerbatim}[commandchars=\\\{\}]
\PYG{p}{[} \PYG{o}{/}\PYG{n}{\PYGZus{}objdef} \PYG{p}{\PYGZob{}}\PYG{n}{C3}\PYG{p}{\PYGZcb{}} \PYG{o}{/}\PYG{n+nb}{type} \PYG{o}{/}\PYG{n+nb}{dict} \PYG{o}{/}\PYG{n}{OBJ} \PYG{n}{pdfmark}
 \PYG{p}{[}   \PYG{p}{\PYGZob{}}\PYG{n}{C3}\PYG{p}{\PYGZcb{}} \PYG{o}{\PYGZlt{}\PYGZlt{}}\PYG{o}{/}\PYG{n}{O} \PYG{o}{/}\PYG{n}{Layout} \PYG{o}{/}\PYG{n}{SpaceAfter} \PYG{l+m+mi}{3} \PYG{o}{/}\PYG{n}{SpaceBefore} \PYG{l+m+mi}{3} \PYG{o}{/}\PYG{n}{TextAlign} \PYG{o}{/}\PYG{n}{Left}\PYG{o}{\PYGZgt{}\PYGZgt{}}
     \PYG{o}{/}\PYG{n}{PUT} \PYG{n}{pdfmark}
 \PYG{p}{[} \PYG{o}{/}\PYG{n}{CM3} \PYG{p}{\PYGZob{}}\PYG{n}{C3}\PYG{p}{\PYGZcb{}} \PYG{o}{/}\PYG{n}{StClassMap} \PYG{n}{pdfmark}
\end{sphinxVerbatim}

Layout class for p

\begin{sphinxVerbatim}[commandchars=\\\{\}]
\PYG{p}{[} \PYG{o}{/}\PYG{n}{\PYGZus{}objdef} \PYG{p}{\PYGZob{}}\PYG{n}{C4}\PYG{p}{\PYGZcb{}} \PYG{o}{/}\PYG{n+nb}{type} \PYG{o}{/}\PYG{n+nb}{dict} \PYG{o}{/}\PYG{n}{OBJ} \PYG{n}{pdfmark}
 \PYG{p}{[}   \PYG{p}{\PYGZob{}}\PYG{n}{C4}\PYG{p}{\PYGZcb{}} \PYG{o}{\PYGZlt{}\PYGZlt{}}\PYG{o}{/}\PYG{n}{O} \PYG{o}{/}\PYG{n}{Layout} \PYG{o}{/}\PYG{n}{SpaceAfter} \PYG{l+m+mi}{1} \PYG{o}{/}\PYG{n}{SpaceBefore} \PYG{l+m+mi}{3} \PYG{o}{/}\PYG{n}{TextAlign} \PYG{o}{/}\PYG{n}{Left}\PYG{o}{\PYGZgt{}\PYGZgt{}}
     \PYG{o}{/}\PYG{n}{PUT} \PYG{n}{pdfmark}
 \PYG{p}{[} \PYG{o}{/}\PYG{n}{CM4} \PYG{p}{\PYGZob{}}\PYG{n}{C4}\PYG{p}{\PYGZcb{}} \PYG{o}{/}\PYG{n}{StClassMap} \PYG{n}{pdfmark}

 \PYG{p}{[} \PYG{o}{/}\PYG{n}{Subtype} \PYG{o}{/}\PYG{n}{document} \PYG{o}{/}\PYG{n}{Lang} \PYG{p}{(}\PYG{n}{en}\PYG{o}{\PYGZhy{}}\PYG{n}{US}\PYG{p}{)} \PYG{o}{/}\PYG{n}{StPNE} \PYG{n}{pdfmark}

 \PYG{p}{[} \PYG{o}{/}\PYG{n}{\PYGZus{}objdef} \PYG{p}{\PYGZob{}}\PYG{n}{dta1}\PYG{p}{\PYGZcb{}} \PYG{o}{/}\PYG{n+nb}{type} \PYG{o}{/}\PYG{n+nb}{dict} \PYG{o}{/}\PYG{n}{OBJ} \PYG{n}{pdfmark}

 \PYG{p}{[}   \PYG{p}{\PYGZob{}}\PYG{n}{dta1}\PYG{p}{\PYGZcb{}} \PYG{o}{\PYGZlt{}\PYGZlt{}}\PYG{o}{/}\PYG{n}{O} \PYG{o}{/}\PYG{n}{XML}\PYG{o}{\PYGZhy{}}\PYG{l+m+mf}{1.00} \PYG{o}{/}\PYG{n}{Author} \PYG{p}{(}\PYG{n}{Joe}\PYG{p}{)}\PYG{o}{\PYGZgt{}\PYGZgt{}} \PYG{o}{/}\PYG{n}{PUT} \PYG{n}{pdfmark}
 \PYG{p}{[} \PYG{o}{/}\PYG{n}{Subtype} \PYG{o}{/}\PYG{n}{documenttitle} \PYG{o}{/}\PYG{n}{Class} \PYG{o}{/}\PYG{n}{CM1} \PYG{o}{/}\PYG{n}{StPNE} \PYG{n}{pdfmark}
 \PYG{p}{[} \PYG{o}{/}\PYG{n}{Obj} \PYG{p}{\PYGZob{}}\PYG{n}{dta1}\PYG{p}{\PYGZcb{}} \PYG{o}{/}\PYG{n}{StAttr} \PYG{n}{pdfmark}

 \PYG{p}{[} \PYG{o}{/}\PYG{n}{StBMC} \PYG{n}{pdfmark}

 \PYG{o}{/}\PYG{n}{Helvetica}\PYG{o}{\PYGZhy{}}\PYG{n}{Bold} \PYG{n}{findfont} \PYG{l+m+mi}{24} \PYG{n}{scalefont} \PYG{n}{setfont}
 \PYG{l+m+mi}{216} \PYG{l+m+mi}{720} \PYG{n}{moveto}
 \PYG{p}{(}\PYG{n}{Title} \PYG{n}{of} \PYG{n}{Document}\PYG{p}{)} \PYG{n}{show}

 \PYG{p}{[} \PYG{o}{/}\PYG{n}{EMC} \PYG{n}{pdfmark}
 \PYG{p}{[} \PYG{o}{/}\PYG{n}{StPop} \PYG{n}{pdfmark}

 \PYG{p}{[} \PYG{o}{/}\PYG{n}{Subtype} \PYG{o}{/}\PYG{n}{topic} \PYG{o}{/}\PYG{n}{StPNE} \PYG{n}{pdfmark}
 \PYG{p}{[} \PYG{o}{/}\PYG{n}{Subtype} \PYG{o}{/}\PYG{n}{topichead} \PYG{o}{/}\PYG{n}{Class} \PYG{o}{/}\PYG{n}{CM2} \PYG{o}{/}\PYG{n}{StPNE} \PYG{n}{pdfmark}
 \PYG{p}{[} \PYG{o}{/}\PYG{n}{StBMC} \PYG{n}{pdfmark}

 \PYG{o}{/}\PYG{n}{Helvetica}\PYG{o}{\PYGZhy{}}\PYG{n}{Bold} \PYG{n}{findfont} \PYG{l+m+mi}{18} \PYG{n}{scalefont} \PYG{n}{setfont}
 \PYG{l+m+mi}{72} \PYG{l+m+mi}{690} \PYG{n}{moveto}
 \PYG{p}{(}\PYG{n}{First} \PYG{n}{Topic}\PYG{p}{)} \PYG{n}{show}

 \PYG{p}{[} \PYG{o}{/}\PYG{n}{EMC} \PYG{n}{pdfmark}
 \PYG{p}{[} \PYG{o}{/}\PYG{n}{StPop} \PYG{n}{pdfmark}

 \PYG{p}{[} \PYG{o}{/}\PYG{n}{Subtype} \PYG{o}{/}\PYG{n}{p} \PYG{o}{/}\PYG{n}{Class} \PYG{o}{/}\PYG{n}{CM4} \PYG{o}{/}\PYG{n}{StPNE} \PYG{n}{pdfmark}
 \PYG{p}{[} \PYG{o}{/}\PYG{n}{StBMC} \PYG{n}{pdfmark}

 \PYG{o}{/}\PYG{n}{Helvetica} \PYG{n}{findfont} \PYG{l+m+mi}{12} \PYG{n}{scalefont} \PYG{n}{setfont}
 \PYG{l+m+mi}{72} \PYG{l+m+mi}{674} \PYG{n}{moveto}
 \PYG{p}{(}\PYG{n}{Some} \PYG{n}{text} \PYG{o+ow}{in} \PYG{n}{a} \PYG{n}{paragraph} \PYG{o+ow}{in} \PYG{n}{the} \PYG{n}{first} \PYG{n}{topic}\PYG{o}{.} \PYG{n}{These} \PYG{n}{lines} \PYG{n}{may} \PYG{o+ow}{not} \PYG{n}{be}

   \PYG{n}{justified}\PYG{p}{,} \PYG{n}{but} \PYG{n}{are} \PYG{n}{illustrative}\PYG{o}{.}\PYG{p}{)} \PYG{n}{show}

 \PYG{p}{[} \PYG{o}{/}\PYG{n}{EMC} \PYG{n}{pdfmark}
 \PYG{p}{[} \PYG{o}{/}\PYG{n}{StPop} \PYG{n}{pdfmark}
 \PYG{p}{[} \PYG{o}{/}\PYG{n}{StPop} \PYG{n}{pdfmark}

 \PYG{p}{[} \PYG{o}{/}\PYG{n}{Subtype} \PYG{o}{/}\PYG{n}{topic} \PYG{o}{/}\PYG{n}{StPNE} \PYG{n}{pdfmark}
 \PYG{p}{[} \PYG{o}{/}\PYG{n}{Subtype} \PYG{o}{/}\PYG{n}{topichead} \PYG{o}{/}\PYG{n}{Class} \PYG{o}{/}\PYG{n}{CM2} \PYG{o}{/}\PYG{n}{StPNE} \PYG{n}{pdfmark}
 \PYG{p}{[} \PYG{o}{/}\PYG{n}{StBMC} \PYG{n}{pdfmark}

 \PYG{o}{/}\PYG{n}{Helvetica}\PYG{o}{\PYGZhy{}}\PYG{n}{Bold} \PYG{n}{findfont} \PYG{l+m+mi}{18} \PYG{n}{scalefont} \PYG{n}{setfont}
 \PYG{l+m+mi}{72} \PYG{l+m+mi}{648} \PYG{n}{moveto}
 \PYG{p}{(}\PYG{n}{Second} \PYG{n}{Topic}\PYG{p}{)} \PYG{n}{show}

 \PYG{p}{[} \PYG{o}{/}\PYG{n}{EMC} \PYG{n}{pdfmark}
 \PYG{p}{[} \PYG{o}{/}\PYG{n}{StPop} \PYG{n}{pdfmark}
 \PYG{p}{[} \PYG{o}{/}\PYG{n}{Subtype} \PYG{o}{/}\PYG{n}{p} \PYG{o}{/}\PYG{n}{Class} \PYG{o}{/}\PYG{n}{CM4} \PYG{o}{/}\PYG{n}{StPNE} \PYG{n}{pdfmark}
 \PYG{p}{[} \PYG{o}{/}\PYG{n}{StBMC} \PYG{n}{pdfmark}

 \PYG{o}{/}\PYG{n}{Helvetica} \PYG{n}{findfont} \PYG{l+m+mi}{12} \PYG{n}{scalefont} \PYG{n}{setfont}
 \PYG{l+m+mi}{72} \PYG{l+m+mi}{632} \PYG{n}{moveto}
 \PYG{p}{(}\PYG{n}{This} \PYG{o+ow}{is} \PYG{n}{a} \PYG{n}{paragraph} \PYG{n}{of} \PYG{n}{text} \PYG{o+ow}{in} \PYG{n}{the} \PYG{n}{second} \PYG{n}{topic}\PYG{o}{.} \PYG{p}{)} \PYG{n}{show}

 \PYG{p}{[} \PYG{o}{/}\PYG{n}{EMC} \PYG{n}{pdfmark}
 \PYG{p}{[} \PYG{o}{/}\PYG{n}{Subtype} \PYG{o}{/}\PYG{n}{emph} \PYG{o}{/}\PYG{n}{StPNE} \PYG{n}{pdfmark}
 \PYG{p}{[} \PYG{o}{/}\PYG{n}{StBMC} \PYG{n}{pdfmark}

 \PYG{o}{/}\PYG{n}{Helvetica}\PYG{o}{\PYGZhy{}}\PYG{n}{Oblique} \PYG{n}{findfont} \PYG{l+m+mi}{12} \PYG{n}{scalefont} \PYG{n}{setfont}
 \PYG{p}{(}\PYG{n}{Emphasized} \PYG{p}{)} \PYG{n}{show}

 \PYG{p}{[} \PYG{o}{/}\PYG{n}{EMC} \PYG{n}{pdfmark}
 \PYG{p}{[} \PYG{o}{/}\PYG{n}{StPop} \PYG{n}{pdfmark}
 \PYG{p}{[} \PYG{o}{/}\PYG{n}{StBMC} \PYG{n}{pdfmark}

 \PYG{o}{/}\PYG{n}{Helvetica} \PYG{n}{findfont} \PYG{l+m+mi}{12} \PYG{n}{scalefont} \PYG{n}{setfont}
 \PYG{p}{(}\PYG{n}{words} \PYG{p}{)} \PYG{n}{show}

 \PYG{l+m+mi}{72} \PYG{l+m+mi}{618} \PYG{n}{moveto}
 \PYG{p}{(}\PYG{n}{here}\PYG{o}{.}\PYG{p}{)} \PYG{n}{show}

 \PYG{p}{[} \PYG{o}{/}\PYG{n}{EMC} \PYG{n}{pdfmark}
 \PYG{p}{[} \PYG{o}{/}\PYG{n}{StPop} \PYG{n}{pdfmark}

 \PYG{p}{[} \PYG{o}{/}\PYG{n}{Subtype} \PYG{o}{/}\PYG{n}{topic} \PYG{o}{/}\PYG{n}{StPNE} \PYG{n}{pdfmark}
 \PYG{p}{[} \PYG{o}{/}\PYG{n}{Subtype} \PYG{o}{/}\PYG{n}{topichead2} \PYG{o}{/}\PYG{n}{Class} \PYG{o}{/}\PYG{n}{CM3} \PYG{o}{/}\PYG{n}{StPNE} \PYG{n}{pdfmark}

 \PYG{p}{[} \PYG{o}{/}\PYG{n}{StBMC} \PYG{n}{pdfmark}

 \PYG{o}{/}\PYG{n}{Helvetica}\PYG{o}{\PYGZhy{}}\PYG{n}{Bold} \PYG{n}{findfont} \PYG{l+m+mi}{14} \PYG{n}{scalefont} \PYG{n}{setfont}
 \PYG{l+m+mi}{72} \PYG{l+m+mi}{596} \PYG{n}{moveto}
 \PYG{p}{(}\PYG{n}{Subtopic} \PYG{n}{of} \PYG{n}{second} \PYG{n}{topic}\PYG{p}{)} \PYG{n}{show}

 \PYG{p}{[} \PYG{o}{/}\PYG{n}{EMC} \PYG{n}{pdfmark}
 \PYG{p}{[} \PYG{o}{/}\PYG{n}{StPop} \PYG{n}{pdfmark}

 \PYG{p}{[} \PYG{o}{/}\PYG{n}{Subtype} \PYG{o}{/}\PYG{n}{p} \PYG{o}{/}\PYG{n}{Class} \PYG{o}{/}\PYG{n}{CM4} \PYG{o}{/}\PYG{n}{StPNE} \PYG{n}{pdfmark}
 \PYG{p}{[} \PYG{o}{/}\PYG{n}{StBMC} \PYG{n}{pdfmark}

 \PYG{o}{/}\PYG{n}{Helvetica} \PYG{n}{findfont} \PYG{l+m+mi}{12} \PYG{n}{scalefont} \PYG{n}{setfont}
 \PYG{l+m+mi}{72} \PYG{l+m+mi}{580} \PYG{n}{moveto}
 \PYG{p}{(}\PYG{n}{This} \PYG{n}{paragraph} \PYG{n}{of} \PYG{n}{text} \PYG{o+ow}{is} \PYG{n}{the} \PYG{n}{second} \PYG{n}{topic}\PYG{p}{,} \PYG{n}{first} \PYG{n}{subtopic}\PYG{o}{.} \PYG{p}{)} \PYG{n}{show}
 \PYG{l+m+mi}{72} \PYG{l+m+mi}{566} \PYG{n}{moveto}
 \PYG{p}{(}\PYG{n}{Hyphenated} \PYG{n}{words} \PYG{n}{make} \PYG{n}{up} \PYG{n}{this} \PYG{n}{para}\PYG{p}{)} \PYG{n}{show}

 \PYG{p}{[} \PYG{o}{/}\PYG{n}{EMC} \PYG{n}{pdfmark}
 \PYG{p}{[} \PYG{o}{/}\PYG{n}{Subtype} \PYG{o}{/}\PYG{n}{Span} \PYG{o}{/}\PYG{n}{ActualText} \PYG{o}{\PYGZlt{}}\PYG{n}{FEFF00AD}\PYG{o}{\PYGZgt{}} \PYG{o}{/}\PYG{n}{StPNE} \PYG{n}{pdfmark}
 \PYG{p}{[} \PYG{o}{/}\PYG{n}{StBMC} \PYG{n}{pdfmark}

 \PYG{p}{(}\PYG{o}{\PYGZhy{}}\PYG{p}{)} \PYG{n}{show}
 \PYG{p}{[} \PYG{o}{/}\PYG{n}{EMC} \PYG{n}{pdfmark}
 \PYG{p}{[} \PYG{o}{/}\PYG{n}{StPop} \PYG{n}{pdfmark}
 \PYG{p}{[} \PYG{o}{/}\PYG{n}{StBMC} \PYG{n}{pdfmark}

 \PYG{l+m+mi}{72} \PYG{l+m+mi}{552} \PYG{n}{moveto}
 \PYG{p}{(}\PYG{n}{graph} \PYG{n}{also}\PYG{o}{.}\PYG{p}{)} \PYG{n}{show}

 \PYG{p}{[} \PYG{o}{/}\PYG{n}{EMC} \PYG{n}{pdfmark}
 \PYG{p}{[} \PYG{o}{/}\PYG{n}{StPop} \PYG{n}{pdfmark}
 \PYG{p}{[} \PYG{o}{/}\PYG{n}{StPop} \PYG{n}{pdfmark}
 \PYG{p}{[} \PYG{o}{/}\PYG{n}{StPop} \PYG{n}{pdfmark}
\end{sphinxVerbatim}

Add another topic with line numbers

\begin{sphinxVerbatim}[commandchars=\\\{\}]
 \PYG{p}{[} \PYG{o}{/}\PYG{n}{Subtype} \PYG{o}{/}\PYG{n}{topic} \PYG{o}{/}\PYG{n}{StPNE} \PYG{n}{pdfmark}
 \PYG{p}{[} \PYG{o}{/}\PYG{n}{Subtype} \PYG{o}{/}\PYG{n}{topichead} \PYG{o}{/}\PYG{n}{Class} \PYG{o}{/}\PYG{n}{CM2} \PYG{o}{/}\PYG{n}{StPNE} \PYG{n}{pdfmark}
 \PYG{p}{[} \PYG{o}{/}\PYG{n}{StBMC} \PYG{n}{pdfmark}

 \PYG{o}{/}\PYG{n}{Helvetica}\PYG{o}{\PYGZhy{}}\PYG{n}{Bold} \PYG{n}{findfont} \PYG{l+m+mi}{18} \PYG{n}{scalefont} \PYG{n}{setfont}
 \PYG{l+m+mi}{72} \PYG{l+m+mi}{510} \PYG{n}{moveto}
 \PYG{p}{(}\PYG{n}{Line} \PYG{n}{Numbered} \PYG{n}{Topic}\PYG{p}{)} \PYG{n}{show}

 \PYG{p}{[} \PYG{o}{/}\PYG{n}{EMC} \PYG{n}{pdfmark}
 \PYG{p}{[} \PYG{o}{/}\PYG{n}{StPop} \PYG{n}{pdfmark}

 \PYG{p}{[} \PYG{o}{/}\PYG{n}{Subtype} \PYG{o}{/}\PYG{n}{p} \PYG{o}{/}\PYG{n}{Class} \PYG{o}{/}\PYG{n}{CM4} \PYG{o}{/}\PYG{n}{StPNE} \PYG{n}{pdfmark}

 \PYG{o}{/}\PYG{n}{Helvetica} \PYG{n}{findfont} \PYG{l+m+mi}{12} \PYG{n}{scalefont} \PYG{n}{setfont}
 \PYG{p}{[} \PYG{o}{/}\PYG{n}{Artifact} \PYG{o}{\PYGZlt{}\PYGZlt{}}\PYG{o}{/}\PYG{n}{Type} \PYG{o}{/}\PYG{n}{Layout}\PYG{o}{\PYGZgt{}\PYGZgt{}} \PYG{o}{/}\PYG{n}{BDC} \PYG{n}{pdfmark}
 \PYG{l+m+mi}{48} \PYG{l+m+mi}{494} \PYG{n}{moveto} \PYG{p}{(}\PYG{l+m+mi}{1}\PYG{p}{)} \PYG{n}{show}

 \PYG{p}{[} \PYG{o}{/}\PYG{n}{EMC} \PYG{n}{pdfmark}
 \PYG{p}{[} \PYG{o}{/}\PYG{n}{StBMC} \PYG{n}{pdfmark}

 \PYG{l+m+mi}{72} \PYG{l+m+mi}{494} \PYG{n}{moveto}
 \PYG{p}{(}\PYG{n}{This} \PYG{o+ow}{is} \PYG{n}{some} \PYG{n}{text} \PYG{n}{such} \PYG{k}{as} \PYG{n}{would} \PYG{n}{appear} \PYG{o+ow}{in} \PYG{n}{a} \PYG{n}{legal} \PYG{n}{bill}\PYG{o}{.} \PYG{p}{)} \PYG{n}{show}

 \PYG{p}{[} \PYG{o}{/}\PYG{n}{EMC} \PYG{n}{pdfmark}
 \PYG{p}{[} \PYG{o}{/}\PYG{n}{Artifact} \PYG{o}{\PYGZlt{}\PYGZlt{}}\PYG{o}{/}\PYG{n}{Type} \PYG{o}{/}\PYG{n}{Layout}\PYG{o}{\PYGZgt{}\PYGZgt{}} \PYG{o}{/}\PYG{n}{BDC} \PYG{n}{pdfmark}

 \PYG{l+m+mi}{48} \PYG{l+m+mi}{478} \PYG{n}{moveto} \PYG{p}{(}\PYG{l+m+mi}{2}\PYG{p}{)} \PYG{n}{show}

 \PYG{p}{[} \PYG{o}{/}\PYG{n}{EMC} \PYG{n}{pdfmark}
 \PYG{p}{[} \PYG{o}{/}\PYG{n}{StBMC} \PYG{n}{pdfmark}

 \PYG{l+m+mi}{72} \PYG{l+m+mi}{478} \PYG{n}{moveto}
 \PYG{p}{(}\PYG{n}{Note} \PYG{n}{that} \PYG{n}{this} \PYG{n}{text} \PYG{n}{has} \PYG{n}{line} \PYG{n}{numbers}\PYG{p}{,} \PYG{n}{but} \PYG{n}{that} \PYG{p}{)} \PYG{n}{show}

 \PYG{p}{[} \PYG{o}{/}\PYG{n}{EMC} \PYG{n}{pdfmark}
 \PYG{p}{[} \PYG{o}{/}\PYG{n}{Artifact} \PYG{o}{\PYGZlt{}\PYGZlt{}}\PYG{o}{/}\PYG{n}{Type} \PYG{o}{/}\PYG{n}{Layout}\PYG{o}{\PYGZgt{}\PYGZgt{}} \PYG{o}{/}\PYG{n}{BDC} \PYG{n}{pdfmark}

 \PYG{l+m+mi}{48} \PYG{l+m+mi}{464} \PYG{n}{moveto} \PYG{p}{(}\PYG{l+m+mi}{3}\PYG{p}{)} \PYG{n}{show}

 \PYG{p}{[} \PYG{o}{/}\PYG{n}{EMC} \PYG{n}{pdfmark}
 \PYG{p}{[} \PYG{o}{/}\PYG{n}{StBMC} \PYG{n}{pdfmark}

 \PYG{l+m+mi}{72} \PYG{l+m+mi}{464} \PYG{n}{moveto}
 \PYG{p}{(}\PYG{n}{the} \PYG{n}{numbers} \PYG{n}{disappear} \PYG{n}{when} \PYG{n}{you} \PYG{n}{reflow} \PYG{p}{)} \PYG{n}{show}

 \PYG{p}{[} \PYG{o}{/}\PYG{n}{EMC} \PYG{n}{pdfmark}
 \PYG{p}{[} \PYG{o}{/}\PYG{n}{Artifact} \PYG{o}{\PYGZlt{}\PYGZlt{}}\PYG{o}{/}\PYG{n}{Type} \PYG{o}{/}\PYG{n}{Layout}\PYG{o}{\PYGZgt{}\PYGZgt{}} \PYG{o}{/}\PYG{n}{BDC} \PYG{n}{pdfmark}

 \PYG{l+m+mi}{48} \PYG{l+m+mi}{450} \PYG{n}{moveto} \PYG{p}{(}\PYG{l+m+mi}{4}\PYG{p}{)} \PYG{n}{show}

 \PYG{p}{[} \PYG{o}{/}\PYG{n}{EMC} \PYG{n}{pdfmark}

 \PYG{p}{[} \PYG{o}{/}\PYG{n}{StBMC} \PYG{n}{pdfmark}

 \PYG{l+m+mi}{72} \PYG{l+m+mi}{450} \PYG{n}{moveto}
 \PYG{p}{(}\PYG{n}{the} \PYG{n}{text} \PYG{o+ow}{or} \PYG{n}{save} \PYG{n}{the} \PYG{n}{text} \PYG{k}{as} \PYG{n}{XML}\PYG{o}{.}\PYG{p}{)} \PYG{n}{show}

 \PYG{p}{[} \PYG{o}{/}\PYG{n}{EMC} \PYG{n}{pdfmark}
 \PYG{p}{[} \PYG{o}{/}\PYG{n}{StPop} \PYG{n}{pdfmark}
 \PYG{p}{[} \PYG{o}{/}\PYG{n}{StPop} \PYG{n}{pdfmark}


\PYG{o}{\PYGZpc{}} \PYG{o}{==}\PYG{o}{==}\PYG{o}{==}\PYG{o}{==}\PYG{o}{==}\PYG{o}{==}\PYG{o}{==}\PYG{o}{==}\PYG{o}{==}\PYG{o}{==}\PYG{o}{==}\PYG{o}{==}\PYG{o}{==}\PYG{o}{==}\PYG{o}{==}\PYG{o}{==}\PYG{o}{==}\PYG{o}{==}\PYG{o}{==}\PYG{o}{==}\PYG{o}{==}\PYG{o}{==}\PYG{o}{==}\PYG{o}{==}\PYG{o}{==}\PYG{o}{==}\PYG{o}{==}\PYG{o}{==}\PYG{o}{==}\PYG{o}{==}\PYG{o}{==}\PYG{o}{==}\PYG{o}{=}
\PYG{o}{\PYGZpc{}} \PYG{n}{Create} \PYG{n}{a} \PYG{n}{simple} \PYG{n}{link} \PYG{n}{example}
\PYG{o}{\PYGZpc{}} \PYG{o}{==}\PYG{o}{==}\PYG{o}{==}\PYG{o}{==}\PYG{o}{==}\PYG{o}{==}\PYG{o}{==}\PYG{o}{==}\PYG{o}{==}\PYG{o}{==}\PYG{o}{==}\PYG{o}{==}\PYG{o}{==}\PYG{o}{==}\PYG{o}{==}\PYG{o}{==}\PYG{o}{==}\PYG{o}{==}\PYG{o}{==}\PYG{o}{==}\PYG{o}{==}\PYG{o}{==}\PYG{o}{==}\PYG{o}{==}\PYG{o}{==}\PYG{o}{==}\PYG{o}{==}\PYG{o}{==}\PYG{o}{==}\PYG{o}{==}\PYG{o}{==}\PYG{o}{==}\PYG{o}{=}

 \PYG{p}{[} \PYG{o}{/}\PYG{n}{Subtype} \PYG{o}{/}\PYG{n}{P} \PYG{o}{/}\PYG{n}{StPNE} \PYG{n}{pdfmark}
 \PYG{p}{[} \PYG{o}{/}\PYG{n}{Subtype} \PYG{o}{/}\PYG{n}{Link} \PYG{o}{/}\PYG{n}{StPNE} \PYG{n}{pdfmark}

 \PYG{p}{[} \PYG{o}{/}\PYG{n}{\PYGZus{}objdef} \PYG{p}{\PYGZob{}}\PYG{n}{annotObj}\PYG{p}{\PYGZcb{}} \PYG{o}{/}\PYG{n}{Rect} \PYG{p}{[}\PYG{l+m+mi}{70} \PYG{l+m+mi}{398} \PYG{l+m+mi}{202} \PYG{l+m+mi}{412}\PYG{p}{]}
     \PYG{o}{/}\PYG{n}{Action} \PYG{o}{\PYGZlt{}\PYGZlt{}} \PYG{o}{/}\PYG{n}{Subtype} \PYG{o}{/}\PYG{n}{URI} \PYG{o}{/}\PYG{n}{URI} \PYG{p}{(}\PYG{n}{http}\PYG{p}{:}\PYG{o}{/}\PYG{o}{/}\PYG{n}{www}\PYG{o}{.}\PYG{n}{adobe}\PYG{o}{.}\PYG{n}{com}\PYG{p}{)} \PYG{o}{\PYGZgt{}\PYGZgt{}}
     \PYG{o}{/}\PYG{n}{Border} \PYG{p}{[}\PYG{l+m+mi}{0} \PYG{l+m+mi}{0} \PYG{l+m+mi}{0}\PYG{p}{]}
     \PYG{o}{/}\PYG{n}{Subtype} \PYG{o}{/}\PYG{n}{Link}
     \PYG{o}{/}\PYG{n}{ANN} \PYG{n}{pdfmark}

 \PYG{p}{[} \PYG{o}{/}\PYG{n}{Obj} \PYG{p}{\PYGZob{}}\PYG{n}{annotObj}\PYG{p}{\PYGZcb{}} \PYG{o}{/}\PYG{n}{StOBJ} \PYG{n}{pdfmark}

 \PYG{p}{[} \PYG{o}{/}\PYG{n}{StBMC} \PYG{n}{pdfmark}
     \PYG{l+m+mi}{0} \PYG{l+m+mi}{0} \PYG{l+m+mi}{1} \PYG{n}{setrgbcolor}
     \PYG{l+m+mi}{72} \PYG{l+m+mi}{400} \PYG{n}{moveto}

 \PYG{p}{(}\PYG{n}{http}\PYG{p}{:}\PYG{o}{/}\PYG{o}{/}\PYG{n}{www}\PYG{o}{.}\PYG{n}{adobe}\PYG{o}{.}\PYG{n}{com}\PYG{o}{.}\PYG{p}{)} \PYG{n}{show}

 \PYG{p}{[} \PYG{o}{/}\PYG{n}{EMC} \PYG{n}{pdfmark}

 \PYG{p}{[} \PYG{o}{/}\PYG{n}{StPop} \PYG{n}{pdfmark}
 \PYG{p}{[} \PYG{o}{/}\PYG{n}{StPop} \PYG{n}{pdfmark}


\PYG{o}{\PYGZpc{}} \PYG{n}{Set} \PYG{n}{the} \PYG{n}{tab} \PYG{n}{order} \PYG{k}{for} \PYG{n}{the} \PYG{n}{page} \PYG{n}{to} \PYG{n}{structure} \PYG{n}{order}\PYG{o}{.}

\PYG{p}{[}   \PYG{p}{\PYGZob{}}\PYG{n}{ThisPage}\PYG{p}{\PYGZcb{}} \PYG{o}{\PYGZlt{}\PYGZlt{}} \PYG{o}{/}\PYG{n}{Tabs} \PYG{o}{/}\PYG{n}{S} \PYG{o}{\PYGZgt{}\PYGZgt{}} \PYG{o}{/}\PYG{n}{PUT} \PYG{n}{pdfmark}

\PYG{o}{\PYGZpc{}} \PYG{o}{==}\PYG{o}{==}\PYG{o}{==}\PYG{o}{==}\PYG{o}{==}\PYG{o}{==}\PYG{o}{==}\PYG{o}{==}\PYG{o}{==}\PYG{o}{==}\PYG{o}{==}\PYG{o}{==}\PYG{o}{==}\PYG{o}{==}\PYG{o}{==}\PYG{o}{==}\PYG{o}{==}\PYG{o}{==}\PYG{o}{==}\PYG{o}{==}\PYG{o}{==}\PYG{o}{==}\PYG{o}{==}\PYG{o}{==}\PYG{o}{==}\PYG{o}{==}\PYG{o}{==}\PYG{o}{==}\PYG{o}{==}\PYG{o}{==}\PYG{o}{==}\PYG{o}{==}\PYG{o}{=}
\PYG{o}{\PYGZpc{}} \PYG{n}{Create} \PYG{n}{figure} \PYG{k}{with} \PYG{n}{a} \PYG{n}{bounding} \PYG{n}{box}
\PYG{o}{\PYGZpc{}} \PYG{o}{==}\PYG{o}{==}\PYG{o}{==}\PYG{o}{==}\PYG{o}{==}\PYG{o}{==}\PYG{o}{==}\PYG{o}{==}\PYG{o}{==}\PYG{o}{==}\PYG{o}{==}\PYG{o}{==}\PYG{o}{==}\PYG{o}{==}\PYG{o}{==}\PYG{o}{==}\PYG{o}{==}\PYG{o}{==}\PYG{o}{==}\PYG{o}{==}\PYG{o}{==}\PYG{o}{==}\PYG{o}{==}\PYG{o}{==}\PYG{o}{==}\PYG{o}{==}\PYG{o}{==}\PYG{o}{==}\PYG{o}{==}\PYG{o}{==}\PYG{o}{==}\PYG{o}{==}\PYG{o}{=}

 \PYG{p}{[} \PYG{o}{/}\PYG{n}{Subtype} \PYG{o}{/}\PYG{n}{Figure} \PYG{o}{/}\PYG{n}{Alt} \PYG{p}{(}\PYG{n}{Logo}\PYG{o}{.}\PYG{p}{)} \PYG{o}{/}\PYG{n}{Title} \PYG{p}{(}\PYG{n}{Company} \PYG{n}{Logo}\PYG{p}{)} \PYG{o}{/}\PYG{n}{StPNE} \PYG{n}{pdfmark}


\PYG{o}{\PYGZpc{}} \PYG{n}{Generate} \PYG{n}{attribute} \PYG{n}{dictionary} \PYG{k}{for} \PYG{n}{figure}

\PYG{p}{[} \PYG{o}{/}\PYG{n}{\PYGZus{}objdef} \PYG{p}{\PYGZob{}}\PYG{n}{layoutObj}\PYG{p}{\PYGZcb{}} \PYG{o}{/}\PYG{n+nb}{type} \PYG{o}{/}\PYG{n+nb}{dict} \PYG{o}{/}\PYG{n}{OBJ} \PYG{n}{pdfmark}
\PYG{p}{[}   \PYG{p}{\PYGZob{}}\PYG{n}{layoutObj}\PYG{p}{\PYGZcb{}} \PYG{o}{\PYGZlt{}\PYGZlt{}}\PYG{o}{/}\PYG{n}{O} \PYG{o}{/}\PYG{n}{Layout} \PYG{o}{/}\PYG{n}{Height} \PYG{l+m+mi}{70} \PYG{o}{/}\PYG{n}{Width} \PYG{l+m+mi}{140} \PYG{o}{/}\PYG{n}{BBox} \PYG{p}{[}\PYG{l+m+mi}{90} \PYG{l+m+mi}{290} \PYG{l+m+mi}{250} \PYG{l+m+mi}{360}\PYG{p}{]}
     \PYG{o}{/}\PYG{n}{Placement} \PYG{o}{/}\PYG{n}{Block}\PYG{o}{\PYGZgt{}\PYGZgt{}} \PYG{o}{/}\PYG{n}{PUT} \PYG{n}{pdfmark}

\PYG{o}{\PYGZpc{}} \PYG{n}{Attach} \PYG{n}{attributes} \PYG{n}{to} \PYG{n}{figure}

\PYG{p}{[} \PYG{o}{/}\PYG{n}{Obj} \PYG{p}{\PYGZob{}}\PYG{n}{layoutObj}\PYG{p}{\PYGZcb{}} \PYG{o}{/}\PYG{n}{StAttr} \PYG{n}{pdfmark}

 \PYG{p}{[} \PYG{o}{/}\PYG{n}{StBMC} \PYG{n}{pdfmark}

 \PYG{o}{/}\PYG{n}{Helvetica} \PYG{n}{findfont} \PYG{l+m+mi}{48} \PYG{n}{scalefont} \PYG{n}{setfont}
 \PYG{l+m+mi}{0} \PYG{l+m+mi}{0} \PYG{l+m+mi}{0} \PYG{n}{setrgbcolor}
 \PYG{l+m+mi}{90} \PYG{l+m+mi}{290} \PYG{n}{moveto}
 \PYG{l+m+mi}{90} \PYG{l+m+mi}{360} \PYG{n}{lineto}
 \PYG{l+m+mi}{250} \PYG{l+m+mi}{360} \PYG{n}{lineto}
 \PYG{l+m+mi}{250} \PYG{l+m+mi}{290} \PYG{n}{lineto}
 \PYG{n}{closepath}
 \PYG{n}{stroke}
 \PYG{l+m+mi}{100} \PYG{l+m+mi}{300} \PYG{n}{moveto}
 \PYG{l+m+mi}{1} \PYG{l+m+mi}{0} \PYG{l+m+mi}{0} \PYG{n}{setrgbcolor}
 \PYG{p}{(}\PYG{n}{LOGO}\PYG{p}{)} \PYG{n}{false} \PYG{n}{charpath}
 \PYG{l+m+mi}{2} \PYG{n}{setlinewidth} \PYG{n}{stroke}

 \PYG{p}{[} \PYG{o}{/}\PYG{n}{EMC} \PYG{n}{pdfmark}

 \PYG{p}{[} \PYG{o}{/}\PYG{n}{StPop} \PYG{n}{pdfmark}

\PYG{o}{\PYGZpc{}} \PYG{o}{==}\PYG{o}{==}\PYG{o}{==}\PYG{o}{==}\PYG{o}{==}\PYG{o}{==}\PYG{o}{==}\PYG{o}{==}\PYG{o}{==}\PYG{o}{==}\PYG{o}{==}\PYG{o}{==}\PYG{o}{==}\PYG{o}{==}\PYG{o}{==}\PYG{o}{==}\PYG{o}{==}\PYG{o}{==}\PYG{o}{==}\PYG{o}{==}\PYG{o}{==}\PYG{o}{==}\PYG{o}{==}\PYG{o}{==}\PYG{o}{==}\PYG{o}{==}\PYG{o}{==}\PYG{o}{==}\PYG{o}{==}\PYG{o}{==}\PYG{o}{==}\PYG{o}{==}\PYG{o}{=}
\PYG{o}{\PYGZpc{}} \PYG{n}{Simple} \PYG{n}{List} \PYG{n}{Example}
\PYG{o}{\PYGZpc{}} \PYG{o}{==}\PYG{o}{==}\PYG{o}{==}\PYG{o}{==}\PYG{o}{==}\PYG{o}{==}\PYG{o}{==}\PYG{o}{==}\PYG{o}{==}\PYG{o}{==}\PYG{o}{==}\PYG{o}{==}\PYG{o}{==}\PYG{o}{==}\PYG{o}{==}\PYG{o}{==}\PYG{o}{==}\PYG{o}{==}\PYG{o}{==}\PYG{o}{==}\PYG{o}{==}\PYG{o}{==}\PYG{o}{==}\PYG{o}{==}\PYG{o}{==}\PYG{o}{==}\PYG{o}{==}\PYG{o}{==}\PYG{o}{==}\PYG{o}{==}\PYG{o}{==}\PYG{o}{==}\PYG{o}{=}

 \PYG{o}{/}\PYG{n}{Helvetica}\PYG{o}{\PYGZhy{}}\PYG{n}{Bold} \PYG{n}{findfont} \PYG{l+m+mi}{18} \PYG{n}{scalefont} \PYG{n}{setfont}
 \PYG{l+m+mi}{0} \PYG{l+m+mi}{0} \PYG{l+m+mi}{0} \PYG{n}{setrgbcolor}

 \PYG{p}{[} \PYG{o}{/}\PYG{n}{Subtype} \PYG{o}{/}\PYG{n}{L} \PYG{o}{/}\PYG{n}{Lang} \PYG{p}{(}\PYG{n}{en}\PYG{o}{\PYGZhy{}}\PYG{n}{US}\PYG{p}{)} \PYG{o}{/}\PYG{n}{Title} \PYG{p}{(}\PYG{n}{Some} \PYG{n}{salutations}\PYG{p}{)} \PYG{o}{/}\PYG{n}{StPNE} \PYG{n}{pdfmark}


\PYG{o}{\PYGZpc{}} \PYG{n}{Create} \PYG{n}{a} \PYG{n+nb}{list} \PYG{n}{attribute} \PYG{n}{which} \PYG{n}{specifies} \PYG{n}{the} \PYG{n+nb}{type} \PYG{n}{of} \PYG{n}{label} \PYG{n}{to} \PYG{n}{use}

\PYG{p}{[} \PYG{o}{/}\PYG{n}{\PYGZus{}objdef} \PYG{p}{\PYGZob{}}\PYG{n}{firstAttrObj}\PYG{p}{\PYGZcb{}} \PYG{o}{/}\PYG{n+nb}{type} \PYG{o}{/}\PYG{n+nb}{dict} \PYG{o}{/}\PYG{n}{OBJ} \PYG{n}{pdfmark}
 \PYG{p}{[}   \PYG{p}{\PYGZob{}}\PYG{n}{firstAttrObj}\PYG{p}{\PYGZcb{}} \PYG{o}{\PYGZlt{}\PYGZlt{}}\PYG{o}{/}\PYG{n}{O} \PYG{o}{/}\PYG{n}{List} \PYG{o}{/}\PYG{n}{ListNumbering} \PYG{o}{/}\PYG{n}{LowerRoman}\PYG{o}{\PYGZgt{}\PYGZgt{}} \PYG{o}{/}\PYG{n}{PUT} \PYG{n}{pdfmark}

\PYG{o}{\PYGZpc{}} \PYG{n}{Create} \PYG{n}{an} \PYG{n}{attribute} \PYG{n}{specifying} \PYG{n}{the} \PYG{n}{writing} \PYG{n}{direction}

\PYG{p}{[} \PYG{o}{/}\PYG{n}{\PYGZus{}objdef} \PYG{p}{\PYGZob{}}\PYG{n}{secondAttrObj}\PYG{p}{\PYGZcb{}} \PYG{o}{/}\PYG{n+nb}{type} \PYG{o}{/}\PYG{n+nb}{dict} \PYG{o}{/}\PYG{n}{OBJ} \PYG{n}{pdfmark}
 \PYG{p}{[}   \PYG{p}{\PYGZob{}}\PYG{n}{secondAttrObj}\PYG{p}{\PYGZcb{}} \PYG{o}{\PYGZlt{}\PYGZlt{}}\PYG{o}{/}\PYG{n}{O} \PYG{o}{/}\PYG{n}{Layout} \PYG{o}{/}\PYG{n}{WritingMode} \PYG{o}{/}\PYG{n}{LrTb}\PYG{o}{\PYGZgt{}\PYGZgt{}} \PYG{o}{/}\PYG{n}{PUT} \PYG{n}{pdfmark}

\PYG{o}{\PYGZpc{}} \PYG{n}{Set} \PYG{n}{attribute} \PYG{n+nb}{dict} \PYG{n}{on} \PYG{n+nb}{list}

\PYG{p}{[} \PYG{o}{/}\PYG{n}{Obj} \PYG{p}{\PYGZob{}}\PYG{n}{firstAttrObj}\PYG{p}{\PYGZcb{}} \PYG{o}{/}\PYG{n}{StAttr} \PYG{n}{pdfmark}
 \PYG{p}{[} \PYG{o}{/}\PYG{n}{Obj} \PYG{p}{\PYGZob{}}\PYG{n}{secondAttrObj}\PYG{p}{\PYGZcb{}} \PYG{o}{/}\PYG{n}{StAttr} \PYG{n}{pdfmark}

 \PYG{o}{/}\PYG{n}{Helvetica}\PYG{o}{\PYGZhy{}}\PYG{n}{Oblique} \PYG{n}{findfont} \PYG{l+m+mi}{12} \PYG{n}{scalefont} \PYG{n}{setfont}

 \PYG{p}{[} \PYG{o}{/}\PYG{n}{Subtype} \PYG{o}{/}\PYG{n}{LI} \PYG{o}{/}\PYG{n}{StPNE} \PYG{n}{pdfmark}
     \PYG{p}{[} \PYG{o}{/}\PYG{n}{Subtype} \PYG{o}{/}\PYG{n}{Lbl} \PYG{o}{/}\PYG{n}{StPNE} \PYG{n}{pdfmark}
         \PYG{p}{[} \PYG{o}{/}\PYG{n}{StBMC} \PYG{n}{pdfmark}
             \PYG{l+m+mi}{48} \PYG{l+m+mi}{238} \PYG{n}{moveto}
             \PYG{p}{(}\PYG{n}{i} \PYG{p}{)} \PYG{n}{show}
         \PYG{p}{[} \PYG{o}{/}\PYG{n}{EMC} \PYG{n}{pdfmark}
     \PYG{p}{[} \PYG{o}{/}\PYG{n}{StPop} \PYG{n}{pdfmark}
     \PYG{p}{[} \PYG{o}{/}\PYG{n}{Subtype} \PYG{o}{/}\PYG{n}{LBody} \PYG{o}{/}\PYG{n}{Lang} \PYG{p}{(}\PYG{n}{en}\PYG{o}{\PYGZhy{}}\PYG{n}{cockney}\PYG{p}{)} \PYG{o}{/}\PYG{n}{StPNE} \PYG{n}{pdfmark}
         \PYG{p}{[} \PYG{o}{/}\PYG{n}{StBMC} \PYG{n}{pdfmark}
             \PYG{l+m+mi}{72} \PYG{l+m+mi}{238} \PYG{n}{moveto}
             \PYG{p}{(}\PYG{n}{whatcha}\PYG{p}{)} \PYG{n}{show}
         \PYG{p}{[} \PYG{o}{/}\PYG{n}{EMC} \PYG{n}{pdfmark}
     \PYG{p}{[} \PYG{o}{/}\PYG{n}{StPop} \PYG{n}{pdfmark}
 \PYG{p}{[} \PYG{o}{/}\PYG{n}{StPop} \PYG{n}{pdfmark}

 \PYG{p}{[} \PYG{o}{/}\PYG{n}{Subtype} \PYG{o}{/}\PYG{n}{LI} \PYG{o}{/}\PYG{n}{StPNE} \PYG{n}{pdfmark}
     \PYG{p}{[} \PYG{o}{/}\PYG{n}{Subtype} \PYG{o}{/}\PYG{n}{Lbl} \PYG{o}{/}\PYG{n}{StPNE} \PYG{n}{pdfmark}
         \PYG{p}{[} \PYG{o}{/}\PYG{n}{StBMC} \PYG{n}{pdfmark}
             \PYG{l+m+mi}{48} \PYG{l+m+mi}{226} \PYG{n}{moveto}
             \PYG{p}{(}\PYG{n}{ii} \PYG{p}{)} \PYG{n}{show}
         \PYG{p}{[} \PYG{o}{/}\PYG{n}{EMC} \PYG{n}{pdfmark}
     \PYG{p}{[} \PYG{o}{/}\PYG{n}{StPop} \PYG{n}{pdfmark}
     \PYG{p}{[} \PYG{o}{/}\PYG{n}{Subtype} \PYG{o}{/}\PYG{n}{LBody} \PYG{o}{/}\PYG{n}{Lang} \PYG{p}{(}\PYG{n}{fr}\PYG{p}{)} \PYG{o}{/}\PYG{n}{StPNE} \PYG{n}{pdfmark}
         \PYG{p}{[} \PYG{o}{/}\PYG{n}{StBMC} \PYG{n}{pdfmark}
             \PYG{l+m+mi}{72} \PYG{l+m+mi}{226} \PYG{n}{moveto}
             \PYG{p}{(}\PYG{n}{bon} \PYG{n}{jour}\PYG{p}{)} \PYG{n}{show}
         \PYG{p}{[} \PYG{o}{/}\PYG{n}{EMC} \PYG{n}{pdfmark}
     \PYG{p}{[} \PYG{o}{/}\PYG{n}{StPop} \PYG{n}{pdfmark}
 \PYG{p}{[} \PYG{o}{/}\PYG{n}{StPop} \PYG{n}{pdfmark}

 \PYG{p}{[} \PYG{o}{/}\PYG{n}{StPop} \PYG{n}{pdfmark}

\PYG{o}{\PYGZpc{}} \PYG{o}{==}\PYG{o}{==}\PYG{o}{==}\PYG{o}{==}\PYG{o}{==}\PYG{o}{==}\PYG{o}{==}\PYG{o}{==}\PYG{o}{==}\PYG{o}{==}\PYG{o}{==}\PYG{o}{==}\PYG{o}{==}\PYG{o}{==}\PYG{o}{==}\PYG{o}{==}\PYG{o}{==}\PYG{o}{==}\PYG{o}{==}\PYG{o}{==}\PYG{o}{==}\PYG{o}{==}\PYG{o}{==}\PYG{o}{==}\PYG{o}{==}\PYG{o}{==}\PYG{o}{==}\PYG{o}{==}\PYG{o}{==}\PYG{o}{==}\PYG{o}{==}\PYG{o}{==}\PYG{o}{=}
\PYG{o}{\PYGZpc{}} \PYG{n}{Simple} \PYG{n}{Table} \PYG{n}{Example}
\PYG{o}{\PYGZpc{}} \PYG{o}{==}\PYG{o}{==}\PYG{o}{==}\PYG{o}{==}\PYG{o}{==}\PYG{o}{==}\PYG{o}{==}\PYG{o}{==}\PYG{o}{==}\PYG{o}{==}\PYG{o}{==}\PYG{o}{==}\PYG{o}{==}\PYG{o}{==}\PYG{o}{==}\PYG{o}{==}\PYG{o}{==}\PYG{o}{==}\PYG{o}{==}\PYG{o}{==}\PYG{o}{==}\PYG{o}{==}\PYG{o}{==}\PYG{o}{==}\PYG{o}{==}\PYG{o}{==}\PYG{o}{==}\PYG{o}{==}\PYG{o}{==}\PYG{o}{==}\PYG{o}{==}\PYG{o}{==}\PYG{o}{=}

\PYG{o}{\PYGZpc{}} \PYG{n}{Create} \PYG{n}{a} \PYG{n}{table} \PYG{n}{element}

\PYG{p}{[} \PYG{o}{/}\PYG{n}{Subtype} \PYG{o}{/}\PYG{n}{Table} \PYG{o}{/}\PYG{n}{Lang} \PYG{p}{(}\PYG{n}{en}\PYG{o}{\PYGZhy{}}\PYG{n}{US}\PYG{p}{)} \PYG{o}{/}\PYG{n}{StPNE} \PYG{n}{pdfmark}


\PYG{o}{\PYGZpc{}} \PYG{n}{Place} \PYG{n}{the} \PYG{n}{frame} \PYG{n}{of} \PYG{n}{the} \PYG{n}{table} \PYG{o+ow}{in} \PYG{n}{an} \PYG{n}{artifact}

    \PYG{p}{[} \PYG{o}{/}\PYG{n}{Artifact} \PYG{o}{\PYGZlt{}\PYGZlt{}}\PYG{o}{/}\PYG{n}{Type} \PYG{o}{/}\PYG{n}{Layout} \PYG{o}{/}\PYG{n}{BBox} \PYG{p}{[}\PYG{l+m+mi}{40} \PYG{l+m+mi}{175} \PYG{l+m+mi}{340} \PYG{l+m+mi}{220}\PYG{p}{]} \PYG{o}{\PYGZgt{}\PYGZgt{}} \PYG{o}{/}\PYG{n}{BDC} \PYG{n}{pdfmark}
         \PYG{l+m+mi}{40} \PYG{l+m+mi}{220} \PYG{n}{moveto} \PYG{l+m+mi}{340} \PYG{l+m+mi}{220} \PYG{n}{lineto} \PYG{l+m+mi}{340} \PYG{l+m+mi}{175} \PYG{n}{lineto} \PYG{l+m+mi}{40} \PYG{l+m+mi}{175} \PYG{n}{lineto} \PYG{n}{closepath}
         \PYG{l+m+mi}{40} \PYG{l+m+mi}{196} \PYG{n}{moveto} \PYG{l+m+mi}{340} \PYG{l+m+mi}{196} \PYG{n}{lineto}
         \PYG{l+m+mi}{190} \PYG{l+m+mi}{220} \PYG{n}{moveto} \PYG{l+m+mi}{190} \PYG{l+m+mi}{175} \PYG{n}{lineto}
         \PYG{n}{stroke}
     \PYG{p}{[} \PYG{o}{/}\PYG{n}{EMC} \PYG{n}{pdfmark}


    \PYG{o}{\PYGZpc{}} \PYG{n}{Create} \PYG{n}{a} \PYG{n}{table} \PYG{n}{attribute} \PYG{n}{which} \PYG{n}{specifies} \PYG{n}{the} \PYG{n+nb}{type} \PYG{n}{of} \PYG{n}{label} \PYG{n}{to} \PYG{n}{use}

    \PYG{p}{[} \PYG{o}{/}\PYG{n}{\PYGZus{}objdef} \PYG{p}{\PYGZob{}}\PYG{n}{tableattrObj}\PYG{p}{\PYGZcb{}} \PYG{o}{/}\PYG{n+nb}{type} \PYG{o}{/}\PYG{n+nb}{dict} \PYG{o}{/}\PYG{n}{OBJ} \PYG{n}{pdfmark}
     \PYG{p}{[}   \PYG{p}{\PYGZob{}}\PYG{n}{tableattrObj}\PYG{p}{\PYGZcb{}} \PYG{o}{\PYGZlt{}\PYGZlt{}}\PYG{o}{/}\PYG{n}{O} \PYG{o}{/}\PYG{n}{Layout} \PYG{o}{/}\PYG{n}{Placement} \PYG{o}{/}\PYG{n}{Block} \PYG{o}{/}\PYG{n}{SpaceAfter} \PYG{l+m+mi}{10}

   \PYG{o}{/}\PYG{n}{BorderColor} \PYG{p}{[}\PYG{l+m+mi}{0} \PYG{l+m+mi}{0} \PYG{l+m+mi}{0}\PYG{p}{]}\PYG{o}{\PYGZgt{}\PYGZgt{}} \PYG{o}{/}\PYG{n}{PUT} \PYG{n}{pdfmark}


\PYG{o}{\PYGZpc{}} \PYG{n}{Attach} \PYG{n}{attribute} \PYG{n}{to} \PYG{n}{table}

    \PYG{p}{[} \PYG{o}{/}\PYG{n}{Obj} \PYG{p}{\PYGZob{}}\PYG{n}{tableattrObj}\PYG{p}{\PYGZcb{}} \PYG{o}{/}\PYG{n}{StAttr} \PYG{n}{pdfmark}


\PYG{o}{\PYGZpc{}} \PYG{n}{Create} \PYG{n}{an} \PYG{n}{attribute} \PYG{n+nb}{object} \PYG{k}{with} \PYG{n}{the} \PYG{n}{common} \PYG{n}{settings} \PYG{k}{for} \PYG{n}{each} \PYG{n}{table} \PYG{n}{data} \PYG{n}{cell}

    \PYG{p}{[} \PYG{o}{/}\PYG{n}{\PYGZus{}objdef} \PYG{p}{\PYGZob{}}\PYG{n}{tableCellsObj}\PYG{p}{\PYGZcb{}} \PYG{o}{/}\PYG{n+nb}{type} \PYG{o}{/}\PYG{n+nb}{dict} \PYG{o}{/}\PYG{n}{OBJ} \PYG{n}{pdfmark}
     \PYG{p}{[}   \PYG{p}{\PYGZob{}}\PYG{n}{tableCellsObj}\PYG{p}{\PYGZcb{}} \PYG{o}{\PYGZlt{}\PYGZlt{}}\PYG{o}{/}\PYG{n}{O} \PYG{o}{/}\PYG{n}{Layout} \PYG{o}{/}\PYG{n}{Width} \PYG{l+m+mi}{150} \PYG{o}{/}\PYG{n}{BorderStyle} \PYG{o}{/}\PYG{n}{Solid}

   \PYG{o}{/}\PYG{n}{BorderThickness} \PYG{l+m+mi}{2} \PYG{o}{/}\PYG{n}{BorderColor} \PYG{p}{[}\PYG{l+m+mi}{0} \PYG{l+m+mi}{0} \PYG{l+m+mi}{0}\PYG{p}{]}\PYG{o}{\PYGZgt{}\PYGZgt{}} \PYG{o}{/}\PYG{n}{PUT} \PYG{n}{pdfmark}

\PYG{o}{\PYGZpc{}} \PYG{n}{Add} \PYG{n}{it} \PYG{n}{to} \PYG{n}{the} \PYG{n}{classmap}

    \PYG{p}{[} \PYG{o}{/}\PYG{n}{CommonTableInfo} \PYG{p}{\PYGZob{}}\PYG{n}{widthObj}\PYG{p}{\PYGZcb{}} \PYG{o}{/}\PYG{n}{StClassMap} \PYG{n}{pdfmark}

     \PYG{p}{[} \PYG{o}{/}\PYG{n}{Subtype} \PYG{o}{/}\PYG{n}{THead} \PYG{o}{/}\PYG{n}{StPNE} \PYG{n}{pdfmark}
     \PYG{p}{[} \PYG{o}{/}\PYG{n}{Subtype} \PYG{o}{/}\PYG{n}{TR} \PYG{o}{/}\PYG{n}{StPNE} \PYG{n}{pdfmark}
         \PYG{p}{[} \PYG{o}{/}\PYG{n}{Subtype} \PYG{o}{/}\PYG{n}{TH} \PYG{o}{/}\PYG{n}{Class} \PYG{o}{/}\PYG{n}{CommonTableInfo} \PYG{o}{/}\PYG{n}{StPNE} \PYG{n}{pdfmark}
             \PYG{p}{[} \PYG{o}{/}\PYG{n}{StBMC} \PYG{n}{pdfmark}
                 \PYG{l+m+mi}{48} \PYG{l+m+mi}{200} \PYG{n}{moveto}
                 \PYG{p}{(}\PYG{n}{Item}\PYG{p}{)} \PYG{n}{show}
             \PYG{p}{[} \PYG{o}{/}\PYG{n}{EMC} \PYG{n}{pdfmark}
         \PYG{p}{[} \PYG{o}{/}\PYG{n}{StPop} \PYG{n}{pdfmark}
         \PYG{p}{[} \PYG{o}{/}\PYG{n}{Subtype} \PYG{o}{/}\PYG{n}{TH} \PYG{o}{/}\PYG{n}{Class} \PYG{o}{/}\PYG{n}{CommonTableInfo} \PYG{o}{/}\PYG{n}{StPNE} \PYG{n}{pdfmark}
             \PYG{p}{[} \PYG{o}{/}\PYG{n}{StBMC} \PYG{n}{pdfmark}
                 \PYG{l+m+mi}{200} \PYG{l+m+mi}{200} \PYG{n}{moveto}
                 \PYG{p}{(}\PYG{n}{Description}\PYG{p}{)} \PYG{n}{show}
             \PYG{p}{[} \PYG{o}{/}\PYG{n}{EMC} \PYG{n}{pdfmark}
         \PYG{p}{[} \PYG{o}{/}\PYG{n}{StPop} \PYG{n}{pdfmark}
     \PYG{p}{[} \PYG{o}{/}\PYG{n}{StPop} \PYG{n}{pdfmark}
     \PYG{p}{[} \PYG{o}{/}\PYG{n}{StPop} \PYG{n}{pdfmark}

     \PYG{p}{[} \PYG{o}{/}\PYG{n}{Subtype} \PYG{o}{/}\PYG{n}{TBody} \PYG{o}{/}\PYG{n}{StPNE} \PYG{n}{pdfmark}
     \PYG{p}{[} \PYG{o}{/}\PYG{n}{Subtype} \PYG{o}{/}\PYG{n}{TR} \PYG{o}{/}\PYG{n}{StPNE} \PYG{n}{pdfmark}
         \PYG{p}{[} \PYG{o}{/}\PYG{n}{Subtype} \PYG{o}{/}\PYG{n}{TD} \PYG{o}{/}\PYG{n}{Class} \PYG{o}{/}\PYG{n}{CommonTableInfo} \PYG{o}{/}\PYG{n}{StPNE} \PYG{n}{pdfmark}
             \PYG{p}{[} \PYG{o}{/}\PYG{n}{StBMC} \PYG{n}{pdfmark}
                 \PYG{l+m+mi}{48} \PYG{l+m+mi}{180} \PYG{n}{moveto}
                 \PYG{p}{(}\PYG{n}{Thing}\PYG{p}{)} \PYG{n}{show}
             \PYG{p}{[} \PYG{o}{/}\PYG{n}{EMC} \PYG{n}{pdfmark}
         \PYG{p}{[} \PYG{o}{/}\PYG{n}{StPop} \PYG{n}{pdfmark}

         \PYG{p}{[} \PYG{o}{/}\PYG{n}{Subtype} \PYG{o}{/}\PYG{n}{TD} \PYG{o}{/}\PYG{n}{Class} \PYG{o}{/}\PYG{n}{CommonTableInfo} \PYG{o}{/}\PYG{n}{StPNE} \PYG{n}{pdfmark}
             \PYG{p}{[} \PYG{o}{/}\PYG{n}{StBMC} \PYG{n}{pdfmark}
                 \PYG{l+m+mi}{200} \PYG{l+m+mi}{180} \PYG{n}{moveto}
                 \PYG{p}{(}\PYG{n}{Things}\PYG{p}{)} \PYG{n}{show}
             \PYG{p}{[} \PYG{o}{/}\PYG{n}{EMC} \PYG{n}{pdfmark}
         \PYG{p}{[} \PYG{o}{/}\PYG{n}{StPop} \PYG{n}{pdfmark}
     \PYG{p}{[} \PYG{o}{/}\PYG{n}{StPop} \PYG{n}{pdfmark}
     \PYG{p}{[} \PYG{o}{/}\PYG{n}{StPop} \PYG{n}{pdfmark}

 \PYG{p}{[} \PYG{o}{/}\PYG{n}{StPop} \PYG{n}{pdfmark}


 \PYG{p}{[} \PYG{o}{/}\PYG{n}{StPop} \PYG{n}{pdfmark}

\PYG{o}{\PYGZpc{}} \PYG{n}{Now} \PYG{n}{that} \PYG{n}{the} \PYG{n}{text} \PYG{o+ow}{is} \PYG{n}{done}\PYG{p}{,} \PYG{n}{let}\PYG{l+s+s1}{\PYGZsq{}}\PYG{l+s+s1}{s make the outlines.}
\PYG{o}{\PYGZpc{}} \PYG{n}{The} \PYG{n}{first} \PYG{n}{bookmark} \PYG{n}{magnifies} \PYG{l+m+mi}{400} \PYG{n}{percent}\PYG{p}{,} \PYG{k}{while} \PYG{n}{the} \PYG{n}{others} \PYG{n}{go} \PYG{n}{to} \PYG{n}{their}
\PYG{o}{\PYGZpc{}} \PYG{n}{line} \PYG{o+ow}{in} \PYG{n}{the} \PYG{n}{text}\PYG{o}{.}

\PYG{p}{[} \PYG{o}{/}\PYG{n}{Count} \PYG{l+m+mi}{4} \PYG{o}{/}\PYG{n}{Page} \PYG{l+m+mi}{1} \PYG{o}{/}\PYG{n}{View} \PYG{p}{[}\PYG{o}{/}\PYG{n}{XYZ} \PYG{l+m+mi}{216} \PYG{l+m+mi}{744} \PYG{l+m+mf}{4.0}\PYG{p}{]} \PYG{o}{/}\PYG{n}{Title} \PYG{p}{(}\PYG{n}{Title} \PYG{n}{of} \PYG{n}{Document}\PYG{p}{)}
 \PYG{o}{/}\PYG{n}{OUT} \PYG{n}{pdfmark}
 \PYG{p}{[} \PYG{o}{/}\PYG{n}{Page} \PYG{l+m+mi}{1} \PYG{o}{/}\PYG{n}{View} \PYG{p}{[}\PYG{o}{/}\PYG{n}{XYZ} \PYG{l+m+mi}{0} \PYG{l+m+mi}{704} \PYG{l+m+mf}{1.0}\PYG{p}{]} \PYG{o}{/}\PYG{n}{Title} \PYG{p}{(}\PYG{n}{First} \PYG{n}{Topic}\PYG{p}{)} \PYG{o}{/}\PYG{n}{OUT} \PYG{n}{pdfmark}
 \PYG{p}{[} \PYG{o}{/}\PYG{n}{Count} \PYG{o}{\PYGZhy{}}\PYG{l+m+mi}{1} \PYG{o}{/}\PYG{n}{Page} \PYG{l+m+mi}{1} \PYG{o}{/}\PYG{n}{View} \PYG{p}{[}\PYG{o}{/}\PYG{n}{XYZ} \PYG{l+m+mi}{0} \PYG{l+m+mi}{662} \PYG{l+m+mf}{1.0}\PYG{p}{]} \PYG{o}{/}\PYG{n}{Title} \PYG{p}{(}\PYG{n}{Second} \PYG{n}{Topic}\PYG{p}{)} \PYG{o}{/}\PYG{n}{OUT} \PYG{n}{pdfmark}
 \PYG{p}{[} \PYG{o}{/}\PYG{n}{Page} \PYG{l+m+mi}{1} \PYG{o}{/}\PYG{n}{View} \PYG{p}{[}\PYG{o}{/}\PYG{n}{XYZ} \PYG{l+m+mi}{0} \PYG{l+m+mi}{610} \PYG{l+m+mf}{1.0}\PYG{p}{]} \PYG{o}{/}\PYG{n}{Title} \PYG{p}{(}\PYG{n}{Subtopic} \PYG{n}{of} \PYG{n}{second} \PYG{n}{Topic}\PYG{p}{)} \PYG{o}{/}\PYG{n}{OUT} \PYG{n}{pdfmark}
 \PYG{p}{[} \PYG{o}{/}\PYG{n}{Page} \PYG{l+m+mi}{1} \PYG{o}{/}\PYG{n}{View} \PYG{p}{[}\PYG{o}{/}\PYG{n}{XYZ} \PYG{l+m+mi}{0} \PYG{l+m+mi}{530} \PYG{l+m+mf}{1.0}\PYG{p}{]} \PYG{o}{/}\PYG{n}{Title} \PYG{p}{(}\PYG{n}{Line} \PYG{n}{Numbered} \PYG{n}{Topic}\PYG{p}{)} \PYG{o}{/}\PYG{n}{OUT} \PYG{n}{pdfmark}
 \PYG{p}{[} \PYG{o}{/}\PYG{n}{PageMode} \PYG{o}{/}\PYG{n}{UseOutlines} \PYG{o}{/}\PYG{n}{Page} \PYG{l+m+mi}{1} \PYG{o}{/}\PYG{n}{View} \PYG{p}{[}\PYG{o}{/}\PYG{n}{XYZ} \PYG{n}{null} \PYG{n}{null} \PYG{n}{null}\PYG{p}{]} \PYG{o}{/}\PYG{n}{DOCVIEW} \PYG{n}{pdfmark}

\PYG{o}{\PYGZpc{}} \PYG{n}{And} \PYG{k}{finally} \PYG{n}{the} \PYG{n}{rolemap}\PYG{p}{,} \PYG{k}{with} \PYG{n}{every} \PYG{n}{tag} \PYG{n}{that} \PYG{n}{we} \PYG{n}{have} \PYG{n}{used} \PYG{n}{defined}\PYG{o}{.}

\PYG{p}{[} \PYG{o}{/}\PYG{n}{document} \PYG{o}{/}\PYG{n}{Document}
     \PYG{o}{/}\PYG{n}{documenttitle} \PYG{o}{/}\PYG{n}{H}
     \PYG{o}{/}\PYG{n}{p} \PYG{o}{/}\PYG{n}{P}
     \PYG{o}{/}\PYG{n}{emph} \PYG{o}{/}\PYG{n}{Span}
     \PYG{o}{/}\PYG{n}{topic} \PYG{o}{/}\PYG{n}{Div}
     \PYG{o}{/}\PYG{n}{topic2} \PYG{o}{/}\PYG{n}{Div}
     \PYG{o}{/}\PYG{n}{topichead} \PYG{o}{/}\PYG{n}{H1}
     \PYG{o}{/}\PYG{n}{topichead2} \PYG{o}{/}\PYG{n}{H2}
     \PYG{o}{/}\PYG{n}{StRoleMap} \PYG{n}{pdfmark}

 \PYG{n}{showpage}
 \PYG{p}{(}\PYG{o}{\PYGZpc{}}\PYG{o}{\PYGZpc{}}\PYG{p}{[}\PYG{n}{Page}\PYG{p}{:} \PYG{l+m+mi}{1}\PYG{p}{]}\PYG{o}{\PYGZpc{}}\PYG{o}{\PYGZpc{}}\PYG{p}{)} \PYG{o}{=}
\end{sphinxVerbatim}


\chapter{Examples}
\label{\detokenize{pdfmark_Examples:examples}}\label{\detokenize{pdfmark_Examples::doc}}
This section provides several examples illustrating various uses of the pdfmark operator.


\section{Building an Output Intents array}
\label{\detokenize{pdfmark_Examples:building-an-output-intents-array}}
The following Windows and Mac OS examples demonstrate how to build an Output Intents array, which is useful in color processing. The hard\sphinxhyphen{}coded file and directory path should be applicable to most users.
\begin{enumerate}
\sphinxsetlistlabels{\arabic}{enumi}{enumii}{}{.}%
\item {} 
Output Intents array in Windows

\end{enumerate}

Define the profile object. The file is set up using a Windows path. You can also use a Mac OS or embed the profile data in the PostScript.*

\begin{sphinxVerbatim}[commandchars=\\\{\}]
  \PYG{p}{[} \PYG{o}{/}\PYG{n}{NamespacePush} \PYG{n}{pdfmark}
    \PYG{p}{[} \PYG{o}{/}\PYG{n}{\PYGZus{}objdef} \PYG{p}{\PYGZob{}}\PYG{n}{Profile}\PYG{p}{\PYGZcb{}} \PYG{o}{/}\PYG{n+nb}{type} \PYG{o}{/}\PYG{n}{stream} \PYG{o}{/}\PYG{n}{OBJ} \PYG{n}{pdfmark}
    \PYG{p}{[} \PYG{p}{\PYGZob{}}\PYG{n}{Profile}\PYG{p}{\PYGZcb{}} \PYG{o}{\PYGZlt{}\PYGZlt{}}\PYG{o}{/}\PYG{n}{N} \PYG{l+m+mi}{4}\PYG{o}{\PYGZgt{}\PYGZgt{}} \PYG{o}{/}\PYG{n}{PUT} \PYG{n}{pdfmark}
    \PYG{p}{[} \PYG{p}{\PYGZob{}}\PYG{n}{Profile}\PYG{p}{\PYGZcb{}}
    \PYG{p}{(}\PYG{n}{c}\PYG{p}{:}\PYG{o}{/}\PYG{n}{Program} \PYG{n}{Files}\PYG{o}{/}\PYG{n}{Common}
    \PYG{n}{Files}\PYG{o}{/}\PYG{n}{Adobe}\PYG{o}{/}\PYG{n}{Color}\PYG{o}{/}\PYG{n}{Profiles}\PYG{o}{/}\PYG{n}{Recommended}\PYG{o}{/}\PYG{n}{EuroscaleCoated}\PYG{o}{.}\PYG{n}{icc}\PYG{p}{)}
    \PYG{p}{(}\PYG{n}{r}\PYG{p}{)} \PYG{n}{file} \PYG{o}{/}\PYG{n}{PUT} \PYG{n}{pdfmark}

\PYG{o}{\PYGZpc{}} \PYG{n}{Build} \PYG{n}{the} \PYG{n}{OutputIntent} \PYG{n}{objects}
    \PYG{p}{[} \PYG{o}{/}\PYG{n}{\PYGZus{}objdef} \PYG{p}{\PYGZob{}}\PYG{n}{OIDict}\PYG{p}{\PYGZcb{}} \PYG{o}{/}\PYG{n+nb}{type} \PYG{o}{/}\PYG{n+nb}{dict} \PYG{o}{/}\PYG{n}{OBJ} \PYG{n}{pdfmark}
    \PYG{p}{[} \PYG{o}{/}\PYG{n}{\PYGZus{}objdef} \PYG{p}{\PYGZob{}}\PYG{n}{OIArray}\PYG{p}{\PYGZcb{}} \PYG{o}{/}\PYG{n+nb}{type} \PYG{o}{/}\PYG{n}{array} \PYG{o}{/}\PYG{n}{OBJ} \PYG{n}{pdfmark}
    \PYG{p}{[} \PYG{p}{\PYGZob{}}\PYG{n}{OIDict}\PYG{p}{\PYGZcb{}} \PYG{o}{\PYGZlt{}\PYGZlt{}} \PYG{o}{/}\PYG{n}{Type} \PYG{o}{/}\PYG{n}{OutputIntent} \PYG{o}{/}\PYG{n}{OutputCondition} \PYG{p}{(}\PYG{n}{Test}\PYG{p}{)} \PYG{o}{/}\PYG{n}{S}
    \PYG{o}{/}\PYG{n}{GTS\PYGZus{}PDFX} \PYG{o}{/}\PYG{n}{OutputConditionIdentifier} \PYG{p}{(}\PYG{n}{Custom}\PYG{p}{)} \PYG{o}{/}\PYG{n}{DestOutputProfile}
    \PYG{p}{\PYGZob{}}\PYG{n}{Profile}\PYG{p}{\PYGZcb{}} \PYG{o}{\PYGZgt{}\PYGZgt{}} \PYG{o}{/}\PYG{n}{PUT} \PYG{n}{pdfmark}
    \PYG{p}{[} \PYG{p}{\PYGZob{}}\PYG{n}{OIArray}\PYG{p}{\PYGZcb{}} \PYG{l+m+mi}{0} \PYG{p}{\PYGZob{}}\PYG{n}{OIDict}\PYG{p}{\PYGZcb{}} \PYG{o}{/}\PYG{n}{PUT} \PYG{n}{pdfmark}
    \PYG{o}{\PYGZpc{}} \PYG{n}{Store} \PYG{n}{the} \PYG{n}{OutputIntents} \PYG{n}{array} \PYG{o+ow}{in} \PYG{n}{the} \PYG{n}{catalog}\PYG{o}{.}

 \PYG{p}{[} \PYG{p}{\PYGZob{}}\PYG{n}{Catalog}\PYG{p}{\PYGZcb{}}\PYG{o}{\PYGZlt{}\PYGZlt{}} \PYG{o}{/}\PYG{n}{OutputIntents} \PYG{p}{\PYGZob{}}\PYG{n}{OIArray}\PYG{p}{\PYGZcb{}} \PYG{o}{\PYGZgt{}\PYGZgt{}} \PYG{o}{/}\PYG{n}{PUT} \PYG{n}{pdfmark}
  \PYG{p}{[} \PYG{o}{/}\PYG{n}{NamespacePop} \PYG{n}{pdfmark}
\end{sphinxVerbatim}
\begin{enumerate}
\sphinxsetlistlabels{\arabic}{enumi}{enumii}{}{.}%
\item {} 
Output Intents array in Mac OS

\end{enumerate}

Define the profile object. The file is set up using a Mac OS path. You can also use a Windows or embed the profile data in the PostScript.

\begin{sphinxVerbatim}[commandchars=\\\{\}]
\PYG{p}{[} \PYG{o}{/}\PYG{n}{NamespacePush} \PYG{n}{pdfmark}
  \PYG{p}{[} \PYG{o}{/}\PYG{n}{\PYGZus{}objdef} \PYG{p}{\PYGZob{}}\PYG{n}{Profile}\PYG{p}{\PYGZcb{}} \PYG{o}{/}\PYG{n+nb}{type} \PYG{o}{/}\PYG{n}{stream} \PYG{o}{/}\PYG{n}{OBJ} \PYG{n}{pdfmark}
  \PYG{p}{[} \PYG{p}{\PYGZob{}}\PYG{n}{Profile}\PYG{p}{\PYGZcb{}} \PYG{o}{\PYGZlt{}\PYGZlt{}}\PYG{o}{/}\PYG{n}{N} \PYG{l+m+mi}{4}\PYG{o}{\PYGZgt{}\PYGZgt{}} \PYG{o}{/}\PYG{n}{PUT} \PYG{n}{pdfmark}
  \PYG{p}{[} \PYG{p}{\PYGZob{}}\PYG{n}{Profile}\PYG{p}{\PYGZcb{}}
  \PYG{p}{(}\PYG{o}{/}\PYG{n}{Library}\PYG{o}{/}\PYG{n}{Application} \PYG{n}{Support}\PYG{o}{/}\PYG{n}{Adobe}\PYG{o}{/}\PYG{n}{Color}\PYG{o}{/}\PYG{n}{Profiles}\PYG{o}{/}\PYG{n}{JapanStandard}\PYG{o}{.}\PYG{n}{icc}\PYG{p}{)}
  \PYG{p}{(}\PYG{n}{r}\PYG{p}{)} \PYG{n}{file} \PYG{o}{/}\PYG{n}{PUT} \PYG{n}{pdfmark}
  \PYG{o}{\PYGZpc{}} \PYG{n}{Build} \PYG{n}{the} \PYG{n}{OutputIntent} \PYG{n}{objects}

  \PYG{p}{[} \PYG{o}{/}\PYG{n}{\PYGZus{}objdef} \PYG{p}{\PYGZob{}}\PYG{n}{OIDict}\PYG{p}{\PYGZcb{}} \PYG{o}{/}\PYG{n+nb}{type} \PYG{o}{/}\PYG{n+nb}{dict} \PYG{o}{/}\PYG{n}{OBJ} \PYG{n}{pdfmark}
  \PYG{p}{[} \PYG{o}{/}\PYG{n}{\PYGZus{}objdef} \PYG{p}{\PYGZob{}}\PYG{n}{OIArray}\PYG{p}{\PYGZcb{}} \PYG{o}{/}\PYG{n+nb}{type} \PYG{o}{/}\PYG{n}{array} \PYG{o}{/}\PYG{n}{OBJ} \PYG{n}{pdfmark}
  \PYG{p}{[} \PYG{p}{\PYGZob{}}\PYG{n}{OIDict}\PYG{p}{\PYGZcb{}} \PYG{o}{\PYGZlt{}\PYGZlt{}} \PYG{o}{/}\PYG{n}{Type} \PYG{o}{/}\PYG{n}{OutputIntent} \PYG{o}{/}\PYG{n}{OutputCondition} \PYG{p}{(}\PYG{n}{Test}\PYG{p}{)} \PYG{o}{/}\PYG{n}{S}
  \PYG{o}{/}\PYG{n}{GTS\PYGZus{}PDFX} \PYG{o}{/}\PYG{n}{OutputConditionIdentifier} \PYG{p}{(}\PYG{n}{Custom}\PYG{p}{)} \PYG{o}{/}\PYG{n}{DestOutputProfile}
  \PYG{p}{\PYGZob{}}\PYG{n}{Profile}\PYG{p}{\PYGZcb{}} \PYG{o}{\PYGZgt{}\PYGZgt{}} \PYG{o}{/}\PYG{n}{PUT} \PYG{n}{pdfmark}
  \PYG{p}{[} \PYG{p}{\PYGZob{}}\PYG{n}{OIArray}\PYG{p}{\PYGZcb{}} \PYG{l+m+mi}{0} \PYG{p}{\PYGZob{}}\PYG{n}{OIDict}\PYG{p}{\PYGZcb{}} \PYG{o}{/}\PYG{n}{PUT} \PYG{n}{pdfmark}

  \PYG{o}{\PYGZpc{}} \PYG{n}{Store} \PYG{n}{the} \PYG{n}{OutputIntents} \PYG{n}{array} \PYG{o+ow}{in} \PYG{n}{the} \PYG{n}{catalog}\PYG{o}{.}
  \PYG{p}{[} \PYG{p}{\PYGZob{}}\PYG{n}{Catalog}\PYG{p}{\PYGZcb{}}\PYG{o}{\PYGZlt{}\PYGZlt{}} \PYG{o}{/}\PYG{n}{OutputIntents} \PYG{p}{\PYGZob{}}\PYG{n}{OIArray}\PYG{p}{\PYGZcb{}} \PYG{o}{\PYGZgt{}\PYGZgt{}} \PYG{o}{/}\PYG{n}{PUT} \PYG{n}{pdfmark}
\PYG{p}{[} \PYG{o}{/}\PYG{n}{NamespacePop} \PYG{n}{pdfmark}
\end{sphinxVerbatim}


\section{Named object examples}
\label{\detokenize{pdfmark_Examples:named-object-examples}}
The following examples demonstrate how to work with named objects.

\sphinxstylestrong{Creating user\sphinxhyphen{}defined named objects}

\begin{sphinxVerbatim}[commandchars=\\\{\}]
\PYG{p}{[}    \PYG{o}{/}\PYG{n}{\PYGZus{}objdef} \PYG{p}{\PYGZob{}}\PYG{n}{myarrayname}\PYG{p}{\PYGZcb{}} \PYG{o}{/}\PYG{n+nb}{type}\PYG{o}{/} \PYG{n}{array} \PYG{o}{/}\PYG{n}{OBJ} \PYG{n}{pdfmark}
\PYG{p}{[} \PYG{o}{/}\PYG{n}{\PYGZus{}objdef} \PYG{p}{\PYGZob{}}\PYG{n}{mydictname}\PYG{p}{\PYGZcb{}} \PYG{o}{/}\PYG{n+nb}{type} \PYG{o}{/}\PYG{n+nb}{dict} \PYG{o}{/}\PYG{n}{OBJ} \PYG{n}{pdfmark}
\PYG{p}{[} \PYG{o}{/}\PYG{n}{\PYGZus{}objdef} \PYG{p}{\PYGZob{}}\PYG{n}{mystreamname}\PYG{p}{\PYGZcb{}} \PYG{o}{/}\PYG{n+nb}{type} \PYG{o}{/}\PYG{n}{stream} \PYG{o}{/}\PYG{n}{OBJ} \PYG{n}{pdfmark}
\end{sphinxVerbatim}
\begin{enumerate}
\sphinxsetlistlabels{\arabic}{enumi}{enumii}{}{.}%
\item {} 
Adding values to named objects

\end{enumerate}

\begin{sphinxVerbatim}[commandchars=\\\{\}]
\PYGZpc{} Insert 132 at location 0

[   \PYGZob{}myarrayname\PYGZcb{} 0 132 /PUT pdfmark
[   \PYGZob{}myarrayname\PYGZcb{} 100 /APPEND pdfmark
[   \PYGZob{}myarrayname\PYGZcb{} /name2 /APPEND pdfmark
[   \PYGZob{}myarrayname\PYGZcb{} 2 [200 300] /PUTINTERVAL pdfmark

\PYGZpc{} At the end of the above examples, the array \PYGZob{}myarrayname\PYGZcb{}
\PYGZpc{} has the value [132 100 200 300 /name2]
\PYGZpc{} Insert key–value pair into dictionary

[   \PYGZob{}mydictname\PYGZcb{} \PYGZlt{}\PYGZlt{} /TheKey 366 \PYGZgt{}\PYGZgt{} /PUT pdfmark

\PYGZpc{} Insert string into stream object

[   \PYGZob{}mystreamname\PYGZcb{} (any string) /PUT pdfmark

\PYGZpc{} Use predefined named objects
\PYGZpc{} Insert key–value pair into Catalog

[   \PYGZob{}Catalog\PYGZcb{} \PYGZlt{}\PYGZlt{} /Answer 42 \PYGZgt{}\PYGZgt{} /PUT pdfmark

\PYGZpc{} Insert key–value pair into Page 37\PYGZsq{}s dictionary

[   \PYGZob{}Page37\PYGZcb{} \PYGZlt{}\PYGZlt{} /SpecialKey (special string) \PYGZgt{}\PYGZgt{} /PUT pdfmark

\PYGZpc{} Insert key–value pair into the current page\PYGZsq{}s dictionary

[   \PYGZob{}ThisPage\PYGZcb{} \PYGZlt{}\PYGZlt{} /NewKey (new string) \PYGZgt{}\PYGZgt{} /PUT pdfmark
\end{sphinxVerbatim}
\begin{enumerate}
\sphinxsetlistlabels{\arabic}{enumi}{enumii}{}{.}%
\item {} 
Creating an annotation as a named object and adding content to it

\end{enumerate}

\begin{sphinxVerbatim}[commandchars=\\\{\}]
\PYG{o}{\PYGZpc{}} \PYG{n}{Create} \PYG{n}{text} \PYG{n}{annotation}

\PYG{p}{[} \PYG{o}{/}\PYG{n}{\PYGZus{}objdef} \PYG{p}{\PYGZob{}}\PYG{n}{MikesAnnot}\PYG{p}{\PYGZcb{}} \PYG{o}{/}\PYG{n}{Contents} \PYG{p}{(}\PYG{n}{a} \PYG{n}{simple} \PYG{n}{text} \PYG{n}{annot}\PYG{p}{)}
    \PYG{o}{/}\PYG{n}{Rect} \PYG{p}{[}\PYG{l+m+mi}{100} \PYG{l+m+mi}{100} \PYG{l+m+mi}{200} \PYG{l+m+mi}{200}\PYG{p}{]} \PYG{o}{/}\PYG{n}{Subtype} \PYG{o}{/}\PYG{n}{Text} \PYG{o}{/}\PYG{n}{ANN} \PYG{n}{pdfmark}

\PYG{o}{\PYGZpc{}} \PYG{n}{Add} \PYG{n}{another} \PYG{n}{key} \PYG{n}{to} \PYG{n}{this} \PYG{n}{text} \PYG{n}{annotation}

\PYG{p}{[}   \PYG{p}{\PYGZob{}}\PYG{n}{MikesAnnot}\PYG{p}{\PYGZcb{}} \PYG{o}{\PYGZlt{}\PYGZlt{}} \PYG{o}{/}\PYG{n}{AnotherKey} \PYG{p}{(}\PYG{n}{another} \PYG{n}{string} \PYG{n}{value}\PYG{p}{)} \PYG{o}{\PYGZgt{}\PYGZgt{}} \PYG{o}{/}\PYG{n}{PUT} \PYG{n}{pdfmark}
\end{sphinxVerbatim}
\begin{enumerate}
\sphinxsetlistlabels{\arabic}{enumi}{enumii}{}{.}%
\item {} 
Using a named object as a value

\end{enumerate}

This example creates a text annotation on the current page with extra keys in the annotation dictionary. These keys, \sphinxcode{\sphinxupquote{MyPrivateAnnotArrayData}} and \sphinxcode{\sphinxupquote{MyPrivateAnnotDictData}} , have values that are indirect references to the array and dictionary objects created by the previous pdfmark entries.

\begin{sphinxVerbatim}[commandchars=\\\{\}]
\PYG{p}{[}    \PYG{o}{/}\PYG{n}{\PYGZus{}objdef} \PYG{p}{\PYGZob{}}\PYG{n}{myarray}\PYG{p}{\PYGZcb{}} \PYG{o}{/}\PYG{n+nb}{type} \PYG{o}{/}\PYG{n}{array} \PYG{o}{/}\PYG{n}{OBJ} \PYG{n}{pdfmark}
\PYG{p}{[} \PYG{o}{/}\PYG{n}{\PYGZus{}objdef} \PYG{p}{\PYGZob{}}\PYG{n}{mydict}\PYG{p}{\PYGZcb{}} \PYG{o}{/}\PYG{n+nb}{type} \PYG{o}{/}\PYG{n+nb}{dict} \PYG{o}{/}\PYG{n}{OBJ} \PYG{n}{pdfmark}
\PYG{p}{[} \PYG{o}{/}\PYG{n}{MyPrivateAnnotArrayData} \PYG{p}{\PYGZob{}}\PYG{n}{myarray}\PYG{p}{\PYGZcb{}}
    \PYG{o}{/}\PYG{n}{MyPrivateAnnotDictData} \PYG{p}{\PYGZob{}}\PYG{n}{mydict}\PYG{p}{\PYGZcb{}}
    \PYG{o}{/}\PYG{n}{SubType} \PYG{o}{/}\PYG{n}{Text}
    \PYG{o}{/}\PYG{n}{Rect} \PYG{p}{[}\PYG{l+m+mi}{500} \PYG{l+m+mi}{500} \PYG{l+m+mi}{550} \PYG{l+m+mi}{550}\PYG{p}{]}
    \PYG{o}{/}\PYG{n}{Contents} \PYG{p}{(}\PYG{n}{Here} \PYG{o+ow}{is} \PYG{n}{a} \PYG{n}{text} \PYG{n}{annotation}\PYG{p}{)}
    \PYG{o}{/}\PYG{n}{ANN} \PYG{n}{pdfmark}
\end{sphinxVerbatim}
\begin{enumerate}
\sphinxsetlistlabels{\arabic}{enumi}{enumii}{}{.}%
\item {} 
Putting a file’s contents into a text annotation

\end{enumerate}

\begin{sphinxVerbatim}[commandchars=\\\{\}]
\PYG{o}{/}\PYG{n}{F} \PYG{p}{(}\PYG{n}{file}\PYG{l+s+s1}{\PYGZsq{}}\PYG{l+s+s1}{s platform dependent path name) (r) file def}
\PYG{p}{[} \PYG{o}{/}\PYG{n}{\PYGZus{}objdef} \PYG{p}{\PYGZob{}}\PYG{n}{mystream}\PYG{p}{\PYGZcb{}} \PYG{o}{/}\PYG{n+nb}{type} \PYG{o}{/}\PYG{n}{stream} \PYG{o}{/}\PYG{n}{OBJ} \PYG{n}{pdfmark}
\PYG{p}{[}   \PYG{p}{\PYGZob{}}\PYG{n}{mystream}\PYG{p}{\PYGZcb{}} \PYG{n}{F} \PYG{o}{/}\PYG{n}{PUT} \PYG{n}{pdfmark}
\PYG{p}{[} \PYG{o}{/}\PYG{n}{MyPrivateAnnotmyStreamData} \PYG{p}{\PYGZob{}}\PYG{n}{mystream}\PYG{p}{\PYGZcb{}}
    \PYG{o}{/}\PYG{n}{SubType} \PYG{o}{/}\PYG{n}{Text}
    \PYG{o}{/}\PYG{n}{Rect} \PYG{p}{[}\PYG{l+m+mi}{500} \PYG{l+m+mi}{500} \PYG{l+m+mi}{550} \PYG{l+m+mi}{550}\PYG{p}{]}
    \PYG{o}{/}\PYG{n}{Contents} \PYG{p}{(}\PYG{n}{Here} \PYG{o+ow}{is} \PYG{n}{a} \PYG{n}{text} \PYG{n}{annotation}\PYG{p}{)}
    \PYG{o}{/}\PYG{n}{ANN} \PYG{n}{pdfmark}
\end{sphinxVerbatim}
\begin{enumerate}
\sphinxsetlistlabels{\arabic}{enumi}{enumii}{}{.}%
\item {} 
Using OBJ to add an open action to a PDF File

\end{enumerate}

\begin{sphinxVerbatim}[commandchars=\\\{\}]
\PYG{o}{\PYGZpc{}} \PYG{n}{Go} \PYG{n}{to} \PYG{n}{the} \PYG{n}{fifth} \PYG{n}{page} \PYG{n}{of} \PYG{n}{a} \PYG{n}{document} \PYG{n}{upon} \PYG{n}{opening} \PYG{n}{it}\PYG{o}{.}
\PYG{o}{\PYGZpc{}} \PYG{n}{First} \PYG{o+ow}{and} \PYG{n}{third} \PYG{n}{lines} \PYG{n}{can} \PYG{n}{be} \PYG{n}{reused}\PYG{o}{.}
\PYG{o}{\PYGZpc{}} \PYG{n}{Second} \PYG{n}{line} \PYG{n}{specifies} \PYG{n}{the} \PYG{n}{GoTo} \PYG{n}{action}\PYG{p}{,} \PYG{n}{which} \PYG{n}{can} \PYG{n}{be} \PYG{n}{customized} \PYG{n}{easily}\PYG{o}{.}

\PYG{p}{[} \PYG{o}{/}\PYG{n}{\PYGZus{}objdef} \PYG{p}{\PYGZob{}}\PYG{n}{MyAction}\PYG{p}{\PYGZcb{}} \PYG{o}{/}\PYG{n+nb}{type} \PYG{o}{/}\PYG{n+nb}{dict} \PYG{o}{/}\PYG{n}{OBJ} \PYG{n}{pdfmark}
\PYG{p}{[} \PYG{p}{\PYGZob{}}\PYG{n}{MyAction}\PYG{p}{\PYGZcb{}} \PYG{o}{\PYGZlt{}\PYGZlt{}} \PYG{o}{/}\PYG{n}{S} \PYG{o}{/}\PYG{n}{GoTo} \PYG{o}{/}\PYG{n}{D} \PYG{p}{[} \PYG{p}{\PYGZob{}}\PYG{n}{Page5}\PYG{p}{\PYGZcb{}} \PYG{o}{/}\PYG{n}{FitH} \PYG{l+m+mi}{770} \PYG{p}{]} \PYG{o}{\PYGZgt{}\PYGZgt{}} \PYG{o}{/}\PYG{n}{PUT} \PYG{n}{pdfmark}
\PYG{p}{[} \PYG{p}{\PYGZob{}}\PYG{n}{Catalog}\PYG{p}{\PYGZcb{}} \PYG{o}{\PYGZlt{}\PYGZlt{}} \PYG{o}{/}\PYG{n}{OpenAction} \PYG{p}{\PYGZob{}}\PYG{n}{MyAction}\PYG{p}{\PYGZcb{}} \PYG{o}{\PYGZgt{}\PYGZgt{}} \PYG{o}{/}\PYG{n}{PUT} \PYG{n}{pdfmark}
\end{sphinxVerbatim}

\#. Using OBJ to create a base URI

\begin{sphinxVerbatim}[commandchars=\\\{\}]
\PYG{o}{\PYGZpc{}} \PYG{n}{Create} \PYG{n}{a} \PYG{n}{dictionary} \PYG{n+nb}{object}
\PYG{p}{[} \PYG{o}{/}\PYG{n}{\PYGZus{}objdef} \PYG{p}{\PYGZob{}}\PYG{n}{myURIdict}\PYG{p}{\PYGZcb{}} \PYG{o}{/}\PYG{n+nb}{type} \PYG{o}{/}\PYG{n+nb}{dict} \PYG{o}{/}\PYG{n}{OBJ} \PYG{n}{pdfmark}
\PYG{o}{\PYGZpc{}} \PYG{n}{Add} \PYG{n}{a} \PYG{l+s+s2}{\PYGZdq{}}\PYG{l+s+s2}{Base}\PYG{l+s+s2}{\PYGZdq{}} \PYG{n}{key}\PYG{o}{\PYGZhy{}}\PYG{n}{value} \PYG{n}{pair} \PYG{n}{to} \PYG{n}{the} \PYG{n}{dictionary} \PYG{n}{we} \PYG{n}{just} \PYG{n}{created}
\PYG{p}{[}   \PYG{p}{\PYGZob{}}\PYG{n}{myURIdict}\PYG{p}{\PYGZcb{}} \PYG{o}{\PYGZlt{}\PYGZlt{}} \PYG{o}{/}\PYG{n}{Base} \PYG{p}{(}\PYG{n}{http}\PYG{p}{:}\PYG{o}{/}\PYG{o}{/}\PYG{n}{www}\PYG{o}{.}\PYG{n}{adobe}\PYG{o}{.}\PYG{n}{com}\PYG{p}{)} \PYG{o}{\PYGZgt{}\PYGZgt{}} \PYG{o}{/}\PYG{n}{PUT} \PYG{n}{pdfmark}
\PYG{o}{\PYGZpc{}} \PYG{n}{Add} \PYG{n}{our} \PYG{n}{dictionary} \PYG{n}{to} \PYG{n}{the} \PYG{n}{PDF} \PYG{n}{file}\PYG{l+s+s1}{\PYGZsq{}}\PYG{l+s+s1}{s Catalog dictionary}
\PYG{p}{[}   \PYG{p}{\PYGZob{}}\PYG{n}{Catalog}\PYG{p}{\PYGZcb{}} \PYG{o}{\PYGZlt{}\PYGZlt{}} \PYG{o}{/}\PYG{n}{URI} \PYG{p}{\PYGZob{}}\PYG{n}{myURIdict}\PYG{p}{\PYGZcb{}} \PYG{o}{\PYGZgt{}\PYGZgt{}} \PYG{o}{/}\PYG{n}{PUT} \PYG{n}{pdfmark}
\end{sphinxVerbatim}
\begin{enumerate}
\sphinxsetlistlabels{\arabic}{enumi}{enumii}{}{.}%
\item {} 
Using OBJ and PUT pdfmarks to create an alternate image

\end{enumerate}

This example shows how to create alternate images. In this case, an image is created that has one Alternate. The Alternate is stored as a JPEG file on a web server, and is the default image used when printing.

\begin{sphinxVerbatim}[commandchars=\\\{\}]
\PYG{o}{\PYGZpc{}} \PYG{n}{Give} \PYG{n}{the} \PYG{n+nb}{next} \PYG{n}{image} \PYG{n}{a} \PYG{n}{name}\PYG{p}{,} \PYG{n}{so} \PYG{n}{we} \PYG{n}{can} \PYG{n}{add} \PYG{n}{an} \PYG{n}{Alternates} \PYG{n}{array} \PYG{n}{to} \PYG{n}{it} \PYG{n}{later}
\PYG{p}{[} \PYG{o}{/}\PYG{n}{\PYGZus{}objdef} \PYG{p}{\PYGZob{}}\PYG{n}{myImage}\PYG{p}{\PYGZcb{}} \PYG{o}{/}\PYG{n}{NI} \PYG{n}{pdfmark}

\PYG{o}{\PYGZpc{}} \PYG{n}{Create} \PYG{n}{the} \PYG{n}{base} \PYG{n}{image} \PYG{p}{(}\PYG{n}{just} \PYG{n}{a} \PYG{l+m+mi}{2}\PYG{n}{x1} \PYG{n}{pixel} \PYG{n}{grayscale} \PYG{n}{image} \PYG{k}{for} \PYG{n}{this} \PYG{n}{sample}\PYG{p}{)}

\PYG{o}{\PYGZlt{}\PYGZlt{}}
\PYG{o}{/}\PYG{n}{Width} \PYG{l+m+mi}{2}
\PYG{o}{/}\PYG{n}{Height} \PYG{l+m+mi}{1}
\PYG{o}{/}\PYG{n}{ImageMatrix} \PYG{p}{[}\PYG{l+m+mi}{1} \PYG{l+m+mi}{0} \PYG{l+m+mi}{0} \PYG{l+m+mi}{1} \PYG{l+m+mi}{0} \PYG{l+m+mi}{0}\PYG{p}{]}
\PYG{o}{/}\PYG{n}{ImageType} \PYG{l+m+mi}{1}
\PYG{o}{/}\PYG{n}{Decode} \PYG{p}{[}\PYG{l+m+mi}{0} \PYG{l+m+mi}{1}\PYG{p}{]}
\PYG{o}{/}\PYG{n}{BitsPerComponent} \PYG{l+m+mi}{8}
\PYG{o}{/}\PYG{n}{DataSource} \PYG{p}{(}\PYG{l+m+mi}{1}\PYG{n}{Z}\PYG{p}{)}
\PYG{o}{\PYGZgt{}\PYGZgt{}} \PYG{n}{image}

\PYG{o}{\PYGZpc{}} \PYG{n}{Create} \PYG{n}{a} \PYG{n}{stream} \PYG{k}{for} \PYG{n}{the} \PYG{n}{Alternate} \PYG{n}{Image}

\PYG{p}{[}\PYG{o}{/}\PYG{n}{\PYGZus{}objdef} \PYG{p}{\PYGZob{}}\PYG{n}{myPrintingImageStream}\PYG{p}{\PYGZcb{}} \PYG{o}{/}\PYG{n+nb}{type} \PYG{o}{/}\PYG{n}{stream} \PYG{o}{/}\PYG{n}{OBJ} \PYG{n}{pdfmark}

\PYG{o}{\PYGZpc{}} \PYG{n}{Add} \PYG{n}{the} \PYG{n}{necessary} \PYG{n}{key}\PYG{o}{\PYGZhy{}}\PYG{n}{value} \PYG{n}{pairs} \PYG{n}{to} \PYG{n}{the} \PYG{n}{stream} \PYG{n}{dictionary} \PYG{n}{to} \PYG{n}{make} \PYG{n}{it} \PYG{n}{a}
\PYG{o}{\PYGZpc{}} \PYG{n}{valid} \PYG{n}{image} \PYG{n}{XObject}\PYG{o}{.}
\PYG{o}{\PYGZpc{}} \PYG{n}{This} \PYG{n}{particular} \PYG{n}{image} \PYG{n}{XObject} \PYG{n}{uses} \PYG{n}{the} \PYG{n}{external} \PYG{n}{streams} \PYG{n}{capability} \PYG{n}{of} \PYG{n}{PDF}
\PYG{o}{\PYGZpc{}} \PYG{n}{to} \PYG{n}{point} \PYG{n}{to} \PYG{n}{an} \PYG{n}{image} \PYG{n}{stored} \PYG{n}{on} \PYG{n}{an} \PYG{n}{IIP} \PYG{n}{server}\PYG{p}{,} \PYG{n}{retrieving} \PYG{n}{it} \PYG{k}{as} \PYG{n}{a} \PYG{n}{JPEG} \PYG{n}{file}\PYG{o}{.}
\PYG{o}{\PYGZpc{}} \PYG{n}{Since} \PYG{n+nb}{all} \PYG{n}{stream} \PYG{n}{data} \PYG{o+ow}{is} \PYG{n}{stored} \PYG{n}{on} \PYG{n}{a} \PYG{n}{web} \PYG{n}{server}\PYG{p}{,} \PYG{n}{we} \PYG{n}{don}\PYG{l+s+s1}{\PYGZsq{}}\PYG{l+s+s1}{t explicitly add}
\PYG{o}{\PYGZpc{}} \PYG{n}{data} \PYG{n}{to} \PYG{n}{the} \PYG{n}{stream}\PYG{o}{.} \PYG{n}{As} \PYG{n}{a} \PYG{n}{result}\PYG{p}{,} \PYG{n}{the} \PYG{n}{stream} \PYG{n}{ends} \PYG{n}{up} \PYG{k}{with} \PYG{n}{a} \PYG{n}{length} \PYG{n}{of} \PYG{n}{zero}\PYG{p}{,}
\PYG{o}{\PYGZpc{}} \PYG{n}{which} \PYG{o+ow}{is} \PYG{n}{OK} \PYG{k}{for} \PYG{n}{external} \PYG{n}{streams}\PYG{o}{.}

\PYG{p}{[}   \PYG{p}{\PYGZob{}}\PYG{n}{myPrintingImageStream}\PYG{p}{\PYGZcb{}}
        \PYG{o}{\PYGZlt{}\PYGZlt{}}
        \PYG{o}{/}\PYG{n}{Type} \PYG{o}{/}\PYG{n}{XObject} \PYG{o}{/}\PYG{n}{Subtype} \PYG{o}{/}\PYG{n}{Image} \PYG{o}{/}\PYG{n}{Width} \PYG{l+m+mi}{150} \PYG{o}{/}\PYG{n}{Height} \PYG{l+m+mi}{150}
        \PYG{o}{/}\PYG{n}{FFilter} \PYG{o}{/}\PYG{n}{DCTDecode} \PYG{o}{/}\PYG{n}{ColorSpace} \PYG{o}{/}\PYG{n}{DeviceRGB} \PYG{o}{/}\PYG{n}{BitsPerComponent} \PYG{l+m+mi}{8}
        \PYG{o}{/}\PYG{n}{F} \PYG{o}{\PYGZlt{}\PYGZlt{}} \PYG{o}{/}\PYG{n}{FS} \PYG{o}{/}\PYG{n}{URL} \PYG{o}{/}\PYG{n}{F} \PYG{p}{(}\PYG{n}{http}\PYG{p}{:}\PYG{o}{/}\PYG{o}{/}\PYG{n}{www}\PYG{o}{.}\PYG{n}{mycompany}\PYG{o}{.}\PYG{n}{com}\PYG{o}{/}\PYG{n}{myfile}\PYG{o}{.}\PYG{n}{jpg}\PYG{p}{)} \PYG{o}{\PYGZgt{}\PYGZgt{}}
        \PYG{o}{\PYGZgt{}\PYGZgt{}}
    \PYG{o}{/}\PYG{n}{PUT} \PYG{n}{pdfmark}

\PYG{o}{\PYGZpc{}} \PYG{n}{Add} \PYG{n}{an} \PYG{n}{Alternates} \PYG{n}{array} \PYG{n}{to} \PYG{n}{the} \PYG{n}{base} \PYG{n}{image}

    \PYG{p}{[}   \PYG{p}{\PYGZob{}}\PYG{n}{myImage}\PYG{p}{\PYGZcb{}}
        \PYG{o}{\PYGZlt{}\PYGZlt{}}
        \PYG{o}{/}\PYG{n}{Alternates}
            \PYG{p}{[} \PYG{o}{\PYGZlt{}\PYGZlt{}}\PYG{o}{/}\PYG{n}{Image} \PYG{p}{\PYGZob{}}\PYG{n}{myPrintingImageStream}\PYG{p}{\PYGZcb{}} \PYG{o}{/}\PYG{n}{DefaultForPrinting} \PYG{n}{true} \PYG{o}{\PYGZgt{}\PYGZgt{}} \PYG{p}{]}
        \PYG{o}{\PYGZgt{}\PYGZgt{}}
    \PYG{o}{/}\PYG{n}{PUT} \PYG{n}{pdfmark}
\end{sphinxVerbatim}

There are two possibilities for alternate images:
\begin{itemize}
\item {} 
Alternate image data is outside the PDF file

\item {} 
Alternate image data is inside the PDF file

\end{itemize}

The above example shows only how to construct the first type. Note also that if the Alternate uses a different color space than the base image, it is possible that the PDF file may not contain the appropriate \sphinxcode{\sphinxupquote{ProcSet}} references in the Resources dictionary to print the page to PostScript. For example, if the base image is grayscale and the Alternate is \sphinxcode{\sphinxupquote{DeviceRGB}} , it is likely that the page’s Resources contains only the \sphinxcode{\sphinxupquote{ImageB}} ProcSet (for grayscale images) and not the \sphinxcode{\sphinxupquote{ImageC}} ProcSet (for color images).


\section{Forms examples}
\label{\detokenize{pdfmark_Examples:forms-examples}}
The examples in this section show how to use the Forms pdfmark suite.
\begin{enumerate}
\sphinxsetlistlabels{\arabic}{enumi}{enumii}{}{.}%
\item {} 
Define the AcroForm dictionary at the document Catalog

\end{enumerate}

The AcroForm dictionary includes these required entries (see the \sphinxstyleemphasis{PDF Reference} for more information):
\begin{itemize}
\item {} 
\sphinxcode{\sphinxupquote{Fields}} (the array from where all widgets in the form can be found)

\item {} 
\sphinxcode{\sphinxupquote{DA}} (Default Appearance)

\item {} 
\sphinxcode{\sphinxupquote{DR}} (Default Resources)

\item {} 
\sphinxcode{\sphinxupquote{NeedAppearances}} , set to true to indicate that when the document is opened, all widgets are traversed to generate their display and to add them to the \sphinxcode{\sphinxupquote{Fields}} array.

\end{itemize}

It also includes definitions of common objects that are used by the widgets such as fonts, encoding arrays, and Form \sphinxstyleemphasis{XObjects} for button faces.

\begin{sphinxVerbatim}[commandchars=\\\{\}]
\PYG{p}{[}    \PYG{o}{/}\PYG{n}{\PYGZus{}objdef} \PYG{p}{\PYGZob{}}\PYG{n}{pdfDocEncoding}\PYG{p}{\PYGZcb{}} \PYG{o}{/}\PYG{n+nb}{type} \PYG{o}{/}\PYG{n+nb}{dict} \PYG{o}{/}\PYG{n}{OBJ} \PYG{n}{pdfmark}
\PYG{p}{[}   \PYG{p}{\PYGZob{}}\PYG{n}{pdfDocEncoding}\PYG{p}{\PYGZcb{}}
        \PYG{o}{\PYGZlt{}\PYGZlt{}}
        \PYG{o}{/}\PYG{n}{Type} \PYG{o}{/}\PYG{n}{Encoding}
        \PYG{o}{/}\PYG{n}{Differences}
            \PYG{p}{[}
            \PYG{l+m+mi}{24}  \PYG{o}{/}\PYG{n}{breve} \PYG{o}{/}\PYG{n}{caron} \PYG{o}{/}\PYG{n}{circumflex} \PYG{o}{/}\PYG{n}{dotaccent} \PYG{o}{/}\PYG{n}{hungarumlaut} \PYG{o}{/}\PYG{n}{ogonek} \PYG{o}{/}\PYG{n}{ring}
                \PYG{o}{/}\PYG{n}{tilde}
            \PYG{l+m+mi}{39} \PYG{o}{/}\PYG{n}{quotesingle}
            \PYG{l+m+mi}{96} \PYG{o}{/}\PYG{n}{grave}
            \PYG{l+m+mi}{128} \PYG{o}{/}\PYG{n}{bullet} \PYG{o}{/}\PYG{n}{dagger} \PYG{o}{/}\PYG{n}{daggerdbl} \PYG{o}{/}\PYG{n}{ellipsis} \PYG{o}{/}\PYG{n}{emdash} \PYG{o}{/}\PYG{n}{endash} \PYG{o}{/}\PYG{n}{florin}
                \PYG{o}{/}\PYG{n}{fraction} \PYG{o}{/}\PYG{n}{guilsinglleft} \PYG{o}{/}\PYG{n}{guilsinglright} \PYG{o}{/}\PYG{n}{minus} \PYG{o}{/}\PYG{n}{perthousand}
                \PYG{o}{/}\PYG{n}{quotedblbase} \PYG{o}{/}\PYG{n}{quotedblleft} \PYG{o}{/}\PYG{n}{quotedblright} \PYG{o}{/}\PYG{n}{quoteleft} \PYG{o}{/}\PYG{n}{quoteright}
                \PYG{o}{/}\PYG{n}{quotesinglbase} \PYG{o}{/}\PYG{n}{trademark} \PYG{o}{/}\PYG{n}{fi} \PYG{o}{/}\PYG{n}{fl} \PYG{o}{/}\PYG{n}{Lslash} \PYG{o}{/}\PYG{n}{OE} \PYG{o}{/}\PYG{n}{Scaron} \PYG{o}{/}\PYG{n}{Ydieresis}
                \PYG{o}{/}\PYG{n}{Zcaron} \PYG{o}{/}\PYG{n}{dotlessi} \PYG{o}{/}\PYG{n}{lslash} \PYG{o}{/}\PYG{n}{oe} \PYG{o}{/}\PYG{n}{scaron} \PYG{o}{/}\PYG{n}{zcaron}
            \PYG{l+m+mi}{164} \PYG{o}{/}\PYG{n}{currency}
            \PYG{l+m+mi}{166} \PYG{o}{/}\PYG{n}{brokenbar}
            \PYG{l+m+mi}{168} \PYG{o}{/}\PYG{n}{dieresis} \PYG{o}{/}\PYG{n}{copyright} \PYG{o}{/}\PYG{n}{ordfeminine}
            \PYG{l+m+mi}{172} \PYG{o}{/}\PYG{n}{logicalnot} \PYG{o}{/}\PYG{o}{.}\PYG{n}{notdef} \PYG{o}{/}\PYG{n}{registered} \PYG{o}{/}\PYG{n}{macron} \PYG{o}{/}\PYG{n}{degree} \PYG{o}{/}\PYG{n}{plusminus}
                \PYG{o}{/}\PYG{n}{twosuperior} \PYG{o}{/}\PYG{n}{threesuperior} \PYG{o}{/}\PYG{n}{acute} \PYG{o}{/}\PYG{n}{mu}
            \PYG{l+m+mi}{183} \PYG{o}{/}\PYG{n}{periodcentered} \PYG{o}{/}\PYG{n}{cedilla} \PYG{o}{/}\PYG{n}{onesuperior} \PYG{o}{/}\PYG{n}{ordmasculine}
            \PYG{l+m+mi}{188} \PYG{o}{/}\PYG{n}{onequarter} \PYG{o}{/}\PYG{n}{onehalf} \PYG{o}{/}\PYG{n}{threequarters}
            \PYG{l+m+mi}{192} \PYG{o}{/}\PYG{n}{Agrave} \PYG{o}{/}\PYG{n}{Aacute} \PYG{o}{/}\PYG{n}{Acircumflex} \PYG{o}{/}\PYG{n}{Atilde} \PYG{o}{/}\PYG{n}{Adieresis} \PYG{o}{/}\PYG{n}{Aring} \PYG{o}{/}\PYG{n}{AE}
                \PYG{o}{/}\PYG{n}{Ccedilla} \PYG{o}{/}\PYG{n}{Egrave} \PYG{o}{/}\PYG{n}{Eacute} \PYG{o}{/}\PYG{n}{Ecircumflex} \PYG{o}{/}\PYG{n}{Edieresis} \PYG{o}{/}\PYG{n}{Igrave} \PYG{o}{/}\PYG{n}{Iacute}
                \PYG{o}{/}\PYG{n}{Icircumflex} \PYG{o}{/}\PYG{n}{Idieresis} \PYG{o}{/}\PYG{n}{Eth} \PYG{o}{/}\PYG{n}{Ntilde} \PYG{o}{/}\PYG{n}{Ograve} \PYG{o}{/}\PYG{n}{Oacute} \PYG{o}{/}\PYG{n}{Ocircumflex}
                \PYG{o}{/}\PYG{n}{Otilde} \PYG{o}{/}\PYG{n}{Odieresis} \PYG{o}{/}\PYG{n}{multiply} \PYG{o}{/}\PYG{n}{Oslash} \PYG{o}{/}\PYG{n}{Ugrave} \PYG{o}{/}\PYG{n}{Uacute} \PYG{o}{/}\PYG{n}{Ucircumflex}
                \PYG{o}{/}\PYG{n}{Udieresis} \PYG{o}{/}\PYG{n}{Yacute} \PYG{o}{/}\PYG{n}{Thorn} \PYG{o}{/}\PYG{n}{germandbls} \PYG{o}{/}\PYG{n}{agrave} \PYG{o}{/}\PYG{n}{aacute} \PYG{o}{/}\PYG{n}{acircumflex}
                \PYG{o}{/}\PYG{n}{atilde} \PYG{o}{/}\PYG{n}{adieresis} \PYG{o}{/}\PYG{n}{aring} \PYG{o}{/}\PYG{n}{ae} \PYG{o}{/}\PYG{n}{ccedilla} \PYG{o}{/}\PYG{n}{egrave} \PYG{o}{/}\PYG{n}{eacute}
                \PYG{o}{/}\PYG{n}{ecircumflex} \PYG{o}{/}\PYG{n}{edieresis} \PYG{o}{/}\PYG{n}{igrave} \PYG{o}{/}\PYG{n}{iacute} \PYG{o}{/}\PYG{n}{icircumflex} \PYG{o}{/}\PYG{n}{idieresis}
                \PYG{o}{/}\PYG{n}{eth} \PYG{o}{/}\PYG{n}{ntilde} \PYG{o}{/}\PYG{n}{ograve} \PYG{o}{/}\PYG{n}{oacute} \PYG{o}{/}\PYG{n}{ocircumflex} \PYG{o}{/}\PYG{n}{otilde} \PYG{o}{/}\PYG{n}{odieresis}
                \PYG{o}{/}\PYG{n}{divide} \PYG{o}{/}\PYG{n}{oslash} \PYG{o}{/}\PYG{n}{ugrave} \PYG{o}{/}\PYG{n}{uacute} \PYG{o}{/}\PYG{n}{ucircumflex} \PYG{o}{/}\PYG{n}{udieresis} \PYG{o}{/}\PYG{n}{yacute}
                \PYG{o}{/}\PYG{n}{thorn} \PYG{o}{/}\PYG{n}{ydieresis}
            \PYG{p}{]}
        \PYG{o}{\PYGZgt{}\PYGZgt{}}
    \PYG{o}{/}\PYG{n}{PUT} \PYG{n}{pdfmark}


\PYG{p}{[} \PYG{o}{/}\PYG{n}{\PYGZus{}objdef} \PYG{p}{\PYGZob{}}\PYG{n}{ZaDb}\PYG{p}{\PYGZcb{}} \PYG{o}{/}\PYG{n+nb}{type} \PYG{o}{/}\PYG{n+nb}{dict} \PYG{o}{/}\PYG{n}{OBJ} \PYG{n}{pdfmark}
\PYG{p}{[}   \PYG{p}{\PYGZob{}}\PYG{n}{ZaDb}\PYG{p}{\PYGZcb{}}
        \PYG{o}{\PYGZlt{}\PYGZlt{}}
        \PYG{o}{/}\PYG{n}{Type} \PYG{o}{/}\PYG{n}{Font}
        \PYG{o}{/}\PYG{n}{Subtype} \PYG{o}{/}\PYG{n}{Type1}
        \PYG{o}{/}\PYG{n}{Name} \PYG{o}{/}\PYG{n}{ZaDb}
        \PYG{o}{/}\PYG{n}{BaseFont} \PYG{o}{/}\PYG{n}{ZapfDingbats}
        \PYG{o}{\PYGZgt{}\PYGZgt{}}
    \PYG{o}{/}\PYG{n}{PUT} \PYG{n}{pdfmark}


\PYG{p}{[} \PYG{o}{/}\PYG{n}{\PYGZus{}objdef} \PYG{p}{\PYGZob{}}\PYG{n}{Helv}\PYG{p}{\PYGZcb{}} \PYG{o}{/}\PYG{n+nb}{type} \PYG{o}{/}\PYG{n+nb}{dict} \PYG{o}{/}\PYG{n}{OBJ} \PYG{n}{pdfmark}
\PYG{p}{[}   \PYG{p}{\PYGZob{}}\PYG{n}{Helv}\PYG{p}{\PYGZcb{}}
        \PYG{o}{\PYGZlt{}\PYGZlt{}}
        \PYG{o}{/}\PYG{n}{Type} \PYG{o}{/}\PYG{n}{Font}
        \PYG{o}{/}\PYG{n}{Subtype} \PYG{o}{/}\PYG{n}{Type1}
        \PYG{o}{/}\PYG{n}{Name} \PYG{o}{/}\PYG{n}{Helv}
        \PYG{o}{/}\PYG{n}{BaseFont} \PYG{o}{/}\PYG{n}{Helvetica}
        \PYG{o}{/}\PYG{n}{Encoding} \PYG{p}{\PYGZob{}}\PYG{n}{pdfDocEncoding}\PYG{p}{\PYGZcb{}}
        \PYG{o}{\PYGZgt{}\PYGZgt{}}
    \PYG{o}{/}\PYG{n}{PUT} \PYG{n}{pdfmark}


\PYG{p}{[} \PYG{o}{/}\PYG{n}{\PYGZus{}objdef} \PYG{p}{\PYGZob{}}\PYG{n}{aform}\PYG{p}{\PYGZcb{}} \PYG{o}{/}\PYG{n+nb}{type} \PYG{o}{/}\PYG{n+nb}{dict} \PYG{o}{/}\PYG{n}{OBJ} \PYG{n}{pdfmark}


\PYG{o}{\PYGZpc{}} \PYG{n}{Define} \PYG{n}{Fields} \PYG{n}{array} \PYG{n}{of} \PYG{n}{Acroform} \PYG{n}{dictionary}\PYG{o}{.} \PYG{n}{It} \PYG{n}{will} \PYG{n}{contain} \PYG{n}{entries} \PYG{k}{for}
\PYG{o}{\PYGZpc{}} \PYG{n}{each} \PYG{n}{of} \PYG{n}{the} \PYG{n}{widgets} \PYG{n}{defined} \PYG{n}{below}\PYG{o}{.}
\PYG{o}{\PYGZpc{}} \PYG{n}{NOTE}\PYG{p}{:} \PYG{n}{It} \PYG{o+ow}{is} \PYG{o+ow}{not} \PYG{n}{necessary} \PYG{n}{to} \PYG{n}{explicitly} \PYG{n}{assign} \PYG{n}{the} \PYG{n}{widget} \PYG{n}{annotations}
\PYG{o}{\PYGZpc{}} \PYG{n}{to} \PYG{n}{the} \PYG{n}{Fields} \PYG{n}{array}\PYG{p}{;} \PYG{n}{Acrobat} \PYG{n}{does} \PYG{n}{it} \PYG{n}{automatically} \PYG{n}{when} \PYG{n}{the} \PYG{n}{file} \PYG{o+ow}{is} \PYG{n}{opened}\PYG{o}{.}


\PYG{p}{[} \PYG{o}{/}\PYG{n}{\PYGZus{}objdef} \PYG{p}{\PYGZob{}}\PYG{n}{afields}\PYG{p}{\PYGZcb{}} \PYG{o}{/}\PYG{n+nb}{type} \PYG{o}{/}\PYG{n}{array} \PYG{o}{/}\PYG{n}{OBJ} \PYG{n}{pdfmark}


\PYG{p}{[}   \PYG{p}{\PYGZob{}}\PYG{n}{aform}\PYG{p}{\PYGZcb{}}
        \PYG{o}{\PYGZlt{}\PYGZlt{}}
        \PYG{o}{/}\PYG{n}{Fields} \PYG{p}{\PYGZob{}}\PYG{n}{afields}\PYG{p}{\PYGZcb{}}
        \PYG{o}{/}\PYG{n}{DR} \PYG{o}{\PYGZlt{}\PYGZlt{}} \PYG{o}{/}\PYG{n}{Font} \PYG{o}{\PYGZlt{}\PYGZlt{}} \PYG{o}{/}\PYG{n}{ZaDb} \PYG{p}{\PYGZob{}}\PYG{n}{ZaDb}\PYG{p}{\PYGZcb{}} \PYG{o}{/}\PYG{n}{Helv} \PYG{p}{\PYGZob{}}\PYG{n}{Helv}\PYG{p}{\PYGZcb{}} \PYG{o}{\PYGZgt{}\PYGZgt{}} \PYG{o}{\PYGZgt{}\PYGZgt{}}
        \PYG{o}{/}\PYG{n}{DA} \PYG{p}{(}\PYG{o}{/}\PYG{n}{Helv} \PYG{l+m+mi}{0} \PYG{n}{Tf} \PYG{l+m+mi}{0} \PYG{n}{g}\PYG{p}{)}
        \PYG{o}{/}\PYG{n}{NeedAppearances} \PYG{n}{true}
        \PYG{o}{\PYGZgt{}\PYGZgt{}}
    \PYG{o}{/}\PYG{n}{PUT} \PYG{n}{pdfmark}


\PYG{o}{\PYGZpc{}} \PYG{n}{Put} \PYG{n}{Acroform} \PYG{n}{entry} \PYG{o+ow}{in} \PYG{n}{catalog} \PYG{n}{dictionary}

\PYG{p}{[}   \PYG{p}{\PYGZob{}}\PYG{n}{Catalog}\PYG{p}{\PYGZcb{}} \PYG{o}{\PYGZlt{}\PYGZlt{}} \PYG{o}{/}\PYG{n}{AcroForm} \PYG{p}{\PYGZob{}}\PYG{n}{aform}\PYG{p}{\PYGZcb{}} \PYG{o}{\PYGZgt{}\PYGZgt{}} \PYG{o}{/}\PYG{n}{PUT} \PYG{n}{pdfmark}
\end{sphinxVerbatim}
\begin{enumerate}
\sphinxsetlistlabels{\arabic}{enumi}{enumii}{}{.}%
\item {} 
Define the Widget annotations, which are also field dictionaries for this form

\end{enumerate}

This is the collection of all individual widget annotations. It is possible to have multiple instances of these sections, such as for defining a single widget on each instance.

\begin{sphinxVerbatim}[commandchars=\\\{\}]
[    /Subtype /Widget
    /Rect [216 647 361 684]
    /F 4
    /T (SL Text)
    /FT /Tx
    /DA (/Helv 14 Tf 0 0 1 rg)
    /V (5)
    /AA \PYGZlt{}\PYGZlt{}
        /K \PYGZlt{}\PYGZlt{} /S /JavaScript /JS (AFNumber\PYGZus{}Keystroke(2, 0, 0, 0, \PYGZdq{}\PYGZdl{}\PYGZdq{}, true);)\PYGZgt{}\PYGZgt{}
        /F \PYGZlt{}\PYGZlt{} /S /JavaScript /JS (AFNumber\PYGZus{}Format(2, 0, 0, 0, \PYGZdq{}\PYGZdl{}\PYGZdq{}, true); \PYGZgt{}\PYGZgt{}
        \PYGZgt{}\PYGZgt{}
    /ANN pdfmark
    [ /Subtype /Widget
    /Rect [216 503 361 612]
    /F 4
    /T (Ping Result)
    /FT /Tx
    /DA (/Helv 0 Tf 0 0 1 rg)
    /Ff 4096
    /ANN pdfmark



[ /Subtype /Widget
    /Rect [216 432 252 468]
    /F 4
    /T (Check Box)
    /FT /Btn
    /DA (/ZaDb 0 Tf 0 g)
    /AS /Off
    /MK \PYGZlt{}\PYGZlt{} /CA (4)\PYGZgt{}\PYGZgt{}
    /AP \PYGZlt{}\PYGZlt{} /N \PYGZlt{}\PYGZlt{} /Oui /null \PYGZgt{}\PYGZgt{} \PYGZgt{}\PYGZgt{}
    /ANN pdfmark


[ /Subtype /Widget
    /Rect [216 360 252 396]
    /F 4
    /T (Radio)
    /FT /Btn
    /DA (/ZaDb 0 Tf 0 g)
    /Ff 49152
    /AS /Off
    /MK \PYGZlt{}\PYGZlt{} /CA (8)\PYGZgt{}\PYGZgt{}
    /AP \PYGZlt{}\PYGZlt{} /N \PYGZlt{}\PYGZlt{} /V1 /null \PYGZgt{}\PYGZgt{} \PYGZgt{}\PYGZgt{}
    /ANN pdfmark


[ /Subtype /Widget
    /Rect [ 261 360 297 396 ]
    /F 4
    /T (Radio)
    /FT /Btn
    /DA (/ZaDb 0 Tf 0 g)
    /Ff 49152
    /AS /Off
    /MK \PYGZlt{}\PYGZlt{} /CA (8)\PYGZgt{}\PYGZgt{}
    /AP \PYGZlt{}\PYGZlt{} /N \PYGZlt{}\PYGZlt{} /V2 /null \PYGZgt{}\PYGZgt{} \PYGZgt{}\PYGZgt{}
    /ANN pdfmark


[ /Subtype /Widget
    /Rect [ 306 360 342 396 ]
    /F 4
    /T (Radio)
    /FT /Btn
    /DA (/ZaDb 0 Tf 0 g)
    /Ff 49152
    /AS /Off
    /MK \PYGZlt{}\PYGZlt{} /CA (8)\PYGZgt{}\PYGZgt{}
    /AP \PYGZlt{}\PYGZlt{} /N \PYGZlt{}\PYGZlt{} /V3 /null \PYGZgt{}\PYGZgt{} \PYGZgt{}\PYGZgt{}
    /ANN pdfmark


[ /Subtype /Widget
    /Rect [ 351 360 387 396 ]
    /F 4
    /T (Radio)
    /FT /Btn
    /DA (/ZaDb 0 Tf 0 g)
    /Ff 49152
    /AS /Off
    /MK \PYGZlt{}\PYGZlt{} /CA (8)\PYGZgt{}\PYGZgt{}
    /AP \PYGZlt{}\PYGZlt{} /N \PYGZlt{}\PYGZlt{} /V4 /null \PYGZgt{}\PYGZgt{} \PYGZgt{}\PYGZgt{}
    /ANN pdfmark


[ /Subtype /Widget
    /Rect [216 287 361 324]
    /F 4
    /T (Pop Down)
    /FT /Ch
    /Ff 131072
    /Opt [ [(1)(First)] [(2)(Second)] [(3)(Third)] [(4)(Fourth)] [(5)(Fifth)]]
    /DV (5)
    /V (5)
    /DA (/TiIt 18 Tf 0 0 1 rg)
    /ANN pdfmark


[ /Subtype /Widget
    /Rect [216 215 361 252]
    /F 4
    /T (Combo)
    /FT /Ch
    /Ff 917504
    /Opt [ (Black)(Blue)(Green)(Pink)(Red)(White)]
    /DA (/TiRo 18 Tf 0 g )
    /V (Black)
    /DV (Black)
    /ANN pdfmark


[ /Subtype /Widget
    /Rect [216 107 253 180]
    /F 4
    /T (ListBox)
    /FT /Ch
    /DA (/Helv 10 Tf 1 0 0 rg)
    /Opt [(1)(2)(3)(4)(5)]
    /DV (3)
    /V (3)
    /ANN pdfmark


\PYGZpc{} Example of how the /MK dictionary is used.
\PYGZpc{} Notice that the text will be shown upside\PYGZhy{}down (180 degree rotation).

[ /Subtype /Widget
    /Rect [ 430 110 570 150 ]
    /F 4
    /T (Clear)
    /FT /Btn
    /H /P
    /DA (/HeBo 18 Tf 0 0 1 rg)
    /Ff 65536
    /MK \PYGZlt{}\PYGZlt{}
        /BC [ 1 0 0 ]
        /BG [ 0.75 0.45 0.75 ]
        /CA (Clear)
        /AC (Done!)
        /R 180
    \PYGZgt{}\PYGZgt{}
    /BS \PYGZlt{}\PYGZlt{}
        /W 3
        /S /I
        \PYGZgt{}\PYGZgt{}
    /A \PYGZlt{}\PYGZlt{} /S /ResetForm \PYGZgt{}\PYGZgt{}
    /ANN pdfmark
\end{sphinxVerbatim}


\section{Structure examples}
\label{\detokenize{pdfmark_Examples:structure-examples}}
This section provides examples of various uses of the structure pdfmark suite. The first example shows an entire structure tree, consisting of one section containing two paragraphs. It illustrates both how to create the tree structure and how the structure is related to the page content of the PDF file. The second example shows the parts of the output PDF file that result from the PostScript language code. Other examples follow.
\begin{enumerate}
\sphinxsetlistlabels{\arabic}{enumi}{enumii}{}{.}%
\item {} 
A simple structure

\end{enumerate}

This example has one section with two paragraphs, all on one page.

\begin{sphinxVerbatim}[commandchars=\\\{\}]
   \PYG{o}{\PYGZpc{}} \PYG{n}{On} \PYG{n}{the} \PYG{n}{first} \PYG{n}{page}\PYG{p}{:}
   \PYG{o}{\PYGZpc{}} \PYG{n}{Start} \PYG{n}{a} \PYG{n}{section} \PYG{k}{with} \PYG{n}{the} \PYG{n}{unnamed} \PYG{n}{Structure} \PYG{n}{Tree} \PYG{k}{as} \PYG{n}{parent}\PYG{o}{.}
   \PYG{o}{\PYGZpc{}} \PYG{n}{Push} \PYG{n}{the} \PYG{n}{Section} \PYG{n}{element} \PYG{n}{onto} \PYG{n}{the} \PYG{n}{implicit} \PYG{n}{parent} \PYG{n}{stack} \PYG{k}{as}
   \PYG{o}{\PYGZpc{}} \PYG{n}{current} \PYG{n}{implicit} \PYG{n}{parent}\PYG{o}{.}

   \PYG{p}{[} \PYG{o}{/}\PYG{n}{Subtype} \PYG{o}{/}\PYG{n}{Section} \PYG{o}{/}\PYG{n}{StPNE} \PYG{n}{pdfmark}

   \PYG{o}{\PYGZpc{}} \PYG{n}{Start} \PYG{n}{a} \PYG{n}{paragraph} \PYG{k}{with} \PYG{n}{the} \PYG{n}{Section} \PYG{k}{as} \PYG{n}{implicit} \PYG{n}{parent}\PYG{o}{.}
   \PYG{o}{\PYGZpc{}} \PYG{n}{Push} \PYG{n}{the} \PYG{n}{Paragraph} \PYG{n}{element} \PYG{n}{on} \PYG{n}{top} \PYG{n}{of} \PYG{n}{the} \PYG{n}{implicit} \PYG{n}{parent}
   \PYG{o}{\PYGZpc{}} \PYG{n}{stack} \PYG{k}{as} \PYG{n}{the} \PYG{n}{current} \PYG{n}{implicit} \PYG{n}{parent}\PYG{o}{.}

   \PYG{p}{[} \PYG{o}{/}\PYG{n}{Subtype} \PYG{o}{/}\PYG{n}{P} \PYG{o}{/}\PYG{n}{StPNE} \PYG{n}{pdfmark}

   \PYG{o}{\PYGZpc{}} \PYG{n}{Begin} \PYG{n}{the} \PYG{n}{marked} \PYG{n}{content} \PYG{n}{holding} \PYG{n}{the} \PYG{n}{text} \PYG{n}{of} \PYG{n}{the}
   \PYG{o}{\PYGZpc{}} \PYG{n}{first} \PYG{n}{paragraph}\PYG{o}{.} \PYG{n}{It} \PYG{o+ow}{is} \PYG{n}{implicitly} \PYG{n}{added} \PYG{n}{to} \PYG{n}{the} \PYG{n}{Paragraph}
   \PYG{o}{\PYGZpc{}} \PYG{n}{element}\PYG{o}{.}

   \PYG{p}{[} \PYG{o}{/}\PYG{n}{StBMC} \PYG{n}{pdfmark}

   \PYG{o}{\PYGZpc{}} \PYG{p}{[}\PYG{n}{PostScript} \PYG{n}{code} \PYG{k}{for} \PYG{n}{the} \PYG{n}{contents} \PYG{n}{of} \PYG{n}{the} \PYG{n}{first} \PYG{n}{paragraph}
   \PYG{o}{\PYGZpc{}} \PYG{n}{goes} \PYG{n}{here}\PYG{o}{.}\PYG{p}{]}

   \PYG{o}{\PYGZpc{}} \PYG{n}{End} \PYG{n}{the} \PYG{n}{marked} \PYG{n}{content} \PYG{n}{holding} \PYG{n}{the} \PYG{n}{text} \PYG{n}{of} \PYG{n}{the} \PYG{n}{first}
   \PYG{o}{\PYGZpc{}} \PYG{n}{paragraph}\PYG{o}{.}

   \PYG{p}{[} \PYG{o}{/}\PYG{n}{EMC} \PYG{n}{pdfmark}

   \PYG{o}{\PYGZpc{}} \PYG{n}{Pop} \PYG{n}{the} \PYG{n}{Paragraph} \PYG{n}{element} \PYG{n}{off} \PYG{n}{the} \PYG{n}{implicit} \PYG{n}{parent} \PYG{n}{stack}\PYG{o}{.}
   \PYG{o}{\PYGZpc{}} \PYG{n}{This} \PYG{n}{exposes} \PYG{n}{the} \PYG{n}{Section} \PYG{n}{element} \PYG{k}{as} \PYG{n}{implicit} \PYG{n}{parent} \PYG{n}{again}\PYG{o}{.}

   \PYG{p}{[} \PYG{o}{/}\PYG{n}{StPop} \PYG{n}{pdfmark}

   \PYG{o}{\PYGZpc{}} \PYG{n}{And} \PYG{n}{now} \PYG{k}{for} \PYG{n}{the} \PYG{n}{second} \PYG{n}{paragraph}\PYG{p}{:}

   \PYG{p}{[} \PYG{o}{/}\PYG{n}{Subtype} \PYG{o}{/}\PYG{n}{P} \PYG{o}{/}\PYG{n}{StPNE} \PYG{n}{pdfmark}


   \PYG{p}{[} \PYG{o}{/}\PYG{n}{StBMC} \PYG{n}{pdfmark}

   \PYG{o}{\PYGZpc{}} \PYG{n}{PostScript} \PYG{n}{code} \PYG{k}{for} \PYG{n}{the} \PYG{n}{contents} \PYG{n}{of} \PYG{n}{the} \PYG{n}{second} \PYG{n}{paragraph} \PYG{n}{goes} \PYG{n}{here}\PYG{o}{.}

   \PYG{p}{[} \PYG{o}{/}\PYG{n}{EMC} \PYG{n}{pdfmark}

\PYG{o}{\PYGZpc{}} \PYG{n}{We}\PYG{l+s+s1}{\PYGZsq{}}\PYG{l+s+s1}{re being tidy by popping both the second Paragraph}
\PYG{o}{|}  \PYG{o}{\PYGZpc{}} \PYG{n}{element} \PYG{o+ow}{and} \PYG{n}{the} \PYG{n}{Section} \PYG{n}{element} \PYG{n}{off} \PYG{n}{the} \PYG{n}{stack}\PYG{o}{.} \PYG{n}{We} \PYG{n}{could} \PYG{n}{have}
\PYG{o}{|} \PYG{o}{\PYGZpc{}} \PYG{n}{left} \PYG{n}{everything} \PYG{n}{hanging} \PYG{n}{at} \PYG{n}{the} \PYG{n}{end} \PYG{n}{of} \PYG{n}{the} \PYG{n}{document}\PYG{p}{,} \PYG{o+ow}{or} \PYG{n}{used}

\PYG{o}{\PYGZpc{}} \PYG{p}{[} \PYG{o}{/}\PYG{n}{StPopAll} \PYG{n}{pdfmark} \PYG{n}{instead}\PYG{o}{.}
\end{sphinxVerbatim}

\begin{sphinxVerbatim}[commandchars=\\\{\}]
\PYG{p}{[} \PYG{o}{/}\PYG{n}{StPop} \PYG{n}{pdfmark}
\PYG{p}{[} \PYG{o}{/}\PYG{n}{StPop} \PYG{n}{pdfmark}
\end{sphinxVerbatim}
\begin{enumerate}
\sphinxsetlistlabels{\arabic}{enumi}{enumii}{}{.}%
\item {} 
PDF output resulting from code in previous example

\end{enumerate}

This example is for illustration purposes only. The PDF code actually produced by Distiller would not include comments and would differ in other ways.

\begin{sphinxVerbatim}[commandchars=\\\{\}]
\PYG{o}{\PYGZpc{}} \PYG{n}{In} \PYG{n}{the} \PYG{n}{Catalog} \PYG{n}{dictionary}\PYG{p}{,} \PYG{n}{under} \PYG{n}{the} \PYG{n}{key} \PYG{n}{StructTreeRoot}\PYG{p}{,}
\PYG{o}{\PYGZpc{}} \PYG{n}{the} \PYG{n}{following} \PYG{n}{dictionary} \PYG{o+ow}{is} \PYG{n}{entered} \PYG{k}{as} \PYG{n+nb}{object} \PYG{l+m+mi}{3} \PYG{l+m+mi}{0}\PYG{p}{:}
\PYG{l+m+mi}{3} \PYG{l+m+mi}{0} \PYG{n}{obj}
\PYG{o}{\PYGZlt{}\PYGZlt{}}\PYG{o}{/}\PYG{n}{Type} \PYG{o}{/}\PYG{n}{StructTreeRoot}
\PYG{o}{\PYGZpc{}} \PYG{n}{The} \PYG{n}{Section} \PYG{n}{element} \PYG{o+ow}{is} \PYG{n}{the} \PYG{n}{only} \PYG{n}{child}\PYG{o}{.}
\PYG{o}{/}\PYG{n}{K} \PYG{p}{[}\PYG{l+m+mi}{4} \PYG{l+m+mi}{0} \PYG{n}{R}\PYG{p}{]}
\PYG{o}{/}\PYG{n}{ParentTree} \PYG{l+m+mi}{100} \PYG{l+m+mi}{0} \PYG{n}{R}
\PYG{o}{\PYGZgt{}\PYGZgt{}} \PYG{n}{endobj}

\PYG{o}{\PYGZpc{}} \PYG{n}{The} \PYG{n}{number} \PYG{n}{tree} \PYG{n}{that} \PYG{n}{locates} \PYG{n}{structure} \PYG{n}{parents} \PYG{n}{of} \PYG{n}{marked} \PYG{n}{content}\PYG{o}{.}
\PYG{l+m+mi}{100} \PYG{l+m+mi}{0} \PYG{n}{obj}
\PYG{o}{\PYGZlt{}\PYGZlt{}}\PYG{o}{/}\PYG{n}{Nums} \PYG{p}{[}\PYG{l+m+mi}{0} \PYG{l+m+mi}{101} \PYG{l+m+mi}{0} \PYG{n}{R}\PYG{p}{]}
\PYG{o}{\PYGZgt{}\PYGZgt{}}
\PYG{n}{endobj}

\PYG{o}{\PYGZpc{}} \PYG{n}{Structure} \PYG{n}{parents} \PYG{k}{for} \PYG{n}{page} \PYG{l+m+mf}{1.}
\PYG{l+m+mi}{101} \PYG{l+m+mi}{0} \PYG{n}{obj}
\PYG{p}{[}\PYG{l+m+mi}{5} \PYG{l+m+mi}{0} \PYG{n}{R} \PYG{l+m+mi}{6} \PYG{l+m+mi}{0} \PYG{n}{R}\PYG{p}{]}
\PYG{n}{endobj}
\PYG{o}{\PYGZpc{}} \PYG{n}{End} \PYG{n}{of} \PYG{n}{parent} \PYG{n}{tree} \PYG{n}{objects}\PYG{o}{.}
\PYG{o}{\PYGZpc{}} \PYG{n}{As} \PYG{n+nb}{object} \PYG{l+m+mi}{4} \PYG{l+m+mi}{0}\PYG{p}{,} \PYG{n}{the} \PYG{n}{following} \PYG{n}{dictionary} \PYG{n}{represents} \PYG{n}{the}
\PYG{o}{\PYGZpc{}} \PYG{n}{Section} \PYG{n}{element}\PYG{p}{:}
\PYG{l+m+mi}{4} \PYG{l+m+mi}{0} \PYG{n}{obj}
\PYG{o}{\PYGZlt{}\PYGZlt{}}\PYG{o}{/}\PYG{n}{Type} \PYG{o}{/}\PYG{n}{StructElement}
\PYG{o}{/}\PYG{n}{S} \PYG{o}{/}\PYG{n}{Section}
\PYG{o}{\PYGZpc{}} \PYG{n}{Parent} \PYG{n}{link} \PYG{n}{refers} \PYG{n}{back} \PYG{n}{to} \PYG{n}{the} \PYG{n}{dictionary} \PYG{n}{representing} \PYG{n}{the}
\PYG{o}{\PYGZpc{}} \PYG{n}{Structure} \PYG{n}{Tree} \PYG{n}{Root}\PYG{o}{.}
\PYG{o}{/}\PYG{n}{P} \PYG{l+m+mi}{3} \PYG{l+m+mi}{0} \PYG{n}{R}
\PYG{o}{\PYGZpc{}} \PYG{n}{The} \PYG{n}{Section} \PYG{n}{element} \PYG{n}{has} \PYG{n}{two} \PYG{n}{Paragraph} \PYG{n}{elements} \PYG{k}{as} \PYG{n}{children}\PYG{o}{.}
\PYG{o}{/}\PYG{n}{K} \PYG{p}{[}\PYG{l+m+mi}{5} \PYG{l+m+mi}{0} \PYG{n}{R} \PYG{l+m+mi}{6} \PYG{l+m+mi}{0} \PYG{n}{R}\PYG{p}{]}
\PYG{o}{\PYGZgt{}\PYGZgt{}} \PYG{n}{endobj}

\PYG{o}{\PYGZpc{}} \PYG{n}{Object} \PYG{l+m+mi}{5} \PYG{l+m+mi}{0}\PYG{p}{,} \PYG{n}{the} \PYG{n}{first} \PYG{n}{Paragraph} \PYG{n}{element}
\PYG{l+m+mi}{5} \PYG{l+m+mi}{0} \PYG{n}{obj}
\PYG{o}{\PYGZlt{}\PYGZlt{}}\PYG{o}{/}\PYG{n}{Type} \PYG{o}{/}\PYG{n}{StructElement}
\PYG{o}{/}\PYG{n}{S} \PYG{o}{/}\PYG{n}{P}
\PYG{o}{/}\PYG{n}{P} \PYG{l+m+mi}{4} \PYG{l+m+mi}{0} \PYG{n}{R}
\PYG{o}{\PYGZpc{}} \PYG{n}{Page} \PYG{o+ow}{in} \PYG{n}{whose} \PYG{n}{content} \PYG{n}{stream} \PYG{n}{integer} \PYG{n}{Marked} \PYG{n}{Content} \PYG{n}{ID}\PYG{l+s+s1}{\PYGZsq{}}\PYG{l+s+s1}{s denote Kids}
\PYG{o}{/}\PYG{n}{Pg} \PYG{l+m+mi}{10} \PYG{l+m+mi}{0} \PYG{n}{R}
\PYG{o}{/}\PYG{n}{K} \PYG{p}{[}\PYG{l+m+mi}{0}\PYG{p}{]}
\PYG{o}{\PYGZgt{}\PYGZgt{}} \PYG{n}{endobj}

\PYG{o}{\PYGZpc{}} \PYG{n}{Object} \PYG{l+m+mi}{6} \PYG{l+m+mi}{0}\PYG{p}{,} \PYG{n}{the} \PYG{n}{second} \PYG{n}{Paragraph} \PYG{n}{element}
\PYG{l+m+mi}{6} \PYG{l+m+mi}{0} \PYG{n}{obj}
\PYG{o}{\PYGZlt{}\PYGZlt{}}\PYG{o}{/}\PYG{n}{Type} \PYG{o}{/}\PYG{n}{StructElement}
\PYG{o}{/}\PYG{n}{S} \PYG{o}{/}\PYG{n}{P}
\PYG{o}{/}\PYG{n}{P} \PYG{l+m+mi}{4} \PYG{l+m+mi}{0} \PYG{n}{R}
\PYG{o}{\PYGZpc{}} \PYG{n}{Page} \PYG{o+ow}{in} \PYG{n}{whose} \PYG{n}{content} \PYG{n}{stream} \PYG{n}{integer} \PYG{n}{Marked} \PYG{n}{Content} \PYG{n}{ID}\PYG{l+s+s1}{\PYGZsq{}}\PYG{l+s+s1}{s denote Kids}
\PYG{o}{/}\PYG{n}{Pg} \PYG{l+m+mi}{10} \PYG{l+m+mi}{0} \PYG{n}{R}
\PYG{o}{/}\PYG{n}{K} \PYG{p}{[}\PYG{l+m+mi}{1}\PYG{p}{]}
\PYG{o}{\PYGZgt{}\PYGZgt{}} \PYG{n}{endobj}

\PYG{o}{\PYGZpc{}} \PYG{n}{Object} \PYG{l+m+mi}{10} \PYG{l+m+mi}{0}\PYG{p}{,} \PYG{n}{the} \PYG{n}{Page} \PYG{n+nb}{object} \PYG{k}{for} \PYG{n}{the} \PYG{n}{page} \PYG{n}{on} \PYG{n}{which} \PYG{n}{both}
\PYG{o}{\PYGZpc{}} \PYG{n}{paragraphs} \PYG{n}{are} \PYG{n}{marked}\PYG{o}{.} \PYG{n}{Only} \PYG{n}{the} \PYG{n}{relevant} \PYG{n}{entries} \PYG{o+ow}{in} \PYG{n}{the}
\PYG{o}{\PYGZpc{}} \PYG{n}{dictionary} \PYG{n}{are} \PYG{n}{shown}\PYG{o}{.}
\PYG{o}{\PYGZpc{}} \PYG{n}{The} \PYG{n}{Resources} \PYG{n}{dictionary} \PYG{n}{of} \PYG{n}{the} \PYG{n}{Contents} \PYG{n}{stream} \PYG{n}{of} \PYG{n}{the} \PYG{n}{page}\PYG{o}{.}
\PYG{o}{\PYGZlt{}\PYGZlt{}}\PYG{o}{/}\PYG{n}{StructParents} \PYG{l+m+mi}{0}
\PYG{o}{\PYGZgt{}\PYGZgt{}}
\PYG{o}{\PYGZpc{}} \PYG{n}{Inside} \PYG{n}{the} \PYG{n}{Contents} \PYG{n}{stream} \PYG{n}{of} \PYG{n}{the} \PYG{n}{page}\PYG{o}{.}
\PYG{o}{/}\PYG{n}{P} \PYG{o}{\PYGZlt{}\PYGZlt{}}\PYG{o}{/}\PYG{n}{MCID} \PYG{l+m+mi}{0}\PYG{o}{\PYGZgt{}\PYGZgt{}} \PYG{n}{BDC}
\PYG{o}{\PYGZpc{}} \PYG{p}{[}\PYG{n}{Paragraph} \PYG{l+m+mi}{1} \PYG{n}{content} \PYG{n}{marking} \PYG{n}{goes} \PYG{n}{here}\PYG{o}{.}\PYG{p}{]}
\PYG{n}{EMC}
\PYG{o}{/}\PYG{n}{P} \PYG{o}{\PYGZlt{}\PYGZlt{}}\PYG{o}{/}\PYG{n}{MCID} \PYG{l+m+mi}{1}\PYG{o}{\PYGZgt{}\PYGZgt{}} \PYG{n}{BDC}
\PYG{o}{\PYGZpc{}} \PYG{p}{[}\PYG{n}{Paragraph} \PYG{l+m+mi}{2} \PYG{n}{content} \PYG{n}{marking} \PYG{n}{goes} \PYG{n}{here}\PYG{p}{]}
\PYG{n}{EMC}
\end{sphinxVerbatim}
\begin{enumerate}
\sphinxsetlistlabels{\arabic}{enumi}{enumii}{}{.}%
\item {} 
A bookmark for a structural element

\end{enumerate}

\begin{sphinxVerbatim}[commandchars=\\\{\}]
 \PYG{p}{[}
\PYG{n}{other} \PYG{o}{/}\PYG{n}{StPNE} \PYG{n}{key}\PYG{o}{\PYGZhy{}}\PYG{n}{value} \PYG{n}{pairs}

        \PYG{o}{/}\PYG{n}{Bookmark}
            \PYG{o}{\PYGZlt{}\PYGZlt{}}
            \PYG{o}{/}\PYG{n}{Title} \PYG{p}{(}\PYG{n}{an} \PYG{n}{element} \PYG{o+ow}{in} \PYG{n}{my} \PYG{n}{structure}\PYG{p}{)}
            \PYG{o}{/}\PYG{n}{Open} \PYG{n}{true}
            \PYG{o}{\PYGZgt{}\PYGZgt{}}
        \PYG{o}{/}\PYG{n}{StPNE} \PYG{n}{pdfmark}
\end{sphinxVerbatim}
\begin{enumerate}
\sphinxsetlistlabels{\arabic}{enumi}{enumii}{}{.}%
\item {} 
Interrupted structure

\end{enumerate}

This example shows a paragraph that is graphically interrupted by a table. The originating application has chosen to write out the PostScript in graphical order, but logically the paragraph is one element and the table is another. To further complicate the situation, the document contains a special element that is a list of tables.

\begin{sphinxVerbatim}[commandchars=\\\{\}]
\PYG{o}{\PYGZpc{}} \PYG{n}{Start} \PYG{n}{a} \PYG{n}{ListOfTables} \PYG{n}{element} \PYG{n}{directly} \PYG{n}{under} \PYG{n}{the} \PYG{n}{Structure}
\PYG{o}{\PYGZpc{}} \PYG{n}{Tree} \PYG{n}{Root}\PYG{o}{.}  \PYG{n}{Give} \PYG{n}{it} \PYG{n}{an} \PYG{n+nb}{object} \PYG{n}{name} \PYG{k}{for} \PYG{n}{later} \PYG{n}{reference}\PYG{o}{.}
\PYG{p}{[} \PYG{o}{/}\PYG{n}{\PYGZus{}objdef} \PYG{p}{\PYGZob{}}\PYG{n}{LOT}\PYG{p}{\PYGZcb{}} \PYG{o}{/}\PYG{n}{Subtype} \PYG{o}{/}\PYG{n}{ListOfTables} \PYG{o}{/}\PYG{n}{StPNE} \PYG{n}{pdfmark}

\PYG{o}{\PYGZpc{}} \PYG{n}{Pop} \PYG{n}{it} \PYG{n}{off} \PYG{n}{the} \PYG{n}{stack} \PYG{n}{so} \PYG{n}{that} \PYG{n}{the} \PYG{n+nb}{next} \PYG{n}{element} \PYG{n}{becomes} \PYG{n}{a}
\PYG{o}{\PYGZpc{}} \PYG{n}{child} \PYG{n}{of} \PYG{n}{the} \PYG{n}{Structure} \PYG{n}{Tree} \PYG{n}{Root}\PYG{o}{.}
\PYG{p}{[} \PYG{o}{/}\PYG{n}{StPop} \PYG{n}{pdfmark}

\PYG{o}{\PYGZpc{}} \PYG{n}{Start} \PYG{n}{the} \PYG{n}{page} \PYG{k}{with} \PYG{n}{the} \PYG{n}{section} \PYG{n}{on} \PYG{n}{it}\PYG{o}{.}

\PYG{o}{\PYGZpc{}} \PYG{n}{Start} \PYG{n}{the} \PYG{n}{section}\PYG{p}{,} \PYG{n}{also} \PYG{n}{making} \PYG{n}{it} \PYG{n}{the} \PYG{n}{default} \PYG{n}{parent} \PYG{n}{element}\PYG{o}{.}
\PYG{p}{[} \PYG{o}{/}\PYG{n}{Subtype} \PYG{o}{/}\PYG{n}{Section} \PYG{o}{/}\PYG{n}{StPNE} \PYG{n}{pdfmark}
\PYG{o}{\PYGZpc{}} \PYG{n}{Start} \PYG{n}{the} \PYG{n}{paragraph}\PYG{o}{.}

\PYG{p}{[} \PYG{o}{/}\PYG{n}{Subtype} \PYG{o}{/}\PYG{n}{P} \PYG{o}{/}\PYG{n}{StPNE} \PYG{n}{pdfmark}

\PYG{o}{\PYGZpc{}} \PYG{n}{Here} \PYG{n}{comes} \PYG{n}{the} \PYG{n}{portion} \PYG{n}{of} \PYG{n}{the} \PYG{n}{paragraph} \PYG{n}{before} \PYG{n}{the} \PYG{n}{table}
\PYG{p}{[} \PYG{o}{/}\PYG{n}{StBMC} \PYG{n}{pdfmark}

\PYG{o}{\PYGZpc{}} \PYG{p}{[}\PYG{n}{Code} \PYG{n}{to} \PYG{n}{write} \PYG{n}{the} \PYG{n}{first} \PYG{n}{portion} \PYG{n}{of} \PYG{n}{the} \PYG{n}{paragraph} \PYG{n}{goes} \PYG{n}{here}\PYG{p}{]}

\PYG{p}{[} \PYG{o}{/}\PYG{n}{EMC} \PYG{n}{pdfmark}

\PYG{o}{\PYGZpc{}} \PYG{n}{Now} \PYG{n}{we}\PYG{l+s+s1}{\PYGZsq{}}\PYG{l+s+s1}{re interrupted by a table that doesn}\PYG{l+s+s1}{\PYGZsq{}}\PYG{n}{t} \PYG{n}{belong} \PYG{n}{to} \PYG{n}{the}
\PYG{o}{\PYGZpc{}} \PYG{n}{paragraph}\PYG{o}{.}  \PYG{n}{Save} \PYG{n}{the} \PYG{n}{context} \PYG{k}{as} \PYG{n}{a} \PYG{n}{conservative} \PYG{n}{move} \PYG{n}{because}
\PYG{o}{\PYGZpc{}} \PYG{n}{we} \PYG{n}{don}\PYG{l+s+s1}{\PYGZsq{}}\PYG{l+s+s1}{t want to worry about what the table code does to the}
\PYG{o}{\PYGZpc{}} \PYG{n}{implicit} \PYG{n}{parent} \PYG{n}{stack}\PYG{o}{.}
\PYG{p}{[} \PYG{o}{/}\PYG{n}{StoreName} \PYG{o}{/}\PYG{n}{S1} \PYG{o}{/}\PYG{n}{StStore} \PYG{n}{pdfmark}

\PYG{o}{\PYGZpc{}} \PYG{n}{The} \PYG{n}{table} \PYG{o+ow}{is} \PYG{n}{an} \PYG{n}{element}\PYG{p}{,} \PYG{o+ow}{and} \PYG{n}{it} \PYG{n}{contains} \PYG{n}{cells} \PYG{k}{as} \PYG{n}{child} \PYG{n}{elements}\PYG{o}{.}
\PYG{p}{[} \PYG{o}{/}\PYG{n}{E} \PYG{p}{\PYGZob{}}\PYG{n}{LOT}\PYG{p}{\PYGZcb{}} \PYG{o}{/}\PYG{n}{StPush} \PYG{n}{pdfmark}
\PYG{p}{[} \PYG{o}{/}\PYG{n}{Subtype} \PYG{o}{/}\PYG{n}{Table} \PYG{o}{/}\PYG{n}{StPNE} \PYG{n}{pdfmark}

\PYG{o}{\PYGZpc{}} \PYG{n}{Code} \PYG{n}{to} \PYG{n}{draw} \PYG{n}{the} \PYG{n}{table} \PYG{o+ow}{and} \PYG{n}{establish} \PYG{n}{its} \PYG{n}{logical} \PYG{n}{substructure} \PYG{n}{here}

\PYG{o}{\PYGZpc{}} \PYG{n}{Pop} \PYG{n}{the} \PYG{n}{table} \PYG{o+ow}{and} \PYG{n}{the} \PYG{n}{List} \PYG{n}{of} \PYG{n}{Tables} \PYG{n}{off} \PYG{n}{the} \PYG{n}{implicit} \PYG{n}{parent} \PYG{n}{stack}\PYG{o}{.}
\PYG{p}{[} \PYG{o}{/}\PYG{n}{StPop} \PYG{n}{pdfmark}
\PYG{p}{[} \PYG{o}{/}\PYG{n}{StPop} \PYG{n}{pdfmark}

\PYG{o}{\PYGZpc{}} \PYG{n}{Resume} \PYG{n}{the} \PYG{n}{paragraph}\PYG{o}{.}  \PYG{n}{It} \PYG{n}{turns} \PYG{n}{out} \PYG{n}{that} \PYG{n}{the} \PYG{n}{table} \PYG{n}{code} \PYG{n}{was}
\PYG{o}{\PYGZpc{}} \PYG{n}{tidy}\PYG{p}{,} \PYG{n}{but} \PYG{n}{it}\PYG{l+s+s1}{\PYGZsq{}}\PYG{l+s+s1}{s probably a good thing that we didn}\PYG{l+s+s1}{\PYGZsq{}}\PYG{n}{t} \PYG{n}{count} \PYG{n}{on}
\PYG{o}{\PYGZpc{}} \PYG{n}{it}\PYG{o}{.} \PYG{n}{Get} \PYG{n}{the} \PYG{n}{implicit} \PYG{n}{parent} \PYG{n}{stack} \PYG{n}{back} \PYG{n}{into} \PYG{n}{a} \PYG{n}{known} \PYG{n}{state}\PYG{o}{.}
\PYG{p}{[} \PYG{o}{/}\PYG{n}{StoreName} \PYG{o}{/}\PYG{n}{S1} \PYG{o}{/}\PYG{n}{StRetrieve} \PYG{n}{pdfmark}

\PYG{p}{[} \PYG{o}{/}\PYG{n}{StBMC} \PYG{n}{pdfmark}

\PYG{o}{\PYGZpc{}} \PYG{p}{[}\PYG{n}{Code} \PYG{n}{to} \PYG{n}{write} \PYG{n}{the} \PYG{n}{second} \PYG{n}{portion} \PYG{n}{of} \PYG{n}{the} \PYG{n}{paragraph}
\PYG{p}{[} \PYG{o}{/}\PYG{n}{EMC} \PYG{n}{pdfmark}

\PYG{o}{\PYGZpc{}} \PYG{n}{Pop} \PYG{n}{the} \PYG{n}{Paragraph} \PYG{o+ow}{and} \PYG{n}{Section} \PYG{n}{elements} \PYG{o+ow}{and} \PYG{n}{the} \PYG{n}{Structure}
\PYG{o}{\PYGZpc{}} \PYG{n}{Tree} \PYG{n}{Root} \PYG{n}{off} \PYG{n}{the} \PYG{n}{stack}\PYG{o}{.}
\PYG{p}{[} \PYG{o}{/}\PYG{n}{StPop} \PYG{n}{pdfmark}
\PYG{p}{[} \PYG{o}{/}\PYG{n}{StPop} \PYG{n}{pdfmark}
\PYG{p}{[} \PYG{o}{/}\PYG{n}{StPop} \PYG{n}{pdfmark}
\end{sphinxVerbatim}
\begin{enumerate}
\sphinxsetlistlabels{\arabic}{enumi}{enumii}{}{.}%
\item {} 
Independence of logical and physical structure

\end{enumerate}

This example shows that the logical structure and the physical nesting of marked content can have different tree structures. In this example there are two Structure Trees. One is the usual hierarchical structure of the document; the other is a list of funny words that occur within the document. The words occur as nested marked content within the marked content forming the contents of a paragraph, but the words become the content of elements in a separate branch of the structure tree from the Paragraph elements.

\begin{sphinxVerbatim}[commandchars=\\\{\}]
\PYG{o}{\PYGZpc{}} \PYG{n}{Set} \PYG{n}{up} \PYG{n}{a} \PYG{n}{List} \PYG{n}{element} \PYG{n}{to} \PYG{n}{hold} \PYG{n}{the} \PYG{n}{Funny} \PYG{n}{Word} \PYG{n}{List}\PYG{o}{.}
\PYG{p}{[} \PYG{o}{/}\PYG{n}{\PYGZus{}objdef} \PYG{p}{\PYGZob{}}\PYG{n}{FWL}\PYG{p}{\PYGZcb{}}    \PYG{o}{/}\PYG{n}{Subtype} \PYG{o}{/}\PYG{n}{List} \PYG{o}{/}\PYG{n}{Title} \PYG{p}{(}\PYG{n}{Funny} \PYG{n}{Words}\PYG{p}{)} \PYG{o}{/}\PYG{n}{StPNE} \PYG{n}{pdfmark}

\PYG{p}{[} \PYG{o}{/}\PYG{n}{StPop} \PYG{n}{p}

\PYG{p}{[}\PYG{o}{/}\PYG{n}{Subtype} \PYG{o}{/}\PYG{n}{Section} \PYG{o}{/}\PYG{n}{StPNE} \PYG{n}{p}

\PYG{p}{[} \PYG{o}{/}\PYG{n}{Subtype} \PYG{o}{/}\PYG{n}{P} \PYG{o}{/}\PYG{n}{StPNE} \PYG{n}{p}

\PYG{o}{\PYGZpc{}} \PYG{n}{Begin} \PYG{n}{PostScript} \PYG{n}{code} \PYG{k}{for} \PYG{n}{the} \PYG{n}{par}
\PYG{p}{[} \PYG{o}{/}\PYG{n}{StBMC} \PYG{n}{p}
\PYG{p}{(}\PYG{n}{John} \PYG{n}{was} \PYG{n}{thrilled} \PYG{n}{to} \PYG{n}{find} \PYG{n}{some}
\PYG{o}{\PYGZpc{}} \PYG{n}{Here}\PYG{l+s+s1}{\PYGZsq{}}\PYG{l+s+s1}{s an occurrence of a funny word comi}
\PYG{o}{\PYGZpc{}} \PYG{n}{Start} \PYG{n}{an} \PYG{n}{element} \PYG{k}{for} \PYG{n}{the} \PYG{n}{funny} \PYG{n}{word} \PYG{n}{l}
\PYG{p}{[} \PYG{o}{/}\PYG{n}{E} \PYG{p}{\PYGZob{}}\PYG{n}{FWL}\PYG{p}{\PYGZcb{}} \PYG{o}{/}\PYG{n}{StPush} \PYG{n}{p}
\PYG{p}{[} \PYG{o}{/}\PYG{n}{Subtype} \PYG{o}{/}\PYG{n}{Word} \PYG{o}{/}\PYG{n}{StPNE} \PYG{n}{p}
\PYG{o}{\PYGZpc{}} \PYG{n}{Fill} \PYG{n}{that} \PYG{n}{element} \PYG{k}{with} \PYG{n}{the} \PYG{n}{funny} \PYG{n}{word} \PYG{k+kn}{from} \PYG{n+nn}{the}
\PYG{o}{\PYGZpc{}} \PYG{n}{page} \PYG{n}{content}\PYG{o}{.} \PYG{n}{This} \PYG{n}{content} \PYG{o+ow}{is} \PYG{n}{still} \PYG{o+ow}{in} \PYG{n}{the}
\PYG{o}{\PYGZpc{}} \PYG{n}{marked} \PYG{n}{content} \PYG{n}{within} \PYG{n}{the} \PYG{n}{paragraph} \PYG{n}{el}
\PYG{p}{[} \PYG{o}{/}\PYG{n}{StBMC} \PYG{n}{pdfmark}
\PYG{p}{(}\PYG{n}{puccoon}\PYG{p}{)} \PYG{n}{show}
\PYG{p}{[} \PYG{o}{/}\PYG{n}{EMC} \PYG{n}{pdfmark}

\PYG{o}{\PYGZpc{}} \PYG{n}{Pop} \PYG{n}{the} \PYG{n}{Word} \PYG{n}{element} \PYG{n}{off} \PYG{n}{the} \PYG{n}{implicit} \PYG{n}{parent} \PYG{n}{stack}\PYG{o}{.}

\PYG{p}{[} \PYG{o}{/}\PYG{n}{StPop} \PYG{n}{pdfmark}

\PYG{o}{\PYGZpc{}} \PYG{n}{Resume} \PYG{n}{paragraph} \PYG{n}{content} \PYG{n}{that}\PYG{l+s+s1}{\PYGZsq{}}\PYG{l+s+s1}{s not in the funny word}
\PYG{o}{\PYGZpc{}} \PYG{p}{(}\PYG{p}{,} \PYG{o+ow}{not} \PYG{n}{knowing} \PYG{n}{that} \PYG{n}{it} \PYG{n}{could} \PYG{n}{also} \PYG{n}{be} \PYG{n}{called} \PYG{p}{)}
\PYG{o}{\PYGZpc{}} \PYG{o}{.}\PYG{o}{.}\PYG{o}{.} \PYG{n}{another} \PYG{n}{funny} \PYG{n}{word} \PYG{o}{.}\PYG{o}{.}\PYG{o}{.}

\PYG{p}{[} \PYG{o}{/}\PYG{n}{E} \PYG{p}{\PYGZob{}}\PYG{n}{FWL}\PYG{p}{\PYGZcb{}} \PYG{o}{/}\PYG{n}{StPush} \PYG{n}{pdfmark}
\PYG{p}{[} \PYG{o}{/}\PYG{n}{Subtype} \PYG{o}{/}\PYG{n}{Word} \PYG{o}{/}\PYG{n}{StPNE} \PYG{n}{pdfmark}
\PYG{p}{[} \PYG{o}{/}\PYG{n}{StBMC} \PYG{n}{pdfmark}
\PYG{p}{(}\PYG{n}{gromwell}\PYG{p}{)} \PYG{n}{show}
\PYG{p}{[} \PYG{o}{/}\PYG{n}{EMC} \PYG{n}{pdfmark}
\PYG{p}{[} \PYG{o}{/}\PYG{n}{StPop} \PYG{n}{pdfmark}
\PYG{p}{(}\PYG{o}{.}\PYG{p}{)} \PYG{n}{show}

\PYG{o}{\PYGZpc{}} \PYG{n}{Close} \PYG{n}{off} \PYG{n}{the} \PYG{n}{marked} \PYG{n}{content} \PYG{k}{for} \PYG{n}{the} \PYG{n}{paragraph}\PYG{o}{.}\PYG{o}{.}\PYG{o}{.}

\PYG{p}{[} \PYG{o}{/}\PYG{n}{EMC} \PYG{n}{pdfmark}

\PYG{o}{\PYGZpc{}} \PYG{o}{.}\PYG{o}{.}\PYG{o}{.}\PYG{o+ow}{and} \PYG{n}{tidy} \PYG{n}{up} \PYG{n}{the} \PYG{n}{stack}

\PYG{p}{[} \PYG{o}{/}\PYG{n}{StPop} \PYG{n}{pdfmark}
\PYG{p}{[} \PYG{o}{/}\PYG{n}{StPop} \PYG{n}{pdfmark}
\PYG{p}{[} \PYG{o}{/}\PYG{n}{StPop} \PYG{n}{pdfmark}
\end{sphinxVerbatim}
\begin{enumerate}
\sphinxsetlistlabels{\arabic}{enumi}{enumii}{}{.}%
\item {} 
Page break within logical structure

\end{enumerate}

This example shows how to handle a logical structure spanning more than one page. It shows a logical paragraph spanning a page break.

\begin{sphinxVerbatim}[commandchars=\\\{\}]
\PYG{o}{\PYGZpc{}}\PYG{o}{\PYGZpc{}}\PYG{n}{Page}\PYG{p}{:} \PYG{l+m+mi}{1} \PYG{l+m+mi}{1}
   \PYG{o}{\PYGZpc{}} \PYG{n}{Begin} \PYG{n}{a} \PYG{n}{Paragraph} \PYG{n}{element}

   \PYG{p}{[} \PYG{o}{/}\PYG{n}{Subtype} \PYG{o}{/}\PYG{n}{P} \PYG{o}{/}\PYG{n}{StPNE} \PYG{n}{pdfmark}

   \PYG{p}{[} \PYG{o}{/}\PYG{n}{StBMC} \PYG{n}{pdfmark}

   \PYG{o}{\PYGZpc{}} \PYG{o}{.}\PYG{o}{.}\PYG{o}{.} \PYG{n}{write} \PYG{n}{the} \PYG{n}{portion} \PYG{n}{of} \PYG{n}{the} \PYG{n}{paragraph} \PYG{n}{that}\PYG{l+s+s1}{\PYGZsq{}}\PYG{l+s+s1}{s on Page 1 ...}

   \PYG{p}{[} \PYG{o}{/}\PYG{n}{EMC} \PYG{n}{pdfmark}
   \PYG{n}{showpage}

   \PYG{o}{\PYGZpc{}}\PYG{o}{\PYGZpc{}}\PYG{n}{Page}\PYG{p}{:} \PYG{l+m+mi}{2} \PYG{l+m+mi}{2}

   \PYG{o}{\PYGZpc{}} \PYG{n}{The} \PYG{n}{Paragraph} \PYG{n}{element} \PYG{o+ow}{is} \PYG{n}{still} \PYG{n}{on} \PYG{n}{the} \PYG{n}{top} \PYG{n}{of} \PYG{n}{the} \PYG{n}{stack}\PYG{p}{,} \PYG{n}{so}
   \PYG{o}{\PYGZpc{}} \PYG{n}{we} \PYG{n}{can} \PYG{n}{just} \PYG{n}{add} \PYG{n}{some} \PYG{n}{more} \PYG{n}{content} \PYG{n}{to} \PYG{n}{it} \PYG{n}{implicitly}\PYG{o}{.}

   \PYG{p}{[} \PYG{o}{/}\PYG{n}{StBMC} \PYG{n}{pdfmark}

   \PYG{o}{\PYGZpc{}} \PYG{o}{.}\PYG{o}{.}\PYG{o}{.} \PYG{n}{write} \PYG{n}{the} \PYG{n}{portion} \PYG{n}{of} \PYG{n}{the} \PYG{n}{paragraph} \PYG{n}{that}\PYG{l+s+s1}{\PYGZsq{}}\PYG{l+s+s1}{s on Page 2 ...}

   \PYG{p}{[} \PYG{o}{/}\PYG{n}{EMC} \PYG{n}{pdfmark}
\end{sphinxVerbatim}
\begin{enumerate}
\sphinxsetlistlabels{\arabic}{enumi}{enumii}{}{.}%
\item {} 
Logical structure out\sphinxhyphen{}of\sphinxhyphen{}order in physical structure

\end{enumerate}

This example shows how to build a logical structure whose elements appear in a different physical order in the document from their logical order. The example is based on a magazine in which an opinion piece starting on the last inside page is continued on an earlier page in the printing order.

\begin{sphinxVerbatim}[commandchars=\\\{\}]
\PYG{o}{\PYGZpc{}}\PYG{o}{\PYGZpc{}}\PYG{n}{Page} \PYG{l+m+mi}{5} \PYG{l+m+mi}{5}
   \PYG{p}{[} \PYG{o}{/}\PYG{n}{Subtype} \PYG{o}{/}\PYG{n}{Section} \PYG{o}{/}\PYG{n}{ID} \PYG{p}{(}\PYG{n}{ID} \PYG{n}{string}\PYG{p}{)} \PYG{o}{/}\PYG{n}{StPNE} \PYG{n}{pdfmark}

   \PYG{o}{\PYGZpc{}} \PYG{n}{Within} \PYG{n}{the} \PYG{n}{Section} \PYG{n}{element}\PYG{p}{,} \PYG{n}{this} \PYG{n}{Paragraph} \PYG{n}{element} \PYG{o+ow}{is} \PYG{n}{actually}
   \PYG{o}{\PYGZpc{}} \PYG{n}{a} \PYG{n}{later} \PYG{n}{paragraph} \PYG{n}{than} \PYG{n}{the} \PYG{n}{Paragraph} \PYG{n}{element} \PYG{n}{that} \PYG{n}{appears}
   \PYG{o}{\PYGZpc{}} \PYG{n}{on} \PYG{n}{the} \PYG{n+nb}{next} \PYG{n}{page}\PYG{o}{.}
   \PYG{p}{[} \PYG{o}{/}\PYG{n}{Subtype} \PYG{o}{/}\PYG{n}{P} \PYG{o}{/}\PYG{n}{StPNE} \PYG{n}{pdfmark}
               \PYG{o}{\PYGZpc{}} \PYG{n}{No} \PYG{o}{/}\PYG{n}{At} \PYG{n}{key}\PYG{p}{,} \PYG{n}{so} \PYG{n}{defaults} \PYG{n}{to} \PYG{n}{being} \PYG{n}{inserted}
               \PYG{o}{\PYGZpc{}} \PYG{k}{as} \PYG{n}{last} \PYG{n}{child} \PYG{n}{of} \PYG{n}{its} \PYG{n}{parent}\PYG{o}{.}

   \PYG{p}{[} \PYG{o}{/}\PYG{n}{StBMC} \PYG{n}{pdfmark}
   \PYG{o}{\PYGZpc{}} \PYG{o}{.}\PYG{o}{.}\PYG{o}{.} \PYG{n}{draw} \PYG{n}{the} \PYG{n}{paragraph}\PYG{o}{.}\PYG{o}{.}\PYG{o}{.}
   \PYG{p}{[} \PYG{o}{/}\PYG{n}{EMC} \PYG{n}{pdfmark}

   \PYG{o}{\PYGZpc{}} \PYG{o}{.}\PYG{o}{.}\PYG{o}{.} \PYG{n}{the} \PYG{n}{rest} \PYG{n}{of} \PYG{n}{the} \PYG{n}{page} \PYG{o}{.}\PYG{o}{.}\PYG{o}{.}
   \PYG{n}{showpage}

   \PYG{o}{\PYGZpc{}} \PYG{n}{Pop} \PYG{n}{the} \PYG{n}{Paragraph} \PYG{n}{element} \PYG{n}{off} \PYG{n}{the} \PYG{n}{stack}
   \PYG{p}{[} \PYG{o}{/}\PYG{n}{StPop} \PYG{n}{pdfmark}
   \PYG{o}{\PYGZpc{}}\PYG{o}{\PYGZpc{}}\PYG{n}{Page} \PYG{l+m+mi}{6} \PYG{l+m+mi}{6}

   \PYG{p}{[} \PYG{o}{/}\PYG{n}{Subtype} \PYG{o}{/}\PYG{n}{P} \PYG{o}{/}\PYG{n}{At} \PYG{l+m+mi}{0} \PYG{o}{/}\PYG{n}{StPNE} \PYG{n}{pdfmark}
   \PYG{o}{\PYGZpc{}} \PYG{n}{Insert} \PYG{k}{as} \PYG{n}{first} \PYG{n}{child} \PYG{n}{of} \PYG{n}{parent}\PYG{o}{.}

   \PYG{p}{[} \PYG{o}{/}\PYG{n}{StBMC} \PYG{n}{pdfmark}
   \PYG{o}{\PYGZpc{}} \PYG{o}{.}\PYG{o}{.}\PYG{o}{.} \PYG{n}{draw} \PYG{n}{the} \PYG{n}{paragraph}\PYG{o}{.}\PYG{o}{.}\PYG{o}{.}
   \PYG{p}{[} \PYG{o}{/}\PYG{n}{EMC} \PYG{n}{pdfmark}

   \PYG{o}{\PYGZpc{}} \PYG{n}{Pop} \PYG{n}{the} \PYG{n}{Paragraph} \PYG{o+ow}{and} \PYG{n}{Section} \PYG{n}{elements} \PYG{n}{off} \PYG{n}{the} \PYG{n}{stack}
   \PYG{p}{[} \PYG{o}{/}\PYG{n}{StPop} \PYG{n}{pdfmark}
   \PYG{p}{[} \PYG{o}{/}\PYG{n}{StPop} \PYG{n}{pdfmark}
\end{sphinxVerbatim}


\chapter{JDF Features}
\label{\detokenize{pdfmark_JDF:jdf-features}}\label{\detokenize{pdfmark_JDF::doc}}
The use of pdfmark in PostScript can include representations of Job Definition Format (JDF) features. JDF is an extensible XML\sphinxhyphen{}based job ticketing format designed for use by the printing industry. Information about JDF can be obtained from \sphinxurl{http://www.cip4.org} .

In particular, pdfmark for JDF allows the PostScript file/stream to specify elements and attributes to be added to a JDF document being used for a job. Applications that support JDF pdfmark include Acrobat Distiller 6.0 and 7.0.

\begin{sphinxadmonition}{note}{Note:}
Distiller 8.0 and later does not support JDF. Any JDF\sphinxhyphen{}related pdfmark commands in the PostScript stream are ignored.
\end{sphinxadmonition}


\section{Syntax}
\label{\detokenize{pdfmark_JDF:syntax}}
\begin{sphinxVerbatim}[commandchars=\\\{\}]
 \PYG{p}{[} \PYG{o}{/}\PYG{n}{Attribute}
\PYG{n}{string}
        \PYG{o}{/}\PYG{n}{Value}
\PYG{n}{string}
        \PYG{o}{/}\PYG{n}{Subtype} \PYG{o}{/}\PYG{n}{CreateAttribute}
        \PYG{o}{/}\PYG{n}{JDF} \PYG{n}{pdfmark}
\end{sphinxVerbatim}

The \sphinxcode{\sphinxupquote{Attribute}} and \sphinxcode{\sphinxupquote{Value}} keys are described in the following table.

\sphinxstylestrong{Keys supported by JDF pdfmark}


\begin{savenotes}\sphinxattablestart
\centering
\begin{tabular}[t]{|\X{10}{100}|\X{10}{100}|\X{80}{100}|}
\hline

Key
&
Type
&
Semantics
\\
\hline
Attribute
&
string
&
An XPath expression that identifies the location of the attribute absolutely from the root of the JDF. If any portion of the hierarchy of elements containing the attribute is not present in the JDF, they are created. XPath is a language for addressing parts of an XML document, as defined in \sphinxstyleemphasis{XML Path Language (XPath) Version 1.0} available from \sphinxurl{http://www.w3.org/TR/xpath} .

JDF pdfmark supports the following subset of XPath expressions:

\begin{sphinxVerbatimintable}[commandchars=\\\{\}]
\PYG{n}{Expression} \PYG{p}{:}\PYG{o}{:=} \PYG{n}{JDFRoot}\PYG{l+s+s1}{\PYGZsq{}}\PYG{l+s+s1}{/}\PYG{l+s+s1}{\PYGZsq{}}\PYG{n}{Attribute} \PYG{o}{|}

 \PYG{n}{JDFRoot}\PYG{l+s+s1}{\PYGZsq{}}\PYG{l+s+s1}{/}\PYG{l+s+s1}{\PYGZsq{}}\PYG{n}{Children}\PYG{l+s+s1}{\PYGZsq{}}\PYG{l+s+s1}{/}\PYG{l+s+s1}{\PYGZsq{}}\PYG{n}{Attribute}
 \PYG{n}{JDFRoot} \PYG{p}{:}\PYG{o}{:=} \PYG{l+s+s1}{\PYGZsq{}}\PYG{l+s+s1}{//JDF}\PYG{l+s+s1}{\PYGZsq{}}
 \PYG{n}{Children} \PYG{p}{:}\PYG{o}{:=} \PYG{n}{Element} \PYG{o}{|} \PYG{n}{Element}\PYG{l+s+s1}{\PYGZsq{}}\PYG{l+s+s1}{/}\PYG{l+s+s1}{\PYGZsq{}}\PYG{n}{Children}
 \PYG{n}{Element} \PYG{p}{:}\PYG{o}{:=} \PYG{n}{element} \PYG{o}{|}

 \PYG{n}{element}\PYG{l+s+s1}{\PYGZsq{}}\PYG{l+s+s1}{[}\PYG{l+s+s1}{\PYGZsq{}}\PYG{n}{FilterExpression}\PYG{l+s+s1}{\PYGZsq{}}\PYG{l+s+s1}{]}\PYG{l+s+s1}{\PYGZsq{}}
 \PYG{n}{FilterExpression} \PYG{p}{:}\PYG{o}{:=}
 \PYG{n}{Filter} \PYG{o}{|} \PYG{n}{Filter} \PYG{l+s+s1}{\PYGZsq{}}\PYG{l+s+s1}{and}\PYG{l+s+s1}{\PYGZsq{}} \PYG{n}{FilterExpression} \PYG{o}{|} \PYG{n}{Filter}

 \PYG{l+s+s1}{\PYGZsq{}}\PYG{l+s+s1}{or}\PYG{l+s+s1}{\PYGZsq{}} \PYG{n}{FilterExpression}
 \PYG{n}{Filter} \PYG{p}{:}\PYG{o}{:=} \PYG{n}{Attribute}\PYG{l+s+s1}{\PYGZsq{}}\PYG{l+s+s1}{=}\PYG{l+s+s1}{\PYGZsq{}}\PYG{n}{Value}
 \PYG{n}{Attribute} \PYG{p}{:}\PYG{o}{:=} \PYG{l+s+s1}{\PYGZsq{}}\PYG{l+s+s1}{@}\PYG{l+s+s1}{\PYGZsq{}}\PYG{n}{attribute}
\end{sphinxVerbatimintable}
\\
\hline
Values
&
string
&\begin{description}
\item[{The value to be assigned to the attribute, using the XPath expression:}] \leavevmode
Value ::= ‘ ” ‘value’ ” ‘

\end{description}
\\
\hline
\end{tabular}
\par
\sphinxattableend\end{savenotes}


\section{XPath Examples}
\label{\detokenize{pdfmark_JDF:xpath-examples}}
The following table presents examples of XPath expressions.

\sphinxstylestrong{XPath expressions}


\begin{savenotes}\sphinxattablestart
\centering
\begin{tabulary}{\linewidth}[t]{|T|T|}
\hline
\sphinxstyletheadfamily 
Expression
&\sphinxstyletheadfamily 
Interpretation
\\
\hline
\sphinxhref{mailto://JDF/@JobID}{//JDF/@JobID}
&
Selects the \sphinxcode{\sphinxupquote{JobID}} attribute in root JDF node.
\\
\hline
//JDF/JDFResourceLinkPool
\sphinxhref{mailto:/ComponentLink/@rRef}{/ComponentLink/@rRef}
&
Selects the \sphinxcode{\sphinxupquote{rRef}} attribute of the \sphinxcode{\sphinxupquote{ComponentLink}} found in the \sphinxcode{\sphinxupquote{ResourcePool}} in the root JDF node.
\\
\hline
//JDF/JDF{[}@Type=”Trapping”{]}/@Status
&
Selects the \sphinxcode{\sphinxupquote{Status}} attribute of the \sphinxcode{\sphinxupquote{Trapping}} node that is a child of the root JDF node.
\\
\hline
//JDF/JDFResourceLinkPool
/ComponentLink{[}@Usage=”Output” and @ProcessUsage=”Good”{]}/@rRef
&
First identifies the \sphinxcode{\sphinxupquote{ResourceLinkPool}} of the root JDF node. It then selects the \sphinxcode{\sphinxupquote{rRef}} attribute of the \sphinxcode{\sphinxupquote{ComponentLink}} with both a \sphinxcode{\sphinxupquote{Usage}} attribute value “Output” and a \sphinxcode{\sphinxupquote{ProcessUsage}} attribute with value “Good”.
\\
\hline
\end{tabulary}
\par
\sphinxattableend\end{savenotes}

\begin{sphinxadmonition}{note}{Note:}
In actual use, all XPath expressions should end with \sphinxcode{\sphinxupquote{@}} attribute because they must define the location of an attribute.
\end{sphinxadmonition}

The JDF pdfmark commands shown in the following example cause supporting applications to modify the current JDF document, as illustrated in the following diagram.
\begin{enumerate}
\sphinxsetlistlabels{\arabic}{enumi}{enumii}{}{.}%
\item {} 
Using JDF pdfmark to set Trapping element and subelement attributes

\end{enumerate}

\begin{sphinxVerbatim}[commandchars=\\\{\}]
\PYG{p}{[} \PYG{o}{/}\PYG{n}{Attribute} \PYG{p}{(}\PYG{o}{/}\PYG{o}{/}\PYG{n}{JDF}\PYG{o}{/}\PYG{n}{JDF}\PYG{p}{[}\PYG{n+nd}{@Type}\PYG{o}{=}\PYG{l+s+s2}{\PYGZdq{}}\PYG{l+s+s2}{Trapping}\PYG{l+s+s2}{\PYGZdq{}}\PYG{p}{]}\PYG{o}{/}\PYG{n+nd}{@Type}\PYG{p}{)}
       \PYG{o}{/}\PYG{n}{Value} \PYG{p}{(}\PYG{n}{Trapping}\PYG{p}{)}
       \PYG{o}{/}\PYG{n}{Subtype} \PYG{o}{/}\PYG{n}{CreateAttribute} \PYG{o}{/}\PYG{n}{JDF} \PYG{n}{pdfmark}
 \PYG{p}{[} \PYG{o}{/}\PYG{n}{Attribute}
       \PYG{p}{(}\PYG{o}{/}\PYG{o}{/}\PYG{n}{JDF}\PYG{o}{/}\PYG{n}{JDF}\PYG{p}{[}\PYG{n+nd}{@Type}\PYG{o}{=}\PYG{l+s+s2}{\PYGZdq{}}\PYG{l+s+s2}{Trapping}\PYG{l+s+s2}{\PYGZdq{}}\PYG{p}{]}\PYG{o}{/}\PYG{n}{ResourceLinkPool}\PYG{o}{/}\PYG{n}{TrappingDetailsLink}\PYG{o}{/}\PYG{n+nd}{@rRef}\PYG{p}{)}
       \PYG{o}{/}\PYG{n}{Value} \PYG{p}{(}\PYG{n}{TD1}\PYG{p}{)}
       \PYG{o}{/}\PYG{n}{Subtype} \PYG{o}{/}\PYG{n}{CreateAttribute} \PYG{o}{/}\PYG{n}{JDF} \PYG{n}{pdfmark}
 \PYG{p}{[} \PYG{o}{/}\PYG{n}{Attribute}
       \PYG{p}{(}\PYG{o}{/}\PYG{o}{/}\PYG{n}{JDF}\PYG{o}{/}\PYG{n}{JDF}\PYG{p}{[}\PYG{n+nd}{@Type}\PYG{o}{=}\PYG{l+s+s2}{\PYGZdq{}}\PYG{l+s+s2}{Trapping}\PYG{l+s+s2}{\PYGZdq{}}\PYG{p}{]}\PYG{o}{/}\PYG{n}{ResourceLinkPool}\PYG{o}{/}\PYG{n}{TrappingDetailsLink}
       \PYG{p}{[}\PYG{n+nd}{@rRef}\PYG{o}{=}\PYG{l+s+s2}{\PYGZdq{}}\PYG{l+s+s2}{TD1}\PYG{l+s+s2}{\PYGZdq{}}\PYG{p}{]}\PYG{o}{/}\PYG{n+nd}{@Usage}\PYG{p}{)}
       \PYG{o}{/}\PYG{n}{Value} \PYG{p}{(}\PYG{n}{Input}\PYG{p}{)}
       \PYG{o}{/}\PYG{n}{Subtype} \PYG{o}{/}\PYG{n}{CreateAttribute} \PYG{o}{/}\PYG{n}{JDF} \PYG{n}{pdfmark}
\end{sphinxVerbatim}

The following shows the JDF structure created through JDF pdfmark in the preceding example.

\noindent\sphinxincludegraphics{{jdfstructure}.png}


\chapter{Distilling Optional Content}
\label{\detokenize{pdfmark_OC_Interface:distilling-optional-content}}\label{\detokenize{pdfmark_OC_Interface::doc}}
The ProcSet entry in a content stream’s resource dictionary holds an array consisting of the names of the procedure sets used in that content stream. This section describes the ProcSet used to build optional content into a PDF document.

Optional content refers to content in a PDF document that can be selectively viewed or hidden. Optional content is a feature that became available with Acrobat 6.

For more information on ProcSet entries and optional content, see the \sphinxstyleemphasis{PDF Reference} .

\begin{sphinxadmonition}{note}{Note:}
While the optional content ProcSet makes extensive use of pdfmark internally, clients of the optional content ProcSet should not have to use pdfmark to add optional content to a PDF file.
\end{sphinxadmonition}


\section{Initialization and termination code}
\label{\detokenize{pdfmark_OC_Interface:initialization-and-termination-code}}
To use the optional content ProcSet, clients must insert the following code into the document setup section of the PostScript job. This places definitions of the optional content ProcSet procedures in \sphinxcode{\sphinxupquote{userdict}} for easy access by the client.

\begin{sphinxVerbatim}[commandchars=\\\{\}]
\PYG{p}{\PYGZob{}}\PYG{o}{/}\PYG{n}{OCProcSet} \PYG{o}{/}\PYG{n}{ProcSet} \PYG{n}{findresource}\PYG{p}{\PYGZcb{}} \PYG{n}{stopped} \PYG{o+ow}{not}
 \PYG{p}{\PYGZob{}}\PYG{o}{/}\PYG{n}{initialize} \PYG{n}{get} \PYG{n}{exec}\PYG{p}{\PYGZcb{}}
 \PYG{p}{\PYGZob{}}
     \PYG{o}{/}\PYG{n}{BeginOC} \PYG{o}{/}\PYG{n}{pop} \PYG{n}{load} \PYG{k}{def}
     \PYG{o}{/}\PYG{n}{EndOC} \PYG{p}{\PYGZob{}}\PYG{p}{\PYGZcb{}} \PYG{k}{def}
     \PYG{o}{/}\PYG{n}{SimpleOC} \PYG{o}{/}\PYG{n}{pop} \PYG{n}{load} \PYG{k}{def}
     \PYG{o}{/}\PYG{n}{SetOCGInitState} \PYG{p}{\PYGZob{}}\PYG{n}{pop} \PYG{n}{pop}\PYG{p}{\PYGZcb{}} \PYG{n}{bind} \PYG{k}{def}
     \PYG{o}{/}\PYG{n}{OCEndPage} \PYG{p}{\PYGZob{}}\PYG{p}{\PYGZcb{}} \PYG{k}{def}
     \PYG{o}{/}\PYG{n}{SetOCGIntent} \PYG{p}{\PYGZob{}}\PYG{n}{pop} \PYG{n}{pop}\PYG{p}{\PYGZcb{}} \PYG{n}{bind} \PYG{k}{def}
     \PYG{o}{/}\PYG{n}{SetOCGUsage} \PYG{p}{\PYGZob{}}\PYG{n}{pop} \PYG{n}{pop}\PYG{p}{\PYGZcb{}} \PYG{n}{bind} \PYG{k}{def}
     \PYG{o}{/}\PYG{n}{AddASEvent} \PYG{p}{\PYGZob{}}\PYG{n}{pop} \PYG{n}{pop} \PYG{n}{pop}\PYG{p}{\PYGZcb{}} \PYG{n}{bind} \PYG{k}{def}
     \PYG{o}{/}\PYG{n}{GetOCGPdfmarkTag} \PYG{p}{\PYGZob{}}\PYG{p}{\PYGZob{}}\PYG{o}{\PYGZhy{}}\PYG{o}{\PYGZhy{}}\PYG{o}{\PYGZhy{}}\PYG{n}{invalidpdfmarkname}\PYG{o}{\PYGZhy{}}\PYG{o}{\PYGZhy{}}\PYG{o}{\PYGZhy{}}\PYG{p}{\PYGZcb{}}\PYG{p}{\PYGZcb{}} \PYG{n}{bind} \PYG{k}{def}
 \PYG{p}{\PYGZcb{}}
 \PYG{n}{ifelse}
\end{sphinxVerbatim}

Also, the following code must be inserted into the trailer section of the PostScript file:

\begin{sphinxVerbatim}[commandchars=\\\{\}]
\PYG{p}{\PYGZob{}}\PYG{o}{/}\PYG{n}{OCProcSet} \PYG{o}{/}\PYG{n}{ProcSet} \PYG{n}{findresource} \PYG{o}{/}\PYG{n}{terminate} \PYG{n}{get} \PYG{n}{exec}\PYG{p}{\PYGZcb{}} \PYG{n}{stopped} \PYG{n}{pop}
\end{sphinxVerbatim}

When using the optional content ProcSet, the optional content group is the primary data item. It is referred to by the group’s name, which is a string object. (See the description of the entries in the optional content group dictionary in the \sphinxstyleemphasis{PDF Reference} .) Clients of the ProcSet do not need to do anything to set up optional content groups—they simply refer to them by their name strings, and the ProcSet takes care of creating them on\sphinxhyphen{}the\sphinxhyphen{}fly. Clients can set the initial state, intent, and usage info for optional content groups, using \sphinxcode{\sphinxupquote{SetOCGInitState}} , \sphinxcode{\sphinxupquote{SetOCGIntent}} , and \sphinxcode{\sphinxupquote{SetOCGUsage}} , respectively.

There are two techniques for using the ProcSet to make content optional: one for non\sphinxhyphen{}nested optional content, and one for nested optional content:
\begin{itemize}
\item {} 
The simplest technique is for non\sphinxhyphen{}nested optional content using the \sphinxcode{\sphinxupquote{SimpleOC}} procedure. Simply pass in a string for the optional content group name, and all marks up to the next \sphinxcode{\sphinxupquote{SimpleOC}} belong to the optional content group with that name. Passing in \sphinxcode{\sphinxupquote{null SimpleOC}} makes subsequent content non\sphinxhyphen{}optional. At the end of the page, before the \sphinxcode{\sphinxupquote{showpage}} , issue \sphinxcode{\sphinxupquote{null SimpleOC}} .

\item {} 
For nested optional content, the technique is for documents that have nested visibility control. For these the ProcSet provides stack\sphinxhyphen{}style optional content control. This is also the style of control used if you have content that requires an Optional Content Membership Dictionary (OCMD), because it belongs to more than one optional content group and can require a visibility policy entry in the OCMD. For this sort of optional content, use the \sphinxcode{\sphinxupquote{BeginOC}} and \sphinxcode{\sphinxupquote{EndOC}} procedures. With this style, you should call \sphinxcode{\sphinxupquote{OCEndPage}} at the end of the page (before \sphinxcode{\sphinxupquote{showpage}} ). This ensures that the marked content is closed properly.

\end{itemize}


\section{Procedure definitions}
\label{\detokenize{pdfmark_OC_Interface:procedure-definitions}}
This section describes the optional content procedures and provides their syntax and examples.


\subsection{AddASEvent}
\label{\detokenize{pdfmark_OC_Interface:addasevent}}
Adds an auto state event to the PDF’s default configuration. See the \sphinxstyleemphasis{PDF Reference} for a description of auto state control for optional content.

\sphinxstylestrong{Syntax}

\begin{sphinxVerbatim}[commandchars=\\\{\}]
\PYG{n}{event\PYGZus{}type} \PYG{n}{event\PYGZus{}categories} \PYG{n}{ocgnames} \PYG{n}{AddASEvent}
\end{sphinxVerbatim}

\sphinxstylestrong{Parameters}


\begin{savenotes}\sphinxattablestart
\centering
\begin{tabulary}{\linewidth}[t]{|T|T|}
\hline

event\_type
&
Must be a PostScript name. Either \sphinxcode{\sphinxupquote{/View}} , \sphinxcode{\sphinxupquote{/Print}} , or \sphinxcode{\sphinxupquote{/Export}} .
\\
\hline
event\_categories
&
Must be a PostScript array of name objects (typically matching keys in optional content group usage dictionaries). For a description of usage dictionaries, see the \sphinxstyleemphasis{PDF Reference} .
\\
\hline
ocgnames
&
Array of valid PostScript string variables representing optional content groups
\\
\hline
\end{tabulary}
\par
\sphinxattableend\end{savenotes}

\sphinxstylestrong{**Example**}

\begin{sphinxVerbatim}[commandchars=\\\{\}]
\PYG{o}{/}\PYG{n}{View} \PYG{p}{[}\PYG{o}{/}\PYG{n}{Zoom}\PYG{p}{]} \PYG{p}{[}\PYG{p}{(}\PYG{l+m+mi}{30}\PYG{p}{,}\PYG{l+m+mi}{000} \PYG{n}{Feet}\PYG{p}{)} \PYG{p}{(}\PYG{l+m+mi}{5}\PYG{p}{,}\PYG{l+m+mi}{000} \PYG{n}{Feet}\PYG{p}{)} \PYG{p}{(}\PYG{l+m+mi}{100} \PYG{n}{Feet}\PYG{p}{)}\PYG{p}{]} \PYG{n}{AddASEvent}
\end{sphinxVerbatim}

The example declares that the three optional content groups named \sphinxcode{\sphinxupquote{30,000 Feet}} , \sphinxcode{\sphinxupquote{5,000 Feet}} , and \sphinxcode{\sphinxupquote{100 Feet}} are to be controlled for on\sphinxhyphen{}screen viewing, based on the current zoom level and the \sphinxcode{\sphinxupquote{/Zoom}} information in each optional content group’s \sphinxcode{\sphinxupquote{Usage}} dictionary.


\subsection{BeginOC}
\label{\detokenize{pdfmark_OC_Interface:beginoc}}
The \sphinxcode{\sphinxupquote{BeginOC}} procedure is used to begin a span of content that belongs to the optional content groups supplied. It is used for nested visibility control when content can belong to more than one optional content group. Both multiple membership (using the array of optional content group names) and stack\sphinxhyphen{}based nesting are supported. The \sphinxcode{\sphinxupquote{EndOC}} calls must come before the \sphinxcode{\sphinxupquote{showpage}} call of any page. Every \sphinxcode{\sphinxupquote{BeginOC}} call should have a matching \sphinxcode{\sphinxupquote{EndOC}} call.

\begin{sphinxadmonition}{note}{Note:}
You cannot mix \sphinxcode{\sphinxupquote{SimpleOC}} and \sphinxcode{\sphinxupquote{BeginOC/EndOC}} on the same page.
\end{sphinxadmonition}
\phantomsection\label{\detokenize{pdfmark_OC_Interface:syntax-1}}
\sphinxstylestrong{Syntax}

\begin{sphinxVerbatim}[commandchars=\\\{\}]
\PYG{n}{ocgname} \PYG{n}{BeginOC}
\PYG{n}{ocgnames} \PYG{n}{BeginOC}
\PYG{n}{ocgname}
\PYG{n}{policy}
 \PYG{n}{BeginOC}
\PYG{n}{ocgnames}
\PYG{n}{policy}
 \PYG{n}{BeginOC}
\end{sphinxVerbatim}
\phantomsection\label{\detokenize{pdfmark_OC_Interface:parameters-1}}
\sphinxstylestrong{Parameters}


\begin{savenotes}\sphinxattablestart
\centering
\phantomsection\label{\detokenize{pdfmark_OC_Interface:section-1}}\nobreak
\begin{tabulary}{\linewidth}[t]{|T|T|}
\hline

ocgname
&
Array of string objects identifying a set of optional content groups.
\\
\hline
ocgnames
&
String object identifying an optional content group.
\\
\hline
policy
&
Optional. One of the following names: \sphinxcode{\sphinxupquote{/AllOn}} , \sphinxcode{\sphinxupquote{/AnyOn}} , \sphinxcode{\sphinxupquote{/AllOff}} , or \sphinxcode{\sphinxupquote{/AnyOff}} , identifying the visibility policy to use. If no \sphinxcode{\sphinxupquote{policy}} is specified, \sphinxcode{\sphinxupquote{/AnyOn}} is used by default.
\\
\hline
\end{tabulary}
\par
\sphinxattableend\end{savenotes}

\begin{sphinxadmonition}{note}{Note:}
\sphinxhref{pdfmark\_OC\_Interface.html\#50454551\_83170}{EndOC}
\end{sphinxadmonition}


\subsection{EndOC}
\label{\detokenize{pdfmark_OC_Interface:endoc}}
The \sphinxcode{\sphinxupquote{EndOC}} procedure is used to end a span of optional content. It is used to close a span of optional content started by \sphinxcode{\sphinxupquote{BeginOC}} . Both multiple membership (using an array of optional content group names) and stack\sphinxhyphen{}based nesting are supported. The \sphinxcode{\sphinxupquote{EndOC}} calls must come before the \sphinxcode{\sphinxupquote{showpage}} call of any page.

Every \sphinxcode{\sphinxupquote{BeginOC}} call should have a matching \sphinxcode{\sphinxupquote{EndOC}} call.

\begin{sphinxadmonition}{note}{Note:}
You cannot mix \sphinxcode{\sphinxupquote{SimpleOC}} and \sphinxcode{\sphinxupquote{BeginOC/EndOC}} on the same page.
\end{sphinxadmonition}
\phantomsection\label{\detokenize{pdfmark_OC_Interface:syntax-2}}
\sphinxstylestrong{Syntax}

\begin{sphinxVerbatim}[commandchars=\\\{\}]
\PYG{n}{EndOC}
\end{sphinxVerbatim}
\phantomsection\label{\detokenize{pdfmark_OC_Interface:see-also-1}}
See also

\sphinxhref{pdfmark\_OC\_Interface.html\#50454551\_51091}{BeginOC}


\subsection{GetOCGPdfmarkTag}
\label{\detokenize{pdfmark_OC_Interface:getocgpdfmarktag}}
The \sphinxcode{\sphinxupquote{GetOCGPdfmarkTag}} returns the object that the ProcSet implementation uses to identify the optional content group object for pdfmark. Using this object, the client can use the \sphinxcode{\sphinxupquote{/PUT}} pdfmark command to add additional key/value pairs to the optional content group dictionary.

The \sphinxcode{\sphinxupquote{GetOCGPdfmarkTag}} is not available in the OCProcSet \sphinxcode{\sphinxupquote{userdict}} by default. To use this procedure, you can add the following to the OCProcSet initialization code within its \sphinxcode{\sphinxupquote{\{/initialize get exec ... end\}}} clause:

\begin{sphinxVerbatim}[commandchars=\\\{\}]
\PYG{n}{userdict} \PYG{n}{begin}
      \PYG{o}{/}\PYG{n}{GetOCGPdfmarkTag} \PYG{n}{dup} \PYG{n}{OCProcSetRes} \PYG{n}{exch} \PYG{n}{get} \PYG{k}{def}
   \PYG{n+nf}{end}
\end{sphinxVerbatim}
\phantomsection\label{\detokenize{pdfmark_OC_Interface:syntax-3}}
\sphinxstylestrong{Syntax}

\begin{sphinxVerbatim}[commandchars=\\\{\}]
\PYG{n}{ocgname}
 \PYG{n}{GetOCGPdfmarkTag}
\PYG{n}{procedure}
\end{sphinxVerbatim}
\phantomsection\label{\detokenize{pdfmark_OC_Interface:parameters-2}}
\sphinxstylestrong{Parameters}


\begin{savenotes}\sphinxattablestart
\centering
\phantomsection\label{\detokenize{pdfmark_OC_Interface:section-2}}\nobreak
\begin{tabulary}{\linewidth}[t]{|T|T|}
\hline

ocgname
&
String object identifying an optional content group
\\
\hline
\end{tabulary}
\par
\sphinxattableend\end{savenotes}

Returns

The optional content group’s \sphinxcode{\sphinxupquote{/OBJ}} pdfmark tag.

\phantomsection\label{\detokenize{pdfmark_OC_Interface:see-also-2}}
See also

pdfmark \sphinxcode{\sphinxupquote{/OBJ}} and \sphinxcode{\sphinxupquote{/PUT}} commands.

\phantomsection\label{\detokenize{pdfmark_OC_Interface:example-1}}
\sphinxstylestrong{**Example**}

\begin{sphinxVerbatim}[commandchars=\\\{\}]
\PYG{p}{[}\PYG{p}{(}\PYG{n}{MyLayer}\PYG{p}{)} \PYG{n}{GetOCGPdfmarkTag} \PYG{o}{\PYGZlt{}\PYGZlt{}}\PYG{o}{/}\PYG{n}{key1} \PYG{p}{(}\PYG{n}{easy} \PYG{k}{as}\PYG{p}{)} \PYG{o}{/}\PYG{n}{key2} \PYG{l+m+mf}{3.14159}\PYG{o}{\PYGZgt{}\PYGZgt{}} \PYG{o}{/}\PYG{n}{PUT} \PYG{n}{pdfmark}
\end{sphinxVerbatim}

This example adds the key/value pairs:

\begin{sphinxVerbatim}[commandchars=\\\{\}]
\PYG{o}{/}\PYG{n}{key1} \PYG{p}{(}\PYG{n}{easy} \PYG{k}{as}\PYG{p}{)}
 \PYG{o}{/}\PYG{n}{key2} \PYG{l+m+mf}{3.14159}
\end{sphinxVerbatim}

to the dictionary for the optional content group with the name \sphinxcode{\sphinxupquote{MyLayer}} .


\subsection{OCEndPage}
\label{\detokenize{pdfmark_OC_Interface:ocendpage}}
The \sphinxcode{\sphinxupquote{OCEndPage}} is called at the end of the page in a multi\sphinxhyphen{}page PostScript file to allow the ProcSet to close any open optional content on the current page. It can be used to close a call to either \sphinxcode{\sphinxupquote{SimpleOC}} or \sphinxcode{\sphinxupquote{BeginOC}} .

\phantomsection\label{\detokenize{pdfmark_OC_Interface:syntax-4}}
\sphinxstylestrong{Syntax}

\begin{sphinxVerbatim}[commandchars=\\\{\}]
\PYG{n}{OCEndPage}
\end{sphinxVerbatim}


\subsection{SetOCGInitState}
\label{\detokenize{pdfmark_OC_Interface:setocginitstate}}
The \sphinxcode{\sphinxupquote{SetOCGInitState}} procedure sets the initial state of an optional content group to be either ON (\sphinxcode{\sphinxupquote{true}} ) or OFF (\sphinxcode{\sphinxupquote{false}} ).

\phantomsection\label{\detokenize{pdfmark_OC_Interface:syntax-5}}
\sphinxstylestrong{Syntax}

\begin{sphinxVerbatim}[commandchars=\\\{\}]
\PYG{n}{ocgname} \PYG{n+nb}{bool} \PYG{n}{SetOCGInitState}
\end{sphinxVerbatim}
\phantomsection\label{\detokenize{pdfmark_OC_Interface:parameters-3}}
\sphinxstylestrong{Parameters}


\begin{savenotes}\sphinxattablestart
\centering
\phantomsection\label{\detokenize{pdfmark_OC_Interface:section-3}}\nobreak
\begin{tabulary}{\linewidth}[t]{|T|T|}
\hline

ocgname
&
Valid PostScript string variable representing an optional content group.
\\
\hline
bool
&
\sphinxcode{\sphinxupquote{true}} or \sphinxcode{\sphinxupquote{false}} . Value of ocgname’s initial state. For a description of an optional content group’s state, see the \sphinxstyleemphasis{PDF Reference} .
\\
\hline
\end{tabulary}
\par
\sphinxattableend\end{savenotes}


\subsection{SetOCGIntent}
\label{\detokenize{pdfmark_OC_Interface:setocgintent}}
The \sphinxcode{\sphinxupquote{SetOCGIntent}} procedure sets the \sphinxcode{\sphinxupquote{Intent}} key in ocgname to intent.

\phantomsection\label{\detokenize{pdfmark_OC_Interface:syntax-6}}
\sphinxstylestrong{Syntax}

\begin{sphinxVerbatim}[commandchars=\\\{\}]
\PYG{n}{ocgname} \PYG{n}{intent} \PYG{n}{SetOCGIntent}
\end{sphinxVerbatim}
\phantomsection\label{\detokenize{pdfmark_OC_Interface:parameters-4}}
\sphinxstylestrong{Parameters}


\begin{savenotes}\sphinxattablestart
\centering
\phantomsection\label{\detokenize{pdfmark_OC_Interface:section-4}}\nobreak
\begin{tabulary}{\linewidth}[t]{|T|T|}
\hline

ocgname
&
Valid PostScript string variable representing an optional content group.
\\
\hline
intent
&
Value of ocgname’s \sphinxcode{\sphinxupquote{Intent}} key, such as \sphinxcode{\sphinxupquote{/Design}} , \sphinxcode{\sphinxupquote{/View}} , \sphinxcode{\sphinxupquote{/All}} , or \sphinxcode{\sphinxupquote{/None}} , or an array of names, excluding \sphinxcode{\sphinxupquote{/All}} and \sphinxcode{\sphinxupquote{/None}} For a description of an optional content group’s \sphinxcode{\sphinxupquote{Intent}} key, see the \sphinxstyleemphasis{PDF Reference} . The default value is \sphinxcode{\sphinxupquote{/View}} .
\\
\hline
\end{tabulary}
\par
\sphinxattableend\end{savenotes}


\subsection{SetOCGUsage}
\label{\detokenize{pdfmark_OC_Interface:setocgusage}}
The \sphinxcode{\sphinxupquote{SetOCGUsage}} procedure sets the \sphinxcode{\sphinxupquote{Usage}} key in ocgname to the dict supplied. This is the top level usage dictionary, not a usage category dictionary. Only one call per optional content group is honored, so the client must collect all usage subdictionaries and issue a single call to set the \sphinxcode{\sphinxupquote{Usage}} dictionary for the optional content group.

\phantomsection\label{\detokenize{pdfmark_OC_Interface:syntax-7}}
\sphinxstylestrong{Syntax}

\begin{sphinxVerbatim}[commandchars=\\\{\}]
\PYG{n}{ocgname} \PYG{n+nb}{dict} \PYG{n}{SetOCGUsage}
\end{sphinxVerbatim}
\phantomsection\label{\detokenize{pdfmark_OC_Interface:parameters-5}}
\sphinxstylestrong{Parameters}


\begin{savenotes}\sphinxattablestart
\centering
\phantomsection\label{\detokenize{pdfmark_OC_Interface:section-5}}\nobreak
\begin{tabulary}{\linewidth}[t]{|T|T|}
\hline

ocgname
&
Valid PostScript string variable representing an optional content group.
\\
\hline
dict
&
Value of ocgname’s \sphinxcode{\sphinxupquote{Usage}} key, which is a dictionary. For a description of an optional content group’s \sphinxcode{\sphinxupquote{Usage}} key, see the \sphinxstyleemphasis{PDF Reference} . By default, there is no \sphinxcode{\sphinxupquote{Usage}} key in the optional content group’s dictionary. This procedure simply adds the key to dict.
\\
\hline
\end{tabulary}
\par
\sphinxattableend\end{savenotes}
\phantomsection\label{\detokenize{pdfmark_OC_Interface:example-2}}
\sphinxstylestrong{**Example**}

\begin{sphinxVerbatim}[commandchars=\\\{\}]
\PYG{p}{(}\PYG{l+m+mi}{30}\PYG{p}{,}\PYG{l+m+mi}{000} \PYG{n}{Feet}\PYG{p}{)} \PYG{o}{\PYGZlt{}\PYGZlt{}}\PYG{o}{/}\PYG{n}{Zoom} \PYG{o}{\PYGZlt{}\PYGZlt{}} \PYG{o}{/}\PYG{n+nb}{max} \PYG{l+m+mf}{0.5} \PYG{o}{\PYGZgt{}\PYGZgt{}} \PYG{o}{\PYGZgt{}\PYGZgt{}} \PYG{n}{SetOCGUsage}
 \PYG{p}{(}\PYG{l+m+mi}{5}\PYG{p}{,}\PYG{l+m+mi}{000} \PYG{n}{Feet}\PYG{p}{)} \PYG{o}{\PYGZlt{}\PYGZlt{}}\PYG{o}{/}\PYG{n}{Zoom} \PYG{o}{\PYGZlt{}\PYGZlt{}} \PYG{o}{/}\PYG{n+nb}{min} \PYG{l+m+mf}{0.5} \PYG{o}{/}\PYG{n+nb}{max} \PYG{l+m+mi}{4}\PYG{o}{\PYGZgt{}\PYGZgt{}} \PYG{o}{\PYGZgt{}\PYGZgt{}} \PYG{n}{SetOCGUsage}
 \PYG{p}{(}\PYG{l+m+mi}{100} \PYG{n}{Feet}\PYG{p}{)} \PYG{o}{\PYGZlt{}\PYGZlt{}}\PYG{o}{/}\PYG{n}{Zoom} \PYG{o}{\PYGZlt{}\PYGZlt{}} \PYG{o}{/}\PYG{n+nb}{min} \PYG{l+m+mi}{4} \PYG{o}{\PYGZgt{}\PYGZgt{}} \PYG{o}{\PYGZgt{}\PYGZgt{}} \PYG{n}{SetOCGUsage}
\end{sphinxVerbatim}

This example specifies, in conjunction with the \sphinxcode{\sphinxupquote{AddASEvent}} example, that the objects in the \sphinxcode{\sphinxupquote{30,000 Feet}} optional content group should be visible when the zoom level is less than 50\%, the objects in the \sphinxcode{\sphinxupquote{5,000 Feet}} optional content group should be visible between 50\% and 400\%, and the objects in the \sphinxcode{\sphinxupquote{100 Feet}} optional content group should be visible when the zoom level is at least 400\%.


\subsection{SimpleOC}
\label{\detokenize{pdfmark_OC_Interface:simpleoc}}
The \sphinxcode{\sphinxupquote{SimpleOC}} procedure ends any current optional content span, and begins a new one where the content belongs to \sphinxcode{\sphinxupquote{ocgname}} .

To use the \sphinxcode{\sphinxupquote{SimpleOC}} procedure, simply pass in a string for the optional content group name, and all marks up to the next \sphinxcode{\sphinxupquote{SimpleOC}} belong to the optional content group with that name. Passing in \sphinxcode{\sphinxupquote{null SimpleOC}} makes subsequent content non\sphinxhyphen{}optional. At the end of the page, before the \sphinxcode{\sphinxupquote{showpage}} , issue \sphinxcode{\sphinxupquote{null SimpleOC}} .

\begin{sphinxadmonition}{note}{Note:}
You cannot mix \sphinxcode{\sphinxupquote{SimpleOC}} and \sphinxcode{\sphinxupquote{BeginOC/EndOC}} on the same page.
\end{sphinxadmonition}
\phantomsection\label{\detokenize{pdfmark_OC_Interface:syntax-8}}
\sphinxstylestrong{Syntax}

\begin{sphinxVerbatim}[commandchars=\\\{\}]
\PYG{n}{ocgname} \PYG{n}{SimpleOC}
\end{sphinxVerbatim}
\phantomsection\label{\detokenize{pdfmark_OC_Interface:parameters-6}}
\sphinxstylestrong{Parameters}


\begin{savenotes}\sphinxattablestart
\centering
\phantomsection\label{\detokenize{pdfmark_OC_Interface:section-6}}\nobreak
\begin{tabulary}{\linewidth}[t]{|T|T|}
\hline

ocgname
&
Valid PostScript string variable representing an optional content group.
\\
\hline
\end{tabulary}
\par
\sphinxattableend\end{savenotes}
\phantomsection\label{\detokenize{pdfmark_OC_Interface:example-3}}
\sphinxstylestrong{Example}

To show content on all layers (at all times):

\begin{sphinxVerbatim}[commandchars=\\\{\}]
\PYG{n}{null} \PYG{n}{SimpleOC}
\end{sphinxVerbatim}



\renewcommand{\indexname}{Index}
\printindex
\end{document}